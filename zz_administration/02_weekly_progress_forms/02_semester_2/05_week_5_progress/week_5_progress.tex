%----------------------- Weekly Progress Document ------------------------
%
% Created by: Shane Reynolds 2020-02-29
%
%-------------------------------------------------------------------------

%-------------- Preamble
\documentclass[12pt]{article}
\input{preamble/weekly_progress_preamble} % this must be left as \input, \include wont work in the preamble

%-------------- Information For The Title Page
\title{	
		Thesis Progress Form\\
		CHARLES DARWIN UNIVERSITY\\
		College of Engineering, IT, and Environment
	  }
\author{}
\date{}

%------------------------ Main Document --------------------------
\begin{document}
	
	\maketitle
	
	\begin{namelist}{xxxxxxxxxxxx}
		\item[{\bf Name:}]
			Shane Reynolds
		\item[{\bf Unit:}]
			ENG720
		\item[{\bf Title:}]
			Automatic generation control of a two area power system using deep reinforcement learning
		\item[{\bf Supervisors:}]
			Charles Yeo \& Stefanija Klaric
		\item[{\bf Time \& Date:} 19/8/2020 \ @ 1pm]
			
	\end{namelist}
	
	\pagestyle{plain} % get rid of fancy headers
	\textheight = 565pt % hack to include page numbers
	
	%-------------- Sections
	\section{Progress since last meeting}
	\begin{itemize}
		\item Completed experiment in which agent memory was filled with tuned PID controller experience 50\% of the time. The controller learned a useful policy much quicker; however, was unable to 
		\item Developed stochastic demand profile to use in final experiments.
	\end{itemize}
	\section{Discussion Points}
	\begin{itemize}
		\item Outlined recent progress to CY, as described above.
		\item Outlined new experiments to CY, as detailed below.
	\end{itemize}
	\section{Plan until the next meeting}
	\begin{itemize}
		\item Run final experiments that will use a stochastic load demand profile to demonstrate that load frequency control can be achieved with a neural network using DDPG; however, this will not beat the current industry PID standard approach to control.
		\item Look at finding a way to present research with a narrative that shows training a neural network for load frequency control (and outperforming a PIC controller) is not possible. This would require some mathematical analysis.
	\end{itemize}
	%-------------- Supervisor sign-off
	\par
	\vspace{\fill}%
	\noindent\rule{0.4\linewidth}{0.5pt}%
	\vspace{1em}%
	\par
	\noindent\textbf{Supervisor}\vspace{1em}%
	\par
	\noindent\today

\end{document}