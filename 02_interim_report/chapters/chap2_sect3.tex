\section{Deep Neural Networks}\label{dnn}
Deep neural networks are the technology responsible for some of the most recent state-of-the-art technological breakthroughs in fields such as audio to text speech recognition systems \cite{Hinton2012}, image classification systems \cite{Krizhevsky2012, Simonyan2014, Szegedy2015, He2016}, text-to-text machine translation (REFERENCE), and robotics \cite{Mnih2015, Lillicrap2015, Schulman2015, Schulman2015highdimensional}.

Deep neural networks are able to adapt to the different needs of diverse research fields due to their unique computational architecture. A deep neural network (DNN) is comprised nodes, called perceptrons, which perform simple linear combination operations on inputs they receive. Additionally, each node is equipped with an activation function. The activation function allows the node to signal when it recognises the linear combination of inputs. DNN architectures connect many nodes together, using weighted edges, to form a computational graph. Modifying the edge weights is referred to as \textit{training the network}, and allows the DNN to change its behaviour. Given this flexibility, a DNN is often thought of as a tool for universal function approximation.


%------------------------ SS: Perceptron Model

\subsection{Perceptron Model}
Rosenblatt is credited with developing the perceptron model that is a fundamental building block for neural network architectures. Motivated by Hebbian theory of synaptic plasticity (i.e. the adaptation of brain neurons during the learning process), Rosenblatt developed a model to emulate the ``perceptual processes of a biological brain'' \cite{Rosenblatt1957}. Rosenblatt's perceptron model consisted of a single node used for binary classification of patterns that are linearly separable \cite{Rosenblatt1958}. The neuron takes a vector of inputs and applies a weight, multiplicatively, to each vector element. The multiplied elements are then summed along with a bias term. Letting input vector elements be $x_i$, weight terms be $w_i$, and the bias term be $b$, the summation operation can be expressed as:
\begin{equation}
	\sum_{i}x_i w_i + b \label{eq:2301}
\end{equation} 

The summation is then passed through an activation function, $f$, to produce the neuron output. Using equation \ref{eq:2301} and letting the neuron output be $y$, the neuron model can be expressed as:
\begin{equation}
	y = f\bigg( \sum_{i}x_i w_i + b \bigg) \label{eq:2302}
\end{equation}

Figure \ref{fig:2301_perceptron_model} provides an shows the computational model of a neuron, expressing \ref{eq:2302}.

\begin{figure}[h]
	\centering
	%------------------------ Set up drawing styles

\tikzset{basic/.style={draw,fill=blue!20,text width=1em,text badly centered}}
\tikzset{input/.style={basic,circle}}
\tikzset{weights/.style={basic,rectangle,fill=blue!20}}
\tikzset{functions/.style={basic,circle,fill=blue!20}}

\begin{tikzpicture}
   	\node[functions] (center) {$f$};
	\node[below of=center,font=\scriptsize,text width=4em] {Activation function};
	\node[right of=center] (right) {$y$};
    	\path[draw,->] (center) -- (right);
	\node[functions,left=3em of center] (left) {$\sum$};
    	\path[draw,->] (left) -- (center);
	\node[weights,left=3em of left] (2) {$w_2$} -- (2) node[input,left of=2] (l2) {$x_2$};
    	\path[draw,->] (l2) -- (2);
    	\path[draw,->] (2) -- (left);
	\node[below of=2] (dots) {$\vdots$} -- (dots) node[left of=dots] (ldots) {$\vdots$};
	\node[weights,below of=dots] (n) {$w_n$} -- (n) node[input,left of=n] (ln) {$x_n$};
    	\path[draw,->] (ln) -- (n);
    	\path[draw,->] (n) -- (left);
	\node[weights,above of=2] (1) {$w_1$} -- (1) node[input,left of=1] (l1) {$x_1$};
    	\path[draw,->] (l1) -- (1);
    	\path[draw,->] (1) -- (left);
	\node[weights,above of=1] (0) {$b$} -- (0) node[input,left of=0] (l0) {$1$};
    	\path[draw,->] (l0) -- (0);
    	\path[draw,->] (0) -- (left);
	\node[below of=ln,font=\scriptsize] {inputs};
	\node[below of=n,font=\scriptsize] {weights};
\end{tikzpicture}
	\caption{text}
	\label{fig:2301_perceptron_model}
\end{figure}


%------------------------ SS: Activation Functions

\subsection{Activation Functions}
The activation function is a key component of Rosenblatt's perceptron --- it determines if the perceptron will activate or not. Rosenblatt's model used a Heaviside step function, which was effective under his simple perceptron learning algorithm; however, as neural networks grew in size, new training algorithms had to be developed. The most widely used learning algorithm today is backpropagation (REFERENCE). This algorithm relies on the calculation of gradients Since the Heaviside function has no gradient, it was ultimately seen as an impediment to network learning once the back-propagation learning algorithm was developed. According to Haykin, a number of other activation function are widely used instead of the Heaviside function \cite{Haykin99}. Some of the most common are:
\begin{enumerate}
	\item Hyperbolic tangent: $f:\mathbb{R} \to [-1,1]$ where
	\begin{equation}
		f(x) = \frac{e^x - e^{-x}}{e^x + e^{-x}}
	\end{equation}
	\item Logistic sigmoid: $f:\mathbb{R} \to [0,1]$ where
	\begin{equation}
		f(x) = \frac{1}{1 + e^{-x}}
	\end{equation}
\end{enumerate}

Both of these activation function suffer from a problem called the vanishing gradient problem. The back-propagation learning algorithm  




%------------------------ SS: Feedforward Network

\subsection{Feedforward Networks}
A typical fully connected feed-forward ANN consists of an input layer, one or more hidden layers, and an output layer, as shown in Figure 4. Hidden layers are made up of multiple nodes. The nodes themselves contain a non-linear activation function, such as a sigmoid or ReLU, and receive weighted input from the previous layers in the model.


%------------------------ SS: Training the Network

\subsection{Training the Network}
Changing the weights in a neuron changes the neurons’s contribution to the model, which in turn affects the overall model output. Weight changes occur during model training, which uses large volumes of labelled data to adjust the weights. Hidden layers are important because they allow highly non-linear models to be constructed, providing an approach for estimating complex phenomena which may be difficult to model with classical approaches, or computationally intractable. Generally, the more hidden layers, the more non-linear the model. Network architectures with multiple hidden layers have become so wide spread that the term Deep Neural Network (DNN) was coined to describe feed-forward ANNs which use two or more hidden layers. It must be noted that whilst increased non-linearity may allow us to model more complex phenomenon, making the ANN deeper does not guarantee increased model performance. This is mainly due to the fact that deeper models may over-fit the data during training, resulting in a failure to generalise on test and validation data sets.

werbos backpropagation


%------------------------ SS: Regularisation

\subsection{Regularisation}
dropout