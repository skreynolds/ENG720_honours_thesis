\chapter{Introduction}
In 2018, approximately 261$\si{\tera\watt\hour}$ of power was generated in the Australian electricity sector. Renewables contributed 19\% of the total generation, an increase from 15\% in 2017. The Department of Industry, Science, Energy and Resources have observed an increase in renewable energy generation year-on-year in the electricity generation market since 2008, as shown in Figure \ref{fig:101_renewable_energy} \cite{Diser2020}.\\

\begin{figure}[ht]
	\centering
	\begin{tikzpicture}
	\begin{axis}[
		no markers,
		tick pos=left,
		width=0.8\textwidth,
		height=0.35\textwidth,
		xlabel=Year,
		x label style={at={(axis description cs:0.5,-0.2)},font=\scriptsize},
		xticklabel style={/pgf/number format/1000 sep=,rotate=60,anchor=east,font=\scriptsize},
		ylabel=Terrawatt hours,
		y label style={font=\scriptsize},
		yticklabel style={/pgf/number format/1000 sep=,font=\scriptsize},
		legend pos={north west},
		legend style={font=\scriptsize}
	]
	\addplot[green!20!gray, ultra thick] table[mark=none, x=year, y=total, col sep=comma] {./figures/101_renewable_energy/renewable_energy.csv};
	\addlegendentry{Total}
	\addplot[green!20!gray, ultra thick, dashed] table[mark=none, x=year, y=renew, col sep=comma] {./figures/101_renewable_energy/renewable_energy.csv};
	\addlegendentry{Renewable}
	\end{axis}
\end{tikzpicture}
	\caption[Renewable power generation versus total power generation 1977 to 2018]{Power generation from renewable sources (dashed line), and total power generation (solid line) in Australia from 1977 to 2018.}
	\label{fig:101_renewable_energy}
\end{figure}

One of the benefits of transitioning from thermal sources of power generation to renewable sources is reduced greenhouse gas emissions \cite{IPCC2012}; however, this transition is not without its drawbacks. With an increased reliance on renewable power generation sources posing challenges for power system stability owing to load management. A recent example is the system failure in Alice Springs, caused by an event cascade that was triggered by cloud cover shadowing a solar array. The system failure left residents in Alice Springs without power for approximately eight hours \cite{UCNT2019}. An independent investigation highlighted that poor control policies were one of the factors that contributed to the blackout. In this instance, the generator provisioned to ramp up in the event of cloud cover was unable to be controlled. Moreover, generators that were still under the control regime were issued operating set points above their rated capacity, that resulted in thermal overload and subsequent tripping from the protection system \cite{Wilkey2019}.

Current control methods use classical feedback loop techniques. These methods can be brittle when faced with system changes, or scenarios which they were not designed for. An improved controller would be one that can learn and adapt its controller to an unseen system or event, given some broad control objective. This research proposes a deep reinforcement learning (DRL) agent for controlling the frequency of a power system consisting of multiple generators, and multiple load demands with individual stochastic profiles.

\section{Research Aim}
The principle aim of this research is to compare the performance of known, optimised feedback loop controller architectures against a DRL based control system when tasked with performing load following ancillary services with regulating generators under AGC for a two area power system. This research will be undertaken in order to understand the feasibility of using DRL agents for two area power system management.

\section{Structure of Interim Report}
The remainder of this report is structured as follows:

Chapter 2 introduces the necessary background to understand work presented later in the report. This includes a formal introduction to reinforcement learning, deep neural networks, and deep reinforcement learning, including associated mathematical preliminaries.

Chapter 3 undertakes a literature review exploring different technologies used to address the load frequency control problem. Topics discussed include feedback loops, fuzzy logic, genetic algorithms, and artificial neural networks. The chapter concludes with a review of DRL applications to load frequency control and the motivation for using these controllers.

Chapter 4 outlines the research approach, providing a discussion on model development and implementation, a framework for preliminary investigations, and concludes by establishing the main experimental approach for assessing DRL controller feasibility for load frequency control.

Chapter 5 details the presents results from a partially completed preliminary investigation.

Chapter 6 provides a discussion on findings from Chapter 5 and evaluates feasibility of the proposed DRL controller. Chapter concludes with recommendations for the continued direction of this work.