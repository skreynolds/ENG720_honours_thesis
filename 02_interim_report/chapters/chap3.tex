\chapter{Literature Review}
Power systems are non-linear; however, traditional control structures for maintaining power system frequency such as load frequency control (LFC) or automatic generation control (AGC), are designed under the assumption that plant modelled using linear ordinary differential equations can capture the dynamic behaviour of existing power systems. If frequency deviations are small, then this assumption is reasonable for the amount of non-linearity present in existing power systems. If frequency deviations are large, or additional non-linearity were introduced to the system, then it is no longer reasonable to assume linear ODE system models.

Owing to an increase in the proportion of photovoltaic power generation, along with an increase in the use of high voltage direct current (HVDC) transmission lines in Australia's power network, power system dynamics are becoming more non-linear (REFERENCE). This is creating a need to explore novel control architectures for AGC in order to improve control performance for the non-linear system. One such control architecture being investigated is deep reinforcement Learning (DRL). To fully grasp this problem two key areas of understanding are necessary. Firstly, it is important to know what LFC is and the control architectures that have previously been explored to address this problem. Secondly, knowledge of DRL and its historical applications to control problems is required to understand strengths, limitations, and underlying assumptions of the architecture.

\section{Automatic Generation Control}\label{agc}
Since Thomas Edison's first commercial power station, commissioned in 1882 at 255-257 Pearl Street New York, controlling power frequency has been a key factor in power generation \cite{Cohn1983}. One of the first attempts to control the frequency for a single generator and load system was with a device called a turbine governor (REFERENCE). The turbine governor was designed with an in-built proportional feedback control loop (REFERENCE). DESCRIBE HOW THIS ACTUALLY WORKS. The literature often refers to this as the primary control loop \cite{Bevrani2011}. Operators often found that primary loop governor control would require constant fine tuning to ensure that the power system was operating at the scheduled frequency (REFERENCE). XXXX undertook a mathematical analysis using first order linear models for the governor, turbine, and generation load control (REFERENCE). The analysis showed that primary control with a governor was successful in arresting frequency deviations from the desired set point, but persistent offset errors from the set point prevented the uptake of the technology \cite{Saadat}. Later research concluded that a secondary control loop to the governor was required to provide sufficient frequency control \cite{Elgerd1970}. The secondary control loop provided integral control (REFERENCE). Mathematical analysis, using first order linear models for plant showed that atuned proportional-integral (PI) controller was able to arrest frequency deviations and subsequently restore the system frequency to it's scheduled value. The PI control scheme constitutes the classical approach to the solving the LFC problem (REFERENCE).

Cohn \cite{Cohn1971} and Aggarwal \textit{et al.} \cite{Aggarwal1968, Aggarwal1968a} undertook pioneering work to develop classical control approaches to work with power systems comprised of two or more control areas. A system made up of more than one power system area required frequency control, but also the control of the power flow over the transmission infrastructure (tie lines) which connected the areas (REFERENCE). Cohn's paper used a first order linear system to model the tie line dynamics. The development of a tie line model allowed Cohn to use PI control architectures for each power system area, albeit with modified input signals (REFERENCE). One of the most important things to come out of his work was development of the feedback signal called area control error (ACE). Cohn used ACE to ensure power systems were restored to the scheduled frequency, given a frequency deviations, and that unscheduled tie line power flows were minimised between neighbouring control areas \cite{Cohn1956}. Classical power system frequency controllers can be designed using Bode and Nyquist diagrams to obtain desired gain and phase margins. Root locus plots can also be used \cite{Ogat2010}. While these approaches are simple, well known, and easy for practical implementation, investigations using these approaches have resulted in control schemes that exhibit poor dynamic performance. This is especially true in the presence of parameter variations and nonlinearities \cite{Kundur1994, Elgerd1970, Bechert1977}.

Frequency controller analysis and design assumes plant models have linear dynamics; however, studies have shown that modern power systems display complex non-linear dynamics \cite{Concordia1957, Kwatny1975, Elgerd1994, Morsali2014}. Modern power systems are large-scale and comprised of multiple power generation sources such as thermal, hydro, and photovoltaic power --- some of the more commonly researched generator non-linearity include governor dead band (GDB) \cite{Concordia1957} and generator ramp constraint (GRC) \cite{Kwatny1975, Elgerd1994}. Moreover, modern power systems use high voltage direct current (HVDC) lines to export power over long distances, and they also feature energy storage systems such as pumped hydro or batteries \cite{Bevrani2011, Glover2012, Kothari2011, Kundur1994}. Both of these features display highly non-linear characteristics (REFERENCE).

Linear ODE power system models capture underlying plant characteristics; however, these models are only valid within certain operating ranges. Non-linear plant characteristics mean that different linear ODE models are required as plant operating conditions change. Governor dead band is observed as a change in generator angular velocity for which there is no change in the governor valve position. GDB is generally attributed to backlash in the governor mechanism, and degrades LFC performance (REFERENCE). GRC is a physical limitation of the turbine that imposes upper and lower boundaries on the rate of change in generating power from the turbine \cite{Morsali2014}. In recent years, frequency control methods using fuzzy logic, genetic algorithms (GAs), and artificial neural networks (ANNs), have attempted way to address the problems that arise due to non-linearity.

\subsection{Fuzzy Logic Control}
Fuzzy logic control schemes are developed directly from power system domain experts or operators who control plant manually. Researchers have shown that a fuzzy gain scheduling PI controller can perform as well as a fixed gain controller, for frequency control of two and multi-area power systems. Moreover, it was found fuzzy controllers are simpler to implement \cite{Chang1997, Cam2005}. Yesil et al. \cite{Yesil2004} proposed a self-tuning fuzzy PID controller for a two area power system and noted improvements in controller transient performance when compared to a fuzzy gain scheduling PI controller.

\subsection{Genetic Algorithms}
Genetric algorithms are stochastic global search algorithms based on natural selection. In the context of power system control, GAs operate on a population of individuals. An individual is a set of control system parameters which are initially drawn at random and without knowledge of the task domain. Successive generations of individuals are developed using genetic operations such as recombination or mutation. An individuals chance of being selected for used in an genetic operation is based on an objective measure of fitness --- strong individuals are retained and weak individuals are discarded \cite{Fleming1993}.

Chang et al. \cite{Chang1998} investigated using GA to determine fuzzy PI controller gains, which resulted in a control scheme which performed favourably when compared to a fixed-gain controller. Rekpreedapong et al. \cite{Rerkpreedapong2003} took this one step further by optimally tuning PI controller gains with GA while using linear matrix inequalities (LMI) constraints from a higher order controller. This research, performed on a three area control system, was motivated by the belief higher order controllers are not practical for industry. Rekpreedapong et al. concluded that the GA tuned PI controller, under LMI constraints, performed almost as well a higher order control system. Research undertaken by Ghosal \cite{Ghoshal2004} concluded that PID control with gains optimised by GA provided better transient performance than PI control with gains optimised in the same way.

\subsection{Artificial Neural Networks}
Artificial neural networks are systems that take input signals and, using many simple processing elements, produce output signals. The processing elements, or neurons, each have a number of internal parameters referred to as weights. Changing a weight will change the behaviour of a neuron. If many weights are changed, the behaviour of the ANN can be changed. The goal is to choose weights of the network in order to achieve the desired input/output relationship --- this is called training the network \cite{Nguyen1990}.

Beaufays et al. \cite{Beaufays1994} demonstrated it was possible to used a neural network for frequency control in one and two-area power systems. The ANN replaced the integral controller in the classical structure; however, employed a state variable vector input containing frequency deviation and tie-line power measurements instead of a single value ACE signal seen with classical controllers. The network was trained using a back propagation through time algorithm, and resulted in better transient performance when compared with a classical PI controller. Using these results, Demiroren et al. \cite{Demiroren2001} went further by including non-linearity in the plant models. Specifically, governor deadband, reheater effects, and generating rate constraints are included and it was shown that the results obtained using the ANN controller outperformed the results of a standard PI classical control model for a two-area power system. Research undertaken a year later confirmed these results for a larger four-area power system with thermal and hydro generation sources \cite{Zeynelgil2002}. 

% % % % % % % % % % % % % % % % % % % % % % % % % % % % % % % % % % % % % % % % % %
% Need to include some evaluation of the above fuzzy logic, GA, and ANN approaches
% % % % % % % % % % % % % % % % % % % % % % % % % % % % % % % % % % % % % % % % % %

\section{Deep Reinforcement Learning}
