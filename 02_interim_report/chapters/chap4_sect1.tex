\section{Model Development and Implementation}
Models were developed in a modular fashion breaking up the environment and controllers so that only one environmental model had to be created

%---------------- SS: Environment

\subsection{Environment Model} \label{ssec:env_modelling}
The two area power system model described in \textsection \ref{ssec:modelling_two_area_system}, and shown in Figure \ref{fig:208_two_area_pi_control_model}, was converted from the frequency domain to the temporal domain. The main reason for taking this approach is to provide reinforcement learning architectures, outlined in \textsection \ref{sec:reinforcement_learning}, with the ability to export control signals to the power system at each time step. This approach is a common practice when developing environments for reinforcement learning \cite{Brockman2016}. Additionally, frequency domain simulation techniques, such as Laplacian transforms, have strict initial condition assumptions \cite{Ogat2010}, that would limit the reach of research findings to real world applications.

Higher order ordinary differential equations and systems involving higher order ordinary differential equations would provide a more compact system representation; however, to accommodate numerical analysis schemes, such as Runge-Kutta, the environment system was expressed as a first order system of linear ordinary differential equations.

\begin{figure}[h]
	\centering
	\resizebox{\textwidth}{!}{%----------- Create a fancy summing block
\tikzset{add/.style n args={4}{
		minimum width=6mm,
		path picture={
			\draw[black] 
			(path picture bounding box.south east) -- (path picture bounding box.north west)
			(path picture bounding box.south west) -- (path picture bounding box.north east);
			\node at ($(path picture bounding box.south)+(0,0.13)$)     {\tiny #1};
			\node at ($(path picture bounding box.west)+(0.13,0)$)      {\tiny #2};
			\node at ($(path picture bounding box.north)+(0,-0.13)$)    {\tiny #3};
			\node at ($(path picture bounding box.east)+(-0.13,0)$)     {\tiny #4};
		}
	}
}

%----------- Block style 1
\tikzstyle{block1} = [draw, fill=white!80!green, rectangle, 
minimum height=3em, minimum width=6em, node distance=2.5cm]

%----------- Block style 2
\tikzstyle{block2} = [draw, fill=white!80!green, rectangle, 
minimum height=3em, minimum width=3em, node distance=2.5cm]

%----------- Sum style
\tikzstyle{sum} = [draw, fill=white!80!green, circle, node distance=2cm]

%----------- Input style
\tikzstyle{input} = [coordinate, node distance=4cm]

%----------- Output style
\tikzstyle{output} = [coordinate, node distance=4cm]

%----------- Pin style
\tikzstyle{pinstyle} = [pin edge={to-,thin,black}]


\begin{tikzpicture}	
	% Tie line nodes
	\node [sum, add={$-$}{}{+}{ }] (sum1) {};
	\node [block1, right of=sum1, label=above:{Tie Line}] (tieline) {$\frac{2\pi T_{12}}{s}$};
	\node [output, right of=tieline, node distance=3cm] (out) {};
	\node [coordinate, above of=out, node distance=5cm] (c1) {};
	\node [coordinate, below of=out, node distance=5cm] (c2) {};
	\node [block2, left of=c2] (a12) {$-a_{12}$};
	
	% Position a reference coordinate for drawing
	\node [coordinate, left of=sum1, node distance=2.5cm] (c3) {};
	\node [coordinate, above of=c3, node distance=0.75cm] (c4) {};
	\node [coordinate, below of=c3, node distance=0.75cm] (c5) {};
	
	% Create nodes for upper leg
	\node [block1, above of=c3, node distance=3.5cm, label=above:{Gen. Load 1}] (genload1) {$\frac{K_{gl1}}{T_{gl1}s+1}$};
	\node [coordinate, right of=genload1, node distance=1.5cm] (c6) {};
	\node [sum, left of=genload1, add={$-$}{+}{$-$}{}, node distance=2.5cm] (sum2) {};
	\node [coordinate, below of=sum2] (p11) {};
	\node [coordinate, left of=p11, node distance=0.5cm, label=left:{$\Delta P_{L1}(s)$}] (p12) {};
	\node [block1, left of=sum2, node distance=3.5cm, label=above:{Turbine 1}] (turbine1) {$\frac{K_{t1}}{T_{t1}s+1}$};
	\node [block1, left of=turbine1, node distance=4.5cm, label=above:{Governor 1}] (governor1) {$\frac{K_{g1}}{T_{g1}s+1}$};
	\node [coordinate, left of=governor1, node distance=3cm] (c8) {};
	\node [coordinate, left of=c5, node distance=13.5cm] (c10) {};
	\node [coordinate, left of=c1, node distance=21.5cm] (c12) {};
	
	
	
	% Create nodes for lower leg
	\node [block1, below of=c3, node distance=3.5cm, label=above:{Gen. Load 2}] (genload2) {$\frac{K_{gl1}}{T_{gl1}s+1}$};
	\node [coordinate, right of=genload2, node distance=1.5cm] (c7) {};
	\node [sum, left of=genload2, add={$-$}{+}{$-$}{}, node distance=2.5cm] (sum3) {};
	\node [coordinate, above of=sum3] (p21) {};
	\node [coordinate, left of=p21, node distance=0.5cm, label=left:{$\Delta P_{L2}(s)$}] (p22) {};
	\node [block1, left of=sum3, node distance=3.5cm, label=above:{Turbine 2}] (turbine2) {$\frac{K_{t2}}{T_{t2}s+1}$};
	\node [block1, left of=turbine2, node distance=4.5cm, label=above:{Governor 2}] (governor2) {$\frac{K_{g2}}{T_{g2}s+1}$};
	\node [coordinate, left of=governor2, node distance=3cm] (c9) {};
	\node [coordinate, left of=c4, node distance=13.5cm] (c11) {};
	\node [coordinate, left of=a12, node distance=19cm] (c13) {};
	
	
	% Connect the tieline nodes
	\draw [->] (sum1) -- (tieline);
	\draw (tieline) -- node [label=below:{$x_5(t)$}] {} (out);
	
	% Connect nodes in upper block
	\draw (out) -- (c1);
	\draw [->] (c1) -| (sum2);
	
	\draw [->] (governor1) -- node [label=above:{$x_2(t)$}] {} (turbine1);
	\draw [->] (turbine1) -- node [label=above:{$x_3(t)$}] {} (sum2);
	\draw [->] (sum2) -- (genload1);
	\draw [->] (genload1) -| node [label=above:{$x_4(t)$}] {} (sum1);
	\draw (c6) |- (c4);
	\draw [->] (c8) -- node [label=above:{$u_1(t)$}] {} (governor1);
	\draw (p12) -- (p11);
	\draw [->] (p11) -- (sum2);
	\draw [->] (c4) -- (c11);
	\draw [->] (c1) -- (c12);
	
	
	% Connect nodes in lower block
	\draw (out) -- (c2);
	\draw [->] (c2) -- (a12);
	\draw [->] (a12) -| (sum3);
	\draw [->] (governor2) -- node [label=below:{$x_7(t)$}] {} (turbine2);
	\draw [->] (turbine2) -- node [label=below:{$x_8(t)$}] {} (sum3);
	\draw [->] (sum3) -- (genload2);
	\draw [->] (genload2) -| node [label=below:{$x_9(t)$}] {} (sum1);
	\draw (c7) |- (c5);
	\draw [->] (c9) -- node [label=below:{$u_2(t)$}] {} (governor2);
	\draw (p22) -- (p21);
	\draw [->] (p21) -- (sum3);
	\draw [->] (c5) -- (c10);
	\draw [->] (a12) -- (c13);
	
\end{tikzpicture}}
	\caption{text}
	\label{4101_two_area_power_system_temporal_model}
\end{figure}

Suppose variables are assigned to the power system according to Figure \ref{4101_two_area_power_system_temporal_model}. The first order system of linear ordinary differential equations for area 1 is:
\begin{align}
	\dot{x}_2(t) &= \frac{1}{T_{sg_1}}\big( K_{sg_1} u_1(t) - x_2(t) \big) \label{eq:4101} \\
	\dot{x}_3(t) &= \frac{1}{T_{t_1}} \big( K_{t_1} x_2(t) - x_3(t) \big) \label{eq:4102} \\
	\dot{x}_4(t) &= \frac{1}{T_{gl_1}} \bigg( K_{gl_1} \big( x_3(t) - x_5(t) - \Delta p_{L1}(t) \big) - x_4(t) \bigg) \label{eq:4103}  \\
	\dot{x}_5(t) &= 2 \pi T_{12} \big( x_4(t) - x_9(t) \big) \label{eq:4104} \\
	\dot{x}_7(t) &= \frac{1}{T_{sg_2}}\big( K_{sg_2} u_2(t) - x_7(t) \big) \label{eq:4105} \\
	\dot{x}_8(t) &= \frac{1}{T_{t_2}} \big( K_{t_2} x_7(t) - x_8(t) \big) \label{eq:4106} \\
	\dot{x}_9(t) &= \frac{1}{T_{gl_2}} \bigg( K_{gl_2} \big( x_8(t) - x_5(t) - \Delta p_{L2}(t) \big) - x_9(t) \bigg) \label{eq:4107}
\end{align}

A full derivation of the first order linear system described by equations \ref{eq:4101} through \ref{eq:4107} is described in Appendix \ref{app:C1_power_system_ode}. The system of equation were implemented as a method \verb|int_power_system_sim| in a Python class \verb|TwoAreaPowerSystemEnv|. Implementation for \verb|int_power_system_sim| is detailed in Appendix \ref{app:implementation_of_power_system_model}.



%---------------- SS: PI Controller

\subsection{Classical PI Controller Model}
The proportional integral (PI) controller model described in \textsection \ref{ssec:control_two_area_system}, and shown in Figure \ref{fig:208_two_area_pi_control_model}, was also converted from the frequency domain to the temporal domain to ensure compatibility with the environment model.

\begin{figure}[h]
	\begin{minipage}[b]{0.5\textwidth}
		\resizebox{7.0cm}{!}{%----------- Create a fancy summing block
\tikzset{add/.style n args={4}{
		minimum width=6mm,
		path picture={
			\draw[black] 
			(path picture bounding box.south east) -- (path picture bounding box.north west)
			(path picture bounding box.south west) -- (path picture bounding box.north east);
			\node at ($(path picture bounding box.south)+(0,0.13)$)     {\tiny #1};
			\node at ($(path picture bounding box.west)+(0.13,0)$)      {\tiny #2};
			\node at ($(path picture bounding box.north)+(0,-0.13)$)    {\tiny #3};
			\node at ($(path picture bounding box.east)+(-0.13,0)$)     {\tiny #4};
		}
	}
}

%----------- Block style 1
\tikzstyle{block1} = [draw, fill=white!80!green, rectangle, 
minimum height=3em, minimum width=6em, node distance=2.5cm]

%----------- Block style 2
\tikzstyle{block2} = [draw, fill=white!80!blue, rectangle, 
minimum height=3em, minimum width=3em, node distance=2.5cm]

%----------- Sum style
\tikzstyle{sum} = [draw, fill=white!80!blue, circle, node distance=2cm]

%----------- Input style
\tikzstyle{input} = [coordinate, node distance=4cm]

%----------- Output style
\tikzstyle{output} = [coordinate, node distance=4cm]

%----------- Pin style
\tikzstyle{pinstyle} = [pin edge={to-,thin,black}]

\begin{tikzpicture}	
	% Initial position node
	\node [coordinate] (c1) {};
	
	
	% Create nodes for upper leg
	\node [sum, above of=c1, add={$-$}{}{+}{}, node distance=3.5cm] (sum4) {};
	\node [coordinate, above of=sum4, node distance=1.25cm] (c10) {};
	\node [coordinate, right of=c10, node distance=2cm] (c12) {};
	\node [coordinate, right of=sum4, node distance=2cm] (c2) {};
	\node [coordinate, above of=c2] (c4) {};
	\node [block2, below of=sum4, node distance=1.75cm] (r1) {$R_1$};
	\node [coordinate, below of=r1] (c6) {};
	\node [coordinate, right of=c6, node distance=2cm] (c8) {};
	\node [block2, left of=sum4] (int1) {$\frac{K_{i_1}}{s}$};
	\node [sum, left of=int1, add={+}{ }{+}{ }] (sum6) {};
	\node [block2, below of=sum6, node distance=1.75cm] (b1) {$b_1$};
	
	
	% Connect nodes
	\draw [->] (sum4) -- node [at end, label=right:{$u_1(t)$}] {} (c2);
	\draw [->] (r1) -- (sum4);
	\draw [->] (b1) -- (sum6);
	\draw [->] (sum6) -- (int1);
	\draw [->] (int1) -- node [label=above:{$x_1(t)$}] {} (sum4);
	\draw [->] (c8) -| node [at start, label=right:{$x_4(t)$}] {} (r1);
	\draw [->] (c8) -| (b1);
	\draw [->] (c12) -| node [at start, label=right:{$x_5(t)$}] {} (sum4);
	\draw [->] (c12) -| (sum6);
	
\end{tikzpicture}}
		\caption{}
		\label{fig:4102_two_area_pi_controller_temporal_1}
	\end{minipage}
	\hspace{0.1cm}
	\begin{minipage}[b]{0.5\textwidth}
		\resizebox{7.2cm}{!}{%----------- Create a fancy summing block
\tikzset{add/.style n args={4}{
		minimum width=6mm,
		path picture={
			\draw[black] 
			(path picture bounding box.south east) -- (path picture bounding box.north west)
			(path picture bounding box.south west) -- (path picture bounding box.north east);
			\node at ($(path picture bounding box.south)+(0,0.13)$)     {\tiny #1};
			\node at ($(path picture bounding box.west)+(0.13,0)$)      {\tiny #2};
			\node at ($(path picture bounding box.north)+(0,-0.13)$)    {\tiny #3};
			\node at ($(path picture bounding box.east)+(-0.13,0)$)     {\tiny #4};
		}
	}
}

%----------- Block style 1
\tikzstyle{block1} = [draw, fill=white!80!blue, rectangle, 
minimum height=3em, minimum width=6em, node distance=2.5cm]

%----------- Block style 2
\tikzstyle{block2} = [draw, fill=white!80!blue, rectangle, 
minimum height=3em, minimum width=3em, node distance=2.5cm]

%----------- Sum style
\tikzstyle{sum} = [draw, fill=white!80!blue, circle, node distance=2cm]

%----------- Input style
\tikzstyle{input} = [coordinate, node distance=4cm]

%----------- Output style
\tikzstyle{output} = [coordinate, node distance=4cm]

%----------- Pin style
\tikzstyle{pinstyle} = [pin edge={to-,thin,black}]

\begin{tikzpicture}	
	% Initial position node
	\node [coordinate] (c1) {};
	
	
	% Create nodes for lower leg
	\node [sum, below of=c1, add={+}{}{$-$}{}, node distance=3.5cm] (sum5) {};
	\node [coordinate, below of=sum5, node distance=1.25cm] (c11) {};
	\node [coordinate, right of=c11, node distance=2cm] (c13) {};
	\node [coordinate, right of=sum5, node distance=2cm] (c3) {};
	\node [coordinate, above of=c3] (c5) {};
	\node [block2, above of=sum5, node distance=1.75cm] (r2) {$R_2$};
	\node [coordinate, above of=r2] (c7) {};
	\node [coordinate, right of=c7, node distance=2cm] (c9) {};
	\node [block2, left of=sum5] (int2) {$\frac{K_{i_2}}{s}$};
	\node [sum, left of=int2, add={+}{ }{+}{ }] (sum7) {};
	\node [block2, above of=sum7, node distance=1.75cm] (b2) {$b_2$};
	
	
	% Connect nodes
	\draw [->] (sum5) -- node [at end, label=right:{$u_2(t)$}] {} (c3);
	\draw [->] (r2) -- (sum5);
	\draw [->] (b2) -- (sum7);
	\draw [->] (sum7) -- (int2);
	\draw [->] (int2) -- node [label=below:{$x_6(t)$}] {} (sum5);
	\draw [->] (c9) -| node [at start, label=right:{$x_9(t)$}] {} (r2);
	\draw [->] (c9) -| (b2);
	\draw [->] (c13) -| node [at start, label=right:{$-x_5(t)$}] {} (sum5);
	\draw [->] (c13) -| (sum7);
	
\end{tikzpicture}}
		\caption{}
		\label{fig:4103_two_area_pi_controller_temporal_2}
	\end{minipage}
\end{figure}

Suppose variables are assigned to the controller for area 1 and the controller for area 2 according to Figures \ref{fig:4102_two_area_pi_controller_temporal_1} and \ref{fig:4103_two_area_pi_controller_temporal_2}, respectively. The first order system of linear differential equations for the PI controller are:
\begin{align}
	\dot{x}_1(t) &= b_1 \Delta f_1(t) + x_5(t) \label{eq:4108} \\
	\dot{x}_6(t) &= b_2 \Delta f_2(t) - x_5(t) \label{eq:4109}
\end{align}

Note that once the PI controller has been stepped forward in time during simulation the actual control signals exported from the controller are $u_1(t)$ and $u_2(t)$. These are expressed as:
\begin{align}
	u_1(t) &= x_1(t) + x_5(t) - R_1 x_4(t) \\
	u_2(t) &= x_6(t) - x_5(t) - R_2 x_9(t) 
\end{align}

A full derivation of the first order linear system described by equations \ref{eq:4108} and \ref{eq:4109} is described in Appendix \ref{sec:temporal_domain_for_pi_controller}. The system of equations were implemented as a method \verb|int_control_system_sim| in a Python class \verb|ClassicalPiController|. Implementation for \verb|int_control_system_sim| is detailed in Appendix \ref{sec:implementation_of_controller}.

%---------------- SS: DDPG NN Controller

\subsection{DDPG Neural Network Controller}
Deep deterministic policy gradient (DDPG), described in \textsection \ref{ssec:deep_deterministic_policy_gradient}, is used to train neural networks to perform control actions given a state observation. Recall that DDPG uses actor and critic networks, both of which also have target networks. The implication is that a total of four neural networks are required for a single DDPG controller instance. To accommodate this requirement, two Python class were created: one for the actor called \verb|Actor|, and another for the critic called \verb|Critic|. Both actor and critic networks were structured using an input layer, two hidden layers and an output layer --- an identical structure to experiments conducted by Lillicrap \textit{et alias}. Neural network hidden layers used rectified linear units (ReLU) for activation functions, and the final output layer of the actor network used a $\tanh$ activation function.







%---------------- SS: DDPG Training

\subsection{DDPG Training Algorithm}

The DDPG algorithm, shown in listing \ref{}, was implemented as a Python class \verb||

%---------------- SS: Simulation

\subsection{Simulation}
Method implementation details are shown in Appendix XXXX. The method is used as an argument for another method of the same class, called \verb|step|. When \verb|step| is called with \verb|int_power_system_sim| as an argument, the system simulation will iterate a single time step, which is performed using function \verb|odeint| from Python's Scipy library.




Policy network architecture can significantly impact results for DDPG. Furthermore, certain activation functions such as rectified linear units (ReLU) have been shown to cause worsened learning performance due to the dying relu problem.