\section{Model Development and Implementation}

%---------------- SS: Environment

\subsection{Environment} \label{ssec:env_modelling}
The environment model described in \textsection \ref{ssec:modelling_two_area_system}, and shown in Figure \ref{fig:208_two_area_pi_control_model}, exists in the frequency domain. Classical approaches for analysing systems in the frequency domain, using Laplacian transforms for example, are not suitable for this research for two reasons. Firstly, according to reinforcement learning architectures outlined in \textsection \ref{sec:reinforcement_learning}, control signals from the neural network agent need to be introduced to the system at each time step. This would be difficult to achieve in the frequency domain, unless the neural network is also expressed in the frequency domain, and its Laplacian transform is known. Secondly, techniques such as Laplacian transforms assume that the system initial conditions are zero \cite{Ogat2010}. Whilst the second assumption would work when modelling variable deviations from initial conditions, this approach would limit the ability to extend research findings to real world conditions.

Given this, conversion of the environment system model from the frequency to the temporal domain is desirable. Moreover, in order to accommodate numerical analysis schemes, such as Runge-Kutta, the environment system should be expressed as a first order system of ordinary linear differential equations. A full derivation of the first order linear system is described in Appendix \ref{app:C1_power_system_ode}.

The first order system of linear ordinary differential equations for area 1 is:
\begin{align}
	\dot{x}_2(t) &= \frac{1}{T_{sg_1}}\big( K_{sg_1} u_1(t) - x_2(t) \big) \label{eq:4101} \\
	\dot{x}_3(t) &= \frac{1}{T_{t_1}} \big( K_{t_1} x_2(t) - x_3(t) \big) \label{eq:4102} \\
	\dot{x}_4(t) &= \frac{1}{T_{gl_1}} \bigg( K_{gl_1} \big( x_3(t) - x_5(t) - \Delta p_{L1}(t) \big) - x_4(t) \label{eq:4103} \bigg)
\end{align}
	
The first order system of linear ordinary differential equations for area 2 is:
\begin{align}
	\dot{x}_7(t) &= \frac{1}{T_{sg_2}}\big( K_{sg_2} u_2(t) - x_7(t) \big) \label{eq:4104} \\
	\dot{x}_8(t) &= \frac{1}{T_{t_2}} \big( K_{t_2} x_7(t) - x_8(t) \label{eq:4105} \big)\\
	\dot{x}_9(t) &= \frac{1}{T_{gl_2}} \bigg( K_{gl_2} \big( x_8(t) - x_5(t) - \Delta p_{L2}(t) \big) - x_9(t) \label{eq:4106} \bigg)
\end{align}

The first order linear ordinary differential equation for the tie-line connecting areas 1 and 2 is:
\begin{equation}
	\dot{x}_5(t) = 2 \pi T_{12} \big( x_4(t) - x_9(t) \big) \label{eq:4107}
\end{equation}

Equations \ref{eq:4101} through \ref{eq:4107} were implemented as a method called \verb|int_power_system_sim| in a Python class called \verb|TwoAreaPowerSystemEnv|.

Implementation for \verb|int_power_system_sim| is detailed in Appendix \ref{app:implementation_of_power_system_model}.

%---------------- SS: PI Controller

\subsection{Classical PI Controller}
The proportional integral (PI) controller model described in \textsection \ref{ssec:control_two_area_system}, and shown in Figure \ref{fig:208_two_area_pi_control_model} needs to be expressed in the temporal domain for the same reasons as those described in section \ref{ssec:env_modelling}. A full derivation of the first order linear system is described in Appendix \ref{sec:temporal_domain_for_pi_controller}.

The first order system of linear differential equations for the PI controller are:
\begin{align}
	\dot{x}_1(t) &= b_1 \Delta f_1(t) + x_5(t) \\
	\dot{x}_6(t) &= b_2 \Delta f_2(t) - x_5(t)
\end{align}


%---------------- SS: DDPG Controller

\subsection{DDPG Neural Networks}

\subsubsection{Neural Network Architecture}

\subsubsection{title}


%---------------- SS: DDPG Controller

\subsection{Simulation}
Method implementation details are shown in Appendix XXXX. The method is used as an argument for another method of the same class, called \verb|step|. When \verb|step| is called with \verb|int_power_system_sim| as an argument, the system simulation will iterate a single time step, which is performed using function \verb|odeint| from Python's Scipy library.