\section{Generator Load Model with Tie Line Input}\label{app:generator_load_model_with_tie_line}
The derivation presented here borrows heavily from Kothari \cite{Kothari2011}. Analysis in Appendix \ref{app:gen_load_model} reasoned that the incremental power input to the generator load system was given by $\Delta P_G - \Delta P_L$ and that this difference was made up from the rate of increase of stored kinetic energy in the generator rotor, and the additional power drawn by frequency dependent loads such as motors. In the presence of a second power area, this power difference could also come from the power transfer over the tie line. Using this idea, the incremental power balance equation for area 1 can be written as:
\begin{equation}
	\Delta P_{G1} - \Delta P_{L1} = \frac{2 H_1}{(f_1)_0} \frac{d}{dt} \Delta f_1 + B_1 \Delta f_1 + \Delta P_{tie, 1}. \label{eq:B201}
\end{equation}

Additional parameters introduced in Equation \ref{eq:B201} include: the generator inertia, $H_1$; and the static rate of change in the load demand with respect to frequency due to motors, $B_1$. Taking the Laplace transform of Equation \ref{eq:B201} and rearranging provides:
\begin{equation}
	\Delta F_1(s) = [\Delta P_{G1}(s) - \Delta P_{L1}(s) - \Delta P_{tie,1}(s)] \times \bigg( \frac{K_{gl1}}{1 + T_{gl1}s} \bigg), \label{eq:B202}
\end{equation}
where
\begin{align}
	K_{gl1} &= \frac{1}{B_1} \\
	T_{gl1} &= \frac{2H_1}{B_1 (f_1)_0}.
\end{align}

Equation \ref{eq:B202} is an updated model of the generator-load system which provides for input from the tie line. Note that power could flow either way depending on the scheduled contract for provision of power between area 1 and area 2. Figure \ref{fig:B201_generator_load_model_1_with_tie_line} shows the generator-load block diagram with the additional tie line input. Given the symmetry of a two-area power system, an identical update can be applied to the generator load model for area 2.

\begin{figure}[h]
	\centering
	%----------- Create a fancy summing block
\tikzset{add/.style n args={4}{
		minimum width=6mm,
		path picture={
			\draw[black] 
			(path picture bounding box.south east) -- (path picture bounding box.north west)
			(path picture bounding box.south west) -- (path picture bounding box.north east);
			\node at ($(path picture bounding box.south)+(0,0.13)$)     {\tiny #1};
			\node at ($(path picture bounding box.west)+(0.13,0)$)      {\tiny #2};
			\node at ($(path picture bounding box.north)+(0,-0.13)$)    {\tiny #3};
			\node at ($(path picture bounding box.east)+(-0.13,0)$)     {\tiny #4};
		}
	}
}

%----------- Block style 1
\tikzstyle{block1} = [draw, fill=green!20, rectangle, 
minimum height=3em, minimum width=6em, node distance=2.5cm]

%----------- Block style 2
\tikzstyle{block2} = [draw, fill=green!20, rectangle, 
minimum height=3em, minimum width=3em, node distance=2.5cm]

%----------- Sum style
\tikzstyle{sum} = [draw, fill=green!20, circle, node distance=2cm]

%----------- Input style
\tikzstyle{input} = [coordinate, node distance=4cm]

%----------- Output style
\tikzstyle{output} = [coordinate, node distance=4cm]

\begin{tikzpicture}
	
	% Draw the nodes first
	\node [input] (input) {};
	\node [sum,add={$-$}{+}{$-$}{ },right of=input] (sum) {};
	\node [block1, right of=sum, label=above:{Gen. Load}] (genload) {$\frac{K_{gl}}{1 + T_{gl}s}$};
	\node [input, above of=sum, node distance=2cm] (tieline) {};
	\node [input, below of=sum, node distance=2cm] (powerdemand) {};
	\node [output, right of=genload] (output) {};
	
	% Connect the nodes
	\draw [->] (input) -- node [label=above:{$\Delta P_G(s)$}] {} (sum);
	\draw [->] (sum) -- (genload);
	\draw [->] (powerdemand) -- node [label=right:{$\Delta P_{L, 1}(s)$}] {} (sum);
	\draw [->] (tieline) -- node [label=right:{$\Delta P_{tie, 1}$}] {} (sum);
	\draw [->] (genload) -- node [label=above:{$\Delta F(s)$}] {} (output);
	
\end{tikzpicture}
	\caption[Generator-load model for a two or more area power system]{Generator-load block diagram now has an additional input from the tie line connecting power area 1 and power area 2}
	\label{fig:B201_generator_load_model_1_with_tie_line}
\end{figure}

The same analysis can be performed for power area 2, which results in the following expression:
\begin{equation}
	\Delta F_2(s) = [\Delta P_{G2}(s) - \Delta P_{L2}(s) - \Delta P_{tie,2}(s)] \times \bigg( \frac{K_{gl2}}{1 + T_{gl2}s} \bigg), \label{eq:B203}
\end{equation}
where
\begin{align}
	K_{gl2} &= \frac{1}{B_2} \\
	T_{gl2} &= \frac{2H_2}{B_2 (f_2)_0}
\end{align}