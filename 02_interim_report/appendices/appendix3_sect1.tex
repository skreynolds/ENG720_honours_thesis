\section{Temporal Domain Model for a Two Area Power System}

To develop a system of first order, linear, ordinary differential equations for a two area power system, consider only the governor, turbine, generator-load, and tie-line components of Figure \ref{fig:206_single_area_pi_control_model}. Let $X_2(s)$, $X_3(s)$, $X_4(s)$, $X_5(s)$, $X_7(s)$, $X_8(s)$, and $X_9(s)$ represent model variables, according to Figure \ref{fig:C101}.

\begin{figure}[h]
	\centering
	\resizebox{\textwidth}{!}{%----------- Create a fancy summing block
\tikzset{add/.style n args={4}{
		minimum width=6mm,
		path picture={
			\draw[black] 
			(path picture bounding box.south east) -- (path picture bounding box.north west)
			(path picture bounding box.south west) -- (path picture bounding box.north east);
			\node at ($(path picture bounding box.south)+(0,0.13)$)     {\tiny #1};
			\node at ($(path picture bounding box.west)+(0.13,0)$)      {\tiny #2};
			\node at ($(path picture bounding box.north)+(0,-0.13)$)    {\tiny #3};
			\node at ($(path picture bounding box.east)+(-0.13,0)$)     {\tiny #4};
		}
	}
}

%----------- Block style 1
\tikzstyle{block1} = [draw, fill=white!80!green, rectangle, 
minimum height=3em, minimum width=6em, node distance=2.5cm]

%----------- Block style 2
\tikzstyle{block2} = [draw, fill=white!80!green, rectangle, 
minimum height=3em, minimum width=3em, node distance=2.5cm]

%----------- Sum style
\tikzstyle{sum} = [draw, fill=white!80!green, circle, node distance=2cm]

%----------- Input style
\tikzstyle{input} = [coordinate, node distance=4cm]

%----------- Output style
\tikzstyle{output} = [coordinate, node distance=4cm]

%----------- Pin style
\tikzstyle{pinstyle} = [pin edge={to-,thin,black}]


\begin{tikzpicture}	
	% Tie line nodes
	\node [sum, add={$-$}{}{+}{ }] (sum1) {};
	\node [block1, right of=sum1, label=above:{Tie Line}] (tieline) {$\frac{2\pi T_{12}}{s}$};
	\node [output, right of=tieline, node distance=3cm] (out) {};
	\node [coordinate, above of=out, node distance=5cm] (c1) {};
	\node [coordinate, below of=out, node distance=5cm] (c2) {};
	\node [block2, left of=c2] (a12) {$-a_{12}$};
	
	% Position a reference coordinate for drawing
	\node [coordinate, left of=sum1, node distance=2.5cm] (c3) {};
	\node [coordinate, above of=c3, node distance=0.75cm] (c4) {};
	\node [coordinate, below of=c3, node distance=0.75cm] (c5) {};
	
	% Create nodes for upper leg
	\node [block1, above of=c3, node distance=3.5cm, label=above:{Gen. Load 1}] (genload1) {$\frac{K_{gl1}}{T_{gl1}s+1}$};
	\node [coordinate, right of=genload1, node distance=1.5cm] (c6) {};
	\node [sum, left of=genload1, add={$-$}{+}{$-$}{}, node distance=2.5cm] (sum2) {};
	\node [coordinate, below of=sum2] (p11) {};
	\node [coordinate, left of=p11, node distance=0.5cm, label=left:{$\Delta P_{L1}(s)$}] (p12) {};
	\node [block1, left of=sum2, node distance=3.5cm, label=above:{Turbine 1}] (turbine1) {$\frac{K_{t1}}{T_{t1}s+1}$};
	\node [block1, left of=turbine1, node distance=4.5cm, label=above:{Governor 1}] (governor1) {$\frac{K_{g1}}{T_{g1}s+1}$};
	\node [coordinate, left of=governor1, node distance=3cm] (c8) {};
	\node [coordinate, left of=c5, node distance=13.5cm] (c10) {};
	\node [coordinate, left of=c1, node distance=21.5cm] (c12) {};
	
	
	
	% Create nodes for lower leg
	\node [block1, below of=c3, node distance=3.5cm, label=above:{Gen. Load 2}] (genload2) {$\frac{K_{gl1}}{T_{gl1}s+1}$};
	\node [coordinate, right of=genload2, node distance=1.5cm] (c7) {};
	\node [sum, left of=genload2, add={$-$}{+}{$-$}{}, node distance=2.5cm] (sum3) {};
	\node [coordinate, above of=sum3] (p21) {};
	\node [coordinate, left of=p21, node distance=0.5cm, label=left:{$\Delta P_{L2}(s)$}] (p22) {};
	\node [block1, left of=sum3, node distance=3.5cm, label=above:{Turbine 2}] (turbine2) {$\frac{K_{t2}}{T_{t2}s+1}$};
	\node [block1, left of=turbine2, node distance=4.5cm, label=above:{Governor 2}] (governor2) {$\frac{K_{g2}}{T_{g2}s+1}$};
	\node [coordinate, left of=governor2, node distance=3cm] (c9) {};
	\node [coordinate, left of=c4, node distance=13.5cm] (c11) {};
	\node [coordinate, left of=a12, node distance=19cm] (c13) {};
	
	
	% Connect the tieline nodes
	\draw [->] (sum1) -- (tieline);
	\draw (tieline) -- node [label=below:{$X_5(s)$}] {} (out);
	
	% Connect nodes in upper block
	\draw (out) -- (c1);
	\draw [->] (c1) -| (sum2);
	
	\draw [->] (governor1) -- node [label=above:{$X_2(s)$}] {} (turbine1);
	\draw [->] (turbine1) -- node [label=above:{$X_3(s)$}] {} (sum2);
	\draw [->] (sum2) -- (genload1);
	\draw [->] (genload1) -| node [label=above:{$X_4(s)$}] {} (sum1);
	\draw (c6) |- (c4);
	\draw [->] (c8) -- node [label=above:{$U_1(s)$}] {} (governor1);
	\draw (p12) -- (p11);
	\draw [->] (p11) -- (sum2);
	\draw [->] (c4) -- (c11);
	\draw [->] (c1) -- (c12);
	
	
	% Connect nodes in lower block
	\draw (out) -- (c2);
	\draw [->] (c2) -- (a12);
	\draw [->] (a12) -| (sum3);
	\draw [->] (governor2) -- node [label=below:{$X_7(s)$}] {} (turbine2);
	\draw [->] (turbine2) -- node [label=below:{$X_8(s)$}] {} (sum3);
	\draw [->] (sum3) -- (genload2);
	\draw [->] (genload2) -| node [label=below:{$X_9(s)$}] {} (sum1);
	\draw (c7) |- (c5);
	\draw [->] (c9) -- node [label=below:{$U_2(s)$}] {} (governor2);
	\draw (p22) -- (p21);
	\draw [->] (p21) -- (sum3);
	\draw [->] (c5) -- (c10);
	\draw [->] (a12) -- (c13);
	
\end{tikzpicture}}
	\caption[Combined governor, turbine, generator-load, and tie-line for a two power area system]{The governor, turbine, generator-load, and tie-line models for both power areas.}
	\label{fig:C101}
\end{figure}

Note that $X_4(s)$ represents the frequency output of area 1; $X_9(s)$ represents the frequency output of area 2; and $X_5(s)$ represents the power flow over the tie-line connecting areas 1 and 2. The system also receives two external control signals that act as inputs the governors in each area. These are denoted as $U_1(s)$ and $U_2(s)$ for control area 1 and control area 2, respectively. Finally, the system experiences changes in load demand as individuals and industry switch appliances on and off --- these are represented by $\Delta P_{L1}(s)$ and $\Delta P_{L2}(s)$ for areas 1 and 2, respectively.

Now, using the governor block transfer function in area 1, the following expression can be written:
\begin{equation}
	X_2(s) = \frac{K_{sg1}}{T_{sg1}s + 1} U_1(s) \label{eq:C011}
\end{equation}

Rearranging \ref{eq:C011} and taking the inverse Laplace transform yields the following first order differential equation:
\begin{equation}
	\dot{x}_2(t) = \frac{1}{T_{sg1}}\big( K_{sg1} u_1(t) - x_2(t) \big) \label{eq:C012}
\end{equation}

Since the two areas of the system are symmetrical, a similar analysis for the same section in area 2 yields:
\begin{equation}
	\dot{x}_7(t) = \frac{1}{T_{sg2}}\big( K_{sg2} u_2(t) - x_7(t) \big) \label{eq:C013}
\end{equation}

Next, the turbine block transfer function in area 1 can be used to write the following expression:
\begin{equation}
	X_3(s) = \frac{K_{t1}}{T_{t1}s + 1} X_2(s) \label{eq:C014}
\end{equation}

Rearranging and taking the inverse Laplace transform:
\begin{equation}
	\dot{x}_3(t) = \frac{1}{T_{t1}} \big( K_{t1} x_2(t) - x_3(t) \big) \label{eq:C015}
\end{equation}

Given symmetry in the system a similar equation can be written for area 1:
\begin{equation}
	\dot{x}_8(t) = \frac{1}{T_{t2}} \big( K_{t2} x_7(t) - x_8(t) \big) \label{eq:C016}
\end{equation}

The same analysis that was undertaken for the governor and turbine can be performed for the generator-load demand block. For area 1, this yields the following first order, ordinary differential equation:
\begin{equation}
	\dot{x}_4(t) = \frac{1}{T_{gl1}} \bigg( K_{gl1} \big( x_3(t) - x_5(t) - \Delta p_{L1}(t) \big) - x_4(t) \bigg) \label{eq:C017}
\end{equation}

Since the areas are symmetrical, using \ref{eq:C017} the generator-load model for area two can be written as:
\begin{equation}
	\dot{x}_9(t) = \frac{1}{T_{gl2}} \bigg( K_{gl2} \big( x_8(t) - x_5(t) - \Delta p_{L2}(t) \big) - x_9(t) \bigg) \label{eq:C018}
\end{equation}

Finally, the ordinary differential equation for the tie-line block can be expressed as:
\begin{equation}
	\dot{x}_5(t) = 2 \pi T_{12} \big( x_4(t) - x_9(t) \big)
\end{equation}