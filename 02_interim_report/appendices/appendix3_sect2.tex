\section{Temporal Domain Model for a PI Controller of a Two Area Power System}

The approach taken in section \ref{app:C1_power_system_ode} for deriving the system of differential equations to model the power system, can also be used to model the proportional-integral (PI) feedback controllers. Consider the the proportional and integral feedback loop components of Figure \ref{fig:206_single_area_pi_control_model}. Let $X_1(s)$ and $X_6(s)$ represent the output from the integral control block, in area 1 and area 2, as shown in Figures \ref{fig:C201_two_area_pi_control_ode_derivation_1} and \ref{fig:C202_two_area_pi_control_ode_derivation_2}, respectively. Note that the controllers receive change in frequency $\Delta F_1(s)$ and $\Delta F_2(s)$ as inputs from the power system model, as well as the change in the tie-line power, $X_5(s)$.

\begin{figure}[h]
	\begin{minipage}[b]{0.5\textwidth}
		\resizebox{7.0cm}{!}{% Set up the standalone document class
\documentclass{standalone}

% Input the preamble (<3)
% Preamble document

% Import tikz package
\usepackage{tikz}

% Import tikz libraries
\usetikzlibrary{shapes, arrows}
\usetikzlibrary{positioning, calc}

%----------- Create a fancy summing block
\tikzset{add/.style n args={4}{
		minimum width=6mm,
		path picture={
			\draw[black] 
			(path picture bounding box.south east) -- (path picture bounding box.north west)
			(path picture bounding box.south west) -- (path picture bounding box.north east);
			\node at ($(path picture bounding box.south)+(0,0.13)$)     {\tiny #1};
			\node at ($(path picture bounding box.west)+(0.13,0)$)      {\tiny #2};
			\node at ($(path picture bounding box.north)+(0,-0.13)$)    {\tiny #3};
			\node at ($(path picture bounding box.east)+(-0.13,0)$)     {\tiny #4};
		}
	}
}

%----------- Block style 1
\tikzstyle{block1} = [draw, fill=blue!20, rectangle, 
minimum height=3em, minimum width=6em, node distance=2.5cm]

%----------- Block style 2
\tikzstyle{block2} = [draw, fill=blue!20, rectangle, 
minimum height=3em, minimum width=3em, node distance=2.5cm]

%----------- Sum style
\tikzstyle{sum} = [draw, fill=blue!20, circle, node distance=2cm]

%----------- Input style
\tikzstyle{input} = [coordinate, node distance=4cm]

%----------- Output style
\tikzstyle{output} = [coordinate, node distance=4cm]

%----------- Pin style
\tikzstyle{pinstyle} = [pin edge={to-,thin,black}]

\begin{document}
\begin{tikzpicture}	
	% Initial position node
	\node [coordinate] (c1) {};
	
	
	% Create nodes for upper leg
	\node [sum, above of=c1, add={$-$}{}{+}{}, node distance=3.5cm] (sum4) {};
	\node [coordinate, above of=sum4, node distance=1.25cm] (c10) {};
	\node [coordinate, right of=c10, node distance=2cm] (c12) {};
	\node [coordinate, right of=sum4, node distance=2cm] (c2) {};
	\node [coordinate, above of=c2] (c4) {};
	\node [block2, below of=sum4, node distance=1.75cm] (r1) {$R_1$};
	\node [coordinate, below of=r1] (c6) {};
	\node [coordinate, right of=c6, node distance=2cm] (c8) {};
	\node [block2, left of=sum4] (int1) {$\frac{K_{i_1}}{s}$};
	\node [sum, left of=int1, add={+}{ }{+}{ }] (sum6) {};
	\node [block2, below of=sum6, node distance=1.75cm] (b1) {$b_1$};
	
	
	% Connect nodes
	\draw [->] (sum4) -- node [at end, label=above:{$X_1(s)$}] {} (c2);
	\draw [->] (r1) -- (sum4);
	\draw [->] (b1) -- (sum6);
	\draw [->] (sum6) -- (int1);
	\draw [->] (int1) -- (sum4);
	\draw [->] (c8) -| node [at start, label=above:{$F_1(s)$}] {} (r1);
	\draw [->] (c8) -| (b1);
	\draw [->] (c12) -| node [at start, label=above:{$T(s)$}] {} (sum4);
	\draw [->] (c12) -| (sum6);
	
\end{tikzpicture}
\end{document}}
		\caption{Proportional integral controller for power area 1, with assigned variables for temporal domain modelling}
		\label{fig:C201_two_area_pi_control_ode_derivation_1}
	\end{minipage}
	\hspace{0.1cm}
	\begin{minipage}[b]{0.5\textwidth}
		\resizebox{7.2cm}{!}{%----------- Create a fancy summing block
\tikzset{add/.style n args={4}{
		minimum width=6mm,
		path picture={
			\draw[black] 
			(path picture bounding box.south east) -- (path picture bounding box.north west)
			(path picture bounding box.south west) -- (path picture bounding box.north east);
			\node at ($(path picture bounding box.south)+(0,0.13)$)     {\tiny #1};
			\node at ($(path picture bounding box.west)+(0.13,0)$)      {\tiny #2};
			\node at ($(path picture bounding box.north)+(0,-0.13)$)    {\tiny #3};
			\node at ($(path picture bounding box.east)+(-0.13,0)$)     {\tiny #4};
		}
	}
}

%----------- Block style 1
\tikzstyle{block1} = [draw, fill=white!80!blue, rectangle, 
minimum height=3em, minimum width=6em, node distance=2.5cm]

%----------- Block style 2
\tikzstyle{block2} = [draw, fill=white!80!blue, rectangle, 
minimum height=3em, minimum width=3em, node distance=2.5cm]

%----------- Sum style
\tikzstyle{sum} = [draw, fill=white!80!blue, circle, node distance=2cm]

%----------- Input style
\tikzstyle{input} = [coordinate, node distance=4cm]

%----------- Output style
\tikzstyle{output} = [coordinate, node distance=4cm]

%----------- Pin style
\tikzstyle{pinstyle} = [pin edge={to-,thin,black}]

\begin{tikzpicture}	
	% Initial position node
	\node [coordinate] (c1) {};
	
	
	% Create nodes for lower leg
	\node [sum, below of=c1, add={+}{}{$-$}{}, node distance=3.5cm] (sum5) {};
	\node [coordinate, below of=sum5, node distance=1.25cm] (c11) {};
	\node [coordinate, right of=c11, node distance=2cm] (c13) {};
	\node [coordinate, right of=sum5, node distance=2cm] (c3) {};
	\node [coordinate, above of=c3] (c5) {};
	\node [block2, above of=sum5, node distance=1.75cm] (r2) {$R_2$};
	\node [coordinate, above of=r2] (c7) {};
	\node [coordinate, right of=c7, node distance=2cm] (c9) {};
	\node [block2, left of=sum5] (int2) {$\frac{K_{i_2}}{s}$};
	\node [sum, left of=int2, add={+}{ }{+}{ }] (sum7) {};
	\node [block2, above of=sum7, node distance=1.75cm] (b2) {$b_2$};
	
	
	% Connect nodes
	\draw [->] (sum5) -- node [at end, label=right:{$U_2(s)$}] {} (c3);
	\draw [->] (r2) -- (sum5);
	\draw [->] (b2) -- (sum7);
	\draw [->] (sum7) -- (int2);
	\draw [->] (int2) -- node [label=below:{$X_6(s)$}] {} (sum5);
	\draw [->] (c9) -| node [at start, label=right:{$\Delta F_2(s)$}] {} (r2);
	\draw [->] (c9) -| (b2);
	\draw [->] (c13) -| node [at start, label=right:{$-X_5(s)$}] {} (sum5);
	\draw [->] (c13) -| (sum7);
	
\end{tikzpicture}}
		\caption{Proportional integral controller for power area 2, with assigned variables for temporal domain modelling}
		\label{fig:C202_two_area_pi_control_ode_derivation_2}
	\end{minipage}
\end{figure}

Considering the integral block for area 1 in Figure \ref{fig:C201_two_area_pi_control_ode_derivation_1}, the following expression can be written:
\begin{equation}
	X_1(s) = \frac{K_{i_1}}{s} \times \big( b_1 \Delta F_1(s) + X_5(s) \big) \label{eq:C201}
\end{equation}

Rearranging and taking the inverse Laplace transform provides the following differential equation:
\begin{equation}
	\dot{x}_1(t) = b_1 \Delta f_1(t) + x_5(t) \label{eq:C202}
\end{equation}

Taking a similar approach for the integral block for area 2 yields:
\begin{equation}
	\dot{x}_6(t) = b_2 \Delta f_2(t) - x_5(t) \label{eq:C203}
\end{equation}

Note that equations \ref{eq:C202} and \ref{eq:C203} are not the control outputs that are output to the system, rather, are the differential equations modelling the integral control blocks. The control output for area 1 and area 2 are shown in equation \ref{eq:C204} and \ref{eq:C205} below.
\begin{align}
	u_1(t) &= x_1(t) + x_5(t) - R_1 \Delta f_1(t) \label{eq:C204} \\
	u_2(t) &= x_6(t) - x_5(t) - R_2 \Delta f_2(t) \label{eq:C205}
\end{align}