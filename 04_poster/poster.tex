%%%%%%%%%%%%%%%%%%%%%%%%%%%%%%%%%%%%%%%%%
% Jacobs Landscape Poster
% LaTeX Template
% Version 1.1 (14/06/14)
%
% Created by:
% Computational Physics and Biophysics Group, Jacobs University
% https://teamwork.jacobs-university.de:8443/confluence/display/CoPandBiG/LaTeX+Poster
% 
% Further modified by:
% Shane Reynolds
%
% License:
% CC BY-NC-SA 3.0 (http://creativecommons.org/licenses/by-nc-sa/3.0/)
%
%%%%%%%%%%%%%%%%%%%%%%%%%%%%%%%%%%%%%%%%%

%----------------------------------------------------------------------------------------
%	PACKAGES AND OTHER DOCUMENT CONFIGURATIONS
%----------------------------------------------------------------------------------------

\documentclass[final]{beamer}

\usepackage[scale=1.24]{beamerposter} % Use the beamerposter package for laying out the poster

\usetheme{confposter} % Use the confposter theme supplied with this template

\setbeamercolor{block title}{fg=black,bg=white} % Colors of the block titles
\setbeamercolor{block body}{fg=black,bg=white} % Colors of the body of blocks

\setbeamercolor{itemize item}{fg=black}
\setbeamertemplate{itemize item}[circle]

%\setbeamercolor{block alerted title}{fg=white,bg=green!20!gray} % Colors of the highlighted block titles
%\setbeamercolor{block alerted body}{fg=black,bg=dgreen!10} % Colors of the body of highlighted blocks
% Many more colors are available for use in beamerthemeconfposter.sty

\newenvironment{variableblock}[3]{%
  \setbeamercolor{block alerted body}{#2}
  \setbeamercolor{block alerted title}{#3}
  \begin{alertblock}{#1}}{\end{alertblock}}

%-----------------------------------------------------------
% Define the column widths and overall poster size
% To set effective sepwid, onecolwid and twocolwid values, first choose how many columns you want and how much separation you want between columns
% In this template, the separation width chosen is 0.024 of the paper width and a 4-column layout
% onecolwid should therefore be (1-(# of columns+1)*sepwid)/# of columns e.g. (1-(4+1)*0.024)/4 = 0.22
% Set twocolwid to be (2*onecolwid)+sepwid = 0.464
% Set threecolwid to be (3*onecolwid)+2*sepwid = 0.708

\newlength{\sepwid}
\newlength{\onecolwid}
\newlength{\twocolwid}
\newlength{\threecolwid}
\setlength{\paperwidth}{48in} % A0 width: 46.8in
\setlength{\paperheight}{36in} % A0 height: 33.1in
\setlength{\sepwid}{0.024\paperwidth} % Separation width (white space) between columns
\setlength{\onecolwid}{0.22\paperwidth} % Width of one column
\setlength{\twocolwid}{0.464\paperwidth} % Width of two columns
\setlength{\threecolwid}{0.708\paperwidth} % Width of three columns
\setlength{\topmargin}{-0.5in} % Reduce the top margin size
%-----------------------------------------------------------

\usepackage{graphicx}  % Required for including images

\usepackage[linesnumbered, ruled]{algorithm2e}

\usepackage{booktabs} % Top and bottom rules for tables

\usepackage{amsmath} % Required for math font

\usepackage{tikz} % Required for drawing sik pictures

\usepackage{pgfplots} % Required for drawing sik plots

% Import tikz libraries
\usetikzlibrary{shapes, arrows}
\usetikzlibrary{positioning, calc}

\graphicspath{{./figures/}}

%----------------------------------------------------------------------------------------
%	TITLE SECTION 
%----------------------------------------------------------------------------------------

\title{Automatic Generation Control for a Two Area Power \\ System Using Deep Reinforcement Learning} % Poster title

\author{Shane Reynolds, Charles Yeo, \& Stefanija Klaric} % Author(s)

\institute{Charles Darwin University} % Institution(s)

%----------------------------------------------------------------------------------------

\begin{document}
\addtobeamertemplate{headline}{} 
{
\begin{tikzpicture}[remember picture,overlay] 
\node [shift={(7cm,-10cm)}] at (current page.north west) {\includegraphics[scale=3.5]{01_cdu_logo}}; 
\end{tikzpicture} 
}

\addtobeamertemplate{block end}{}{\vspace*{2ex}} % White space under blocks
\addtobeamertemplate{block alerted end}{}{\vspace*{2ex}} % White space under highlighted (alert) blocks

\setlength{\belowcaptionskip}{2ex} % White space under figures
\setlength\belowdisplayshortskip{2ex} % White space under equations

\begin{frame}[t] % The whole poster is enclosed in one beamer frame

\begin{columns}[t] % The whole poster consists of three major columns, the second of which is split into two columns twice - the [t] option aligns each column's content to the top

\begin{column}{\sepwid}\end{column} % Empty spacer column

\begin{column}{\onecolwid} % The first column

%----------------------------------------------------------------------------------------
%	Background
%----------------------------------------------------------------------------------------
\vspace{-1.5cm}
\begin{block}{Problem Statement \& Aim}
An increase in photovoltaic power generation, and battery energy storage systems is causing Australian power system dynamics to become more non-linear, driving a need to explore novel control architectures to improve frequency control performance. This research aims to investigate the feasibility of controlling the power system frequency with a neural network.
\end{block}


%----------------------------------------------------------------------------------------
%	Section 1
%----------------------------------------------------------------------------------------
\vspace{-1cm}
\begin{block}{Reinforcement Learning}
Reinforcement learning is a branch of machine learning concerned with an agent's sequential decision making to maximise cumulative expected reward.
	\begin{figure}
		% Additional styles
\tikzstyle{block} = [rectangle, draw, text width=8em, text centered, rounded corners, minimum height=4em, line width=4pt]
    
\tikzstyle{line} = [draw, -latex]



\begin{tikzpicture}[node distance = 6em, auto, thick]
    \node [block, fill=blue!25!gray, text=white] (Agent) {\textbf{Agent}};
    \node [block, below of=Agent, fill=green!25!gray, text=white] (Environment) {\textbf{Environment}};
    
     \path [line, line width=4pt] (Agent.0) --++ (4em,0em) |- node [near start]{$\boldsymbol{a_t}$} (Environment.0);
     \path [line, line width=4pt] (Environment.190) --++ (-6em,0em) |- node [near start] {$\boldsymbol{s_t}$} (Agent.170);
     \path [line, line width=4pt] (Environment.170) --++ (-4.25em,0em) |- node [near start, right] {$\boldsymbol{r_t}$} (Agent.190);
\end{tikzpicture}
		\caption{ \ The agent exists in some environment and at each time step observes state $\boldsymbol{s_t \in S}$; and takes an action $\boldsymbol{a_t \in A}$. Following this, the agent then receives a reward $\boldsymbol{r_t \in R: S \times A \times S \to [R_{min}, R_{max}]}$.}
	\end{figure}
\end{block}

\vspace{-1.5cm}

\begin{variableblock}{The Environment}{fg=black,bg=dgreen!10}{fg=white,bg=green!20!gray}
The control objective is to maintain inter-area power transfer, whilst regulating the frequency of each area.
	\begin{figure}
	\centering
	% Set up the standalone document class
\documentclass{standalone}

% Input the preamble (<3)
% Preamble document

% Import tikz package
\usepackage{tikz}

% Import tikz libraries
\usetikzlibrary{shapes, arrows}
\usetikzlibrary{positioning, calc}

%----------- Create a fancy summing block
\tikzset{add/.style n args={4}{
		minimum width=6mm,
		path picture={
			\draw[black] 
			(path picture bounding box.south east) -- (path picture bounding box.north west)
			(path picture bounding box.south west) -- (path picture bounding box.north east);
			\node at ($(path picture bounding box.south)+(0,0.13)$)     {\tiny #1};
			\node at ($(path picture bounding box.west)+(0.13,0)$)      {\tiny #2};
			\node at ($(path picture bounding box.north)+(0,-0.13)$)    {\tiny #3};
			\node at ($(path picture bounding box.east)+(-0.13,0)$)     {\tiny #4};
		}
	}
}

%----------- Block style 1
\tikzstyle{block1} = [draw, fill=blue!20, rectangle, 
minimum height=3em, minimum width=6em, node distance=2.5cm]

%----------- Block style 2
\tikzstyle{block2} = [draw, fill=blue!20, rectangle, 
minimum height=3em, minimum width=3em, node distance=2.5cm]

%----------- Sum style
\tikzstyle{sum} = [draw, fill=blue!20, circle, node distance=2cm]

%----------- Input style
\tikzstyle{input} = [coordinate, node distance=4cm]

%----------- Output style
\tikzstyle{output} = [coordinate, node distance=4cm]

%----------- Pin style
\tikzstyle{pinstyle} = [pin edge={to-,thin,black}]

\begin{document}
\begin{tikzpicture}

	% Tie line nodes
	\node [draw, circle, fill=blue!20] (area1) {Area 1};
	\node [draw, circle, right of=area1, fill=blue!20, node distance=4cm] (area2) {Area 2}; 

	% Connect the tieline nodes
	\draw (area1) -- node [label=above:{Tie Line}] {} (area2);
	
\end{tikzpicture}
\end{document}
	\caption{ \ Two power areas connected via a transmission line. Each power area consists of: a governor controlled generator and a stochastic load demand.}
	\end{figure}
	\vspace{-1cm}

\end{variableblock}

%------------------------------------------------



%----------------------------------------------------------------------------------------

\end{column} % End of the first column

\begin{column}{\sepwid}\end{column} % Empty spacer column

\begin{column}[t]{\twocolwid} % Begin a column which is two columns wide (column 2)



%----------------------------------------------------------------------------------------
%	Section 4
%----------------------------------------------------------------------------------------
\vspace{-1.5cm}
\begin{variableblock}{The Environment: A More Detailed View}{fg=black,bg=dgreen!10}{fg=white,bg=green!20!gray}
\makebox[52cm][c]{
	\begin{minipage}{0.45\linewidth}
		\begin{figure}
			\resizebox{40cm}{!}{% Set up the standalone document class
\documentclass{standalone}

% Input the preamble (<3)
% Preamble document

% Import tikz package
\usepackage{tikz}

% Import tikz libraries
\usetikzlibrary{shapes, arrows}
\usetikzlibrary{positioning, calc}

%----------- Create a fancy summing block
\tikzset{add/.style n args={4}{
		minimum width=6mm,
		path picture={
			\draw[black] 
			(path picture bounding box.south east) -- (path picture bounding box.north west)
			(path picture bounding box.south west) -- (path picture bounding box.north east);
			\node at ($(path picture bounding box.south)+(0,0.13)$)     {\tiny #1};
			\node at ($(path picture bounding box.west)+(0.13,0)$)      {\tiny #2};
			\node at ($(path picture bounding box.north)+(0,-0.13)$)    {\tiny #3};
			\node at ($(path picture bounding box.east)+(-0.13,0)$)     {\tiny #4};
		}
	}
}

%----------- Block style 1
\tikzstyle{block1} = [draw, fill=blue!20, rectangle, 
minimum height=3em, minimum width=6em, node distance=2.5cm]

%----------- Block style 2
\tikzstyle{block2} = [draw, fill=blue!20, rectangle, 
minimum height=3em, minimum width=3em, node distance=2.5cm]

%----------- Sum style
\tikzstyle{sum} = [draw, fill=blue!20, circle, node distance=2cm]

%----------- Input style
\tikzstyle{input} = [coordinate, node distance=4cm]

%----------- Output style
\tikzstyle{output} = [coordinate, node distance=4cm]

%----------- Pin style
\tikzstyle{pinstyle} = [pin edge={to-,thin,black}]

\begin{document}
\begin{tikzpicture}	
	% Tie line nodes
	\node [sum, add={$-$}{}{+}{ }] (sum1) {};
	\node [block1, right of=sum1, label=above:{Tie Line}] (tieline) {$\frac{2\pi T_{12}}{s}$};
	\node [output, right of=tieline, node distance=3cm] (out) {};
	\node [coordinate, above of=out, node distance=5cm] (c1) {};
	\node [coordinate, below of=out, node distance=5cm] (c2) {};
	\node [block2, left of=c2] (a12) {$-a_{12}$};
	
	% Position a reference coordinate for drawing
	\node [coordinate, left of=sum1, node distance=2.5cm] (c3) {};
	\node [coordinate, above of=c3, node distance=0.75cm] (c4) {};
	\node [coordinate, below of=c3, node distance=0.75cm] (c5) {};
	
	% Create nodes for upper leg
	\node [block1, above of=c3, node distance=3.5cm, label=above:{Gen. Load 1}] (genload1) {$\frac{K_{gl1}}{T_{gl1}s+1}$};
	\node [coordinate, right of=genload1, node distance=1.5cm] (c6) {};
	\node [sum, left of=genload1, add={$-$}{+}{$-$}{}, node distance=2.5cm] (sum2) {};
	\node [coordinate, below of=sum2] (p11) {};
	\node [coordinate, left of=p11, node distance=0.5cm, label=left:{$\Delta P_{L1}(s)$}] (p12) {};
	\node [block1, left of=sum2, node distance=3.5cm, label=above:{Turbine 1}] (turbine1) {$\frac{K_{t1}}{T_{t1}s+1}$};
	\node [block1, left of=turbine1, node distance=4.5cm, label=above:{Governor 1}] (governor1) {$\frac{K_{g1}}{T_{g1}s+1}$};
	\node [coordinate, left of=governor1, node distance=3cm] (c8) {};
	\node [coordinate, left of=c5, node distance=13.5cm] (c10) {};
	\node [coordinate, left of=c1, node distance=21.5cm] (c12) {};
	
	
	
	% Create nodes for lower leg
	\node [block1, below of=c3, node distance=3.5cm, label=above:{Gen. Load 2}] (genload2) {$\frac{K_{gl1}}{T_{gl1}s+1}$};
	\node [coordinate, right of=genload2, node distance=1.5cm] (c7) {};
	\node [sum, left of=genload2, add={$-$}{+}{$-$}{}, node distance=2.5cm] (sum3) {};
	\node [coordinate, above of=sum3] (p21) {};
	\node [coordinate, left of=p21, node distance=0.5cm, label=left:{$\Delta P_{L2}(s)$}] (p22) {};
	\node [block1, left of=sum3, node distance=3.5cm, label=above:{Turbine 2}] (turbine2) {$\frac{K_{t2}}{T_{t2}s+1}$};
	\node [block1, left of=turbine2, node distance=4.5cm, label=above:{Governor 2}] (governor2) {$\frac{K_{g2}}{T_{g2}s+1}$};
	\node [coordinate, left of=governor2, node distance=3cm] (c9) {};
	\node [coordinate, left of=c4, node distance=13.5cm] (c11) {};
	\node [coordinate, left of=a12, node distance=19cm] (c13) {};
	
	
	% Connect the tieline nodes
	\draw [->] (sum1) -- (tieline);
	\draw (tieline) -- node [label=below:{$X_5(s)$}] {} (out);
	
	% Connect nodes in upper block
	\draw (out) -- (c1);
	\draw [->] (c1) -| (sum2);
	
	\draw [->] (governor1) -- node [label=above:{$X_2(s)$}] {} (turbine1);
	\draw [->] (turbine1) -- node [label=above:{$X_3(s)$}] {} (sum2);
	\draw [->] (sum2) -- (genload1);
	\draw [->] (genload1) -| node [label=above:{$X_4(s)$}] {} (sum1);
	\draw (c6) |- (c4);
	\draw [->] (c8) -- node [label=above:{$U_1(s)$}] {} (governor1);
	\draw (p12) -- (p11);
	\draw [->] (p11) -- (sum2);
	\draw [->] (c4) -- (c11);
	\draw [->] (c1) -- (c12);
	
	
	% Connect nodes in lower block
	\draw (out) -- (c2);
	\draw [->] (c2) -- (a12);
	\draw [->] (a12) -| (sum3);
	\draw [->] (governor2) -- node [label=below:{$X_7(s)$}] {} (turbine2);
	\draw [->] (turbine2) -- node [label=below:{$X_8(s)$}] {} (sum3);
	\draw [->] (sum3) -- (genload2);
	\draw [->] (genload2) -| node [label=below:{$X_9(s)$}] {} (sum1);
	\draw (c7) |- (c5);
	\draw [->] (c9) -- node [label=below:{$U_2(s)$}] {} (governor2);
	\draw (p22) -- (p21);
	\draw [->] (p21) -- (sum3);
	\draw [->] (c5) -- (c10);
	\draw [->] (a12) -- (c13);
	
\end{tikzpicture}
\end{document}}
		\end{figure}
	\end{minipage}
	\hspace{15.5cm}
	\begin{minipage}{0.27\linewidth}
		\begin{figure}
			\vspace{0.5cm}
			\setlength{\belowcaptionskip}{-10pt}
			\caption{ \ Block diagram of two area power system connected via a tie-line. Generators are modelled using governor and  turbine models. Governors, turbines, generator loads, and tie-lines are all modelled using first order linear systems. Simulation is undertaken in the temporal domain.}
		\end{figure}
	\end{minipage}
}
\end{variableblock}
\vspace{-1.5cm}
\begin{block}{Results Comparison from Preliminary Experiments}
	\begin{minipage}{0.5\textwidth}
		\begin{figure}\scriptsize
			\hspace{3cm}
			% This file was created by tikzplotlib v0.9.1.
\begin{tikzpicture}[trim axis left]

\definecolor{color0}{rgb}{0.12156862745098,0.466666666666667,0.705882352941177}
\definecolor{color1}{rgb}{1,0.498039215686275,0.0549019607843137}

\pgfplotsset{scaled y ticks=false}


\begin{axis}[
tick align=outside,
tick pos=left,
x grid style={white!69.0196078431373!black},
xmin=-1.50000000000009, xmax=31.500000000002,
xtick style={color=black},
y grid style={white!69.0196078431373!black},
ymin=-0.0224612300423489, ymax=0.00730709319820797,
ytick style={color=black},
yticklabel style={
        /pgf/number format/fixed,
        /pgf/number format/precision=5
},
width=25cm,
height=10cm,
xlabel=time (seconds),
ylabel=Frequency (Hz)
]
\addplot [line width=3pt, green!20!gray]
table {%
0 0
0.01 0
0.02 0
0.03 0
0.04 0
0.05 0
0.06 0
0.07 0
0.08 0
0.09 0
0.1 0
0.11 0
0.12 0
0.13 0
0.14 0
0.15 0
0.16 0
0.17 0
0.18 0
0.19 0
0.2 0
0.21 0
0.22 0
0.23 0
0.24 0
0.25 0
0.26 0
0.27 0
0.28 0
0.29 0
0.3 0
0.31 0
0.32 0
0.33 0
0.34 0
0.35 0
0.36 0
0.37 0
0.38 0
0.39 0
0.4 0
0.41 0
0.42 0
0.43 0
0.44 0
0.45 0
0.46 0
0.47 0
0.48 0
0.49 0
0.5 0
0.51 0
0.52 0
0.53 0
0.54 0
0.55 0
0.56 0
0.57 0
0.58 0
0.59 0
0.6 0
0.61 0
0.62 0
0.63 0
0.64 0
0.65 0
0.66 0
0.67 0
0.68 0
0.69 0
0.7 0
0.71 0
0.72 0
0.73 0
0.74 0
0.75 0
0.76 0
0.77 0
0.78 0
0.79 0
0.8 0
0.81 0
0.820000000000001 0
0.830000000000001 0
0.840000000000001 0
0.850000000000001 0
0.860000000000001 0
0.870000000000001 0
0.880000000000001 0
0.890000000000001 0
0.900000000000001 0
0.910000000000001 0
0.920000000000001 0
0.930000000000001 0
0.940000000000001 0
0.950000000000001 0
0.960000000000001 0
0.970000000000001 0
0.980000000000001 0
0.990000000000001 0
1 -6.20824004696096e-10
1.01 -0.000599804935089449
1.02 -0.00119907507020158
1.03 -0.00179752917333065
1.04 -0.00239483168752664
1.05 -0.00299060965330246
1.06 -0.0035844546477612
1.07 -0.00417592916044445
1.08 -0.00476457203700884
1.09 -0.00534990321841046
1.1 -0.00593142791365752
1.11 -0.0065086403093356
1.12 -0.00708102687294505
1.13 -0.00764806929888523
1.14 -0.00820924713944498
1.15 -0.00876404015770972
1.16 -0.00931193043460291
1.17 -0.00985240425820529
1.18 -0.010384953819942
1.19 -0.0109090787391257
1.2 -0.0114242874346331
1.21 -0.0119300983601258
1.22 -0.0124260411171524
1.23 -0.0129116574586643
1.24 -0.0133865021938943
1.25 -0.0138501440041666
1.26 -0.0143021661780033
1.27 -0.014742167272839
1.28 -0.0151697617097358
1.29 -0.0155845803066927
1.3 -0.0159862707554451
1.31 -0.0163744980460401
1.32 -0.0167489448429481
1.33 -0.0171093118160071
1.34 -0.0174553179291005
1.35 -0.0177867006891191
1.36 -0.0181032163574599
1.37 -0.0184046401260522
1.38 -0.0186907662596788
1.39 -0.0189614082061646
1.4 -0.0192163986758395
1.41 -0.0194555896915384
1.42 -0.0196788526102796
1.43 -0.0198860781176575
1.44 -0.0200771761963022
1.45 -0.0202520760677041
1.46 -0.0204107261104341
1.47 -0.0205530937542481
1.48 -0.0206791653509738
1.49 -0.0207889460232166
1.5 -0.0208824594900378
1.51 -0.0209597478726123
1.52 -0.0210208714788517
1.53 -0.0210659085679938
1.54 -0.0210949550957385
1.55 -0.0211081244405054
1.56 -0.0211055471113838
1.57 -0.0210873704383387
1.58 -0.021053758245239
1.59 -0.0210048905054337
1.6 -0.0209409629785557
1.61 -0.0208621868369109
1.62 -0.0207687882752429
1.63 -0.0206610081056492
1.64 -0.0205391013384232
1.65 -0.0204033367492368
1.66 -0.0202539963849012
1.67 -0.0200913752094865
1.68 -0.0199157806176659
1.69 -0.0197275319393628
1.7 -0.0195269599529585
1.71 -0.0193144063891981
1.72 -0.0190902234264388
1.73 -0.0188547731778899
1.74 -0.0186084271714953
1.75 -0.0183515658231156
1.76 -0.0180845779036665
1.77 -0.0178078600008771
1.78 -0.0175218159769454
1.79 -0.0172268564203725
1.8 -0.0169233980959289
1.81 -0.0166118633916627
1.82 -0.0162926797637
1.83 -0.0159662791796423
1.84 -0.0156330975612198
1.85 -0.0152935742268575
1.86 -0.0149481513348054
1.87 -0.0145972733274811
1.88 -0.0142413863776684
1.89 -0.013880937837207
1.9 -0.0135163756888059
1.91 -0.0131481480016023
1.92 -0.0127767023910808
1.93 -0.0124024854839615
1.94 -0.0120259423886515
1.95 -0.0116475161718502
1.96 -0.0112676473418824
1.97 -0.0108867733393261
1.98 -0.0105053280354897
1.99 -0.0101237412392771
2 -0.00974243821297125
2.01 -0.00936183919745007
2.02 -0.00898235894733461
2.03 -0.0086044062765558
2.04 -0.0082283836148098
2.05 -0.00785468657535736
2.06 -0.00748370353446157
2.07 -0.00711581519578193
2.08 -0.00675139425808641
2.09 -0.00639080499805628
2.1 -0.00603440290085287
2.11 -0.00568253430445305
2.12 -0.00533553605807665
2.13 -0.0049937351950123
2.14 -0.00465744862013147
2.15 -0.00432698281236135
2.16 -0.00400263354236993
2.17 -0.00368468560569651
2.18 -0.00337341257154243
2.19 -0.00306907654741639
2.2 -0.00277192795980941
2.21 -0.0024822053510542
2.22 -0.00220013519250341
2.23 -0.00192593171414118
2.24 -0.00165979675072184
2.25 -0.00140191960450959
2.26 -0.00115247692467226
2.27 -0.000911632603362458
2.28 -0.000679537688498825
2.29 -0.000456330313240723
2.29999999999999 -0.000242135642129242
2.30999999999999 -3.70658338486513e-05
2.31999999999999 0.000158779979457218
2.32999999999999 0.000345315696388275
2.33999999999999 0.000522468234150686
2.34999999999999 0.000690177500144704
2.35999999999999 0.00084839634744394
2.36999999999999 0.000997090515621051
2.37999999999999 0.00113623855329664
2.38999999999999 0.0012658317247547
2.39999999999999 0.00138587390394453
2.40999999999999 0.00149638145190855
2.41999999999999 0.00159738307915531
2.42999999999999 0.00168891969315537
2.43999999999999 0.00177104423111326
2.44999999999999 0.00184382147812196
2.45999999999999 0.00190732792183416
2.46999999999999 0.00196165178622707
2.47999999999999 0.00200689229538038
2.48999999999999 0.0020431596715634
2.49999999999999 0.0020705748956772
2.50999999999999 0.00208926945455804
2.51999999999999 0.00209938507597944
2.52999999999999 0.00210107345204816
2.53999999999999 0.002094495951605
2.54999999999999 0.00207982332219134
2.55999999999999 0.00205723538211287
2.56999999999999 0.00202692070311458
2.57999999999999 0.00198907628417225
2.58999999999999 0.00194390721690139
2.59999999999999 0.00189162634308319
2.60999999999999 0.00183245390480826
2.61999999999999 0.00176661718774031
2.62999999999999 0.00169435015800467
2.63999999999999 0.00161589309320912
2.64999999999999 0.00153149220810705
2.65999999999999 0.00144139927541541
2.66999999999999 0.00134587124230211
2.67999999999999 0.00124516984305894
2.68999999999999 0.00113956120847756
2.69999999999999 0.00102931547244663
2.70999999999999 0.000914706376288181
2.71999999999999 0.000796010871351132
2.72999999999999 0.000673508720378468
2.73999999999999 0.00054748209816333
2.74999999999999 0.000418215192006679
2.75999999999999 0.000285993802486533
2.76999999999998 0.000151104945045282
2.77999999999998 1.38364528974718e-05
2.78999999999998 -0.000125523418244083
2.79999999999998 -0.000266686383131384
2.80999999999998 -0.000409364517146857
2.81999999999998 -0.000553270643191168
2.82999999999998 -0.000698118714819796
2.83999999999998 -0.000843624195162716
2.84999999999998 -0.000989504431169413
2.85999999999998 -0.00113547902272993
2.86999999999998 -0.00128127018623205
2.87999999999998 -0.00142660311212407
2.88999999999998 -0.00157120631606294
2.89999999999998 -0.00171481198323777
2.90999999999998 -0.00185715630546675
2.91999999999998 -0.0019979798106756
2.92999999999998 -0.00213702768442753
2.93999999999998 -0.002274050083068
2.94999999999998 -0.00240880243818
2.95999999999998 -0.00254104575200257
2.96999999999998 -0.00267054688348841
2.97999999999998 -0.00279707882469062
2.98999999999998 -0.00292042096718226
2.99999999999998 -0.00304035935822703
3.00999999999998 -0.00315668694643378
3.01999999999998 -0.00326920381664274
3.02999999999998 -0.00337771741380613
3.03999999999998 -0.00348204275564128
3.04999999999998 -0.00358200263385007
3.05999999999998 -0.0036774278037138
3.06999999999998 -0.00376815716188857
3.07999999999998 -0.00385403791224242
3.08999999999998 -0.00393492571959095
3.09999999999998 -0.00401068485116643
3.10999999999998 -0.00408118830584149
3.11999999999998 -0.00414631793090938
3.12999999999998 -0.0042059645262336
3.13999999999998 -0.00426002793589267
3.14999999999998 -0.00430841712720164
3.15999999999998 -0.00435105025707846
3.16999999999998 -0.00438785472595898
3.17999999999998 -0.00441876721961255
3.18999999999998 -0.00444373373514415
3.19999999999998 -0.00446270959802892
3.20999999999998 -0.00447565946523311
3.21999999999998 -0.00448255731577869
3.22999999999998 -0.00448338642880349
3.23999999999997 -0.00447813934917143
3.24999999999997 -0.00446681784068277
3.25999999999997 -0.00444943282692012
3.26999999999997 -0.00442600431973534
3.27999999999997 -0.00439656133532557
3.28999999999997 -0.00436114179774506
3.29999999999997 -0.00431979242952093
3.30999999999997 -0.00427256998542847
3.31999999999997 -0.00421953778052145
3.32999999999997 -0.00416076834767293
3.33999999999997 -0.00409634251186089
3.34999999999997 -0.00402634923381546
3.35999999999997 -0.00395088544581628
3.36999999999997 -0.00387005588188806
3.37999999999997 -0.00378397290724588
3.38999999999997 -0.00369275637567997
3.39999999999997 -0.00359653268860509
3.40999999999997 -0.00349543557059216
3.41999999999997 -0.00338960562032342
3.42999999999997 -0.00327918997186837
3.43999999999997 -0.00316434205731013
3.44999999999997 -0.00304522136228964
3.45999999999997 -0.00292199317481146
3.46999999999997 -0.00279482832766527
3.47999999999997 -0.00266390281117687
3.48999999999997 -0.00252939748714201
3.49999999999997 -0.00239149828907784
3.50999999999997 -0.0022503955815361
3.51999999999997 -0.00210628389052851
3.52999999999997 -0.00195936162176884
3.53999999999997 -0.00180983077191651
3.54999999999997 -0.0016578966350299
3.55999999999997 -0.00150376750539006
3.56999999999997 -0.00134765437744442
3.57999999999997 -0.00118977064344378
3.58999999999997 -0.00103032891787928
3.59999999999997 -0.000869549872670158
3.60999999999997 -0.000707652187363879
3.61999999999997 -0.000544855728184272
3.62999999999997 -0.000381381238593514
3.63999999999997 -0.000217450030708302
3.64999999999997 -5.32836778285742e-05
3.65999999999997 0.000110896291623577
3.66999999999997 0.000274868698432642
3.67999999999997 0.000438413014848592
3.68999999999997 0.000601309664426964
3.69999999999997 0.000763340319099222
3.70999999999996 0.000924288193135567
3.71999999999996 0.00108393833362723
3.72999999999996 0.0012420779071484
3.73999999999996 0.00139849648225157
3.74999999999996 0.00155298630745587
3.75999999999996 0.00170534258439347
3.76999999999996 0.00185536373578626
3.77999999999996 0.00200285166793178
3.78999999999996 0.00214761202738551
3.79999999999996 0.0022894544515349
3.80999999999996 0.00242819281276937
3.81999999999996 0.00256364545595971
3.82999999999996 0.00269563542904389
3.83999999999996 0.00282399070619741
3.84999999999996 0.00294854440359831
3.85999999999996 0.00306913498753567
3.86999999999996 0.00318560647448808
3.87999999999996 0.00329780862300728
3.88999999999996 0.00340559711719455
3.89999999999996 0.00350883374156981
3.90999999999996 0.00360738654714454
3.91999999999996 0.0037011300085222
3.92999999999996 0.00378994517186203
3.93999999999996 0.00387371979355395
3.94999999999996 0.00395234846946565
3.95999999999996 0.00402573275463446
3.96999999999996 0.0040937812732901
3.97999999999996 0.00415640981910618
3.98999999999996 0.00421354144559161
3.99999999999996 0.00426510654654482
4.00999999999996 0.00431104292650664
4.01999999999996 0.00435129586115921
4.02999999999996 0.00438581814763041
4.03999999999996 0.00441457014467423
4.04999999999996 0.00443751980270837
4.05999999999996 0.00445464268377633
4.06999999999996 0.00446592197322277
4.07999999999996 0.00447134847498853
4.08999999999996 0.00447092060329592
4.09999999999996 0.00446464464673244
4.10999999999996 0.00445253418648235
4.11999999999996 0.00443461034586673
4.12999999999996 0.00441090172501595
4.13999999999996 0.00438144434208721
4.14999999999996 0.00434628156505436
4.15999999999996 0.00430546403421719
4.16999999999996 0.00425904957560252
4.17999999999996 0.00420710310546587
4.18999999999996 0.00414969652616012
4.19999999999995 0.00408690861374305
4.20999999999995 0.00401882489790146
4.21999999999995 0.00394553753698013
4.22999999999995 0.00386714518338983
4.23999999999995 0.00378375285389456
4.24999999999995 0.00369547181217314
4.25999999999995 0.00360241856259226
4.26999999999995 0.00350471595296653
4.27999999999995 0.00340249269161272
4.28999999999995 0.00329588295552423
4.29999999999995 0.00318502621239112
4.30999999999995 0.00307006703098643
4.31999999999995 0.00295115488310913
4.32999999999995 0.00282844393862768
4.33999999999995 0.00270209285450631
4.34999999999995 0.00257226455840815
4.35999999999995 0.00243912602733374
4.36999999999995 0.00230284806168909
4.37999999999995 0.00216360505514386
4.38999999999995 0.0020215747606239
4.39999999999995 0.00187693805277433
4.40999999999995 0.00172987868722497
4.41999999999995 0.00158058305698879
4.42999999999995 0.00142923994632348
4.43999999999995 0.00127604028238626
4.44999999999995 0.00112117688501278
4.45999999999995 0.000964844214950388
4.46999999999995 0.000807235461569877
4.47999999999995 0.000648550615547445
4.48999999999995 0.000488987496423043
4.49999999999995 0.000328744554208681
4.50999999999995 0.000168020614053058
4.51999999999995 7.01462139128355e-06
4.52999999999995 -0.000154074611784431
4.53999999999995 -0.000315048659654247
4.54999999999995 -0.000475709733840413
4.55999999999995 -0.000635860932909991
4.56999999999995 -0.000795306489805999
4.57999999999995 -0.000953852016904347
4.58999999999995 -0.00111130474839825
4.59999999999995 -0.00126747377971501
4.60999999999995 -0.00142217030367456
4.61999999999995 -0.00157520784310449
4.62999999999995 -0.00172640247963031
4.63999999999995 -0.00187557307836667
4.64999999999995 -0.00202254150824036
4.65999999999995 -0.00216713285768297
4.66999999999994 -0.00230917564543689
4.67999999999994 -0.00244850202622659
4.68999999999994 -0.00258494799105343
4.69999999999994 -0.00271835356186927
4.70999999999994 -0.00284856298042708
4.71999999999994 -0.00297542489110627
4.72999999999994 -0.00309879251742872
4.73999999999994 -0.00321852383213038
4.74999999999994 -0.00333448172057924
4.75999999999994 -0.00344653413580643
4.76999999999994 -0.00355455423942545
4.77999999999994 -0.00365842056840961
4.78999999999994 -0.00375801715584102
4.79999999999994 -0.003853233658577
4.80999999999994 -0.00394396547730996
4.81999999999994 -0.00403011386889854
4.82999999999994 -0.00411158605085688
4.83999999999994 -0.00418829529789924
4.84999999999994 -0.0042601610304998
4.85999999999994 -0.00432710889525932
4.86999999999994 -0.00438907083693039
4.87999999999994 -0.00444598516249748
4.88999999999994 -0.0044977959748439
4.89999999999994 -0.0045444553353715
4.90999999999994 -0.00458592093713267
4.91999999999994 -0.00462215701171856
4.92999999999994 -0.00465313435135912
4.93999999999994 -0.00467883032269187
4.94999999999994 -0.00469922887221158
4.95999999999994 -0.00471432052342417
4.96999999999994 -0.00472410236800312
4.97999999999994 -0.00472857804180838
4.98999999999994 -0.00472775769965577
4.99999999999994 -0.00472165797986968
5.00999999999994 -0.00471030196070779
5.01999999999994 -0.00469371910887444
5.02999999999994 -0.00467194522022368
5.03999999999994 -0.00464502235277793
5.04999999999994 -0.00461299875220802
5.05999999999994 -0.00457592876994707
5.06999999999994 -0.00453387277415419
5.07999999999994 -0.00448689705381486
5.08999999999994 -0.00443507371736574
5.09999999999994 -0.00437848058417031
5.10999999999994 -0.00431720106968731
5.11999999999994 -0.00425132407180661
5.12999999999994 -0.00418094385854662
5.13999999999993 -0.00410615997131388
5.14999999999993 -0.00402707718060641
5.15999999999993 -0.003943804157296
5.16999999999993 -0.00385645460940342
5.17999999999993 -0.00376514718744777
5.18999999999993 -0.00367000475357741
5.19999999999993 -0.00357115429074456
5.20999999999993 -0.00346872743401102
5.21999999999993 -0.00336285961822173
5.22999999999993 -0.00325368994875038
5.23999999999993 -0.00314136103808894
5.24999999999993 -0.00302601882666007
5.25999999999993 -0.00290781239478033
5.26999999999993 -0.00278689376880001
5.27999999999993 -0.00266341772293999
5.28999999999993 -0.00253754157770604
5.29999999999993 -0.00240942499546831
5.30999999999993 -0.00227922977364922
5.31999999999993 -0.00214711963589068
5.32999999999993 -0.00201326002153099
5.33999999999993 -0.00187781787369954
5.34999999999993 -0.00174096142632524
5.35999999999993 -0.00160285999034549
5.36999999999993 -0.0014636837393984
5.37999999999993 -0.00132360349527647
5.38999999999993 -0.00118279051341722
5.39999999999993 -0.00104141626870399
5.40999999999993 -0.000899652241847281
5.41999999999993 -0.000757669706614728
5.42999999999993 -0.000615639518175104
5.43999999999993 -0.000473731902818848
5.44999999999993 -0.000332116249314907
5.45999999999993 -0.000190960902159837
5.46999999999993 -5.04329569720495e-05
5.47999999999993 8.93019417202205e-05
5.48999999999993 0.000228079800041996
5.49999999999993 0.000365738471476317
5.50999999999993 0.000502117850449643
5.51999999999993 0.00063706006256323
5.52999999999993 0.000770409651257955
5.53999999999993 0.000902013760688953
5.54999999999993 0.00103172231459537
5.55999999999993 0.00115938819095742
5.56999999999993 0.00128486739223842
5.57999999999993 0.00140801921101611
5.58999999999993 0.00152870639081464
5.59999999999993 0.00164679528195464
5.60999999999992 0.00176215599224679
5.61999999999992 0.00187466253236073
5.62999999999992 0.00198419295570918
5.63999999999992 0.00209062949269439
5.64999999999992 0.00219385867917211
5.65999999999992 0.00229377147899574
5.66999999999992 0.00239026340051196
5.67999999999992 0.00248323460688686
5.68999999999992 0.00257259002015007
5.69999999999992 0.00265823941885293
5.70999999999992 0.00274009752924485
5.71999999999992 0.00281808410988128
5.72999999999992 0.00289212402958467
5.73999999999992 0.0029621473386891
5.74999999999992 0.00302808933350765
5.75999999999992 0.00308989061397087
5.76999999999992 0.00314749713439325
5.77999999999992 0.0032008602473337
5.78999999999992 0.00324993674052499
5.79999999999992 0.00329468886685635
5.80999999999992 0.00333508436740206
5.81999999999992 0.00337109648937566
5.82999999999992 0.00340270399230959
5.83999999999992 0.00342989115028734
5.84999999999992 0.00345264774830502
5.85999999999992 0.00347096907076114
5.86999999999992 0.00348485588315292
5.87999999999992 0.00349431440704464
5.88999999999992 0.00349935628838909
5.89999999999992 0.00349999855930359
5.90999999999992 0.00349626359343003
5.91999999999992 0.0034881790550528
5.92999999999992 0.00347577784222114
5.93999999999992 0.00345909802425477
5.94999999999992 0.00343818277426739
5.95999999999992 0.00341308029787488
5.96999999999992 0.00338384376045218
5.97999999999992 0.00335053121826293
5.98999999999992 0.00331320556689449
5.99999999999992 0.00327193453511134
6.00999999999992 0.00322679047194985
6.01999999999992 0.00317784838969669
6.02999999999992 0.0031251889301446
6.03999999999992 0.00306889694808106
6.04999999999992 0.00300906138554603
6.05999999999992 0.00294577514168215
6.06999999999992 0.00287913493811646
6.07999999999991 0.00280924092472588
6.08999999999991 0.00273619668547566
6.09999999999991 0.00266010981217613
6.10999999999991 0.00258109103627024
6.11999999999991 0.00249925412753527
6.12999999999991 0.00241471576218468
6.13999999999991 0.00232759537632771
6.14999999999991 0.00223801501117397
6.15999999999991 0.00214609915289263
6.16999999999991 0.00205197456862247
6.17999999999991 0.00195577013950163
6.18999999999991 0.00185761669128764
6.19999999999991 0.00175764681976139
6.20999999999991 0.00165599472178684
6.21999999999991 0.00155279602337261
6.22999999999991 0.00144818760072067
6.23999999999991 0.00134230740398042
6.24999999999991 0.00123529428039433
6.25999999999991 0.00112728779707863
6.26999999999991 0.00101842806367563
6.27999999999991 0.000908855555109472
6.28999999999991 0.00079871093467343
6.29999999999991 0.000688134877673387
6.30999999999991 0.000577267895849757
6.31999999999991 0.000466250162796697
6.32999999999991 0.000355221340595178
6.33999999999991 0.000244320407873236
6.34999999999991 0.000133685489503834
6.35999999999991 2.34536881477034e-05
6.36999999999991 -8.62390821554827e-05
6.37999999999991 -0.000195258260139037
6.38999999999991 -0.000303470799033551
6.39999999999991 -0.000410745325644314
6.40999999999991 -0.000516952296601492
6.41999999999991 -0.000621964151602294
6.42999999999991 -0.000725660593380797
6.43999999999991 -0.000827913571499038
6.44999999999991 -0.000928602358815346
6.45999999999991 -0.00102760878570075
6.46999999999991 -0.00112481737599721
6.47999999999991 -0.00122011547893183
6.48999999999991 -0.00131339339684441
6.49999999999991 -0.00140454450859067
6.50999999999991 -0.0014934653884889
6.51999999999991 -0.0015800559206836
6.52999999999991 -0.00166421940880534
6.53999999999991 -0.00174586268081171
6.5499999999999 -0.00182489618890096
6.5599999999999 -0.00190123410439557
6.5699999999999 -0.00197478730522858
6.5799999999999 -0.00204548490615167
6.5899999999999 -0.00211325279471569
6.5999999999999 -0.00217802090857016
6.6099999999999 -0.00223972330370486
6.6199999999999 -0.00229829821722321
6.6299999999999 -0.00235368812459102
6.6399999999999 -0.00240583979131233
6.6499999999999 -0.00245470431899126
6.6599999999999 -0.00250023718574636
6.6699999999999 -0.00254239828095193
6.6799999999999 -0.00258115193428784
6.6899999999999 -0.00261646693908733
6.6999999999999 -0.00264831656997999
6.7099999999999 -0.0026766785964438
6.7199999999999 -0.00270153528659064
6.7299999999999 -0.00272287340762525
6.7399999999999 -0.00274068422175761
6.7499999999999 -0.00275496347556291
6.7599999999999 -0.00276571138379215
6.7699999999999 -0.00277293260769355
6.7799999999999 -0.00277663622791939
6.7899999999999 -0.00277683571211181
6.7999999999999 -0.00277354887728733
6.8099999999999 -0.00276679784718096
6.8199999999999 -0.00275660900477616
6.8299999999999 -0.00274301294036844
6.8399999999999 -0.00272604439572029
6.8499999999999 -0.00270574220530371
6.8599999999999 -0.00268214923655104
6.8699999999999 -0.00265531233316555
6.8799999999999 -0.00262528226707587
6.8899999999999 -0.00259211362150006
6.8999999999999 -0.00255586474066894
6.9099999999999 -0.0025165967870219
6.9199999999999 -0.00247437445991288
6.9299999999999 -0.00242926630670533
6.9399999999999 -0.00238134418285538
6.9499999999999 -0.00233068314459918
6.9599999999999 -0.0022773613382474
6.9699999999999 -0.00222145940927249
6.9799999999999 -0.00216306121370674
6.9899999999999 -0.00210225362588827
6.9999999999999 -0.00203912613635554
7.00999999999989 -0.00197377067024289
7.01999999999989 -0.00190628148247331
7.02999999999989 -0.00183675504006929
7.03999999999989 -0.00176528989673935
7.04999999999989 -0.00169198656258793
7.05999999999989 -0.00161694737041245
7.06999999999989 -0.00154027633942891
7.07999999999989 -0.00146207903696684
7.08999999999989 -0.00138246243852087
7.09999999999989 -0.00130153478646362
7.10999999999989 -0.00121940544767849
7.11999999999989 -0.00113618477034365
7.12999999999989 -0.00105198394008244
7.13999999999989 -0.000966914835684872
7.14999999999989 -0.000881089884597699
7.15999999999989 -0.000794621918376466
7.16999999999989 -0.000707624028288304
7.17999999999989 -0.000620209421252096
7.18999999999989 -0.000532491276299447
7.19999999999989 -0.000444582601737799
7.20999999999989 -0.000356596093194438
7.21999999999989 -0.00026864399271485
7.22999999999989 -0.000180837949082838
7.23999999999989 -9.32888795676861e-05
7.24999999999989 -6.10683323587312e-06
7.25999999999989 8.05991440051282e-05
7.26999999999989 0.000166721142457096
7.27999999999989 0.000252156637665218
7.28999999999989 0.000336796596390564
7.29999999999989 0.000420537295889735
7.30999999999989 0.000503276654801194
7.31999999999989 0.000584914354700945
7.32999999999989 0.000665351958822972
7.33999999999989 0.000744493027812859
7.34999999999989 0.000822243232385642
7.35999999999989 0.000898510462763617
7.36999999999989 0.000973204934773446
7.37999999999989 0.00104623929225828
7.38999999999989 0.00111752870621502
7.39999999999989 0.00118699097066564
7.40999999999989 0.00125454659396195
7.41999999999989 0.00132011888652417
7.42999999999989 0.00138363404461002
7.43999999999989 0.00144502123002631
7.44999999999989 0.00150421264570028
7.45999999999989 0.00156113646160955
7.46999999999989 0.00161573844811654
7.47999999999988 0.00166796039515402
7.48999999999988 0.00171774739637853
7.49999999999988 0.00176504790293984
7.50999999999988 0.0018098137727652
7.51999999999988 0.00185200031531569
7.52999999999988 0.00189156633177879
7.53999999999988 0.00192847415066709
7.54999999999988 0.00196268965879968
7.55999999999988 0.00199418232793785
7.56999999999988 0.00202292523586171
7.57999999999988 0.00204889508358017
7.58999999999988 0.00207207220966843
7.59999999999988 0.00209244059477779
7.60999999999988 0.00210998786576541
7.61999999999988 0.00212470529468979
7.62999999999988 0.00213658779278543
7.63999999999988 0.00214563389991501
7.64999999999988 0.00215184576955495
7.65999999999988 0.00215522914938708
7.66999999999988 0.00215579335759259
7.67999999999988 0.00215355125497954
7.68999999999988 0.00214851921313195
7.69999999999988 0.00214071707886493
7.70999999999988 0.00213016813544367
7.71999999999988 0.00211689906135571
7.72999999999988 0.00210093988810687
7.73999999999988 0.00208232394654373
7.74999999999988 0.00206108779547985
7.75999999999988 0.00203727118366369
7.76999999999988 0.00201091700812386
7.77999999999988 0.00198207128432461
7.78999999999988 0.001950781828365
7.79999999999988 0.00191710019292825
7.80999999999988 0.0018810807356852
7.81999999999988 0.00184278053642522
7.82999999999988 0.00180225931126111
7.83999999999988 0.00175957932400581
7.84999999999988 0.00171480529482668
7.85999999999988 0.00166800373718787
7.86999999999988 0.00161924411007678
7.87999999999988 0.00156859801312588
7.88999999999988 0.00151613909179281
7.89999999999988 0.00146194297054566
7.90999999999988 0.00140608712173824
7.91999999999988 0.00134865074843961
7.92999999999988 0.00128971468263182
7.93999999999988 0.00122936127917223
7.94999999999987 0.00116767430682384
7.95999999999987 0.00110473883711284
7.96999999999987 0.00104064113150293
7.97999999999987 0.000975468527233935
7.98999999999987 0.00090930932209475
7.99999999999987 0.000842252658356019
8.00999999999987 0.000774388406061043
8.01999999999987 0.000705807045857621
8.02999999999987 0.000636599551542444
8.03999999999987 0.000566857272482888
8.04999999999987 0.000496671816075962
8.05999999999987 0.000426134930400468
8.06999999999987 0.000355338387215376
8.07999999999987 0.000284373865454576
8.08999999999987 0.000213332835366784
8.09999999999987 0.000142306443445659
8.10999999999987 7.13853982943479e-05
8.11999999999987 6.59857565842528e-07
8.12999999999987 -6.97806838820047e-05
8.13999999999987 -0.000139850805210543
8.14999999999987 -0.000209459774956198
8.15999999999987 -0.000278520749688337
8.16999999999987 -0.00034694797627391
8.17999999999987 -0.000414656895971967
8.18999999999987 -0.000481564246497663
8.19999999999987 -0.000547588161941186
8.20999999999987 -0.000612648270429573
8.21999999999987 -0.000676665789421994
8.22999999999987 -0.000739563618531636
8.23999999999987 -0.000801266429770983
8.24999999999987 -0.000861700755119758
8.25999999999987 -0.000920795071318704
8.26999999999987 -0.000978479881796109
8.27999999999987 -0.0010346877955724
8.28999999999987 -0.00108935360284962
8.29999999999987 -0.00114241434862255
8.30999999999987 -0.00119380940187249
8.31999999999987 -0.00124348052188914
8.32999999999987 -0.00129137192124582
8.33999999999987 -0.00133743032536279
8.34999999999987 -0.00138160502859717
8.35999999999987 -0.00142384794680209
8.36999999999987 -0.00146411366630185
8.37999999999987 -0.00150235219478862
8.38999999999987 -0.00153853113148808
8.39999999999987 -0.00157261339737164
8.40999999999987 -0.00160456474580239
8.41999999999986 -0.00163435379307395
8.42999999999986 -0.00166195204518922
8.43999999999986 -0.00168733392086326
8.44999999999986 -0.00171047677073958
8.45999999999986 -0.00173136089281494
8.46999999999986 -0.00174996954487427
8.47999999999986 -0.00176628895173926
8.48999999999986 -0.00178030830796584
8.49999999999986 -0.00179201977931951
8.50999999999986 -0.00180141849937683
8.51999999999986 -0.00180850256243773
8.52999999999986 -0.00181327301279681
8.53999999999986 -0.0018157338304369
8.54999999999986 -0.00181589191323062
8.55999999999986 -0.00181375705576794
8.56999999999986 -0.00180934192498063
8.57999999999986 -0.00180266203282225
8.58999999999986 -0.0017937357064186
8.59999999999986 -0.00178258405096769
8.60999999999986 -0.00176923090555191
8.61999999999986 -0.00175370280725719
8.62999999999986 -0.00173602894993056
8.63999999999986 -0.00171624114289856
8.64999999999986 -0.00169437377320341
8.65999999999986 -0.00167046378140815
8.66999999999986 -0.00164455006632706
8.67999999999986 -0.00161667379237684
8.68999999999986 -0.00158687879244762
8.69999999999986 -0.0015552111428222
8.70999999999986 -0.00152171909405082
8.71999999999986 -0.00148645299948285
8.72999999999986 -0.00144946524153963
8.73999999999986 -0.00141081015581826
8.74999999999986 -0.00137054351940971
8.75999999999986 -0.00132872339734451
8.76999999999986 -0.00128540957814996
8.77999999999986 -0.00124066346295995
8.78999999999986 -0.00119454798648718
8.79999999999986 -0.0011471275337895
8.80999999999986 -0.00109846785422423
8.81999999999986 -0.00104863597940121
8.82999999999986 -0.00099770013545651
8.83999999999986 -0.000945729639301679
8.84999999999986 -0.000892794806543536
8.85999999999986 -0.000838966859432882
8.86999999999986 -0.000784317833618063
8.87999999999986 -0.000728920483983416
8.88999999999985 -0.000672848189794361
8.89999999999985 -0.000616174859335745
8.90999999999985 -0.000558974834208195
8.91999999999985 -0.000501322793433962
8.92999999999985 -0.000443293657513949
8.93999999999985 -0.000384962492571719
8.94999999999985 -0.000326404414715169
8.95999999999985 -0.000267694494743351
8.96999999999985 -0.000208907663322885
8.97999999999985 -0.000150118616755732
8.98999999999985 -9.14017234582041e-05
8.99999999999985 -3.28309312683047e-05
9.00999999999985 2.55203243028524e-05
9.01999999999985 8.35792107619375e-05
9.02999999999985 0.00014127677580298
9.03999999999985 0.000198538903520057
9.04999999999985 0.000255295104574844
9.05999999999985 0.000311475837877357
9.06999999999985 0.000367012594672128
9.07999999999985 0.000421837980903006
9.08999999999985 0.000475885797764912
9.09999999999985 0.000529091120353115
9.10999999999985 0.000581390374322429
9.11999999999985 0.000632721410471155
9.12999999999985 0.000683023577166984
9.13999999999985 0.00073223779053474
9.14999999999985 0.000780306602328778
9.15999999999985 0.000827174265415604
9.16999999999985 0.00087278679646283
9.17999999999985 0.000917092036956854
9.18999999999985 0.000960039710998503
9.19999999999985 0.00100158148045361
9.20999999999985 0.00104167099747818
9.21999999999985 0.00108026395427324
9.22999999999985 0.00111731813001731
9.23999999999985 0.00115279343492791
9.24999999999985 0.00118665195140671
9.25999999999985 0.00121885088829974
9.26999999999985 0.0012493640696118
9.27999999999985 0.0012781603788664
9.28999999999985 0.00130521100880888
9.29999999999985 0.00133048948757102
9.30999999999985 0.00135397170176536
9.31999999999985 0.00137563591649521
9.32999999999985 0.00139546279227105
9.33999999999985 0.00141343539882871
9.34999999999985 0.00142953922598494
9.35999999999984 0.00144376219237113
9.36999999999984 0.00145609464721752
9.37999999999984 0.00146652937332167
9.38999999999984 0.00147506158548325
9.39999999999984 0.00148168892592673
9.40999999999984 0.00148641145675435
9.41999999999984 0.00148923164948628
9.42999999999984 0.00149015437176504
9.43999999999984 0.00148918687129692
9.44999999999984 0.00148633875496134
9.45999999999984 0.00148162196563481
9.46999999999984 0.00147505075766903
9.47999999999984 0.00146664166942775
9.48999999999984 0.00145641349345246
9.49999999999984 0.00144438724455574
9.50999999999984 0.00143058612635071
9.51999999999984 0.0014150354971814
9.52999999999984 0.00139776283748148
9.53999999999984 0.00137879772374261
9.54999999999984 0.00135817169861324
9.55999999999984 0.00133591761490014
9.56999999999984 0.00131207077615943
9.57999999999984 0.00128666834445051
9.58999999999984 0.00125974928451275
9.59999999999984 0.00123135430599897
9.60999999999984 0.00120152580383278
9.61999999999984 0.00117030779676157
9.62999999999984 0.00113774560945676
9.63999999999984 0.00110388623574552
9.64999999999984 0.00106877824485185
9.65999999999984 0.0010324715504204
9.66999999999984 0.000995017347696172
9.67999999999984 0.000956468046548749
9.68999999999984 0.000916877201846541
9.69999999999984 0.000876299441927932
9.70999999999984 0.000834790395593225
9.71999999999984 0.000792406617890752
9.72999999999984 0.000749205514895125
9.73999999999984 0.00070524526783527
9.74999999999984 0.000660584760629904
9.75999999999984 0.000615283498232654
9.76999999999984 0.000569401526763872
9.77999999999984 0.000522999355723668
9.78999999999984 0.000476137879843199
9.79999999999984 0.000428878300708699
9.80999999999984 0.000381282048282437
9.81999999999984 0.000333410702436891
9.82999999999983 0.000285325914613641
9.83999999999983 0.00023708932971419
9.84999999999983 0.000188762508327309
9.85999999999983 0.000140406849394422
9.86999999999983 9.20835132091202e-05
9.87999999999983 4.38533453939232e-05
9.88999999999983 -4.22319810482498e-06
9.89999999999983 -5.20861259955194e-05
9.90999999999983 -9.96759853255754e-05
9.91999999999983 -0.000146933934163415
9.92999999999983 -0.000193805135612527
9.93999999999983 -0.000240229292625882
9.94999999999983 -0.000286149815350018
9.95999999999983 -0.000331510999699932
9.96999999999983 -0.000376258093959556
9.97999999999983 -0.000420337363852729
9.98999999999983 -0.000463696156012699
9.99999999999983 -0.000506282959779911
10.0099999999998 -0.000548047467260037
10.0199999999998 -0.000588940631575461
10.0299999999998 -0.000628914723246592
10.0399999999998 -0.000667923384640857
10.0499999999998 -0.000705921682296621
10.0599999999998 -0.000742866157362893
10.0699999999998 -0.000778714874204813
10.0799999999998 -0.000813427466358761
10.0899999999998 -0.000846965180478304
10.0999999999998 -0.000879290918029244
10.1099999999998 -0.000910369274689687
10.1199999999998 -0.0009401598003089
10.1299999999998 -0.000968637432383786
10.1399999999998 -0.000995772124711121
10.1499999999998 -0.00102153566112673
10.1599999999998 -0.0010459016829772
10.1699999999998 -0.00106884571411795
10.1799999999998 -0.00109034518341877
10.1899999999998 -0.00111037944476166
10.1999999999998 -0.00112892979451891
10.2099999999998 -0.00114597948650334
10.2199999999998 -0.00116151374438606
10.2299999999998 -0.00117551977158153
10.2399999999998 -0.00118798675942843
10.2499999999998 -0.00119890589067875
10.2599999999998 -0.00120827034192755
10.2699999999998 -0.00121607528386175
10.2799999999998 -0.00122231787860177
10.2899999999998 -0.00122699727395598
10.2999999999998 -0.00123011459541521
10.3099999999998 -0.00123167293598113
10.3199999999998 -0.00123167734359579
10.3299999999998 -0.00123013480621586
10.3399999999998 -0.00122705423458603
10.3499999999998 -0.00122244644278191
10.3599999999998 -0.001216324126617
10.3699999999998 -0.00120870184004906
10.3799999999998 -0.00119959596979458
10.3899999999998 -0.00118902470849569
10.3999999999998 -0.0011770080270711
10.4099999999998 -0.00116356764752744
10.4199999999998 -0.00114872701915179
10.4299999999998 -0.00113251130674797
10.4399999999998 -0.00111494677520048
10.4499999999998 -0.00109606151968201
10.4599999999998 -0.00107588529014164
10.4699999999998 -0.00105444933396529
10.4799999999998 -0.00103178634936328
10.4899999999998 -0.00100793043719807
10.4999999999998 -0.000982917051309483
10.5099999999998 -0.000956782947398294
10.5199999999998 -0.00092956581235388
10.5299999999998 -0.000901304910020429
10.5399999999998 -0.000872040628589724
10.5499999999998 -0.000841814428852033
10.5599999999998 -0.000810668791075025
10.5699999999998 -0.000778647159077296
10.5799999999998 -0.000745793882423534
10.5899999999998 -0.000712154157227079
10.5999999999998 -0.000677773965849966
10.6099999999998 -0.000642700015696486
10.6199999999998 -0.00060697967724797
10.6299999999998 -0.000570660921460712
10.6399999999998 -0.000533792256633873
10.6499999999998 -0.000496422664845904
10.6599999999998 -0.000458601537760285
10.6699999999998 -0.000420378612823202
10.6799999999998 -0.000381803909214266
10.6899999999998 -0.000342927663125138
10.6999999999998 -0.000303800263471485
10.7099999999998 -0.000264472188679314
10.7199999999998 -0.000224993941329008
10.7299999999998 -0.000185415983906064
10.7399999999998 -0.000145788675373281
10.7499999999998 -0.000106162208139414
10.7599999999998 -6.65865455067837e-05
10.7699999999998 -2.71113596776424e-05
10.7799999999998 1.22140296020315e-05
10.7899999999998 5.13407156844729e-05
10.7999999999998 9.02202648730018e-05
10.8099999999998 0.000128804775414271
10.8199999999998 0.00016704693555465
10.8299999999998 0.000204900080593005
10.8399999999998 0.00024232190573987
10.8499999999998 0.000279263943058864
10.8599999999998 0.000315681785078567
10.8699999999998 0.000351531884760017
10.8799999999998 0.000386771606715114
10.8899999999998 0.000421359277024456
10.8999999999998 0.000455254231599564
10.9099999999998 0.000488416863036509
10.9199999999998 0.000520808665909417
10.9299999999998 0.000552392280409047
10.9399999999998 0.000583131534214497
10.9499999999998 0.000612991483231111
10.9599999999998 0.00064193844995652
10.9699999999998 0.000669933998864622
10.9799999999998 0.000696953082136159
10.9899999999998 0.000722966088776144
10.9999999999998 0.00074794481225959
11.0099999999998 0.000771862479840828
11.0199999999998 0.000794693779914225
11.0299999999998 0.000816414887401842
11.0399999999998 0.000837003487146393
11.0499999999998 0.000856438795290178
11.0599999999998 0.00087470157862381
11.0699999999998 0.000891774171891166
11.0799999999998 0.000907640493234607
11.0899999999998 0.000922286057344735
11.0999999999998 0.000935697986221989
11.1099999999998 0.000947865018170779
11.1199999999998 0.000958777514952468
11.1299999999998 0.000968427466799934
11.1399999999998 0.000976808494020223
11.1499999999998 0.000983915848419613
11.1599999999998 0.000989746411946237
11.1699999999998 0.000994298693292091
11.1799999999998 0.000997572822472357
11.1899999999998 0.000999570543404075
11.1999999999998 0.00100029520451101
11.2099999999998 0.000999751747387591
11.2199999999998 0.00099794669356283
11.2299999999998 0.000994888129377038
11.2399999999998 0.000990585688929095
11.2499999999998 0.000985050535831495
11.2599999999998 0.00097829534299218
11.2699999999998 0.000970334271192795
11.2799999999998 0.000961182946747945
11.2899999999998 0.00095085843901779
11.2999999999998 0.000939379239445999
11.3099999999998 0.000926765246471214
11.3199999999998 0.00091303753131548
11.3299999999998 0.000898218263948649
11.3399999999998 0.000882331161393508
11.3499999999998 0.000865401182911744
11.3599999999998 0.000847454492481596
11.3699999999998 0.000828518419978735
11.3799999999998 0.000808621421106119
11.3899999999998 0.000787793036121549
11.3999999999998 0.000766063698828781
11.4099999999998 0.000743464905264325
11.4199999999998 0.000720029235848534
11.4299999999998 0.000695790167285268
11.4399999999998 0.000670782030727629
11.4499999999998 0.00064503996700993
11.4599999999998 0.00061859988003931
11.4699999999998 0.000591498388885998
11.4799999999998 0.00056377277886341
11.4899999999998 0.000535460951804529
11.4999999999998 0.000506601375662026
11.5099999999998 0.000477233033542929
11.5199999999998 0.000447395372270723
11.5299999999998 0.000417128250558328
11.5399999999998 0.00038647188686994
11.5499999999998 0.000355466807046172
11.5599999999998 0.000324153791764852
11.5699999999998 0.000292573823907893
11.5799999999998 0.000260768035903815
11.5899999999998 0.000228777657114424
11.5999999999998 0.000196643961333088
11.6099999999998 0.000164408214461729
11.6199999999998 0.000132111622432408
11.6299999999998 9.97952794389247e-05
11.6399999999998 6.75001165428969e-05
11.6499999999998 3.52668507179086e-05
11.6599999999998 3.13593439449729e-06
11.6699999999998 -2.88524944323342e-05
11.6799999999998 -6.06586614716403e-05
11.6899999999998 -9.22432047239924e-05
11.6999999999998 -0.000123567222326379
11.7099999999998 -0.000154592319598845
11.7199999999998 -0.00018528065523672
11.7299999999998 -0.000215594986593878
11.7399999999998 -0.000245498714003796
11.7499999999998 -0.000274955924086829
11.7599999999998 -0.00030393143199341
11.7699999999998 -0.000332390822534617
11.7799999999998 -0.000360300490153205
11.7899999999998 -0.000387627677689754
11.7999999999998 -0.000414340513900453
11.8099999999998 -0.000440408049684799
11.8199999999998 -0.000465800292967466
11.8299999999998 -0.000490488242259983
11.8399999999998 -0.000514443918772006
11.8499999999998 -0.000537640397095405
11.8599999999998 -0.000560051834420414
11.8699999999998 -0.000581653498250909
11.8799999999998 -0.000602421792590642
11.8899999999998 -0.000622334282574405
11.8999999999998 -0.000641369717520064
11.9099999999998 -0.000659508052379704
11.9199999999998 -0.000676730467570066
11.9299999999998 -0.000693019387164815
11.9399999999998 -0.000708358495433186
11.9499999999998 -0.000722732751711902
11.9599999999998 -0.00073612840359936
11.9699999999998 -0.000748532998463381
11.9799999999998 -0.000759935393256034
11.9899999999998 -0.000770325762631275
11.9999999999998 -0.000779695605363544
12.0099999999998 -0.000788037749619298
12.0199999999998 -0.000795346354954356
12.0299999999998 -0.000801616913978034
12.0399999999998 -0.000806846252139257
12.0499999999998 -0.000811032525705427
12.0599999999998 -0.000814175218081797
12.0699999999998 -0.000816275134487731
12.0799999999998 -0.000817334395009869
12.0899999999998 -0.000817356426056656
12.0999999999998 -0.000816345950244471
12.1099999999998 -0.000814308974753389
12.1199999999998 -0.000811252778202241
12.1299999999998 -0.000807185896110798
12.1399999999998 -0.000802118105047445
12.1499999999998 -0.000796060405616482
12.1599999999998 -0.000789025004463191
12.1699999999998 -0.00078102529493584
12.1799999999998 -0.000772075840678562
12.1899999999998 -0.000762192359837235
12.1999999999998 -0.000751391719159772
12.2099999999998 -0.000739691421513665
12.2199999999998 -0.000727110343343211
12.2299999999998 -0.00071366840212824
12.2399999999998 -0.000699386513873565
12.2499999999998 -0.000684286561870729
12.2599999999998 -0.00066839136441894
12.2699999999998 -0.000651724641544197
12.2799999999998 -0.000634310965685707
12.2899999999998 -0.000616175557589494
12.2999999999998 -0.000597344677470745
12.3099999999998 -0.000577845332120829
12.3199999999998 -0.00055770524243756
12.3299999999998 -0.000536952807806668
12.3399999999998 -0.000515617068643571
12.3499999999998 -0.000493727667697945
12.3599999999998 -0.000471314810438691
12.3699999999998 -0.000448409224710301
12.3799999999998 -0.000425042119790242
12.3899999999998 -0.000401245144945481
12.3999999999998 -0.000377050347569306
12.4099999999998 -0.000352490130969656
12.4199999999998 -0.000327597211874684
12.4299999999998 -0.000302404577717638
12.4399999999998 -0.000276945443760822
12.4499999999998 -0.000251253210116984
12.4599999999998 -0.000225361418725329
12.4699999999998 -0.000199303710338389
12.4799999999998 -0.000173113781575436
12.4899999999998 -0.000146825342097325
12.4999999999998 -0.000120472071957043
12.5099999999998 -9.40875791796826e-05
12.5199999999998 -6.77053576248382e-05
12.5299999999998 -4.13587451837116e-05
12.5399999999998 -1.5080882362538e-05
12.5499999999998 1.10953286968424e-05
12.5599999999998 3.71372647090632e-05
12.5699999999998 6.30126212014245e-05
12.5799999999998 8.86894518858158e-05
12.5899999999998 0.000114136207398641
12.5999999999998 0.000139321773363041
12.6099999999998 0.000164215507728238
12.6199999999998 0.000188787277341968
12.6299999999998 0.000213007493713324
12.6399999999998 0.000236847147924303
12.6499999999998 0.000260277844649916
12.6599999999998 0.000283271835248135
12.6699999999998 0.000305802049881257
12.6799999999998 0.000327842128632508
12.6899999999998 0.00034936645158463
12.6999999999998 0.000370350167821034
12.7099999999998 0.000390769223330445
12.7199999999998 0.000410600387784527
12.7299999999998 0.000429821280133416
12.7399999999998 0.000448410393019949
12.7499999999998 0.000466347115978384
12.7599999999998 0.000483611757393901
12.7699999999998 0.000500185565200803
12.7799999999998 0.000516050746299041
12.7899999999998 0.00053119048467044
12.7999999999998 0.000545588958177727
12.8099999999998 0.000559231354031266
12.8199999999998 0.000572103882910077
12.8299999999998 0.000584193791725642
12.8399999999998 0.000595489375018711
12.8499999999998 0.000605979984981143
12.8599999999998 0.000615656040096698
12.8699999999998 0.000624509032396473
12.8799999999998 0.000632531533326602
12.8899999999998 0.00063971719841113
12.8999999999998 0.000646060770268486
12.9099999999998 0.000651558079876441
12.9199999999998 0.000656206047233107
12.9299999999998 0.000660002680317999
12.9399999999998 0.000662947072692092
12.9499999999998 0.000665039399749036
12.9599999999998 0.000666280913632477
12.9699999999998 0.000666673936837708
12.9799999999998 0.000666221854520026
12.9899999999998 0.000664929105537636
12.9999999999998 0.000662801172264773
13.0099999999998 0.000659844569222594
13.0199999999998 0.0006560668305948
13.0299999999998 0.000651476496728856
13.0399999999998 0.000646083099786722
13.0499999999998 0.000639897148836309
13.0599999999998 0.000632930114955381
13.0699999999998 0.000625194417606967
13.0799999999998 0.000616703415624639
13.0899999999998 0.0006074711952053
13.0999999999998 0.000597512654842556
13.1099999999998 0.000586843691385702
13.1199999999998 0.000575481028421658
13.1299999999998 0.000563442191152152
13.1399999999998 0.000550745480405771
13.1499999999998 0.000537409945816143
13.1599999999998 0.000523455358199445
13.1699999999998 0.000508902078121328
13.1799999999998 0.000493771180264855
13.1899999999998 0.000478084455917094
13.1999999999998 0.00046186429169699
13.2099999999998 0.000445133641525074
13.2199999999998 0.000427915996667109
13.2299999999998 0.000410235354564144
13.2399999999998 0.000392116186800873
13.2499999999998 0.000373583406410736
13.2599999999998 0.000354662334644755
13.2699999999998 0.000335378667295329
13.2799999999998 0.000315758440647355
13.2899999999998 0.000295827997118484
13.2999999999998 0.000275613950644262
13.3099999999998 0.000255143151860262
13.3199999999998 0.000234442653130861
13.3299999999998 0.000213539673472963
13.3399999999998 0.000192461563421772
13.3499999999998 0.000171235769884984
13.3599999999998 0.000149889801031104
13.3699999999998 0.00012845119125702
13.3799999999998 0.00010694746627941
13.3899999999998 8.54061083941295e-05
13.3999999999998 6.38545219471043e-05
13.4099999999998 4.23199990596893e-05
13.4199999999998 2.08296856510152e-05
13.4299999999998 -5.89452200967218e-07
13.4399999999998 -2.19106615185914e-05
13.4499999999998 -4.31074352629305e-05
13.4599999999998 -6.41535447129527e-05
13.4699999999998 -8.50230713451762e-05
13.4799999999998 -0.000105690438176402
13.4899999999998 -0.000126130440532139
13.4999999999998 -0.000146318276204421
13.5099999999998 -0.000166229574963604
13.5199999999998 -0.000185840427389775
13.5299999999998 -0.000205127412990355
13.5399999999998 -0.000224067627571641
13.5499999999998 -0.000242638709832977
13.5599999999998 -0.000260818867153567
13.5699999999998 -0.000278586900542935
13.5799999999998 -0.000295922228727318
13.5899999999998 -0.000312804911338413
13.5999999999998 -0.000329215671208092
13.6099999999998 -0.000345135915701828
13.6199999999998 -0.000360547757095391
13.6299999999998 -0.000375434031968691
13.6399999999998 -0.000389778319595773
13.6499999999998 -0.000403564959312329
13.6599999999998 -0.000416779066843463
13.6699999999998 -0.000429406549575891
13.6799999999998 -0.000441434120760122
13.6899999999998 -0.000452849312629623
13.6999999999998 -0.00046364048842544
13.7099999999998 -0.000473796853316164
13.7199999999998 -0.000483308464204601
13.7299999999998 -0.000492166238414038
13.7399999999998 -0.000500361961248407
13.7499999999998 -0.00050788829242221
13.7599999999998 -0.000514738771357551
13.7699999999998 -0.000520907821347156
13.7799999999998 -0.000526390752899637
13.7899999999998 -0.000531183765010454
13.7999999999998 -0.000535283946203285
13.8099999999998 -0.00053868927421298
13.8199999999997 -0.000541398614518163
13.8299999999997 -0.000543411717767361
13.8399999999997 -0.000544729216109793
13.8499999999997 -0.000545352618444436
13.8599999999997 -0.000545284304603994
13.8699999999997 -0.000544527518494305
13.8799999999997 -0.000543086360215051
13.8899999999997 -0.000540965777195614
13.8999999999997 -0.000538171554392373
13.9099999999997 -0.000534710303614867
13.9199999999997 -0.000530589452086588
13.9299999999997 -0.000525817230419879
13.9399999999997 -0.000520402660340968
13.9499999999997 -0.000514355542863477
13.9599999999997 -0.00050768644856149
13.9699999999997 -0.000500406693323839
13.9799999999997 -0.000492528031245791
13.9899999999997 -0.000484063140691113
13.9999999999997 -0.000475025384191528
14.0099999999997 -0.000465428788254214
14.0199999999997 -0.00045528802245063
14.0299999999997 -0.000444618377811752
14.0399999999997 -0.000433435744556406
14.0499999999997 -0.000421756581277887
14.0599999999997 -0.000409597772127079
14.0699999999997 -0.000396976894446732
14.0799999999997 -0.00038391202101897
14.0899999999997 -0.000370421698332026
14.0999999999997 -0.000356524922762475
14.1099999999997 -0.000342241115536871
14.1199999999997 -0.000327590096872079
14.1299999999997 -0.000312592059505547
14.1399999999997 -0.000297267541742847
14.1499999999997 -0.000281637400109121
14.1599999999997 -0.00026572278166998
14.1699999999997 -0.0002495450960761
14.1799999999997 -0.000233125987379256
14.1899999999997 -0.000216487305663272
14.1999999999997 -0.000199651078532081
14.2099999999997 -0.000182639482494276
14.2199999999997 -0.000165474814283279
14.2299999999997 -0.000148179462151241
14.2399999999997 -0.000130775877174258
14.2499999999997 -0.000113286544606004
14.2599999999997 -9.57339553164048e-05
14.2699999999997 -7.81405773515591e-05
14.2799999999997 -6.05288276506781e-05
14.2899999999997 -4.29210439553692e-05
14.2999999999997 -2.533945694616e-05
14.3099999999997 -7.80616264058814e-06
14.3199999999997 9.65690491326938e-06
14.3299999999997 2.70280006144464e-05
14.3399999999997 4.42855949254695e-05
14.3499999999997 6.14084000952201e-05
14.3599999999997 7.83753959568158e-05
14.3699999999997 9.51658552698273e-05
14.3799999999997 0.000111759368576867
14.3899999999997 0.000128135868545233
14.3999999999997 0.000144275653765281
14.4099999999997 0.000160159411977712
14.4199999999997 0.000175768242703185
14.4299999999997 0.000191083679248219
14.4399999999997 0.000206087710062477
14.4499999999997 0.000220762799423354
14.4599999999997 0.000235091907424808
14.4699999999997 0.000249058509245985
14.4799999999997 0.000262646613685459
14.4899999999997 0.000275840780937825
14.4999999999997 0.000288626139585234
14.5099999999997 0.000300988402796541
14.5199999999997 0.000312913883713742
14.5299999999997 0.000324389510009941
14.5399999999997 0.000335402837604328
14.5499999999997 0.000345942063520651
14.5599999999997 0.000355996037876934
14.5699999999997 0.000365554274995271
14.5799999999997 0.000374606963621784
14.5899999999997 0.000383144976247927
14.5999999999997 0.000391159877525616
14.6099999999997 0.000398643931769751
14.6199999999997 0.000405590109543024
14.6299999999997 0.000411992093319039
14.6399999999997 0.000417844282221053
14.6499999999997 0.000423141795834873
14.6599999999997 0.00042788047720598
14.6699999999997 0.000432056894704426
14.6799999999997 0.000435668342904144
14.6899999999997 0.000438712842846736
14.6999999999997 0.000441189141248609
14.7099999999997 0.00044309670880101
14.7199999999997 0.000444435737571276
14.7299999999997 0.000445207137515471
14.7399999999997 0.000445412532114805
14.7499999999997 0.000445054253151072
14.7599999999997 0.000444135334640054
14.7699999999997 0.000442659505947263
14.7799999999997 0.000440631184118576
14.7899999999997 0.000438055465471802
14.7999999999997 0.000434938116518866
14.8099999999997 0.000431285564332488
14.8199999999997 0.000427104886560968
14.8299999999997 0.000422403801493448
14.8399999999997 0.000417190659068974
14.8499999999997 0.000411474435207186
14.8599999999997 0.000405264563042907
14.8699999999997 0.000398571058123147
14.8799999999997 0.000391404583239316
14.8899999999997 0.000383776350883937
14.8999999999997 0.00037569810643849
14.9099999999997 0.000367182110784675
14.9199999999997 0.000358241122360537
14.9299999999997 0.000348888378683989
14.9399999999997 0.000339137507499895
14.9499999999997 0.000329002614071706
14.9599999999997 0.000318498277182189
14.9699999999997 0.000307639470659716
14.9799999999997 0.000296441544594481
14.9899999999997 0.000284920205298484
14.9999999999997 0.000273091494469516
15.0099999999997 0.000260971767788228
15.0199999999997 0.000248577673078225
15.0299999999997 0.000235926128112835
15.0399999999997 0.000223034298128913
15.0499999999997 0.000209919573095712
15.0599999999997 0.000196599544779947
15.0699999999997 0.000183091983644134
15.0799999999997 0.00016941481561294
15.0899999999997 0.000155586098740675
15.0999999999997 0.000141623999812085
15.1099999999997 0.000127546770907879
15.1199999999997 0.000113372725965858
15.1299999999997 9.91202173681626e-05
15.1399999999997 8.48076125846582e-05
15.1499999999997 7.04532709022185e-05
15.1599999999997 5.60755202692905e-05
15.1699999999997 4.16926342846892e-05
15.1799999999997 2.73228093593485e-05
15.1899999999997 1.29841420792082e-05
15.1999999999997 -1.30539320290625e-06
15.2099999999997 -1.55279665190122e-05
15.2199999999997 -2.96659141543571e-05
15.2299999999997 -4.37017602062659e-05
15.2399999999997 -5.76182378072207e-05
15.2499999999997 -7.13983099870508e-05
15.2599999999997 -8.50251901493664e-05
15.2699999999997 -9.84823621381693e-05
15.2799999999997 -0.000111753599871037
15.2899999999997 -0.000124822986516083
15.2999999999997 -0.000137674933190478
15.3099999999997 -0.000150294197159113
15.3199999999997 -0.000162665899512608
15.3299999999997 -0.000174775542304769
15.3399999999997 -0.000186609025130238
15.3499999999997 -0.000198152661123973
15.3599999999997 -0.00020939319236154
15.3699999999997 -0.000220317804657804
15.3799999999997 -0.000230914141725714
15.3899999999997 -0.000241170318695671
15.3999999999997 -0.000251074934977331
15.4099999999997 -0.000260617086450748
15.4199999999997 -0.000269786376974368
15.4299999999997 -0.000278572929198529
15.4399999999997 -0.000286967394674049
15.4499999999997 -0.000294960963246333
15.4599999999997 -0.000302545371726477
15.4699999999997 -0.000309712911831794
15.4799999999997 -0.000316456437389093
15.4899999999997 -0.000322769370795138
15.4999999999997 -0.000328645708729586
15.5099999999997 -0.000334080027116765
15.5199999999997 -0.00033906748533363
15.5299999999997 -0.000343603829662253
15.5399999999997 -0.000347685395986213
15.5499999999997 -0.000351309111895355
15.5599999999997 -0.000354472497532847
15.5699999999997 -0.000357173666199675
15.5799999999997 -0.000359411324045121
15.5899999999997 -0.000361184769010774
15.5999999999997 -0.000362493889037553
15.6099999999997 -0.000363339159543347
15.6199999999997 -0.000363721640180515
15.6299999999997 -0.000363642970884572
15.6399999999997 -0.000363105367228031
15.6499999999997 -0.000362111615097073
15.6599999999997 -0.000360665064714189
15.6699999999997 -0.000358769624038643
15.6799999999997 -0.000356429751591396
15.6899999999997 -0.000353650448778008
15.6999999999997 -0.000350437251835442
15.7099999999997 -0.000346796223640152
15.7199999999997 -0.000342733945875639
15.7299999999997 -0.000338257512751145
15.7399999999997 -0.000333374493598173
15.7499999999997 -0.000328092782276014
15.7599999999997 -0.000322420873920144
15.7699999999997 -0.000316367718459572
15.7799999999997 -0.000309942707114203
15.7899999999997 -0.000303155658413078
15.7999999999997 -0.000296016803750748
15.8099999999997 -0.000288536772500005
15.8199999999997 -0.000280726567718945
15.8299999999997 -0.000272597483929582
15.8399999999997 -0.0002641612700703
15.8499999999997 -0.000255430003952908
15.8599999999997 -0.000246416077644175
15.8699999999997 -0.00023713218151369
15.8799999999997 -0.000227591287492417
15.8899999999997 -0.00021780663179615
15.8999999999997 -0.000207791697249802
15.9099999999997 -0.000197560195295049
15.9199999999997 -0.000187126047737974
15.9299999999997 -0.000176503368279847
15.9399999999997 -0.000165706443866856
15.9499999999997 -0.000154749715890412
15.9599999999997 -0.000143647761267174
15.9699999999997 -0.000132415273426343
15.9799999999997 -0.000121067043230868
15.9899999999997 -0.000109617939858404
15.9999999999997 -9.80828916674827e-05
16.0099999999997 -8.64768670739061e-05
16.0199999999997 -7.48148554620373e-05
16.0299999999997 -6.31118481553827e-05
16.0399999999997 -5.13828194705991e-05
16.0499999999997 -3.96427078787278e-05
16.0599999999997 -2.79063972971126e-05
16.0699999999997 -1.61886985352924e-05
16.0799999999997 -4.50433091771214e-06
16.0899999999997 7.13209589430602e-06
16.0999999999997 1.87060998590524e-05
16.1099999999997 3.02033442677395e-05
16.1199999999997 4.16096551933497e-05
16.1299999999997 5.29110386539296e-05
16.1399999999997 6.40936974701323e-05
16.1499999999997 7.51440477968834e-05
16.1599999999997 8.60487353099104e-05
16.1699999999997 9.67946510280981e-05
16.1799999999997 0.000107368946753401
16.1899999999997 0.000117759050110515
16.1999999999997 0.000127952679169126
16.2099999999997 0.000137937856632119
16.2199999999997 0.000147702923573879
16.2299999999997 0.000157236552713283
16.2399999999997 0.000166527761205562
16.2499999999997 0.000175565922943215
16.2599999999997 0.000184340780348866
16.2699999999997 0.000192842455646326
16.2799999999997 0.000201061461601346
16.2899999999997 0.000208988711719658
16.2999999999997 0.000216615529891897
16.3099999999998 0.000223933659475811
16.3199999999998 0.000230935271806838
16.3299999999998 0.000237612974128999
16.3399999999998 0.000243959816938712
16.3499999999998 0.000249969300735035
16.3599999999998 0.000255635382170574
16.3699999999998 0.000260952479598085
16.3799999999998 0.000265915478008614
16.3899999999998 0.000270519733357844
16.3999999999998 0.000274761076278103
16.4099999999998 0.000278635815174317
16.4199999999998 0.000282140738703044
16.4299999999998 0.000285273117693726
16.4399999999998 0.000288030706325334
16.4499999999998 0.000290411742707737
16.4599999999998 0.000292414948959713
16.4699999999998 0.000294039530620252
16.4799999999998 0.000295285175456004
16.4899999999998 0.000296152051670527
16.4999999999998 0.000296640805522219
16.5099999999998 0.000296752558359368
16.5199999999998 0.000296488903082638
16.5299999999998 0.000295851900047881
16.5399999999998 0.000294844072425853
16.5499999999998 0.000293468401041079
16.5599999999998 0.000291728318721462
16.5699999999998 0.000289627704206707
16.5799999999998 0.000287170875694597
16.5899999999998 0.000284362584167559
16.5999999999998 0.000281208006783189
16.6099999999998 0.00027771274096503
16.6199999999998 0.000273882800895311
16.6299999999998 0.000269724484597117
16.6399999999998 0.000265244510054452
16.6499999999998 0.000260450011543294
16.6599999999998 0.000255348488485641
16.6699999999998 0.000249947794218776
16.6799999999998 0.000244256124380495
16.6899999999998 0.000238282004925012
16.6999999999998 0.000232034279784848
16.7099999999998 0.000225522049737429
16.7199999999998 0.000218754737872679
16.7299999999998 0.000211742077580092
16.7399999999998 0.000204494064267539
16.7499999999998 0.000197020942769095
16.7599999999998 0.000189333193946225
16.7699999999998 0.000181441520773213
16.7799999999998 0.000173356834052937
16.7899999999998 0.000165090237846757
16.7999999999998 0.000156653014672782
16.8099999999998 0.000148056610512119
16.8199999999998 0.000139312619654597
16.8299999999998 0.000130432769411226
16.8399999999998 0.000121428904717988
16.8499999999998 0.000112312972653921
16.8599999999998 0.000103097006895617
16.8699999999998 9.37931121294413e-05
16.8799999999998 8.4413448442461e-05
16.8899999999998 7.49702157125085e-05
16.8999999999998 6.54756380177869e-05
16.9099999999998 5.59419480859542e-05
16.9199999999998 4.6381371802455e-05
16.9299999999998 3.68061127976659e-05
16.9399999999998 2.72283371321506e-05
16.9499999999999 1.76601580990443e-05
16.9599999999999 8.11362116237096e-06
16.9699999999999 -1.39931095019847e-06
16.9799999999999 -1.0866772981961e-05
16.9899999999999 -2.02770116912607e-05
16.9999999999999 -2.96184001875138e-05
17.0099999999999 -3.88794520419439e-05
17.0199999999999 -4.80488351561628e-05
17.0299999999999 -5.71153853722229e-05
17.0399999999999 -6.60681198080466e-05
17.0499999999999 -7.48962499026365e-05
17.0599999999999 -8.35891941559153e-05
17.0699999999999 -9.2136590548448e-05
17.0799999999999 -0.000100528308626836
17.0899999999999 -0.000108754461241016
17.0999999999999 -0.000116805415920221
17.1099999999999 -0.000124671805874917
17.1199999999999 -0.000132344540612333
17.1299999999999 -0.00013981481615317
17.1399999999999 -0.00014707412484326
17.1499999999999 -0.000154114264741263
17.1599999999999 -0.000160927348578547
17.1699999999999 -0.000167505812280361
17.1799999999999 -0.000173842423039455
17.1899999999999 -0.000179930286934075
17.1999999999999 -0.000185762856082805
17.2099999999999 -0.000191333935329344
17.2199999999999 -0.000196637688450985
17.2299999999999 -0.00020166864388513
17.2399999999999 -0.000206421699968916
17.2499999999999 -0.000210892129687566
17.2599999999999 -0.000215075584927804
17.2699999999999 -0.000218968100233304
17.2799999999999 -0.00022256609605983
17.2899999999999 -0.000225866381528326
17.2999999999999 -0.000228866156674972
17.3099999999999 -0.000231563014201154
17.3199999999999 -0.000233954940782722
17.3299999999999 -0.000236040317663593
17.3399999999999 -0.000237817920975066
17.3499999999999 -0.000239286921479628
17.3599999999999 -0.000240446883820352
17.3699999999999 -0.000241297765280081
17.3799999999999 -0.000241839914055525
17.3899999999999 -0.000242074067052544
17.3999999999999 -0.000242001347210246
17.4099999999999 -0.000241623260363368
17.4199999999999 -0.000240941691654862
17.4299999999999 -0.000239958901514395
17.4399999999999 -0.000238677521224399
17.4499999999999 -0.000237100548105484
17.4599999999999 -0.000235231340371574
17.4699999999999 -0.00023307361174158
17.4799999999999 -0.000230631425972179
17.4899999999999 -0.000227909191659914
17.4999999999999 -0.000224911658153594
17.5099999999999 -0.000221643878150976
17.5199999999999 -0.000218111140552256
17.5299999999999 -0.000214319124834861
17.5399999999999 -0.000210273812371904
17.5499999999999 -0.000205981477420634
17.5599999999999 -0.000201448677791736
17.5699999999999 -0.000196682245211353
17.5799999999999 -0.000191689275388244
17.59 -0.000186477110300877
17.6 -0.000181053289650679
17.61 -0.000175425651070457
17.62 -0.000169602250227349
17.63 -0.000163591351011916
17.64 -0.000157401414873392
17.65 -0.000151041089644756
17.66 -0.000144519198019692
17.67 -0.000137844725768659
17.68 -0.000131026809747648
17.69 -0.000124074725736505
17.7 -0.000116997876135177
17.71 -0.000109805777541521
17.72 -0.000102508048231545
17.73 -9.51143955614539e-05
17.74 -8.7634603309762e-05
17.75 -8.00785189771546e-05
17.76 -7.24560410613114e-05
17.77 -6.47771063236098e-05
17.78 -5.70516770642776e-05
17.79 -4.92897284224754e-05
17.8 -4.15012357174812e-05
17.81 -3.36961618470138e-05
17.82 -2.58844447585475e-05
17.83 -1.80759850091712e-05
17.84 -1.02806334295148e-05
17.85 -2.50817890683541e-06
17.86 5.2316636977175e-06
17.87 1.29292654829582e-05
17.88 2.05750952782383e-05
17.89 2.81597312365408e-05
17.9 3.56738722362292e-05
17.91 4.31083490779168e-05
17.92 5.04541354631978e-05
17.93 5.77023587423983e-05
17.94 6.48443104187461e-05
17.95 7.18714563968683e-05
17.96 7.87754469637392e-05
17.97 8.55481264907982e-05
17.98 9.21815428461363e-05
17.99 9.86679565062817e-05
18 0.000104999849357396
18.01 0.000111169933175586
18.02 0.000117171157779279
18.03 0.000122996718842391
18.04 0.000128640065360398
18.05 0.000134094906762366
18.06 0.000139355219661191
18.07 0.000144415254235208
18.08 0.000149269540234762
18.09 0.000153912892607935
18.1 0.000158340416740075
18.11 0.000162547513302315
18.12 0.000166529882704775
18.13 0.000170283529150675
18.14 0.000173804764288138
18.15 0.000177090210456956
18.16 0.000180136803528144
18.17 0.000182941795334686
18.18 0.000185502755692345
18.19 0.000187817574010038
18.2 0.000189884460518959
18.21 0.000191701947022991
18.22 0.000193268887267292
18.2300000000001 0.000194584456935689
18.2400000000001 0.000195648153220903
18.2500000000001 0.000196459793993374
18.2600000000001 0.000197019516572497
18.2700000000001 0.000197327776104922
18.2800000000001 0.000197385343555606
18.2900000000001 0.000197193303318535
18.3000000000001 0.000196753050455815
18.3100000000001 0.000196066287576281
18.3200000000001 0.000195135021368621
18.3300000000001 0.000193961558810377
18.3400000000001 0.000192548503085377
18.3500000000001 0.000190898749263421
18.3600000000001 0.000189015479839598
18.3700000000001 0.000186902160328373
18.3800000000001 0.000184562535353403
18.3900000000001 0.000182000626415307
18.4000000000001 0.000179220632870552
18.4100000000001 0.000176227051055461
18.4200000000001 0.000173024645503874
18.4300000000001 0.000169618422931102
18.4400000000001 0.000166013624745829
18.4500000000001 0.000162215719306537
18.4600000000001 0.000158230393932447
18.4700000000001 0.000154063546679445
18.4800000000001 0.00014972124582667
18.4900000000001 0.00014520977420943
18.5000000000001 0.000140535618842571
18.5100000000001 0.000135705440272576
18.5200000000001 0.000130726064155908
18.5300000000001 0.000125604472317889
18.5400000000001 0.000120347793476951
18.5500000000001 0.00011496329372791
18.5600000000001 0.000109458366838247
18.5700000000001 0.000103840524392782
18.5800000000001 9.81173858125417e-05
18.5900000000001 9.22966682686517e-05
18.6000000000001 8.63861765091416e-05
18.6100000000001 8.03937926150145e-05
18.6200000000001 7.43274657007786e-05
18.6300000000001 6.81952015741017e-05
18.6400000000001 6.20050523687606e-05
18.6500000000001 5.57651061647793e-05
18.6600000000001 4.94834766094033e-05
18.6700000000001 4.31682925523729e-05
18.6800000000001 3.68276877087872e-05
18.6900000000001 3.04697903626926e-05
18.7000000000001 2.41027131243682e-05
18.7100000000001 1.77345427541268e-05
18.7200000000001 1.13733300652775e-05
18.7300000000001 5.02707991871055e-06
18.7400000000001 -1.29625867859369e-06
18.7500000000001 -7.58880235908147e-06
18.7600000000001 -1.38427430522601e-05
18.7700000000001 -2.00503575028455e-05
18.7800000000001 -2.6204016637964e-05
18.7900000000001 -3.2296194772269e-05
18.8000000000001 -3.83194786400729e-05
18.8100000000001 -4.42665762438365e-05
18.8200000000001 -5.01303255086962e-05
18.8300000000001 -5.59037027329634e-05
18.8400000000001 -6.15798308248272e-05
18.8500000000001 -6.71519873158649e-05
18.8600000000001 -7.26136121422397e-05
18.8700000000002 -7.79583151848039e-05
18.8800000000002 -8.31798835597204e-05
18.8900000000002 -8.82722886514114e-05
18.9000000000002 -9.32296928800033e-05
18.9100000000002 -9.80464561976024e-05
18.9200000000002 -0.000102717142303262
18.9300000000002 -0.000107236524572534
18.9400000000002 -0.000111599591694729
18.9500000000002 -0.000115801553012139
18.9600000000002 -0.000119837843555837
18.9700000000002 -0.000123704128773143
18.9800000000002 -0.000127396308942164
18.9900000000002 -0.000130910523269343
19.0000000000002 -0.000134243153666284
19.0100000000002 -0.000137390828202604
19.0200000000002 -0.000140350424231978
19.0300000000002 -0.000143119071188961
19.0400000000002 -0.000145694153054638
19.0500000000002 -0.000148073310489561
19.0600000000002 -0.000150254442632907
19.0700000000002 -0.000152235708567194
19.0800000000002 -0.000154015528451632
19.0900000000002 -0.000155592584340834
19.1000000000002 -0.000156965820588485
19.1100000000002 -0.000158134444009021
19.1200000000002 -0.000159097923676843
19.1300000000002 -0.000159855990397105
19.1400000000002 -0.000160408635850924
19.1500000000002 -0.000160756111418443
19.1600000000002 -0.000160898926683977
19.1700000000002 -0.00016083784762836
19.1800000000002 -0.000160573894514819
19.1900000000002 -0.000160108339476392
19.2000000000002 -0.000159442703815391
19.2100000000002 -0.000158578755029436
19.2200000000002 -0.00015751850358545
19.2300000000002 -0.000156264199475532
19.2400000000002 -0.00015481832861344
19.2500000000002 -0.000153183609183532
19.2600000000002 -0.000151362988180156
19.2700000000002 -0.000149359638716983
19.2800000000002 -0.000147176928154368
19.2900000000002 -0.000144818393089039
19.3000000000002 -0.00014228782349463
19.3100000000002 -0.000139589209440275
19.3200000000002 -0.000136726735087831
19.3300000000002 -0.000133704772476738
19.3400000000002 -0.000130527875104718
19.3500000000002 -0.000127200771312677
19.3600000000002 -0.000123728352267202
19.3700000000002 -0.000120115641757937
19.3800000000002 -0.000116367859039402
19.3900000000002 -0.000112490367714009
19.4000000000002 -0.000108488669161309
19.4100000000002 -0.000104368395420686
19.4200000000002 -0.000100135301744775
19.4300000000002 -9.57952589267316e-05
19.4400000000002 -9.13542454574073e-05
19.4500000000002 -8.68183395470082e-05
19.4600000000002 -8.21937110352689e-05
19.4700000000002 -7.74866132087374e-05
19.4800000000002 -7.27033745406893e-05
19.4900000000002 -6.78503903674753e-05
19.5000000000002 -6.29341145140939e-05
19.5100000000003 -5.79610508811128e-05
19.5200000000003 -5.29377450046468e-05
19.5300000000003 -4.78707756008114e-05
19.5400000000003 -4.27667461058578e-05
19.5500000000003 -3.76322762230453e-05
19.5600000000003 -3.24739934871047e-05
19.5700000000003 -2.72985248570876e-05
19.5800000000003 -2.2112488348235e-05
19.5900000000003 -1.69224847133218e-05
19.6000000000003 -1.17350891839099e-05
19.6100000000003 -6.55684328169103e-06
19.6200000000003 -1.39424670998765e-06
19.6300000000003 3.74625066460602e-06
19.6400000000003 8.85825673076206e-06
19.6500000000003 1.39354449415323e-05
19.6600000000003 1.89715620046384e-05
19.6700000000003 2.39604354447592e-05
19.6800000000003 2.88959810288575e-05
19.6900000000003 3.37722100457278e-05
19.7000000000003 3.8583236431284e-05
19.7100000000003 4.33232837312533e-05
19.7200000000003 4.79866918932162e-05
19.7300000000003 5.25679238802275e-05
19.7400000000003 5.70615720984346e-05
19.7500000000003 6.14623646314506e-05
19.7600000000003 6.57651712745016e-05
19.7700000000003 6.99650093616179e-05
19.7800000000003 7.40570493792086e-05
19.7900000000003 7.80366203609441e-05
19.8000000000003 8.18992150568266e-05
19.8100000000003 8.56404948714334e-05
19.8200000000003 8.92562945663838e-05
19.8300000000003 9.27426267220491e-05
19.8400000000003 9.60956859539733e-05
19.8500000000003 9.93118528798297e-05
19.8600000000003 0.000102387697833055
19.8700000000003 0.000105319984319658
19.8800000000003 0.000108105672215033
19.8900000000003 0.000110741920697964
19.9000000000003 0.000113226090919349
19.9100000000003 0.000115555748403524
19.9200000000003 0.000117728665180436
19.9300000000003 0.000119742821647247
19.9400000000003 0.000121596408158326
19.9500000000003 0.000123287826342956
19.9600000000003 0.000124815690150421
19.9700000000003 0.000126178826635927
19.9800000000003 0.000127376276441049
19.9900000000003 0.000128407294020502
20.0000000000003 0.000129271347609733
20.0100000000003 0.000129968118915459
20.0200000000003 0.000130497502539809
20.0300000000003 0.000130859605140613
20.0400000000003 0.000131054744330981
20.0500000000003 0.000131083447321949
20.0600000000003 0.000130946449312838
20.0700000000003 0.00013064469163511
20.0800000000003 0.000130179319657174
20.0900000000003 0.000129551680460116
20.1000000000003 0.000128763320298632
20.1100000000003 0.000127815981868879
20.1200000000003 0.00012671160141933
20.1300000000003 0.00012545230577004
20.1400000000003 0.000124040409372119
20.1500000000004 0.000122478411706982
20.1600000000004 0.00012076899583059
20.1700000000004 0.000118914957059857
20.1800000000004 0.000116919298089228
20.1900000000004 0.000114785194368903
20.2000000000004 0.000112515981730273
20.2100000000004 0.00011011515140308
20.2200000000004 0.000107586344862811
20.2300000000004 0.000104933348515155
20.2400000000004 0.000102160088224597
20.2500000000004 9.92706030505466e-05
20.2600000000004 9.62690738663514e-05
20.2700000000004 9.31598159163614e-05
20.2800000000004 8.99472592213503e-05
20.2900000000004 8.66359429546961e-05
20.3000000000004 8.32305094854302e-05
20.3100000000004 7.97356982050602e-05
20.3200000000004 7.61563391978357e-05
20.3300000000004 7.24973467891465e-05
20.3400000000004 6.8763712994912e-05
20.3500000000004 6.49605008888137e-05
20.3600000000004 6.10928379009685e-05
20.3700000000004 5.71659090598865e-05
20.3800000000004 5.31849501884116e-05
20.3900000000004 4.9155241063771e-05
20.4000000000004 4.50820985513942e-05
20.4100000000004 4.09708697219106e-05
20.4200000000004 3.68269249605107e-05
20.4300000000004 3.26556510777443e-05
20.4400000000004 2.84624444306614e-05
20.4500000000004 2.42527040631095e-05
20.4600000000004 2.00318248739074e-05
20.4700000000004 1.58051908214778e-05
20.4800000000004 1.15781681734561e-05
20.4900000000004 7.35609880962562e-06
20.5000000000004 3.14429358648295e-06
20.5100000000004 -1.05197422846776e-06
20.5200000000004 -5.22747544465327e-06
20.5300000000004 -9.37703135077591e-06
20.5400000000004 -1.34955200238029e-05
20.5500000000004 -1.75778825370837e-05
20.5600000000004 -2.16191290601854e-05
20.5700000000004 -2.56143448432141e-05
20.5800000000004 -2.95586960786142e-05
20.5900000000004 -3.34474356335764e-05
20.6000000000004 -3.72759086464126e-05
20.6100000000004 -4.10395579804268e-05
20.6200000000004 -4.47339295290881e-05
20.6300000000004 -4.83546773664461e-05
20.6400000000004 -5.18975687370174e-05
20.6500000000004 -5.53584888795656e-05
20.6600000000004 -5.87334456794098e-05
20.6700000000004 -6.20185741441715e-05
20.6800000000004 -6.5210140698665e-05
20.6900000000004 -6.83045472932003e-05
20.7000000000004 -7.12983353218802e-05
20.7100000000004 -7.41881893465561e-05
20.7200000000004 -7.69709406226639e-05
20.7300000000004 -7.96435704233895e-05
20.7400000000004 -8.22032131589164e-05
20.7500000000004 -8.46471592877675e-05
20.7600000000004 -8.69728580175386e-05
20.7700000000004 -8.91779197925979e-05
20.7800000000004 -9.12601185666183e-05
20.7900000000005 -9.32173938580867e-05
20.8000000000005 -9.50478525872183e-05
20.8100000000005 -9.67497706930128e-05
20.8200000000005 -9.83215945294566e-05
20.8300000000005 -9.97619420401835e-05
20.8400000000005 -0.000101069603711206
20.8500000000005 -0.000102243543303558
20.8600000000005 -0.000103282898369871
20.8700000000005 -0.000104186980519464
20.8800000000005 -0.000104955275498085
20.8900000000005 -0.000105587443035735
20.9000000000005 -0.000106083316476693
20.9100000000005 -0.000106442902193651
20.9200000000005 -0.000106666378788252
20.9300000000005 -0.000106754096080865
20.9400000000005 -0.000106706573892981
20.9500000000005 -0.000106524500626467
20.9600000000005 -0.000106208731644965
20.9700000000005 -0.000105760287464424
20.9800000000005 -0.00010518035176237
20.9900000000005 -0.000104470269220083
21.0000000000005 -0.000103631543220188
21.0100000000005 -0.000102665833438682
21.0200000000005 -0.000101574953405998
21.0300000000005 -0.000100360868196547
21.0400000000005 -9.90256926380645e-05
21.0500000000005 -9.75716663139729e-05
21.0600000000005 -9.60011486093959e-05
21.0700000000005 -9.43166629649124e-05
21.0800000000005 -9.2520865232392e-05
21.0900000000005 -9.06165396848177e-05
21.1000000000005 -8.86065948850087e-05
21.1100000000005 -8.64940594188199e-05
21.1200000000005 -8.4282077498561e-05
21.1300000000005 -8.19739011058929e-05
21.1400000000005 -7.95728707487818e-05
21.1500000000005 -7.70824549356678e-05
21.1600000000005 -7.45062176104688e-05
21.1700000000005 -7.18478137583454e-05
21.1800000000005 -6.91109846634631e-05
21.1900000000005 -6.62995529552357e-05
21.2000000000005 -6.34174175083484e-05
21.2100000000005 -6.04685482322799e-05
21.2200000000005 -5.74569807725449e-05
21.2300000000005 -5.43868111392434e-05
21.2400000000005 -5.12621902749997e-05
21.2500000000005 -4.80873185725023e-05
21.2600000000005 -4.48664403507014e-05
21.2700000000005 -4.160383829811e-05
21.2800000000005 -3.83038278912207e-05
21.2900000000005 -3.49707517957707e-05
21.3000000000005 -3.16089742584276e-05
21.3100000000005 -2.82228754963073e-05
21.3200000000005 -2.48168460916246e-05
21.3300000000005 -2.13952813987178e-05
21.3400000000005 -1.79625759705383e-05
21.3500000000005 -1.45231180116801e-05
21.3600000000005 -1.10812838648813e-05
21.3700000000005 -7.64143253786369e-06
21.3800000000005 -4.20790027730833e-06
21.3900000000005 -7.84995196592352e-07
21.4000000000005 2.62300803609103e-06
21.4100000000005 6.01187344282973e-06
21.4200000000005 9.37740890792652e-06
21.4300000000006 1.2715471270484e-05
21.4400000000006 1.60219713315524e-05
21.4500000000006 1.92928787699004e-05
21.4600000000006 2.25242269606146e-05
21.4700000000006 2.57121176908737e-05
21.4800000000006 2.88527257674183e-05
21.4900000000006 3.19423035103932e-05
21.5000000000006 3.49771851283972e-05
21.5100000000006 3.79537909697733e-05
21.5200000000006 4.08686316453183e-05
21.5300000000006 4.37183120178292e-05
21.5400000000006 4.64995350540249e-05
21.5500000000006 4.92091055345228e-05
21.5600000000006 5.18439336182959e-05
21.5700000000006 5.44010382571603e-05
21.5800000000006 5.68775504569524e-05
21.5900000000006 5.92707163820504e-05
21.6000000000006 6.15779002999784e-05
21.6100000000006 6.37965873631407e-05
21.6200000000006 6.59243862249097e-05
21.6300000000006 6.79590314875613e-05
21.6400000000006 6.98983859797373e-05
21.6500000000006 7.17404428613579e-05
21.6600000000006 7.34833275541578e-05
21.6700000000006 7.51252994962079e-05
21.6800000000006 7.66647537190542e-05
21.6900000000006 7.81002222463325e-05
21.7000000000006 7.94303753129359e-05
21.7100000000006 8.06540224040784e-05
21.7200000000006 8.17701131138148e-05
21.7300000000006 8.27777378228264e-05
21.7400000000006 8.36761282013339e-05
21.7500000000006 8.44646575166804e-05
21.7600000000006 8.51428407703321e-05
21.7700000000006 8.57103346589912e-05
21.7800000000006 8.61669373546549e-05
21.7900000000006 8.65125881081667e-05
21.8000000000006 8.67473666779714e-05
21.8100000000006 8.68714925861559e-05
21.8200000000006 8.68853242042899e-05
21.8300000000006 8.67893576721433e-05
21.8400000000006 8.65842256531007e-05
21.8500000000006 8.62706959311807e-05
21.8600000000006 8.58496698562385e-05
21.8700000000006 8.53221806467123e-05
21.8800000000006 8.46893915642014e-05
21.8900000000006 8.39525939835697e-05
21.9000000000006 8.31132054016673e-05
21.9100000000006 8.21727674717092e-05
21.9200000000006 8.11329442621533e-05
21.9300000000006 7.99955212757008e-05
21.9400000000006 7.87623557912966e-05
21.9500000000006 7.74354478294771e-05
21.9600000000006 7.60169088340193e-05
21.9700000000006 7.45089562808815e-05
21.9800000000006 7.29139103697051e-05
21.9900000000006 7.12341906026556e-05
22.0000000000006 6.94723122552675e-05
22.0100000000006 6.76308827440781e-05
22.0200000000006 6.57125850096143e-05
22.0300000000006 6.37201950355791e-05
22.0400000000006 6.16565772344265e-05
22.0500000000006 5.95246718243497e-05
22.0600000000006 5.73274910800903e-05
22.0700000000007 5.50681153711864e-05
22.0800000000007 5.27496890611523e-05
22.0900000000007 5.03754163053735e-05
22.1000000000007 4.79485567698565e-05
22.1100000000007 4.54724212855308e-05
22.1200000000007 4.29503674490356e-05
22.1300000000007 4.0385795178864e-05
22.1400000000007 3.77821422346161e-05
22.1500000000007 3.51428797064144e-05
22.1600000000007 3.2471507481143e-05
22.1700000000007 2.97715496919006e-05
22.1800000000007 2.70465501568674e-05
22.1900000000007 2.43000678136831e-05
22.2000000000007 2.15356721553101e-05
22.2100000000007 1.87569386733013e-05
22.2200000000007 1.59674443142796e-05
22.2300000000007 1.31707629554105e-05
22.2400000000007 1.0370460904547e-05
22.2500000000007 7.57009243067744e-06
22.2600000000007 4.7731953302047e-06
22.2700000000007 1.98328653453727e-06
22.2800000000007 -7.96142235635814e-07
22.2900000000007 -3.56162876407744e-06
22.3000000000007 -6.30974457556375e-06
22.3100000000007 -9.03709911015586e-06
22.3200000000007 -1.1740343830197e-05
22.3300000000007 -1.44161762551376e-05
22.3400000000007 -1.70613439194349e-05
22.3500000000007 -1.96726482488635e-05
22.3600000000007 -2.22469483507113e-05
22.3700000000007 -2.47811647134905e-05
22.3800000000007 -2.72722828118722e-05
22.3900000000007 -2.97173566127519e-05
22.4000000000007 -3.21135119784738e-05
22.4100000000007 -3.44579499633678e-05
22.4200000000007 -3.67479499999587e-05
22.4300000000007 -3.89808729712841e-05
22.4400000000007 -4.11541641660179e-05
22.4500000000007 -4.32653561133567e-05
22.4600000000007 -4.53120712942467e-05
22.4700000000007 -4.72920247264539e-05
22.4800000000007 -4.9203026420677e-05
22.4900000000007 -5.10429837052253e-05
22.5000000000007 -5.28099034169316e-05
22.5100000000007 -5.45018939561547e-05
22.5200000000007 -5.61171672039267e-05
22.5300000000007 -5.76540402994549e-05
22.5400000000007 -5.91109372764064e-05
22.5500000000007 -6.04863905565673e-05
22.5600000000007 -6.17790422996695e-05
22.5700000000007 -6.29876456083685e-05
22.5800000000007 -6.411106558754e-05
22.5900000000007 -6.51482802572667e-05
22.6000000000007 -6.60983813190667e-05
22.6100000000007 -6.6960574775127e-05
22.6200000000007 -6.77341814014578e-05
22.6300000000007 -6.84186370757264e-05
22.6400000000007 -6.90134929475434e-05
22.6500000000007 -6.95184154761529e-05
22.6600000000007 -6.99331863180533e-05
22.6700000000007 -7.02577020704497e-05
22.6800000000007 -7.04919738717959e-05
22.6900000000007 -7.06361268609659e-05
22.7000000000007 -7.06903994969143e-05
22.7100000000008 -7.06551427410792e-05
22.7200000000008 -7.05308191052976e-05
22.7300000000008 -7.03180015687175e-05
22.7400000000008 -7.00173723682665e-05
22.7500000000008 -6.96297216689489e-05
22.7600000000008 -6.91559461231921e-05
22.7700000000008 -6.85970473338801e-05
22.7800000000008 -6.79541302464549e-05
22.7900000000008 -6.72284015186788e-05
22.8000000000008 -6.64211679722474e-05
22.8100000000008 -6.5533835383148e-05
22.8200000000008 -6.45678901440114e-05
22.8300000000008 -6.35249022244104e-05
22.8400000000008 -6.24065477188689e-05
22.8500000000008 -6.1214590189106e-05
22.8600000000008 -5.99508780177353e-05
22.8700000000008 -5.86173416683436e-05
22.8800000000008 -5.72159908557677e-05
22.8900000000008 -5.57489116304568e-05
22.9000000000008 -5.42182614754661e-05
22.9100000000008 -5.26262565984069e-05
22.9200000000008 -5.0975196893725e-05
22.9300000000008 -4.926744525715e-05
22.9400000000008 -4.7505424652117e-05
22.9500000000008 -4.56916149557544e-05
22.9600000000008 -4.38285496697039e-05
22.9700000000008 -4.19188125368166e-05
22.9800000000008 -3.99650340863492e-05
22.9900000000008 -3.79698881218659e-05
23.0000000000008 -3.59360881618635e-05
23.0100000000008 -3.3866383840979e-05
23.0200000000008 -3.17635572784265e-05
23.0300000000008 -2.96304194196112e-05
23.0400000000008 -2.74698063564816e-05
23.0500000000008 -2.52845756318707e-05
23.0600000000008 -2.30776025329597e-05
23.0700000000008 -2.08517763788431e-05
23.0800000000008 -1.86099968070895e-05
23.0900000000008 -1.63551700641453e-05
23.1000000000008 -1.40902053043271e-05
23.1100000000008 -1.18180109021351e-05
23.1200000000008 -9.54149078251891e-06
23.1300000000008 -7.26354077371265e-06
23.1400000000008 -4.98704498716448e-06
23.1500000000008 -2.71487222903449e-06
23.1600000000008 -4.49872447678713e-07
23.1700000000008 1.80512677855003e-06
23.1800000000008 4.04732370892192e-06
23.1900000000008 6.2739458591806e-06
23.2000000000008 8.48225336751851e-06
23.2100000000008 1.06695423036729e-05
23.2200000000008 1.28331479172205e-05
23.2300000000008 1.4970447821225e-05
23.2400000000008 1.70788651075361e-05
23.2500000000008 1.91558713901065e-05
23.2600000000008 2.11989897728137e-05
23.2700000000008 2.32057977383935e-05
23.2800000000008 2.51739299551842e-05
23.2900000000008 2.71010809985438e-05
23.3000000000008 2.89850079838719e-05
23.3100000000008 3.08235331083329e-05
23.3200000000008 3.26145460984521e-05
23.3300000000008 3.43560065611108e-05
23.3400000000008 3.60459462351641e-05
23.3500000000009 3.76824711414296e-05
23.3600000000009 3.92637636288134e-05
23.3700000000009 4.07880843144634e-05
23.3800000000009 4.22537739159866e-05
23.3900000000009 4.36592549739275e-05
23.4000000000009 4.50030334628518e-05
23.4100000000009 4.62837002895126e-05
23.4200000000009 4.74999326767532e-05
23.4300000000009 4.86504954319341e-05
23.4400000000009 4.97342420988221e-05
23.4500000000009 5.07501159920582e-05
23.4600000000009 5.16971511134444e-05
23.4700000000009 5.25744729494744e-05
23.4800000000009 5.33812991496655e-05
23.4900000000009 5.41169400854305e-05
23.5000000000009 5.47807992893692e-05
23.5100000000009 5.53723737774166e-05
23.5200000000009 5.58912542453877e-05
23.5300000000009 5.63371251508008e-05
23.5400000000009 5.67097646771683e-05
23.5500000000009 5.70090445794811e-05
23.5600000000009 5.72349299129692e-05
23.5700000000009 5.73874786462688e-05
23.5800000000009 5.74668411603751e-05
23.5900000000009 5.74732596350353e-05
23.6000000000009 5.74070673245997e-05
23.6100000000009 5.72686877258294e-05
23.6200000000009 5.70586336408511e-05
23.6300000000009 5.67775061395265e-05
23.6400000000009 5.64259934272772e-05
23.6500000000009 5.60048696275686e-05
23.6600000000009 5.55149934942817e-05
23.6700000000009 5.49573070816739e-05
23.6800000000009 5.43328344279456e-05
23.6900000000009 5.36426803807545e-05
23.7000000000009 5.28880298915497e-05
23.7100000000009 5.20701147049026e-05
23.7200000000009 5.11902632074949e-05
23.7300000000009 5.02498758723431e-05
23.7400000000009 4.92504231449364e-05
23.7500000000009 4.8193443251611e-05
23.7600000000009 4.70805399332916e-05
23.7700000000009 4.59133801077676e-05
23.7800000000009 4.46936914637345e-05
23.7900000000009 4.34232522413e-05
23.8000000000009 4.21039012498725e-05
23.8100000000009 4.0737535485334e-05
23.8200000000009 3.93261019316516e-05
23.8300000000009 3.78715950659236e-05
23.8400000000009 3.63760542285962e-05
23.8500000000009 3.48415609047271e-05
23.8600000000009 3.3270235940022e-05
23.8700000000009 3.16642367056826e-05
23.8800000000009 3.00257542214526e-05
23.8900000000009 2.83570102439055e-05
23.9000000000009 2.66602543257492e-05
23.9100000000009 2.49377608511819e-05
23.9200000000009 2.31918260519505e-05
23.9300000000009 2.14247650084737e-05
23.9400000000009 1.96389086402312e-05
23.9500000000009 1.78366006895217e-05
23.9600000000009 1.60201947026033e-05
23.9700000000009 1.41920510121503e-05
23.9800000000009 1.23545337249426e-05
23.990000000001 1.05100077186229e-05
24.000000000001 8.66083565132887e-06
24.010000000001 6.80937498796397e-06
24.020000000001 4.95797504680475e-06
24.030000000001 3.10897407011076e-06
24.040000000001 1.26469632235122e-06
24.050000000001 -5.72550780409081e-07
24.060000000001 -2.40047950646862e-06
24.070000000001 -4.21682460635522e-06
24.080000000001 -6.01934606924075e-06
24.090000000001 -7.80583183460439e-06
24.100000000001 -9.57410045592641e-06
24.110000000001 -1.13220037132653e-05
24.120000000001 -1.3047429171643e-05
24.130000000001 -1.47483026822741e-05
24.140000000001 -1.64225908237211e-05
24.150000000001 -1.80683032801953e-05
24.160000000001 -1.96834951542652e-05
24.170000000001 -2.12662692113701e-05
24.180000000001 -2.28147780536249e-05
24.190000000001 -2.43272262204852e-05
24.200000000001 -2.58018722139576e-05
24.210000000001 -2.72370304461721e-05
24.220000000001 -2.86310731072555e-05
24.230000000001 -2.99824319513875e-05
24.240000000001 -3.12895999992838e-05
24.250000000001 -3.25511331553145e-05
24.260000000001 -3.37656517376163e-05
24.270000000001 -3.49318419196689e-05
24.280000000001 -3.60484570819389e-05
24.290000000001 -3.71143190723079e-05
24.300000000001 -3.81283193741141e-05
24.310000000001 -3.90894201807834e-05
24.320000000001 -3.99966553761241e-05
24.330000000001 -4.08491314195046e-05
24.340000000001 -4.16460281352481e-05
24.350000000001 -4.23865994057079e-05
24.360000000001 -4.30701737676165e-05
24.370000000001 -4.36961549114247e-05
24.380000000001 -4.4264022083482e-05
24.390000000001 -4.47733303914561e-05
24.400000000001 -4.522371101311e-05
24.410000000001 -4.56148713042384e-05
24.420000000001 -4.59465948151829e-05
24.430000000001 -4.62187412095696e-05
24.440000000001 -4.6431246087799e-05
24.450000000001 -4.65841207161222e-05
24.460000000001 -4.66774516623196e-05
24.470000000001 -4.67114003392022e-05
24.480000000001 -4.66862024574113e-05
24.490000000001 -4.66021673893195e-05
24.500000000001 -4.64596774462918e-05
24.510000000001 -4.62591870722449e-05
24.520000000001 -4.60012219575265e-05
24.530000000001 -4.56863780790006e-05
24.540000000001 -4.53153206756444e-05
24.550000000001 -4.4888783175766e-05
24.560000000001 -4.44075661066579e-05
24.570000000001 -4.38725360528446e-05
24.580000000001 -4.32846248264104e-05
24.590000000001 -4.26448167436372e-05
24.600000000001 -4.19541535838113e-05
24.610000000001 -4.1213745737062e-05
24.620000000001 -4.04247612620166e-05
24.6300000000011 -3.95884241350247e-05
24.6400000000011 -3.8706012437387e-05
24.6500000000011 -3.77788564831793e-05
24.6600000000011 -3.68083368903045e-05
24.6700000000011 -3.5795881663386e-05
24.6800000000011 -3.4742957569723e-05
24.6900000000011 -3.36510859637769e-05
24.7000000000011 -3.25218297225631e-05
24.7100000000011 -3.13567912905303e-05
24.7200000000011 -3.01576105858193e-05
24.7300000000011 -2.892596282071e-05
24.7400000000011 -2.76635562618266e-05
24.7500000000011 -2.63721299443056e-05
24.7600000000011 -2.50534513489194e-05
24.7700000000011 -2.37093140485733e-05
24.7800000000011 -2.23415353292354e-05
24.7900000000011 -2.09519537896207e-05
24.8000000000011 -1.9542426923521e-05
24.8100000000011 -1.81148286884022e-05
24.8200000000011 -1.66710470637414e-05
24.8300000000011 -1.52129816024689e-05
24.8400000000011 -1.37425409787864e-05
24.8500000000011 -1.22616405356056e-05
24.8600000000011 -1.07721998347781e-05
24.8700000000011 -9.27614021326977e-06
24.8800000000011 -7.77538234837729e-06
24.8900000000011 -6.27184383506336e-06
24.9000000000011 -4.76743677843157e-06
24.9100000000011 -3.26406540434496e-06
24.9200000000011 -1.76362369112895e-06
24.9300000000011 -2.67993025271711e-07
24.9400000000011 1.22096011602312e-06
24.9500000000011 2.70138645261679e-06
24.9600000000011 4.17145615678775e-06
24.9700000000011 5.62936107346507e-06
24.9800000000011 7.07331690267754e-06
24.9900000000011 8.50156534165248e-06
25.0000000000011 9.91237618403271e-06
25.0100000000011 1.13040493737646e-05
25.0200000000011 1.26749170112669e-05
25.0300000000011 1.40233453095669e-05
25.0400000000011 1.5347736498173e-05
25.0500000000011 1.66465306725147e-05
25.0600000000011 1.79182075868694e-05
25.0700000000011 1.91612883887774e-05
25.0800000000011 2.03743372930265e-05
25.0900000000011 2.15559631933534e-05
25.1000000000011 2.27048212101968e-05
25.1100000000011 2.38196141727402e-05
25.1200000000011 2.48990940337328e-05
25.1300000000011 2.59420632156219e-05
25.1400000000011 2.69473758866114e-05
25.1500000000011 2.79139391653664e-05
25.1600000000011 2.88407142531739e-05
25.1700000000011 2.97267174924788e-05
25.1800000000011 3.0571021350803e-05
25.1900000000011 3.13727553291594e-05
25.2000000000011 3.21311067941677e-05
25.2100000000011 3.28453217331958e-05
25.2200000000011 3.35147054319285e-05
25.2300000000011 3.413862307389e-05
25.2400000000011 3.47165002615357e-05
25.2500000000011 3.52478234586378e-05
25.2600000000011 3.57321403537937e-05
25.2700000000012 3.61690601449881e-05
25.2800000000012 3.65582537461892e-05
25.2900000000012 3.68994539126823e-05
25.3000000000012 3.71924552896483e-05
25.3100000000012 3.74371143828517e-05
25.3200000000012 3.7633349451258e-05
25.3300000000012 3.7781140322616e-05
25.3400000000012 3.78805281327509e-05
25.3500000000012 3.79316149894702e-05
25.3600000000012 3.79345635621639e-05
25.3700000000012 3.78895965984089e-05
25.3800000000012 3.77969963691907e-05
25.3900000000012 3.76571040447885e-05
25.4000000000012 3.74703190040464e-05
25.4100000000012 3.72370980808538e-05
25.4200000000012 3.69579547536278e-05
25.4300000000012 3.66334582873483e-05
25.4400000000012 3.62642328454619e-05
25.4500000000012 3.58509566066522e-05
25.4600000000012 3.53943609667212e-05
25.4700000000012 3.48952293261143e-05
25.4800000000012 3.43543782925067e-05
25.4900000000012 3.37726867987779e-05
25.5000000000012 3.31510812599964e-05
25.5100000000012 3.24905341767361e-05
25.5200000000012 3.17920626869176e-05
25.5300000000012 3.10567270683057e-05
25.5400000000012 3.02856291938056e-05
25.5500000000012 2.94799109417356e-05
25.5600000000012 2.86407481005868e-05
25.5700000000012 2.77693556180423e-05
25.5800000000012 2.68669866892219e-05
25.5900000000012 2.59349274173213e-05
25.6000000000012 2.49744951563192e-05
25.6100000000012 2.39870367688426e-05
25.6200000000012 2.2973926827526e-05
25.6300000000012 2.19365657746167e-05
25.6400000000012 2.08763780486238e-05
25.6500000000012 1.97948101839697e-05
25.6600000000012 1.86933288881441e-05
25.6700000000012 1.75734191000872e-05
25.6800000000012 1.64365820330895e-05
25.6900000000012 1.52843332052192e-05
25.7000000000012 1.41182004601541e-05
25.7100000000012 1.29397219811653e-05
25.7200000000012 1.1750444300952e-05
25.7300000000012 1.05519203099634e-05
25.7400000000012 9.3457072658043e-06
25.7500000000012 8.13336480629288e-06
25.7600000000012 6.91645296870816e-06
25.7700000000012 5.69653021772226e-06
25.7800000000012 4.47515148450442e-06
25.7900000000012 3.25386621943267e-06
25.8000000000012 2.03421646082378e-06
25.8100000000012 8.17734922062148e-07
25.8200000000012 -3.94056900537315e-07
25.8300000000012 -1.59965059682669e-06
25.8400000000012 -2.79755268662753e-06
25.8500000000012 -3.98628643602262e-06
25.8600000000012 -5.16439364396542e-06
25.8700000000012 -6.33043639708662e-06
25.8800000000012 -7.48299879062892e-06
25.8900000000012 -8.62068861349e-06
25.9000000000012 -9.7421389954147e-06
25.9100000000013 -1.0846010014429e-05
25.9200000000013 -1.19309902626715e-05
25.9300000000013 -1.29957983688374e-05
25.9400000000013 -1.40391844755206e-05
25.9500000000013 -1.50599316697885e-05
25.9600000000013 -1.60568573654127e-05
25.9700000000013 -1.70288146352216e-05
25.9800000000013 -1.79746934921444e-05
25.9900000000013 -1.88934221175779e-05
26.0000000000013 -1.97839680357235e-05
26.0100000000013 -2.06453392327123e-05
26.0200000000013 -2.14765852193489e-05
26.0300000000013 -2.22767980363991e-05
26.0400000000013 -2.30451132014242e-05
26.0500000000013 -2.3780710596241e-05
26.0600000000013 -2.44828152941697e-05
26.0700000000013 -2.51506983263054e-05
26.0800000000013 -2.57836773861452e-05
26.0900000000013 -2.63811174719663e-05
26.1000000000013 -2.69424314664446e-05
26.1100000000013 -2.74670806530842e-05
26.1200000000013 -2.79545751691103e-05
26.1300000000013 -2.84044743945592e-05
26.1400000000013 -2.88163872773918e-05
26.1500000000013 -2.91899725945305e-05
26.1600000000013 -2.9524939148981e-05
26.1700000000013 -2.98210459030818e-05
26.1800000000013 -3.00781020464352e-05
26.1900000000013 -3.02959670021303e-05
26.2000000000013 -3.04745503690303e-05
26.2100000000013 -3.06138118012682e-05
26.2200000000013 -3.07137608254962e-05
26.2300000000013 -3.07744565965562e-05
26.2400000000013 -3.07960075923659e-05
26.2500000000013 -3.07785712489741e-05
26.2600000000013 -3.07223535369494e-05
26.2700000000013 -3.0627608480543e-05
26.2800000000013 -3.04946376214909e-05
26.2900000000013 -3.03237894299854e-05
26.3000000000013 -3.01154586664875e-05
26.3100000000013 -2.98700857001482e-05
26.3200000000013 -2.95881557937745e-05
26.3300000000013 -2.92701983742831e-05
26.3400000000013 -2.89167863292752e-05
26.3500000000013 -2.85285354301354e-05
26.3600000000013 -2.81060959500336e-05
26.3700000000013 -2.76501571099855e-05
26.3800000000013 -2.71614524709291e-05
26.3900000000013 -2.66407535380735e-05
26.4000000000013 -2.60888686055846e-05
26.4100000000013 -2.55066415603027e-05
26.4200000000013 -2.48949506462513e-05
26.4300000000013 -2.42547071917082e-05
26.4400000000013 -2.35868539572344e-05
26.4500000000013 -2.2892359210194e-05
26.4600000000013 -2.21722267079432e-05
26.4700000000013 -2.1427487527148e-05
26.4800000000013 -2.06591987629373e-05
26.4900000000013 -1.98684421417323e-05
26.5000000000013 -1.90563225800553e-05
26.5100000000013 -1.82239667050547e-05
26.5200000000013 -1.73725213455527e-05
26.5300000000013 -1.6503151999264e-05
26.5400000000013 -1.56170412802443e-05
26.5500000000014 -1.47153873498186e-05
26.5600000000014 -1.3799402333777e-05
26.5700000000014 -1.28703107283612e-05
26.5800000000014 -1.19293477974161e-05
26.5900000000014 -1.09777579629712e-05
26.6000000000014 -1.00167931914557e-05
26.6100000000014 -9.04771137770619e-06
26.6200000000014 -8.07177472888561e-06
26.6300000000014 -7.09024815040705e-06
26.6400000000014 -6.1043976359314e-06
26.6500000000014 -5.11548866347856e-06
26.6600000000014 -4.12478459967352e-06
26.6700000000014 -3.13354511411954e-06
26.6800000000014 -2.14302460586949e-06
26.6900000000014 -1.15447064393352e-06
26.7000000000014 -1.69122423735185e-07
26.7100000000014 8.11790758599977e-07
26.7200000000014 1.78705101227601e-06
26.7300000000014 2.75545333636442e-06
26.7400000000014 3.71580708107295e-06
26.7500000000014 4.66693738399733e-06
26.7600000000014 5.60768657967062e-06
26.7700000000014 6.53691558073972e-06
26.7800000000014 7.4535052291656e-06
26.7900000000014 8.35635761587395e-06
26.8000000000014 9.24439736734166e-06
26.8100000000014 1.01165728976466e-05
26.8200000000014 1.09718576245579e-05
26.8300000000014 1.18092511483067e-05
26.8400000000014 1.26277803917168e-05
26.8500000000014 1.34265007004387e-05
26.8600000000014 1.42044969020814e-05
26.8700000000014 1.49608843231233e-05
26.8800000000014 1.56948097624771e-05
26.8900000000014 1.64054524207019e-05
26.9000000000014 1.70920247839079e-05
26.9100000000014 1.77537734614361e-05
26.9200000000014 1.83899799764848e-05
26.9300000000014 1.89999615089023e-05
26.9400000000014 1.95830715894347e-05
26.9500000000014 2.01387007447844e-05
26.9600000000014 2.06662770928988e-05
26.9700000000014 2.11652668879726e-05
26.9800000000014 2.1635175014719e-05
26.9900000000014 2.20755454315286e-05
27.0000000000014 2.2485961562201e-05
27.0100000000014 2.28660466360063e-05
27.0200000000014 2.32154639758965e-05
27.0300000000014 2.35339172347598e-05
27.0400000000014 2.38211505796779e-05
27.0500000000014 2.40769488245659e-05
27.0600000000014 2.43011375100033e-05
27.0700000000014 2.44935829320287e-05
27.0800000000014 2.4654192119569e-05
27.0900000000014 2.47829127605797e-05
27.1000000000014 2.48797330774538e-05
27.1100000000014 2.49446816521895e-05
27.1200000000014 2.49778272019006e-05
27.1300000000014 2.49792783053705e-05
27.1400000000014 2.49491830814898e-05
27.1500000000014 2.48877288206067e-05
27.1600000000014 2.47951415700831e-05
27.1700000000014 2.46716856757577e-05
27.1800000000014 2.4517663281687e-05
27.1900000000015 2.43334137917153e-05
27.2000000000015 2.41193132986889e-05
27.2100000000015 2.38757739917966e-05
27.2200000000015 2.36032435631458e-05
27.2300000000015 2.33022046619052e-05
27.2400000000015 2.29731737965376e-05
27.2500000000015 2.26166907569358e-05
27.2600000000015 2.22333353964624e-05
27.2700000000015 2.18237187781588e-05
27.2800000000015 2.13884822543669e-05
27.2900000000015 2.09282965123581e-05
27.3000000000015 2.04438605874166e-05
27.3100000000015 1.99359008448182e-05
27.3200000000015 1.9405169932162e-05
27.3300000000015 1.88524432656273e-05
27.3400000000015 1.82785214356317e-05
27.3500000000015 1.76842301231502e-05
27.3600000000015 1.70704166344416e-05
27.3700000000015 1.6437948802215e-05
27.3800000000015 1.57877138339688e-05
27.3900000000015 1.5120617124741e-05
27.4000000000015 1.44375810433126e-05
27.4100000000015 1.3739543697319e-05
27.4200000000015 1.30274576810007e-05
27.4300000000015 1.23022888084512e-05
27.4400000000015 1.15650148347536e-05
27.4500000000015 1.08166241671255e-05
27.4600000000015 1.00581145680333e-05
27.4700000000015 9.2904918521467e-06
27.4800000000015 8.51476857893873e-06
27.4900000000015 7.73196274268882e-06
27.5000000000015 6.94309646162742e-06
27.5100000000015 6.14919466791811e-06
27.5200000000015 5.35128380017324e-06
27.5300000000015 4.5503905001576e-06
27.5400000000015 3.74754031533185e-06
27.5500000000015 2.94375640885992e-06
27.5600000000015 2.14005827868549e-06
27.5700000000015 1.33746048726333e-06
27.5800000000015 5.36971403503988e-07
27.5900000000015 -2.60408041528344e-07
27.6000000000015 -1.05368558365441e-06
27.6100000000015 -1.8418788378697e-06
27.6200000000015 -2.62401649243245e-06
27.6300000000015 -3.39913948332487e-06
27.6400000000015 -4.16630214768788e-06
27.6500000000015 -4.92457335488299e-06
27.6600000000015 -5.67303761384174e-06
27.6700000000015 -6.41079615542828e-06
27.6800000000015 -7.13696798855473e-06
27.6900000000015 -7.85069092884427e-06
27.7000000000015 -8.55112259866945e-06
27.7100000000015 -9.23744139743798e-06
27.7200000000015 -9.90884744104182e-06
27.7300000000015 -1.05645634694231e-05
27.7400000000015 -1.1203835721268e-05
27.7500000000015 -1.18259347748734e-05
27.7600000000015 -1.24301563542974e-05
27.7700000000015 -1.30158220999147e-05
27.7800000000015 -1.35822803025925e-05
27.7900000000015 -1.41289066007273e-05
27.8000000000015 -1.46551046394376e-05
27.8100000000015 -1.51603066912579e-05
27.8200000000015 -1.56439742377366e-05
27.8300000000016 -1.61055985113868e-05
27.8400000000016 -1.65447009974975e-05
27.8500000000016 -1.69608338953596e-05
27.8600000000016 -1.73535805385212e-05
27.8700000000016 -1.77225557737381e-05
27.8800000000016 -1.80674062983391e-05
27.8900000000016 -1.83878109557822e-05
27.9000000000016 -1.86834809892318e-05
27.9100000000016 -1.89541602530412e-05
27.9200000000016 -1.91996253820845e-05
27.9300000000016 -1.94196859189921e-05
27.9400000000016 -1.96141843993382e-05
27.9500000000016 -1.97829963942838e-05
27.9600000000016 -1.99260305120959e-05
27.9700000000016 -2.00432283578135e-05
27.9800000000016 -2.01345644516122e-05
27.9900000000016 -2.02000461062238e-05
28.0000000000016 -2.02397132638429e-05
28.0100000000016 -2.02536382930342e-05
28.0200000000016 -2.02419257462512e-05
28.0300000000016 -2.02047120787076e-05
28.0400000000016 -2.01421653295085e-05
28.0500000000016 -2.00544847662049e-05
28.0600000000016 -1.99419004943255e-05
28.0700000000016 -1.98046730341231e-05
28.0800000000016 -1.96430928680035e-05
28.0900000000016 -1.94574799645609e-05
28.1000000000016 -1.92481832904479e-05
28.1100000000016 -1.90155803340621e-05
28.1200000000016 -1.87600767001716e-05
28.1300000000016 -1.84821008716957e-05
28.1400000000016 -1.81821073710685e-05
28.1500000000016 -1.78605793624347e-05
28.1600000000016 -1.75180249219366e-05
28.1700000000016 -1.71549762774039e-05
28.1800000000016 -1.6771989021044e-05
28.1900000000016 -1.63696412963155e-05
28.2000000000016 -1.59485329601714e-05
28.2100000000016 -1.55092846936238e-05
28.2200000000016 -1.50525339464821e-05
28.2300000000016 -1.4578941143323e-05
28.2400000000016 -1.40891846502597e-05
28.2500000000016 -1.35839599108393e-05
28.2600000000016 -1.30639785288717e-05
28.2700000000016 -1.2529967317617e-05
28.2800000000016 -1.19826673248445e-05
28.2900000000016 -1.14228328391535e-05
28.3000000000016 -1.08512303810426e-05
28.3100000000016 -1.02686376812806e-05
28.3200000000016 -9.6758426486368e-06
28.3300000000016 -9.07364232876854e-06
28.3400000000016 -8.46284185588958e-06
28.3500000000016 -7.84425339876293e-06
28.3600000000016 -7.21869510249875e-06
28.3700000000016 -6.58699002759247e-06
28.3800000000016 -5.94996508761853e-06
28.3900000000016 -5.30844998696712e-06
28.4000000000016 -4.6632761599966e-06
28.4100000000016 -4.01527571295724e-06
28.4200000000016 -3.36528037002671e-06
28.4300000000016 -2.71412042478201e-06
28.4400000000016 -2.06262369841446e-06
28.4500000000016 -1.41161450598295e-06
28.4600000000016 -7.61912631975985e-07
28.4700000000017 -1.14332316439544e-07
28.4800000000017 5.30318747096969e-07
28.4900000000017 1.17124040068101e-06
28.5000000000017 1.80764099679549e-06
28.5100000000017 2.43873835784781e-06
28.5200000000017 3.06376071914921e-06
28.5300000000017 3.68194765427067e-06
28.5400000000017 4.29255098168621e-06
28.5500000000017 4.89483565165021e-06
28.5600000000017 5.48808061227749e-06
28.5700000000017 6.07157965383091e-06
28.5800000000017 6.64464223025639e-06
28.5900000000017 7.20659425702923e-06
28.6000000000017 7.75677888441974e-06
28.6100000000017 8.29455724531636e-06
28.6200000000017 8.81930917678539e-06
28.6300000000017 9.33043391457086e-06
28.6400000000017 9.82735075980339e-06
28.6500000000017 1.03094997171883e-05
28.6600000000017 1.07763421040135e-05
28.6700000000017 1.12273611293438e-05
28.6800000000017 1.16620624428131e-05
28.6900000000017 1.20799746524649e-05
28.7000000000017 1.24806498111353e-05
28.7100000000017 1.28636638709149e-05
28.7200000000017 1.32286171052656e-05
28.7300000000017 1.35751344984194e-05
28.7400000000017 1.39028661017163e-05
28.7500000000017 1.42114873565986e-05
28.7600000000017 1.45006993840094e-05
28.7700000000017 1.47702292399927e-05
28.7800000000017 1.50198301373388e-05
28.7900000000017 1.52492816331591e-05
28.8000000000017 1.54583897823225e-05
28.8100000000017 1.56469872567278e-05
28.8200000000017 1.58149334305629e-05
28.8300000000017 1.59621144311496e-05
28.8400000000017 1.60884431560602e-05
28.8500000000017 1.61938592564721e-05
28.8600000000017 1.62783290868726e-05
28.8700000000017 1.63418456214337e-05
28.8800000000017 1.63844283373755e-05
28.8900000000017 1.64061230656941e-05
28.9000000000017 1.64070018097016e-05
28.9100000000017 1.63871625319138e-05
28.9200000000017 1.63467289099316e-05
28.9300000000017 1.62858500621195e-05
28.9400000000017 1.62047002441242e-05
28.9500000000017 1.61034785176646e-05
28.9600000000017 1.59824083937104e-05
28.9700000000017 1.58417374534664e-05
28.9800000000017 1.56817369532687e-05
28.9900000000017 1.55027014256037e-05
29.0000000000017 1.53049483040771e-05
29.0100000000017 1.50888171267271e-05
29.0200000000017 1.48546635870465e-05
29.0300000000017 1.46028690362957e-05
29.0400000000017 1.43338352934981e-05
29.0500000000017 1.40479840406133e-05
29.0600000000017 1.37457561953306e-05
29.0700000000017 1.34276112624553e-05
29.0800000000017 1.30940266648507e-05
29.0900000000017 1.27454970549082e-05
29.1000000000017 1.2382532362759e-05
29.1100000000018 1.20056586420108e-05
29.1200000000018 1.16154183529406e-05
29.1300000000018 1.12123681397437e-05
29.1400000000018 1.07970781033525e-05
29.1500000000018 1.03701310418041e-05
29.1600000000018 9.93212166842893e-06
29.1700000000018 9.48365581330474e-06
29.1800000000018 9.02534961129527e-06
29.1900000000018 8.55782867898745e-06
29.2000000000018 8.08172728232115e-06
29.2100000000018 7.59768749642958e-06
29.2200000000018 7.10635835905307e-06
29.2300000000018 6.60839501879647e-06
29.2400000000018 6.10445787944334e-06
29.2500000000018 5.59521174150034e-06
29.2600000000018 5.08132494212529e-06
29.2700000000018 4.56346849456705e-06
29.2800000000018 4.04231522823935e-06
29.2900000000018 3.51853893052268e-06
29.3000000000018 2.99281349139342e-06
29.3100000000018 2.46581205195344e-06
29.3200000000018 1.93820615792627e-06
29.3300000000018 1.41066491917311e-06
29.3400000000018 8.83854176267363e-07
29.3500000000018 3.58435675147811e-07
29.3600000000018 -1.64933749136986e-07
29.3700000000018 -6.85602978608317e-07
29.3800000000018 -1.20292740748378e-06
29.3900000000018 -1.71626972530066e-06
29.4000000000018 -2.22500068709858e-06
29.4100000000018 -2.72849986975192e-06
29.4200000000018 -3.22615641356312e-06
29.4300000000018 -3.71736974824284e-06
29.4400000000018 -4.20155030244156e-06
29.4500000000018 -4.67812019600769e-06
29.4600000000018 -5.1465139141831e-06
29.4700000000018 -5.60617896296925e-06
29.4800000000018 -6.05657650492363e-06
29.4900000000018 -6.49718197467913e-06
29.5000000000018 -6.92748567350457e-06
29.5100000000018 -7.34699334224999e-06
29.5200000000018 -7.75522671206632e-06
29.5300000000018 -8.15172403229962e-06
29.5400000000018 -8.53604057500797e-06
29.5500000000018 -8.90774911556953e-06
29.5600000000018 -9.26644038889515e-06
29.5700000000018 -9.61172352078247e-06
29.5800000000018 -9.94322643398574e-06
29.5900000000018 -1.02605962286097e-05
29.6000000000018 -1.05634995364701e-05
29.6100000000018 -1.0851622849098e-05
29.6200000000018 -1.11246728191013e-05
29.6300000000018 -1.13823765346284e-05
29.6400000000018 -1.16244817667205e-05
29.6500000000018 -1.18507571893697e-05
29.6600000000018 -1.20609925721371e-05
29.6700000000018 -1.22549989452212e-05
29.6800000000018 -1.24326087369064e-05
29.6900000000018 -1.25936758833509e-05
29.7000000000018 -1.27380759107354e-05
29.7100000000018 -1.28657059898151e-05
29.7200000000018 -1.29764849627115e-05
29.7300000000018 -1.3070353342523e-05
29.7400000000018 -1.31472732855531e-05
29.7500000000019 -1.32072285364366e-05
29.7600000000019 -1.32502243463996e-05
29.7700000000019 -1.32762873649266e-05
29.7800000000019 -1.32854655051674e-05
29.7900000000019 -1.3277827783469e-05
29.8000000000019 -1.32534641334989e-05
29.8100000000019 -1.32124851955228e-05
29.8200000000019 -1.31550220815461e-05
29.8300000000019 -1.3081226117258e-05
29.8400000000019 -1.2991268562094e-05
29.8500000000019 -1.28853403094369e-05
29.8600000000019 -1.27636515703465e-05
29.8700000000019 -1.26264315471856e-05
29.8800000000019 -1.24739281106306e-05
29.8900000000019 -1.23064075131456e-05
29.9000000000019 -1.21241512969488e-05
29.9100000000019 -1.1927458200015e-05
29.9200000000019 -1.17166454447173e-05
29.9300000000019 -1.14920465714754e-05
29.9400000000019 -1.12540109398139e-05
29.9500000000019 -1.10029032116818e-05
29.9600000000019 -1.0739102817824e-05
29.9700000000019 -1.04630034079925e-05
29.9800000000019 -1.01750122857904e-05
29.9900000000019 -9.87554811752895e-06
30.0000000000019 -9.56504404235647e-06
};
\addlegendentry{PI};
\addplot [line width=3pt, color1]
table {%
0 0
0.01 0
0.02 0
0.03 0
0.04 0
0.05 0
0.06 0
0.07 0
0.08 0
0.09 0
0.1 0
0.11 0
0.12 0
0.13 0
0.14 0
0.15 0
0.16 0
0.17 0
0.18 0
0.19 0
0.2 0
0.21 0
0.22 0
0.23 0
0.24 0
0.25 0
0.26 0
0.27 0
0.28 0
0.29 0
0.3 0
0.31 0
0.32 0
0.33 0
0.34 0
0.35 0
0.36 0
0.37 0
0.38 0
0.39 0
0.4 0
0.41 0
0.42 0
0.43 0
0.44 0
0.45 0
0.46 0
0.47 0
0.48 0
0.49 0
0.5 0
0.51 0
0.52 0
0.53 0
0.54 0
0.55 0
0.56 0
0.57 0
0.58 0
0.59 0
0.6 0
0.61 0
0.62 0
0.63 0
0.64 0
0.65 0
0.66 0
0.67 0
0.68 0
0.69 0
0.7 0
0.71 0
0.72 0
0.73 0
0.74 0
0.75 0
0.76 0
0.77 0
0.78 0
0.79 0
0.8 0
0.81 0
0.820000000000001 0
0.830000000000001 0
0.840000000000001 0
0.850000000000001 0
0.860000000000001 0
0.870000000000001 0
0.880000000000001 0
0.890000000000001 0
0.900000000000001 0
0.910000000000001 0
0.920000000000001 0
0.930000000000001 0
0.940000000000001 0
0.950000000000001 0
0.960000000000001 0
0.970000000000001 0
0.980000000000001 0
0.990000000000001 0
1 0
1.01 -4.56702698575438e-08
1.02 -3.14412282210323e-07
1.03 -1.03224858863571e-06
1.04 -2.42692001312371e-06
1.05 -4.72220514793178e-06
1.06 -8.14114550207929e-06
1.07 -1.29054674852236e-05
1.08 -1.92351943429038e-05
1.09 -2.73483098595994e-05
1.1 -3.74604295610264e-05
1.11 -4.97844613317375e-05
1.12 -6.45302582465675e-05
1.13 -8.19042662493712e-05
1.14 -0.000102109168994071
1.15 -0.000125343531876731
1.16 -0.000151801447050597
1.17 -0.00018167218102023
1.18 -0.000215139826246024
1.19 -0.000252382958049325
1.2 -0.000293574297985703
1.21 -0.000338880384746122
1.22 -0.000388461253549982
1.23 -0.000442470124908258
1.24 -0.000501053103557714
1.25 -0.000564348888296994
1.26 -0.000632488493391376
1.27 -0.000705594982154219
1.28 -0.000783783213258725
1.29 -0.000867159600283394
1.3 -0.000955821884947396
1.31 -0.00104985892444807
1.32 -0.00114935049327104
1.33 -0.00125436709980434
1.34 -0.00136496981805016
1.35 -0.00148121013469254
1.36 -0.00160312981174468
1.37 -0.00173076076496695
1.38 -0.00186412495821491
1.39 -0.00200323431384583
1.4 -0.00214809063928263
1.41 -0.00229868556980549
1.42 -0.00245500052761286
1.43 -0.00261700669716682
1.44 -0.00278466501667225
1.45 -0.00295792618619637
1.46 -0.00313673069158975
1.47 -0.00332100884456962
1.48 -0.00351068083878698
1.49 -0.00370565682162715
1.5 -0.00390583698206329
1.51 -0.00411111165360954
1.52 -0.00432136143270295
1.53 -0.0045364573121556
1.54 -0.00475626082943241
1.55 -0.00498062422949103
1.56 -0.0052093906419006
1.57 -0.00544239427193859
1.58 -0.00567946060534774
1.59 -0.00592040662698311
1.6 -0.0061650410541343
1.61 -0.00641316457871866
1.62 -0.00666457012321627
1.63 -0.00691904310879842
1.64 -0.00717636173505131
1.65 -0.00743629727097169
1.66 -0.00769861439029244
1.67 -0.00796307140439868
1.68 -0.00822942060418756
1.69 -0.0084974086039717
1.7 -0.00876677667869147
1.71 -0.00903726110918077
1.72 -0.0093085935349716
1.73 -0.00958050131411414
1.74 -0.00985270788948236
1.75 -0.0101249331610275
1.76 -0.0103968938634371
1.77 -0.01066830394865
1.78 -0.0109388749727432
1.79 -0.0112083164864504
1.8 -0.0114763364288989
1.81 -0.0117426415239977
1.82 -0.0120069376788832
1.83 -0.0122689303838665
1.84 -0.0125283251133168
1.85 -0.0127848277269192
1.86 -0.0130381448707465
1.87 -0.013287984377585
1.88 -0.0135340556659596
1.89 -0.0137760701373067
1.9 -0.0140137415707465
1.91 -0.0142467865149136
1.92 -0.0144749246763098
1.93 -0.0146978793036483
1.94 -0.0149153775676687
1.95 -0.0151271509359078
1.96 -0.0153329355419192
1.97 -0.0155324725484477
1.98 -0.0157255085040693
1.99 -0.0159117956928221
2 -0.0160910924763613
2.01 -0.0162631636281865
2.02 -0.0164277806594973
2.03 -0.0165847221362496
2.04 -0.0167337739869955
2.05 -0.016874729801105
2.06 -0.0170073911170471
2.07 -0.0171315677126815
2.08 -0.0172470778421629
2.09 -0.0173537485094947
2.1 -0.0174514157133097
2.11 -0.0175399246799069
2.12 -0.0176191300842584
2.13 -0.0176888962587173
2.14 -0.017749097389171
2.15 -0.0177996176984012
2.16 -0.0178403516164308
2.17 -0.0178712039376503
2.18 -0.0178920899645405
2.19 -0.0179029356378194
2.2 -0.0179036776528659
2.21 -0.0178942635622843
2.22 -0.0178746518644997
2.23 -0.017844812078285
2.24 -0.0178047248031464
2.25 -0.0177543817655067
2.26 -0.0176937858506483
2.27 -0.017622951120394
2.28 -0.0175419028165252
2.29 -0.01745067734995
2.29999999999999 -0.017349322275659
2.30999999999999 -0.0172378962535171
2.31999999999999 -0.017116468994963
2.32999999999999 -0.0169851211956915
2.33999999999999 -0.0168439444544621
2.34999999999999 -0.0166930411780935
2.35999999999999 -0.0165325244728114
2.36999999999999 -0.0163625180234051
2.37999999999999 -0.0161831559564623
2.38999999999999 -0.0159945826902335
2.39999999999999 -0.0157969527743333
2.40999999999999 -0.0155904307152294
2.41999999999999 -0.0153751907890753
2.42999999999999 -0.0151514168420423
2.43999999999999 -0.014919302078276
2.44999999999999 -0.0146790488355468
2.45999999999999 -0.0144308684617544
2.46999999999999 -0.0141749815215782
2.47999999999999 -0.0139116163226356
2.48999999999999 -0.0136410093442461
2.49999999999999 -0.0133634049583651
2.50999999999999 -0.0130790551423308
2.51999999999999 -0.0127882191833656
2.52999999999999 -0.0124911633749482
2.53999999999999 -0.0121881607052759
2.54999999999999 -0.0118794905381046
2.55999999999999 -0.0115654382863014
2.56999999999999 -0.0112462950784786
2.57999999999999 -0.0109223574190995
2.58999999999999 -0.0105939268424684
2.59999999999999 -0.0102613095610289
2.60999999999999 -0.00992481610840887
2.61999999999999 -0.00958476097765819
2.62999999999999 -0.00924146225513431
2.63999999999999 -0.00889524125049722
2.64999999999999 -0.00854642212328069
2.65999999999999 -0.00819533150651048
2.66999999999999 -0.00784229812784497
2.67999999999999 -0.00748765242871479
2.68999999999999 -0.00713172618194104
2.69999999999999 -0.00677485210831213
2.70999999999999 -0.00641736349259943
2.71999999999999 -0.00605959379949268
2.72999999999999 -0.00570187628993329
2.73999999999999 -0.00534454363832393
2.74999999999999 -0.00498792755108879
2.75999999999999 -0.00463235838705714
2.76999999999998 -0.00427816478013842
2.77999999999998 -0.00392567326475337
2.78999999999998 -0.00357520790448088
2.79999999999998 -0.00322708992437457
2.80999999999998 -0.00288163734739737
2.81999999999998 -0.00253916463541585
2.82999999999998 -0.00219998233518921
2.83999999999998 -0.00186439672977979
2.84999999999998 -0.00153270949580389
2.85999999999998 -0.00120521736693388
2.86999999999998 -0.000882211804052241
2.87999999999998 -0.000563978672449437
2.88999999999998 -0.000250797926447132
2.89999999999998 5.7056698182097e-05
2.90999999999998 0.000359317983625507
2.91999999999998 0.000655725520916634
2.92999999999998 0.00094602599211228
2.93999999999998 0.00122997344314835
2.94999999999998 0.00150732954726987
2.95999999999998 0.00177786385868231
2.96999999999998 0.00204135405613818
2.97999999999998 0.00229758617618606
2.98999999999998 0.00254635483582348
2.99999999999998 0.00278746344430865
3.00999999999998 0.00302072440390081
3.01999999999998 0.00324595929931345
3.02999999999998 0.00346299907567922
3.03999999999998 0.00367168420484046
3.04999999999998 0.00387186483979472
3.05999999999998 0.00406340095713928
3.06999999999998 0.00424616248737467
3.07999999999998 0.0044200294329424
3.08999999999998 0.00458489197388755
3.09999999999998 0.00474065056103012
3.10999999999998 0.0048872159966561
3.11999999999998 0.00502450950257532
3.12999999999998 0.00515246277544806
3.13999999999998 0.00527101802947697
3.14999999999998 0.00538012802639291
3.15999999999998 0.00547975609273206
3.16999999999998 0.00556987612477479
3.17999999999998 0.00565047258172155
3.18999999999998 0.00572154046112989
3.19999999999998 0.00578308526778103
3.20999999999998 0.00583512296784114
3.21999999999998 0.00587767993055738
3.22999999999998 0.00591079285755505
3.23999999999997 0.00593450869979861
3.24999999999997 0.0059488845622667
3.25999999999997 0.00595398759636447
3.26999999999997 0.0059498948800471
3.27999999999997 0.00593669328554226
3.28999999999997 0.0059144793344106
3.29999999999997 0.00588335903942656
3.30999999999997 0.00584345030795974
3.31999999999997 0.00579487613380529
3.32999999999997 0.00573777019728154
3.33999999999997 0.0056722749888938
3.34999999999997 0.00559854160900058
3.35999999999997 0.00551672956104437
3.36999999999997 0.00542700654145486
3.37999999999997 0.00532954823298803
3.38999999999997 0.00522453814266034
3.39999999999997 0.00511216626202445
3.40999999999997 0.00499263031444395
3.41999999999997 0.00486613512761913
3.42999999999997 0.00473289219839315
3.43999999999997 0.00459311941524984
3.44999999999997 0.00444704077435877
3.45999999999997 0.00429488608952313
3.46999999999997 0.00413689069639515
3.47999999999997 0.00397329498435436
3.48999999999997 0.00380434407396633
3.49999999999997 0.00363028814409521
3.50999999999997 0.00345138162365781
3.51999999999997 0.00326788289322278
3.52999999999997 0.00308005397178049
3.53999999999997 0.0028881601958746
3.54999999999997 0.00269246989406197
3.55999999999997 0.00249325405818141
3.56999999999997 0.00229078601232922
3.57999999999997 0.00208534108018961
3.58999999999997 0.00187719258733167
3.59999999999997 0.00166662322865014
3.60999999999997 0.00145391221502489
3.61999999999997 0.00123933943681376
3.62999999999997 0.0010231851268333
3.63999999999997 0.000805729525174629
3.64999999999997 0.000587252546058484
3.65999999999997 0.000368033446989364
3.66999999999997 0.000148350500504407
3.67999999999997 -7.15193311649118e-05
3.68999999999997 -0.000291300718185713
3.69999999999997 -0.000510720281626398
3.70999999999996 -0.000729506908037226
3.71999999999996 -0.00094739205976098
3.72999999999996 -0.00116411008062261
3.73999999999996 -0.0013793984966628
3.74999999999996 -0.00159299831157998
3.75999999999996 -0.00180465429655127
3.76999999999996 -0.00201411527410899
3.77999999999996 -0.00222113439575707
3.78999999999996 -0.00242546941301932
3.79999999999996 -0.00262688294162014
3.80999999999996 -0.00282514271850724
3.81999999999996 -0.00302002185143558
3.82999999999996 -0.00321129906093341
3.83999999999996 -0.00339875891407068
3.84999999999996 -0.00358219205011439
3.85999999999996 -0.00376139539782446
3.86999999999996 -0.00393617238398743
3.87999999999996 -0.0041063331330459
3.88999999999996 -0.00427169465761873
3.89999999999996 -0.0044320810397194
3.90999999999996 -0.00458732360249194
3.91999999999996 -0.0047372610722959
3.92999999999996 -0.00488173973098484
3.93999999999996 -0.00502061355823467
3.94999999999996 -0.00515374436379172
3.95999999999996 -0.00528100190952229
3.96999999999996 -0.00540226402115892
3.97999999999996 -0.00551741668965081
3.98999999999996 -0.00562635416203904
3.99999999999996 -0.00572897902178935
4.00999999999996 -0.00582520225852801
4.01999999999996 -0.00591494332713832
4.02999999999996 -0.00599813019618698
4.03999999999996 -0.00607469938566092
4.04999999999996 -0.00614459599400559
4.05999999999996 -0.00620777371455361
4.06999999999996 -0.00626419484342007
4.07999999999996 -0.00631383026954371
4.08999999999996 -0.00635665946156727
4.09999999999996 -0.00639267050525298
4.10999999999996 -0.00642186025921436
4.11999999999996 -0.00644423377270094
4.12999999999996 -0.0064598045172029
4.13999999999996 -0.00646859432016435
4.14999999999996 -0.00647063328884705
4.15999999999996 -0.00646595972450315
4.16999999999996 -0.00645462002704155
4.17999999999996 -0.00643666859040992
4.18999999999996 -0.00641216768897437
4.19999999999995 -0.00638118735528696
4.20999999999995 -0.00634380524984481
4.21999999999995 -0.00630010652604824
4.22999999999995 -0.00625018368426875
4.23999999999995 -0.00619413643168278
4.24999999999995 -0.00613207155452776
4.25999999999995 -0.00606410190361618
4.26999999999995 -0.00599034744460079
4.27999999999995 -0.00591093479691688
4.28999999999995 -0.00582599683637805
4.29999999999995 -0.00573567250171466
4.30999999999995 -0.00564010659036509
4.31999999999995 -0.00553944954616248
4.32999999999995 -0.00543385724021395
4.33999999999995 -0.00532349074573674
4.34999999999995 -0.00520851610738762
4.35999999999995 -0.00508910410551723
4.36999999999995 -0.00496543001573138
4.37999999999995 -0.00483767336411753
4.38999999999995 -0.00470601767848254
4.39999999999995 -0.00457065023594328
4.40999999999995 -0.00443176180720864
4.41999999999995 -0.00428954639789181
4.42999999999995 -0.00414420098719166
4.43999999999995 -0.00399592526428246
4.44999999999995 -0.00384492136275236
4.45999999999995 -0.00369139359343035
4.46999999999995 -0.0035355453477114
4.47999999999995 -0.0033775876374996
4.48999999999995 -0.00321772965713743
4.49999999999995 -0.00305618170405383
4.50999999999995 -0.00289315490809815
4.51999999999995 -0.00272886096113903
4.52999999999995 -0.00256351184806272
4.53999999999995 -0.00239731957862205
4.54999999999995 -0.00223049592073848
4.55999999999995 -0.00206325213557735
4.56999999999995 -0.00189579871471302
4.57999999999995 -0.00172834511969806
4.58999999999995 -0.00156109952434565
4.59999999999995 -0.00139426856003046
4.60999999999995 -0.00122805706430858
4.61999999999995 -0.00106266783315121
4.62999999999995 -0.00089830137708213
4.63999999999995 -0.000735155681502113
4.64999999999995 -0.000573425971477527
4.65999999999995 -0.000413304481263418
4.66999999999994 -0.000254980228824893
4.67999999999994 -9.86387956123772e-05
4.68999999999994 5.55378881603969e-05
4.69999999999994 0.000207371752486784
4.70999999999994 0.000356688790570934
4.71999999999994 0.000503319255655039
4.72999999999994 0.000647097851396419
4.73999999999994 0.000787863915636973
4.74999999999994 0.000925461597355301
4.75999999999994 0.00105974002481702
4.76999999999994 0.00119055345817226
4.77999999999994 0.00131776147293568
4.78999999999994 0.00144122909161721
4.79999999999994 0.00156082692402237
4.80999999999994 0.00167643129975264
4.81999999999994 0.00178792439277951
4.82999999999994 0.00189519433797494
4.83999999999994 0.00199813533949171
4.84999999999994 0.00209664777096429
4.85999999999994 0.00219063826729141
4.86999999999994 0.00228001980781176
4.87999999999994 0.0023647117913946
4.88999999999994 0.00244463921318899
4.89999999999994 0.00251973564373065
4.90999999999994 0.00258994010222588
4.91999999999994 0.00265519824544995
4.92999999999994 0.0027154623996715
4.93999999999994 0.00277069158401271
4.94999999999994 0.00282085152525523
4.95999999999994 0.00286591466411331
4.96999999999994 0.00290586015519535
4.97999999999994 0.00294067385183946
4.98999999999994 0.00297034828918214
4.99999999999994 0.0029948826568622
5.00999999999994 0.00301428276335714
5.01999999999994 0.00302856099216281
5.02999999999994 0.00303773624991572
5.03999999999994 0.00304183390658215
5.04999999999994 0.00304088572785744
5.05999999999994 0.00303492979994373
5.06999999999994 0.00302401044691617
5.07999999999994 0.00300817814095308
5.08999999999994 0.00298748940680104
5.09999999999994 0.00296200671876972
5.10999999999994 0.00293179839098791
5.11999999999994 0.00289693846825622
5.12999999999994 0.00285750661730387
5.13999999999993 0.00281358803165423
5.14999999999993 0.0027652733840897
5.15999999999993 0.00271265763567614
5.16999999999993 0.0026558409978875
5.17999999999993 0.0025949288549291
5.18999999999993 0.00253003110384417
5.19999999999993 0.00246126205207508
5.20999999999993 0.0023887408913177
5.21999999999993 0.00231259091980459
5.22999999999993 0.00223293940690849
5.23999999999993 0.00214991742817973
5.24999999999993 0.00206365968637641
5.25999999999993 0.00197430432434541
5.26999999999993 0.00188199273232489
5.27999999999993 0.00178686935097737
5.28999999999993 0.00168908147092962
5.29999999999993 0.00158877902935227
5.30999999999993 0.0014861144039938
5.31999999999993 0.00138124220502378
5.32999999999993 0.0012743190650076
5.33999999999993 0.00116550342731696
5.34999999999993 0.00105495533327042
5.35999999999993 0.000942836208291411
5.36999999999993 0.000829308647367516
5.37999999999993 0.00071453620009114
5.38999999999993 0.000598683155559701
5.39999999999993 0.000481914327411189
5.40999999999993 0.000364394839268442
5.41999999999993 0.000246289910863508
5.42999999999993 0.000127764645110726
5.43999999999993 8.98381639447029e-06
5.44999999999993 -0.000109888339665201
5.45999999999993 -0.000228688334709431
5.46999999999993 -0.000347253634143528
5.47999999999993 -0.000465422861393727
5.48999999999993 -0.000583035999662597
5.49999999999993 -0.000699934590957058
5.50999999999993 -0.00081596193213827
5.51999999999993 -0.000930963267752629
5.52999999999993 -0.00104478597942291
5.53999999999993 -0.00115727977157408
5.54999999999993 -0.00126829685327541
5.55999999999993 -0.00137769211598698
5.56999999999993 -0.00148532330700467
5.57999999999993 -0.00159105119840394
5.58999999999993 -0.00169473975128986
5.59999999999993 -0.00179625627516714
5.60999999999992 -0.00189547158225157
5.61999999999992 -0.00199226013655087
5.62999999999992 -0.0020865001975511
5.63999999999992 -0.00217807395835176
5.64999999999992 -0.00226686767810109
5.65999999999992 -0.00235277180859027
5.66999999999992 -0.00243568111487424
5.67999999999992 -0.0025154947897942
5.68999999999992 -0.00259211656228566
5.69999999999992 -0.00266545479936417
5.70999999999992 -0.00273542260168956
5.71999999999992 -0.00280193789261796
5.72999999999992 -0.00286492350065963
5.73999999999992 -0.00292430723526952
5.74999999999992 -0.0029800219559062
5.75999999999992 -0.00303200563430371
5.76999999999992 -0.00308020140990979
5.77999999999992 -0.00312455763845315
5.78999999999992 -0.00316502793361096
5.79999999999992 -0.00320157120175746
5.80999999999992 -0.00323415166978314
5.81999999999992 -0.00326273890769518
5.82999999999992 -0.00328730783989478
5.83999999999992 -0.00330783875224398
5.84999999999992 -0.00332431729318785
5.85999999999992 -0.0033367344671051
5.86999999999992 -0.00334508662086859
5.87999999999992 -0.00334937542367718
5.88999999999992 -0.00334960784023563
5.89999999999992 -0.00334579609737833
5.90999999999992 -0.00333795764425978
5.91999999999992 -0.00332611510627616
5.92999999999992 -0.00331029623295033
5.93999999999992 -0.00329053384013617
5.94999999999992 -0.00326686574713574
5.95999999999992 -0.00323933470981761
5.96999999999992 -0.00320798835193645
5.97999999999992 -0.00317287909960448
5.98999999999992 -0.00313406413139951
5.99999999999992 -0.00309160537036543
6.00999999999992 -0.00304556928628709
6.01999999999992 -0.00299602508243559
6.02999999999992 -0.00294304742333545
6.03999999999992 -0.00288671512814451
6.04999999999992 -0.0028271110495327
6.05999999999992 -0.0027643219480957
6.06999999999992 -0.00269843836225924
6.07999999999991 -0.0026295542340756
6.08999999999991 -0.00255776690913475
6.09999999999991 -0.00248317767256908
6.10999999999991 -0.00240589092875839
6.11999999999991 -0.00232601410087476
6.12999999999991 -0.00224365750158161
6.13999999999991 -0.00215893418978857
6.14999999999991 -0.00207195981944098
6.15999999999991 -0.00198285248307801
6.16999999999991 -0.00189173255157262
6.17999999999991 -0.0017987225108806
6.18999999999991 -0.00170394679634644
6.19999999999991 -0.00160753162190045
6.20999999999991 -0.00150960481450305
6.21999999999991 -0.00141029564562073
6.22999999999991 -0.0013097346559048
6.23999999999991 -0.00120805348233963
6.24999999999991 -0.00110538468471566
6.25999999999991 -0.00100186157166898
6.26999999999991 -0.000897618026522447
6.27999999999991 -0.000792788333159092
6.28999999999991 -0.000687507002154885
6.29999999999991 -0.000581908597394639
6.30999999999991 -0.000476127563392593
6.31999999999991 -0.000370298053535927
6.32999999999991 -0.000264553759467204
6.33999999999991 -0.000159027741818617
6.34999999999991 -5.3852262508059e-05
6.35999999999991 5.08413811959412e-05
6.36999999999991 0.000154923020637339
6.37999999999991 0.000258263777554773
6.38999999999991 0.000360736223514815
6.39999999999991 0.000462214536773481
6.40999999999991 0.000562574656387694
6.41999999999991 0.000661694433395888
6.42999999999991 0.000759458871543323
6.43999999999991 0.000855745217284463
6.44999999999991 0.000950437929328321
6.45999999999991 0.00104342394551219
6.46999999999991 0.00113459281706554
6.47999999999991 0.00122383683890799
6.48999999999991 0.001311051175838
6.49999999999991 0.00139613398447413
6.50999999999991 0.00147898653081626
6.51999999999991 0.00155951330329951
6.52999999999991 0.00163762212121979
6.53999999999991 0.00171322423841503
6.5499999999999 0.00178623444209335
6.5599999999999 0.00185657114670457
6.5699999999999 0.00192414952078153
6.5799999999999 0.00198890259266039
6.5899999999999 0.00205076021142161
6.5999999999999 0.0021096561809031
6.6099999999999 0.00216552832729217
6.6199999999999 0.00221831856131671
6.6299999999999 0.00226797293497879
6.6399999999999 0.00231444169278139
6.6499999999999 0.00235767931740654
6.6599999999999 0.00239764456981063
6.6699999999999 0.00243430052371025
6.6799999999999 0.0024676145944395
6.6899999999999 0.00249755856216717
6.6999999999999 0.0025241085894701
6.7099999999999 0.0025472452348407
6.7199999999999 0.00256695345657212
6.7299999999999 0.00258322261333067
6.7399999999999 0.00259604646021315
6.7499999999999 0.00260542313832703
6.7599999999999 0.00261135515887606
6.7699999999999 0.0026138493818104
6.7799999999999 0.00261291698911464
6.7899999999999 0.0026085734528259
6.7999999999999 0.00260083849790064
6.8099999999999 0.00258973606008942
6.8199999999999 0.00257529423904414
6.8299999999999 0.00255754524700381
6.8399999999999 0.00253652535361347
6.8499999999999 0.00251227482786934
6.8599999999999 0.00248483787910447
6.8699999999999 0.00245426260105603
6.8799999999999 0.00242060092452187
6.8899999999999 0.00238390849986336
6.8999999999999 0.00234424464827162
6.9099999999999 0.00230167141454017
6.9199999999999 0.00225625430184514
6.9299999999999 0.00220806258182843
6.9399999999999 0.00215716874840198
6.9499999999999 0.00210364841166128
6.9599999999999 0.00204758018841835
6.9699999999999 0.00198904511263188
6.9799999999999 0.00192812735112422
6.9899999999999 0.00186491400935433
6.9999999999999 0.00179949472907191
7.00999999999989 0.00173196150846321
7.01999999999989 0.00166240859926983
7.02999999999989 0.00159093239077703
7.03999999999989 0.00151763128595354
7.04999999999989 0.00144260557265333
7.05999999999989 0.00136595729137589
7.06999999999989 0.00128779010044417
7.07999999999989 0.00120820913915048
7.08999999999989 0.00112732088926347
7.09999999999989 0.00104523303520358
7.10999999999989 0.000962054323147272
7.11999999999989 0.000877894419291845
7.12999999999989 0.000792863767496115
7.13999999999989 0.000707073446501297
7.14999999999989 0.000620635026929133
7.15999999999989 0.000533660428249744
7.16999999999989 0.000446261775907354
7.17999999999989 0.000358551258789548
7.18999999999989 0.000270640987222589
7.19999999999989 0.000182642851673209
7.20999999999989 9.46683823346566e-05
7.21999999999989 6.82860976681752e-06
7.22999999999989 -8.07660732364912e-05
7.23999999999989 -0.00016800604834793
7.24999999999989 -0.000254782606853351
7.25999999999989 -0.000340988083063305
7.26999999999989 -0.000426515985819924
7.27999999999989 -0.000511265435743074
7.28999999999989 -0.000595128695006298
7.29999999999989 -0.000678003564573575
7.30999999999989 -0.000759789519002424
7.31999999999989 -0.00084038782648568
7.32999999999989 -0.000919701666064654
7.33999999999989 -0.000997636241882978
7.34999999999989 -0.00107409889435306
7.35999999999989 -0.00114899920811171
7.36999999999989 -0.00122224911664505
7.37999999999989 -0.00129376300322991
7.38999999999989 -0.001363457798627
7.39999999999989 -0.00143125307553262
7.40999999999989 -0.00149707113844429
7.41999999999989 -0.00156083710998829
7.42999999999989 -0.00162247901329206
7.43999999999989 -0.00168192785031396
7.44999999999989 -0.00173911767604824
7.45999999999989 -0.00179397841225827
7.46999999999989 -0.0018464578248494
7.47999999999988 -0.00189649958341368
7.48999999999988 -0.00194405066786694
7.49999999999988 -0.00198906142093598
7.50999999999988 -0.00203148559617937
7.51999999999988 -0.00207128040149992
7.52999999999988 -0.00210840653811307
7.53999999999988 -0.00214282823494169
7.54999999999988 -0.00217451327841414
7.55999999999988 -0.00220343303791404
7.56999999999988 -0.00222956248575201
7.57999999999988 -0.00225288021325329
7.58999999999988 -0.00227336844400992
7.59999999999988 -0.00229101303727037
7.60999999999988 -0.00230580349103359
7.61999999999988 -0.00231773294002165
7.62999999999988 -0.00232679814864851
7.63999999999988 -0.00233299949949029
7.64999999999988 -0.00233634097731353
7.65999999999988 -0.0023368301487349
7.66999999999988 -0.00233447813760962
7.67999999999988 -0.00232929959628168
7.68999999999988 -0.00232131267288649
7.69999999999988 -0.0023105389749946
7.70999999999988 -0.00229700353006224
7.71999999999988 -0.00228073474349196
7.72999999999988 -0.00226176435580089
7.73999999999988 -0.00224012738790183
7.74999999999988 -0.00221586206834461
7.75999999999988 -0.0021890097955682
7.76999999999988 -0.0021596150959424
7.77999999999988 -0.00212772559382635
7.78999999999988 -0.0020933906602953
7.79999999999988 -0.00205666341062924
7.80999999999988 -0.00201759974120621
7.81999999999988 -0.00197625824612597
7.82999999999988 -0.00193270013092413
7.83999999999988 -0.00188698912347598
7.84999999999988 -0.00183919138219689
7.85999999999988 -0.00178937482251545
7.86999999999988 -0.0017376102932181
7.87999999999988 -0.00168397075529473
7.88999999999988 -0.00162853118715017
7.89999999999988 -0.00157136851820769
7.90999999999988 -0.00151256149706831
7.91999999999988 -0.00145219057401148
7.92999999999988 -0.00139033779914695
7.93999999999988 -0.00132708671625768
7.94999999999987 -0.00126252225367742
7.95999999999987 -0.00119673061298357
7.96999999999987 -0.00112979915600753
7.97999999999987 -0.00106181629051712
7.98999999999987 -0.000992871354845572
7.99999999999987 -0.000923054501695055
8.00999999999987 -0.00085245658131511
8.01999999999987 -0.000781169024239489
8.02999999999987 -0.000709283723753719
8.03999999999987 -0.000636892918258511
8.04999999999987 -0.000564089073688896
8.05999999999987 -0.000490964766145133
8.06999999999987 -0.000417612564888299
8.07999999999987 -0.000344124915850528
8.08999999999987 -0.000270594025808506
8.09999999999987 -0.000197111747364978
8.10999999999987 -0.000123769464882233
8.11999999999987 -5.06579815086466e-05
8.12999999999987 2.21325925631992e-05
8.13999999999987 9.45163776568688e-05
8.14999999999987 0.000166401961630369
8.15999999999987 0.000237701975107046
8.16999999999987 0.000308330168557878
8.17999999999987 0.00037820151621999
8.18999999999987 0.00044723231796884
8.19999999999987 0.000515340299030038
8.20999999999987 0.000582444707419217
8.21999999999987 0.000648466409001026
8.22999999999987 0.00071332798006082
8.23999999999987 0.000776953797286301
8.24999999999987 0.000839270125058816
8.25999999999987 0.000900205199957883
8.26999999999987 0.000959689312386331
8.27999999999987 0.00101765488516002
8.28999999999987 0.00107403654876454
8.29999999999987 0.0011287712146525
8.30999999999987 0.00118179814407977
8.31999999999987 0.00123305901407116
8.32999999999987 0.00128249798002975
8.33999999999987 0.00133006173492477
8.34999999999987 0.00137569956499714
8.35999999999987 0.0014193634019255
8.36999999999987 0.00146100787139982
8.37999999999987 0.0015005829145111
8.38999999999987 0.00153805635480369
8.39999999999987 0.00157339122130214
8.40999999999987 0.00160655339730183
8.41999999999986 0.00163751165047405
8.42999999999986 0.00166623765920532
8.43999999999986 0.00169270603515565
8.44999999999986 0.00171689434202522
8.45999999999986 0.00173878311052507
8.46999999999986 0.00175835585036807
8.47999999999986 0.00177559905704414
8.48999999999986 0.00179050221402876
8.49999999999986 0.00180305779381102
8.50999999999986 0.00181326125402478
8.51999999999986 0.00182111102988873
8.52999999999986 0.00182660852300417
8.53999999999986 0.0018297580865748
8.54999999999986 0.0018305670071348
8.55999999999986 0.00182904548290507
8.56999999999986 0.00182520659895015
8.57999999999986 0.00181906629939776
8.58999999999986 0.0018106433571411
8.59999999999986 0.00179995933620257
8.60999999999986 0.00178703854696338
8.61999999999986 0.00177190800993024
8.62999999999986 0.00175459741406812
8.63999999999986 0.00173513907513339
8.64999999999986 0.00171356789760961
8.65999999999986 0.00168992135042917
8.66999999999986 0.00166423886367784
8.67999999999986 0.00163656214111643
8.68999999999986 0.00160693556871751
8.69999999999986 0.00157540578416521
8.70999999999986 0.00154202160724568
8.71999999999986 0.00150683396790039
8.72999999999986 0.00146989583202592
8.73999999999986 0.00143126212511028
8.74999999999986 0.00139098921541599
8.75999999999986 0.00134913577000233
8.76999999999986 0.00130576218457999
8.77999999999986 0.00126093047188826
8.78999999999986 0.00121470418229002
8.79999999999986 0.0011671483201374
8.80999999999986 0.00111832925731685
8.81999999999986 0.00106831465084782
8.82999999999986 0.00101717335476452
8.83999999999986 0.000964975315842185
8.84999999999986 0.00091179148111756
8.85999999999986 0.000857693705452511
8.86999999999986 0.000802754657904446
8.87999999999986 0.000747047727185088
8.88999999999985 0.000690646926430347
8.89999999999985 0.000633626797468552
8.90999999999985 0.000576062314752243
8.91999999999985 0.000518028789105362
8.92999999999985 0.000459601771427792
8.93999999999985 0.000400856956493254
8.94999999999985 0.000341870086971406
8.95999999999985 0.000282716857801801
8.96999999999985 0.000223472821044256
8.97999999999985 0.000164213291327513
8.98999999999985 0.000105013252016191
8.99999999999985 4.59472622131754e-05
9.00999999999985 -1.29106352866928e-05
9.01999999999985 -7.14870049779854e-05
9.02999999999985 -0.000129712272538907
9.03999999999985 -0.000187511753973662
9.04999999999985 -0.000244814371600656
9.05999999999985 -0.000301550001308893
9.06999999999985 -0.000357649556900692
9.07999999999985 -0.00041304507270506
9.08999999999985 -0.000467669784370108
9.09999999999985 -0.00052145820774507
9.10999999999985 -0.000574346215764377
9.11999999999985 -0.00062627111324861
9.12999999999985 -0.000677171709539574
9.13999999999985 -0.000726988388889401
9.14999999999985 -0.000775663178526504
9.15999999999985 -0.00082313981432399
9.16999999999985 -0.000869363803667021
9.17999999999985 -0.000914282486639798
9.18999999999985 -0.000957845093984301
9.19999999999985 -0.00100000280240599
9.20999999999985 -0.00104070878724629
9.21999999999985 -0.0010799182723771
9.22999999999985 -0.00111758857726544
9.23999999999985 -0.0011536791611595
9.24999999999985 -0.00118815166435088
9.25999999999985 -0.00122096284540454
9.26999999999985 -0.00125208612054794
9.27999999999985 -0.00128148995874762
9.28999999999985 -0.00130914514745527
9.29999999999985 -0.00133502481885508
9.30999999999985 -0.00135910447303366
9.31999999999985 -0.00138136199805862
9.32999999999985 -0.0014017776869567
9.33999999999985 -0.00142033425158686
9.34999999999985 -0.00143701683354432
9.35999999999984 -0.00145181301293943
9.36999999999984 -0.00146471281020877
9.37999999999984 -0.00147570868911612
9.38999999999984 -0.00148479555520942
9.39999999999984 -0.00149197075125953
9.40999999999984 -0.00149723404972304
9.41999999999984 -0.00150058764228631
9.42999999999984 -0.00150203612656783
9.43999999999984 -0.00150158649005259
9.44999999999984 -0.00149924808922244
9.45999999999984 -0.00149503262634541
9.46999999999984 -0.00148895412490803
9.47999999999984 -0.00148102890212317
9.48999999999984 -0.00147127553906187
9.49999999999984 -0.00145971484870804
9.50999999999984 -0.00144636984244506
9.51999999999984 -0.00143126569593979
9.52999999999984 -0.00141442971645245
9.53999999999984 -0.00139589131675814
9.54999999999984 -0.00137568188548628
9.55999999999984 -0.00135383413193229
9.56999999999984 -0.00133038322291184
9.57999999999984 -0.00130536619298665
9.58999999999984 -0.00127882188854218
9.59999999999984 -0.00125079090992076
9.60999999999984 -0.00122131555167681
9.61999999999984 -0.00119043974102567
9.62999999999984 -0.00115820871996682
9.63999999999984 -0.0011246694078592
9.64999999999984 -0.00108987030787925
9.65999999999984 -0.00105386127617251
9.66999999999984 -0.0010166934588677
9.67999999999984 -0.00097841922495073
9.68999999999984 -0.000939092096496868
9.69999999999984 -0.00089876667700411
9.70999999999984 -0.000857498578249446
9.71999999999984 -0.000815344345940021
9.72999999999984 -0.000772361384356347
9.73999999999984 -0.000728607880344804
9.74999999999984 -0.000684142730719668
9.75999999999984 -0.000639025460472026
9.76999999999984 -0.000593316142760465
9.77999999999984 -0.000547075320981842
9.78999999999984 -0.000500363930479687
9.79999999999984 -0.000453243220024529
9.80999999999984 -0.000405774673190206
9.81999999999984 -0.000358019929742412
9.82999999999983 -0.000310040707150967
9.83999999999983 -0.000261898722333029
9.84999999999983 -0.000213655613731913
9.85999999999983 -0.000165372863833075
9.86999999999983 -0.000117111722013049
9.87999999999983 -6.89331283653665e-05
9.88999999999983 -2.08976385447133e-05
9.89999999999983 2.69346513641876e-05
9.90999999999983 7.45041786834786e-05
9.91999999999983 0.000121751986976972
9.92999999999983 0.000168623077870492
9.93999999999983 0.000215057073634406
9.94999999999983 0.000260997256692289
9.95999999999983 0.000306387791058368
9.96999999999983 0.000351173789091529
9.97999999999983 0.000395301376719666
9.98999999999983 0.000438717757062323
9.99999999999983 0.000481371272381266
10.0099999999998 0.000523211464290874
10.0199999999998 0.000564189132161459
10.0299999999998 0.000604256389651809
10.0399999999998 0.000643366719308715
10.0499999999998 0.000681475025041648
10.0599999999998 0.000718537682710876
10.0699999999998 0.000754512588878808
10.0799999999998 0.000789359206913711
10.0899999999998 0.000823038611081555
10.0999999999998 0.000855513528385812
10.1099999999998 0.000886748378111065
10.1199999999998 0.00091670255985169
10.1299999999998 0.000945350800653097
10.1399999999998 0.000972662872950677
10.1499999999998 0.000998610377853765
10.1599999999998 0.0010231667727536
10.1699999999998 0.00104630739645606
10.1799999999998 0.0010680094918202
10.1899999999998 0.00108825222588731
10.1999999999998 0.00110701670748855
10.2099999999998 0.00112428600232277
10.2199999999998 0.00114004514550015
10.2299999999998 0.00115428115155105
10.2399999999998 0.00116698302272623
10.2499999999998 0.00117814175261023
10.2599999999998 0.00118775032866808
10.2699999999998 0.00119580373260837
10.2799999999998 0.00120229893784951
10.2899999999998 0.00120723490388575
10.2999999999998 0.00121061256838372
10.3099999999998 0.00121243483711803
10.3199999999998 0.00121270657150578
10.3299999999998 0.00121143457378336
10.3399999999998 0.00120862756988004
10.3499999999998 0.00120429619005826
10.3599999999998 0.00119845294741539
10.3699999999998 0.00119111221438164
10.3799999999998 0.0011822901974225
10.3899999999998 0.00117200491028896
10.3999999999998 0.0011602761464453
10.4099999999998 0.00114712545194677
10.4199999999998 0.0011325761016784
10.4299999999998 0.001116653087591
10.4399999999998 0.00109938250628217
10.4499999999998 0.00108079228458852
10.4599999999998 0.00106091200538854
10.4699999999998 0.00103977275102846
10.4799999999998 0.00101740705677433
10.4899999999998 0.000993848862702358
10.4999999999998 0.000969133464084768
10.5099999999998 0.000943297460331971
10.5199999999998 0.000916378385439326
10.5299999999998 0.000888415351245459
10.5399999999998 0.000859448596510812
10.5499999999998 0.000829519435177971
10.5599999999998 0.000798670203267155
10.5699999999998 0.000766944202976052
10.5799999999998 0.000734385644905735
10.5899999999998 0.000701039588895752
10.5999999999998 0.000666951883756984
10.6099999999998 0.000632169106097437
10.6199999999998 0.000596738498388166
10.6299999999998 0.00056070790639091
10.6399999999998 0.000524125716054054
10.6499999999998 0.000487040789975337
10.6599999999998 0.000449502403232602
10.6699999999998 0.000411560179603056
10.6799999999998 0.000373264027534041
10.6899999999998 0.000334664075440996
10.6999999999998 0.00029581060743574
10.7099999999998 0.000256754000125449
10.7199999999998 0.000217544657272024
10.7299999999998 0.000178232945553915
10.7399999999998 0.000138869131146439
10.7499999999998 9.95033166967193e-05
10.7599999999998 6.01853787757444e-05
10.7699999999998 2.09649058873616e-05
10.7799999999998 -1.81088628871264e-05
10.7899999999998 -5.69870985319079e-05
10.7999999999998 -9.56214419523214e-05
10.8099999999998 -0.000133964062971636
10.8199999999998 -0.000171967718393726
10.8299999999998 -0.000209585809063855
10.8399999999998 -0.00024677607236811
10.8499999999998 -0.000283490120027084
10.8599999999998 -0.000319683601245528
10.8699999999998 -0.000355313022698709
10.8799999999998 -0.000390335799766559
10.8899999999998 -0.000424710306368844
10.8999999999998 -0.000458395923346253
10.9099999999998 -0.000491353085334423
10.9199999999998 -0.000523543326079269
10.9299999999998 -0.000554929322099044
10.9399999999998 -0.000585474934581354
10.9499999999998 -0.000615145250145122
10.9599999999998 -0.000643906619234987
10.9699999999998 -0.000671720654112875
10.9799999999998 -0.000698562306054374
10.9899999999998 -0.000724401983760126
10.9999999999998 -0.000749211497757891
11.0099999999998 -0.000772964089744391
11.0199999999998 -0.000795634459979633
11.0299999999998 -0.000817198792709197
11.0399999999998 -0.000837634779592808
11.0499999999998 -0.000856921641119825
11.0599999999998 -0.000875040145995421
11.0699999999998 -0.000891972628483832
11.0799999999998 -0.000907703003892881
11.0899999999998 -0.00092221678176409
11.0999999999998 -0.000935501076675129
11.1099999999998 -0.00094754461727536
11.1199999999998 -0.000958337753480105
11.1299999999998 -0.000967872461527353
11.1399999999998 -0.00097614234562745
11.1499999999998 -0.000983142639429546
11.1599999999998 -0.000988870204708516
11.1699999999998 -0.000993323528011727
11.1799999999998 -0.000996502715283516
11.1899999999998 -0.000998409484489393
11.1999999999998 -0.000999047156266731
11.2099999999998 -0.000998420642634794
11.2199999999998 -0.000996536433804875
11.2299999999998 -0.00099340258310352
11.2399999999998 -0.000989028689966679
11.2499999999998 -0.00098342588173966
11.2599999999998 -0.000976606793504984
11.2699999999998 -0.000968585546705244
11.2799999999998 -0.000959377726845096
11.2899999999998 -0.000949000361043
11.2999999999998 -0.000937471897100896
11.3099999999998 -0.00092481218842962
11.3199999999998 -0.000911042260379857
11.3299999999998 -0.00089618423633222
11.3399999999998 -0.000880261784919938
11.3499999999998 -0.00086329981588172
11.3599999999998 -0.000845324442589411
11.3699999999998 -0.000826362943278396
11.3799999999998 -0.000806443721026517
11.3899999999998 -0.000785596262530244
11.3999999999998 -0.0007638509474297
11.4099999999998 -0.00074123921663512
11.4199999999998 -0.000717793594404038
11.4299999999998 -0.000693547500549578
11.4399999999998 -0.000668535208645431
11.4499999999998 -0.000642791801300231
11.4599999999998 -0.000616353123592802
11.4699999999998 -0.00058925573520641
11.4799999999998 -0.000561536861552821
11.4899999999998 -0.000533234344092381
11.4999999999998 -0.00050438658997747
11.5099999999998 -0.000475032521130089
11.5199999999998 -0.00044521152284641
11.5299999999998 -0.000414963392011692
11.5399999999998 -0.000384328285003565
11.5499999999998 -0.00035334666535805
11.5599999999998 -0.0003220592512707
11.5699999999998 -0.000290506963003237
11.5799999999998 -0.000258730870265259
11.5899999999998 -0.000226772139639505
11.5999999999998 -0.000194671982118118
11.6099999999998 -0.000162471600817
11.6199999999998 -0.000130212138934117
11.6299999999998 -9.79346280171962e-05
11.6399999999998 -6.56799366052351e-05
11.6499999999998 -3.348871930742e-05
11.6599999999998 -1.40136638221108e-06
11.6699999999998 3.05420461217303e-05
11.6799999999998 6.23018056013361e-05
11.6899999999998 9.38386114545741e-05
11.6999999999998 0.000125113622894027
11.7099999999998 0.00015608850596292
11.7199999999998 0.000186725479697476
11.7299999999998 0.000216987361381003
11.7399999999998 0.000246837610836535
11.7499999999998 0.000276240373706411
11.7599999999998 0.000305160523668528
11.7699999999998 0.000333563703540704
11.7799999999998 0.000361416365226251
11.7899999999998 0.000388685808455418
11.7999999999998 0.000415340218279229
11.8099999999998 0.000441348701273977
11.8199999999998 0.00046668132040069
11.8299999999998 0.000491309128545068
11.8399999999998 0.000515204200607852
11.8499999999998 0.000538339664168755
11.8599999999998 0.000560689728683201
11.8699999999998 0.000582229713178955
11.8799999999998 0.00060293607242446
11.8899999999998 0.000622786421542853
11.8999999999998 0.000641759559047619
11.9099999999998 0.00065983548827808
11.9199999999998 0.000676995437214944
11.9299999999998 0.000693221876658407
11.9399999999998 0.00070849853675339
11.9499999999998 0.000722810421848765
11.9599999999998 0.000736143823679587
11.9699999999998 0.000748486332863608
11.9799999999998 0.000759826848705588
11.9899999999998 0.000770155587305133
11.9999999999998 0.000779464087966187
12.0099999999998 0.000787745218459989
12.0199999999998 0.000794993177015
12.0299999999998 0.000801203493973973
12.0399999999998 0.000806373031573689
12.0499999999998 0.000810499981918136
12.0599999999998 0.000813583863292891
12.0699999999998 0.000815625514837007
12.0799999999998 0.000816627089592465
12.0899999999998 0.000816592045955631
12.0999999999998 0.000815525137560909
12.1099999999998 0.000813432401634672
12.1199999999998 0.00081032114586907
12.1299999999998 0.000806199933883561
12.1399999999998 0.000801078569372492
12.1499999999998 0.000794968079092888
12.1599999999998 0.000787880694870428
12.1699999999998 0.000779829834262052
12.1799999999998 0.00077083008415309
12.1899999999998 0.000760897184964323
12.1999999999998 0.000750048024754471
12.2099999999998 0.000738300126486093
12.2199999999998 0.000725672386080596
12.2299999999998 0.000712184739546754
12.2399999999998 0.000697858120465698
12.2499999999998 0.000682714428759921
12.2599999999998 0.000666776498421705
12.2699999999998 0.000650068064240012
12.2799999999998 0.000632613712491771
12.2899999999998 0.000614438676816775
12.2999999999998 0.000595569229509482
12.3099999999998 0.000576032388523409
12.3199999999998 0.000555855885017776
12.3299999999998 0.000535068127753012
12.3399999999998 0.000513698165645348
12.3499999999998 0.000491775649083596
12.3599999999998 0.000469330790326007
12.3699999999998 0.000446394323168457
12.3799999999998 0.000422997462013633
12.3899999999998 0.000399171860439475
12.3999999999998 0.000374949569348031
12.4099999999998 0.000350362994765995
12.4199999999998 0.000325444855362654
12.4299999999998 0.000300228139747347
12.4399999999998 0.000274746063606199
12.4499999999998 0.000249032026736475
12.4599999999998 0.000223119570035758
12.4699999999998 0.000197042332502164
12.4799999999998 0.000170834008301292
12.4899999999998 0.000144528303954771
12.4999999999998 0.000118158895704687
12.5099999999998 9.17593871075887e-05
12.5199999999998 6.53632669110663e-05
12.5299999999998 3.90038672651812e-05
12.5399999999998 1.27143223203448e-05
12.5499999999998 -1.34724727375367e-05
12.5599999999998 -3.95239021644559e-05
12.5699999999998 -6.54076695997141e-05
12.5799999999998 -9.10918374229726e-05
12.5899999999998 -0.000116544865479221
12.5999999999998 -0.000141735649125977
12.6099999999998 -0.000166633556557545
12.6199999999998 -0.000191208465362316
12.6299999999998 -0.000215430798270437
12.6399999999998 -0.000239271558050149
12.6499999999998 -0.000262702361512677
12.6599999999998 -0.000285695472586933
12.6699999999998 -0.000308223834425647
12.6799999999998 -0.00033026110050673
12.6899999999998 -0.000351781664696648
12.6999999999998 -0.000372760690236389
12.7099999999998 -0.000393174137630943
12.7199999999998 -0.000412998791411805
12.7299999999998 -0.000432212285717449
12.7399999999998 -0.000450793128692548
12.7499999999998 -0.000468720725671753
12.7599999999998 -0.000485975401124327
12.7699999999998 -0.000502538419337567
12.7799999999998 -0.000518392003818632
12.7899999999998 -0.000533519355396173
12.7999999999998 -0.000547904669004873
12.8099999999998 -0.000561533149137797
12.8199999999998 -0.000574391023953152
12.8299999999998 -0.000586465558023957
12.8399999999998 -0.000597745063720844
12.8499999999998 -0.000608218911220027
12.8599999999998 -0.000617877537130377
12.8699999999998 -0.000626712451735285
12.8799999999998 -0.000634716244846936
12.8899999999998 -0.000641882590456069
12.8999999999998 -0.000648206249735277
12.9099999999998 -0.000653683072290751
12.9199999999998 -0.000658309996810792
12.9299999999998 -0.000662085050014318
12.9399999999998 -0.000665007344238559
12.9499999999998 -0.000667077073678104
12.9599999999998 -0.000668295509290251
12.9699999999998 -0.000668664992384906
12.9799999999998 -0.000668188926921394
12.9899999999998 -0.000666871770540051
12.9999999999998 -0.000664719024364269
13.0099999999998 -0.000661737221620589
13.0199999999998 -0.000657933915143796
13.0299999999998 -0.000653317663867968
13.0399999999998 -0.00064789801846748
13.0499999999998 -0.000641685506439323
13.0599999999998 -0.000634691617198835
13.0699999999998 -0.00062692878844871
13.0799999999998 -0.000618410397161727
13.0899999999998 -0.000609150547372807
13.0999999999998 -0.000599164155265655
13.1099999999998 -0.000588467135408715
13.1199999999998 -0.000577076228955665
13.1299999999998 -0.000565008978514589
13.1399999999998 -0.000552283702152224
13.1499999999998 -0.000538919466564531
13.1599999999998 -0.000524936059446805
13.1699999999998 -0.000510353857985212
13.1799999999998 -0.00049519395333336
13.1899999999998 -0.00047947815310164
13.1999999999998 -0.000463228860010652
13.2099999999998 -0.00044646904385846
13.2199999999998 -0.000429222211560478
13.2299999999998 -0.000411512375975141
13.2399999999998 -0.000393364023867637
13.2499999999998 -0.000374802083210319
13.2599999999998 -0.000355851889946886
13.2699999999998 -0.00033653915431163
13.2799999999998 -0.000316889926776154
13.2899999999998 -0.000296930563685383
13.2999999999998 -0.000276687692638663
13.3099999999998 -0.000256188177668025
13.3199999999998 -0.000235459084263311
13.3299999999998 -0.000214527644292453
13.3399999999998 -0.000193421220863999
13.3499999999998 -0.000172167273178294
13.3599999999998 -0.000150793321412969
13.3699999999998 -0.000129326911687925
13.3799999999998 -0.000107795581154347
13.3899999999998 -8.62268232519189e-05
13.3999999999998 -6.46480531777433e-05
13.4099999999998 -4.30865736099457e-05
13.4199999999998 -2.15695407284658e-05
13.4299999999998 -1.23930574753726e-07
13.4399999999998 2.12234942084823e-05
13.4499999999998 4.24462172166787e-05
13.4599999999998 6.35180006633231e-05
13.4699999999998 8.44129172601673e-05
13.4799999999998 0.000105105381560324
13.4899999999998 0.000125570180726871
13.4999999999998 0.000145782504690656
13.5099999999998 0.00016571797566191
13.5199999999998 0.000185352676961279
13.5299999999998 0.000204663181136887
13.5399999999998 0.000223626577335153
13.5499999999998 0.000242220497894078
13.5599999999998 0.000260423144129003
13.5699999999998 0.000278213311281852
13.5799999999998 0.000295570412606146
13.5899999999998 0.000312474502554216
13.5999999999998 0.000328906299070244
13.6099999999998 0.000344847204921847
13.6199999999998 0.000360279328074798
13.6299999999998 0.000375185501084747
13.6399999999998 0.000389549299484965
13.6499999999998 0.000403355059151462
13.6599999999998 0.000416587892628225
13.6699999999998 0.000429233704396786
13.6799999999998 0.000441279205075615
13.6899999999998 0.000452711924536411
13.6999999999998 0.000463520223925712
13.7099999999998 0.000473693306581745
13.7199999999998 0.000483221227837876
13.7299999999998 0.000492094903705548
13.7399999999998 0.000500306118431022
13.7499999999998 0.000507847530921774
13.7599999999998 0.00051471268003992
13.7699999999998 0.000520895988761528
13.7799999999998 0.000526392767518287
13.7899999999998 0.000531199215464227
13.7999999999998 0.000535312421513307
13.8099999999998 0.00053873036401838
13.8199999999997 0.000541451909299712
13.8299999999997 0.000543476809066968
13.8399999999997 0.000544805696745794
13.8499999999997 0.000545440082722629
13.8599999999997 0.000545382348524372
13.8699999999997 0.000544635739953442
13.8799999999997 0.000543204359204115
13.8899999999997 0.000541093155993982
13.8999999999997 0.000538307917756852
13.9099999999997 0.000534855258964528
13.9199999999997 0.000530742609683293
13.9299999999997 0.00052597820354462
13.9399999999997 0.000520571065466293
13.9499999999997 0.000514530999822583
13.9599999999997 0.000507868580715254
13.9699999999997 0.000500595127720137
13.9799999999997 0.00049272239865299
13.9899999999997 0.00048426307583165
13.9999999999997 0.000475230525885534
14.0099999999997 0.000465638779558102
14.0199999999997 0.000455502510790686
14.0299999999997 0.000444837015112752
14.0399999999997 0.00043365818736531
14.0499999999997 0.000421982490879879
14.0599999999997 0.000409826814616371
14.0699999999997 0.000397208740889301
14.0799999999997 0.000384146347554886
14.0899999999997 0.000370658186273882
14.0999999999997 0.000356763258689438
14.1099999999997 0.000342480991384212
14.1199999999997 0.000327831210016088
14.1299999999997 0.000312834112843845
14.1399999999997 0.000297510243770106
14.1499999999997 0.000281880464988202
14.1599999999997 0.000265965929298539
14.1699999999997 0.000249788052148692
14.1799999999997 0.000233368483444971
14.1899999999997 0.000216729079178962
14.1999999999997 0.000199891872911196
14.2099999999997 0.000182879047151379
14.2199999999997 0.000165712904674274
14.2299999999997 0.000148415839809358
14.2399999999997 0.000131010309741863
14.2499999999997 0.000113518805862263
14.2599999999997 9.59638252008713e-05
14.2699999999997 7.83678419837327e-05
14.2799999999997 6.075327934559e-05
14.2899999999997 4.31424812352525e-05
14.2999999999997 2.55576845482626e-05
14.3099999999997 8.02099152118917e-06
14.3199999999997 -9.44565757858089e-06
14.3299999999997 -2.68205114341666e-05
14.3399999999997 -4.40820342991861e-05
14.3499999999997 -6.12089322241658e-05
14.3599999999997 -7.81801788579029e-05
14.3699999999997 -9.4975040793098e-05
14.3799999999997 -0.000111573102426291
14.3899999999997 -0.000127954290302784
14.3999999999997 -0.000144098896918226
14.4099999999997 -0.000159987603949039
14.4199999999997 -0.000175601504885103
14.4299999999997 -0.000190922127038651
14.4399999999997 -0.00020593145290449
14.4499999999997 -0.000220611940847436
14.4599999999997 -0.000234946545093928
14.4699999999997 -0.000248918735003361
14.4799999999997 -0.000262512513604963
14.4899999999997 -0.000275712435376946
14.4999999999997 -0.000288503623240529
14.5099999999997 -0.000300871784761494
14.5199999999997 -0.000312803227538963
14.5299999999997 -0.000324284873765623
14.5399999999997 -0.000335304273944892
14.5499999999997 -0.000345849619751499
14.5599999999997 -0.000355909756023239
14.5699999999997 -0.00036547419187273
14.5799999999997 -0.00037453311090925
14.5899999999997 -0.000383077380561859
14.5999999999997 -0.000391098560496256
14.6099999999997 -0.000398588910118968
14.6199999999997 -0.00040554139516374
14.6299999999997 -0.000411949693356171
14.6399999999997 -0.00041780819915389
14.6499999999997 -0.000423112027560819
14.6599999999997 -0.000427857017125596
14.6699999999997 -0.000432039731807685
14.6799999999997 -0.000435657461857833
14.6899999999997 -0.000438708224082937
14.6999999999997 -0.000441190761054182
14.7099999999997 -0.000443104539408011
14.7199999999997 -0.000444449747248244
14.7299999999997 -0.000445227290659507
14.7399999999997 -0.000445438789344384
14.7499999999997 -0.000445086571399511
14.7599999999997 -0.000444173667249582
14.7699999999997 -0.000442703802763609
14.7799999999997 -0.000440681391586029
14.7899999999997 -0.000438111526728665
14.7999999999997 -0.000434999971493248
14.8099999999997 -0.000431353149838365
14.8199999999997 -0.000427178136394454
14.8299999999997 -0.000422482646529211
14.8399999999997 -0.000417275027356721
14.8499999999997 -0.000411564252068164
14.8599999999997 -0.000405359751186129
14.8699999999997 -0.000398671537729897
14.8799999999997 -0.000391510272044295
14.8899999999997 -0.000383887164272268
14.8999999999997 -0.000375813957542335
14.9099999999997 -0.00036730291057942
14.9199999999997 -0.000358366779760519
14.9299999999997 -0.000349018800637729
14.9399999999997 -0.000339272599088553
14.9499999999997 -0.000329142278597994
14.9599999999997 -0.000318642416260387
14.9699999999997 -0.000307787984310291
14.9799999999997 -0.000296594331337459
14.9899999999997 -0.000285077162245733
14.9999999999997 -0.000273252517416017
15.0099999999997 -0.000261136751302373
15.0199999999997 -0.000248746510591161
15.0299999999997 -0.000236098712006838
15.0399999999997 -0.000223210519824795
15.0499999999997 -0.000210099323139228
15.0599999999997 -0.000196782712927191
15.0699999999997 -0.000183278458945897
15.0799999999997 -0.000169604486498003
15.0899999999997 -0.000155778853098012
15.0999999999997 -0.00014181972507195
15.1099999999997 -0.00012774535412175
15.1199999999997 -0.000113574053885211
15.1299999999997 -9.93241765220583e-05
15.1399999999997 -8.50140893561023e-05
15.1499999999997 -7.06621516032854e-05
15.1599999999997 -5.62866912149803e-05
15.1699999999997 -4.19059818654979e-05
15.1799999999997 -2.75382201125308e-05
15.1899999999997 -1.32015027587111e-05
15.1999999999997 1.08619555785604e-06
15.2099999999997 1.53070445166859e-05
15.2199999999997 2.94433799846837e-05
15.2299999999997 4.347772557639e-05
15.2399999999997 5.73928138784727e-05
15.2499999999997 7.11716073133438e-05
15.2599999999997 8.47973186170318e-05
15.2699999999997 9.82534309072437e-05
15.2799999999997 0.00011152371731801
15.2899999999997 0.000124592260178115
15.2999999999997 0.000137443469711101
15.3099999999997 0.000150062102235414
15.3199999999997 0.000162433277843905
15.3299999999997 0.000174542497542792
15.3399999999997 0.000186375659830808
15.3499999999997 0.000197919076700188
15.3599999999997 0.000209159489038473
15.3699999999997 0.000220084081428708
15.3799999999997 0.000230680496309742
15.3899999999997 0.000240936847497108
15.3999999999997 0.00025084173304633
15.4099999999997 0.00026038424744558
15.4199999999997 0.000269553993125176
15.4299999999997 0.00027834109127258
15.4399999999997 0.000286736191942486
15.4499999999997 0.000294730483452417
15.4599999999997 0.000302315701055311
15.4699999999997 0.000309484134881529
15.4799999999997 0.000316228637143604
15.4899999999997 0.000322542628598155
15.4999999999997 0.000328420104260284
15.5099999999997 0.000333855638366791
15.5199999999997 0.000338844388585564
15.5299999999997 0.000343382099469487
15.5399999999997 0.000347465105154237
15.5499999999997 0.000351090331464411
15.5599999999997 0.000354255296762022
15.5699999999997 0.000356958112552326
15.5799999999997 0.000359197483175658
15.5899999999997 0.000360972704752772
15.5999999999997 0.000362283663393191
15.6099999999997 0.000363130832674143
15.6199999999997 0.000363515270399348
15.6299999999997 0.000363438614648972
15.6399999999997 0.000362903079134718
15.6499999999997 0.000361911447877712
15.6599999999997 0.000360467069232354
15.6699999999997 0.000358573849287954
15.6799999999997 0.000356236244694802
15.6899999999997 0.000353459254988175
15.6999999999997 0.000350248414536189
15.7099999999997 0.000346609784348866
15.7199999999997 0.000342549944246512
15.7299999999997 0.000338075986578952
15.7399999999997 0.000333195478828543
15.7499999999997 0.00032791631303948
15.7599999999997 0.000322246982511102
15.7699999999997 0.000316196435345503
15.7799999999997 0.000309774060945842
15.7899999999997 0.000302989676035517
15.7999999999997 0.000295853510215451
15.8099999999997 0.000288376191077714
15.8199999999997 0.000280568719914616
15.8299999999997 0.000272442389506248
15.8399999999997 0.000264008947049981
15.8499999999997 0.000255280468634137
15.8599999999997 0.000246269344619837
15.8699999999997 0.000236988263689352
15.8799999999997 0.000227450196105225
15.8899999999997 0.000217668376434334
15.8999999999997 0.000207656285872788
15.9099999999997 0.000197427634254188
15.9199999999997 0.000186996341797874
15.9299999999997 0.000176376520640289
15.9399999999997 0.000165582456185285
15.9499999999997 0.000154628588304979
15.9599999999997 0.00014352949242031
15.9699999999997 0.00013229986048885
15.9799999999997 0.000120954481926492
15.9899999999997 0.000109508224488883
15.9999999999997 9.79760151380422e-05
16.0099999999997 8.6372820919174e-05
16.0199999999997 7.47136298723655e-05
16.0299999999997 6.30134320035475e-05
16.0399999999997 5.12872003388595e-05
16.0499999999997 3.95498720862197e-05
16.0599999999997 2.78163299275566e-05
16.0699999999997 1.61013834649946e-05
16.0799999999997 4.41975084383428e-06
16.0899999999997 -7.21395942522759e-06
16.0999999999997 -1.87852664223604e-05
16.1099999999997 -3.02798345317034e-05
16.1199999999997 -4.16834908900279e-05
16.1299999999997 -5.29822425498677e-05
16.1399999999997 -6.41622933369173e-05
16.1499999999997 -7.5210060381574e-05
16.1599999999997 -8.61121903053639e-05
16.1699999999997 -9.68555750432135e-05
16.1799999999997 -0.000107427367283297
16.1899999999997 -0.000117814995506663
16.1999999999997 -0.000128006178609458
16.2099999999997 -0.000137988940091125
16.2199999999997 -0.000147751621792717
16.2299999999997 -0.000157282897169914
16.2399999999997 -0.000166571784084938
16.2499999999997 -0.000175607657107521
16.2599999999997 -0.000184380259307848
16.2699999999997 -0.000192879713527718
16.2799999999997 -0.000201096533121402
16.2899999999997 -0.00020902163215382
16.2999999999997 -0.000216646335045607
16.3099999999998 -0.000223962385655478
16.3199999999998 -0.000230961955790988
16.3299999999998 -0.000237637653139606
16.3399999999998 -0.000243982528612737
16.3499999999998 -0.000249990083096183
16.3599999999998 -0.00025565427360128
16.3699999999998 -0.000260969518811745
16.3799999999998 -0.00026593070402207
16.3899999999998 -0.000270533185464141
16.3999999999998 -0.000274772794019525
16.4099999999998 -0.000278645838315718
16.4199999999998 -0.000282149107205478
16.4299999999998 -0.000285279871688387
16.4399999999998 -0.000288035886087828
16.4499999999998 -0.000290415388632699
16.4599999999998 -0.000292417101535751
16.4699999999998 -0.000294040230405254
16.4799999999998 -0.000295284463052803
16.4899999999998 -0.000296149967702934
16.4999999999998 -0.000296637390611438
16.5099999999998 -0.000296747853100793
16.5199999999998 -0.000296482948023044
16.5299999999998 -0.000295844735663009
16.5399999999998 -0.000294835739098396
16.5499999999998 -0.000293458939039077
16.5599999999998 -0.000291717768177102
16.5699999999998 -0.000289616105095518
16.5799999999998 -0.000287158267815051
16.5899999999998 -0.000284349007121042
16.5999999999998 -0.000281193499954307
16.6099999999998 -0.000277697343502155
16.6199999999998 -0.000273866551691125
16.6299999999998 -0.000269707422278754
16.6399999999998 -0.000265226672964003
16.6499999999998 -0.000260431437717867
16.6599999999998 -0.000255329215640533
16.6699999999998 -0.000249927859731138
16.6799999999998 -0.000244235565273478
16.6899999999998 -0.000238260857852414
16.6999999999998 -0.000232012581016237
16.7099999999998 -0.000225499835146731
16.7199999999998 -0.000218732042922583
16.7299999999998 -0.000211718937306798
16.7399999999998 -0.000204470513267915
16.7499999999998 -0.000196997015188277
16.7599999999998 -0.000189308923465667
16.7699999999998 -0.000181416940599212
16.7799999999998 -0.000173331976905628
16.7899999999998 -0.000165065135949577
16.7999999999998 -0.000156627699742412
16.8099999999998 -0.000148031113748886
16.8199999999998 -0.000139286971733351
16.8299999999998 -0.000130407000472676
16.8399999999998 -0.000121403044360502
16.8499999999998 -0.000112287049925779
16.8599999999998 -0.000103071050287724
16.8699999999998 -9.37671495684927e-05
16.8799999999998 -8.43875072845467e-05
16.8899999999998 -7.49443227371702e-05
16.8999999999998 -6.54498194225108e-05
16.9099999999998 -5.59162294811001e-05
16.9199999999998 -4.63557782066174e-05
16.9299999999998 -3.67806686334613e-05
16.9399999999998 -2.72030662224278e-05
16.9499999999999 -1.76350836635117e-05
16.9599999999999 -8.08876581463694e-06
16.9699999999999 1.42392520517798e-06
16.9799999999999 1.08911247500326e-05
16.9899999999999 2.03010801908268e-05
16.9999999999999 2.96421652508952e-05
17.0099999999999 3.8902894116356e-05
17.0199999999999 4.80719353043131e-05
17.0299999999999 5.71381252725339e-05
17.0399999999999 6.60904817545119e-05
17.0499999999999 7.49182168043149e-05
17.0599999999999 8.361074953607e-05
17.0699999999999 9.21577185433376e-05
17.0799999999999 0.000100548993984163
17.0899999999999 0.00010877468931804
17.0999999999999 0.000116825172681543
17.1099999999999 0.000124691077889941
17.1199999999999 0.000132363315052417
17.1299999999999 0.00013983308078846
17.1399999999999 0.000147091868039235
17.1499999999999 0.000154131475454973
17.1599999999999 0.000160944016354576
17.1699999999999 0.000167521927246503
17.1799999999999 0.00017385797590212
17.1899999999999 0.000179945268973432
17.1999999999999 0.000185777259147661
17.2099999999999 0.000191347751831784
17.2199999999999 0.000196650911360754
17.2299999999999 0.000201681266723796
17.2399999999999 0.000206433716803793
17.2499999999999 0.000210903535125419
17.2599999999999 0.000215086374108345
17.2699999999999 0.000218978268822478
17.2799999999999 0.000222575640242907
17.2899999999999 0.000225875298002796
17.2999999999999 0.000228874442643263
17.3099999999999 0.000231570667363172
17.3199999999999 0.00023396195932822
17.3299999999999 0.000236046700264384
17.3399999999999 0.000237823666777077
17.3499999999999 0.000239292030094811
17.3599999999999 0.000240451355318456
17.3699999999999 0.000241301600180291
17.3799999999999 0.000241843113317976
17.3899999999999 0.000242076632069718
17.3999999999999 0.000242003279798261
17.4099999999999 0.000241624562753162
17.4199999999999 0.000240942366483277
17.4299999999999 0.00023995895181518
17.4399999999999 0.000238676950419124
17.4499999999999 0.000237099359994379
17.4599999999999 0.000235229539124308
17.4699999999999 0.000233071201887964
17.4799999999999 0.000230628412392831
17.4899999999999 0.000227905579576868
17.4999999999999 0.000224907453120876
17.5099999999999 0.00022163908604422
17.5199999999999 0.000218105767557617
17.5299999999999 0.000214313177442644
17.5399999999999 0.00021026729736693
17.5499999999999 0.0002059744018726
17.5599999999999 0.000201441049045591
17.5699999999999 0.000196674070877678
17.5799999999999 0.000191680563333645
17.59 0.000186467868638302
17.6 0.000181043526729359
17.61 0.000175415375467569
17.62 0.000169591470738311
17.63 0.000163580076640777
17.64 0.00015738965482331
17.65 0.000151028853308551
17.66 0.000144506494970464
17.67 0.000137831565750479
17.68 0.000131013202666315
17.69 0.000124060681650378
17.7 0.000116983405246071
17.71 0.00010979089018569
17.72 0.000102492754870739
17.73 9.5098706774054e-05
17.74 8.76185297820071e-05
17.75 8.00620714944532e-05
17.76 7.24392304996463e-05
17.77 6.47599436410384e-05
17.78 5.70341732925301e-05
17.79 4.92718946586547e-05
17.8 4.14830831158681e-05
17.81 3.36777016109797e-05
17.82 2.58656881325781e-05
17.83 1.8056943271003e-05
17.84 1.02613178823891e-05
17.85 2.48860087187034e-06
17.86 -5.25149288903318e-06
17.87 -1.29493344961474e-05
17.88 -2.05953927830916e-05
17.89 -2.81802459142444e-05
17.9 -3.56945927863623e-05
17.91 -4.31292642253166e-05
17.92 -5.04752339646898e-05
17.93 -5.77236293933911e-05
17.94 -6.48657420596879e-05
17.95 -7.18930379195649e-05
17.96 -7.87971673175307e-05
17.97 -8.5569974688594e-05
17.98 -9.22035079703068e-05
17.99 -9.8690027714406e-05
18 -0.000105022015887863
18.01 -0.000111192184353053
18.02 -0.000117193483019977
18.03 -0.000123019107659289
18.04 -0.000128662507368214
18.05 -0.000134117391682432
18.06 -0.000139377737326168
18.07 -0.000144437794593653
18.08 -0.000149292093355541
18.09 -0.000153935448684492
18.1 -0.000158362966094545
18.11 -0.000162570046389495
18.12 -0.000166552390115936
18.13 -0.000170306001617235
18.14 -0.000173827192685178
18.15 -0.000177112585806588
18.16 -0.000180159117002739
18.17 -0.000182964038259944
18.18 -0.000185524919550225
18.19 -0.000187839650441539
18.2 -0.000189906441326759
18.21 -0.000191723824173938
18.22 -0.000193290652894754
18.2300000000001 -0.000194606103341759
18.2400000000001 -0.000195669672878467
18.2500000000001 -0.000196481179548034
18.2600000000001 -0.000197040760844361
18.2700000000001 -0.000197348872090255
18.2800000000001 -0.000197406284428341
18.2900000000001 -0.000197214082431657
18.3000000000001 -0.000196773661342609
18.3100000000001 -0.000196086723951449
18.3200000000001 -0.000195155277129272
18.3300000000001 -0.000193981628036892
18.3400000000001 -0.00019256838004215
18.3500000000001 -0.000190918428399485
18.3600000000001 -0.00018903495578914
18.3700000000001 -0.000186921427911147
18.3800000000001 -0.000184581589575068
18.3900000000001 -0.000182019462467739
18.4000000000001 -0.000179239246128863
18.4100000000001 -0.000176245437079746
18.4200000000001 -0.000173042800039701
18.4300000000001 -0.000169636341909171
18.4400000000001 -0.000166031304281529
18.4500000000001 -0.000162233155699388
18.4600000000001 -0.00015824758366545
18.4700000000001 -0.000154080486418322
18.4800000000001 -0.000149737932418223
18.4900000000001 -0.000145226204681209
18.5000000000001 -0.000140551790402309
18.5100000000001 -0.000135721350306954
18.5200000000001 -0.000130741710229271
18.5300000000001 -0.000125619852170882
18.5400000000001 -0.000120362905025091
18.5500000000001 -0.000114978135060073
18.5600000000001 -0.000109472936215084
18.5700000000001 -0.00010385482024506
18.5800000000001 -9.81314067394107e-05
18.5900000000001 -9.23104130358379e-05
18.6000000000001 -8.63996440470726e-05
18.6100000000001 -8.0406982016876e-05
18.6200000000001 -7.4340376220504e-05
18.6300000000001 -6.82078326242989e-05
18.6400000000001 -6.20174035185766e-05
18.6500000000001 -5.57771771377054e-05
18.6600000000001 -4.94952672810228e-05
18.6700000000001 -4.31798029480541e-05
18.6800000000001 -3.68389180013242e-05
18.6900000000001 -3.04807408698945e-05
18.7000000000001 -2.41133843066018e-05
18.7100000000001 -1.77449352118121e-05
18.7200000000001 -1.13834445363402e-05
18.7300000000001 -5.03691727599343e-06
18.7400000000001 1.28669742995898e-06
18.7500000000001 7.57951608433873e-06
18.7600000000001 1.38337304897279e-05
18.7700000000001 2.00416172666491e-05
18.7800000000001 2.61955472207966e-05
18.7900000000001 3.22879945481819e-05
18.8000000000001 3.83115458672909e-05
18.8100000000001 4.42589090675972e-05
18.8200000000001 5.01229219641109e-05
18.8300000000001 5.58965607478972e-05
18.8400000000001 6.15729482227965e-05
18.8500000000001 6.71453618189511e-05
18.8600000000001 7.26072413740148e-05
18.8700000000002 7.7952196673269e-05
18.8800000000002 8.31740147402523e-05
18.8900000000002 8.82666668697166e-05
18.9000000000002 9.32243153950786e-05
18.9100000000002 9.80413201846984e-05
18.9200000000002 0.000102712244856848
18.9300000000002 0.000107231862709263
18.9400000000002 0.000111595162356403
18.9500000000002 0.000115797353068666
18.9600000000002 0.000119833869808187
18.9700000000002 0.000123700377956294
18.9800000000002 0.000127392777728043
18.9900000000002 0.000130907208269752
19.0000000000002 0.000134240051435817
19.0100000000002 0.000137387935241551
19.0200000000002 0.00014034773698921
19.0300000000002 0.000143116586064803
19.0400000000002 0.000145691866403723
19.0500000000002 0.000148071218623661
19.0600000000002 0.000150252541823748
19.0700000000002 0.000152233995049248
19.0800000000002 0.000154013998424881
19.0900000000002 0.000155591233973516
19.1000000000002 0.000156964646019808
19.1100000000002 0.000158133441351855
19.1200000000002 0.000159097089020378
19.1300000000002 0.000159855319809487
19.1400000000002 0.000160408125381852
19.1500000000002 0.000160755757101743
19.1600000000002 0.000160898724540136
19.1700000000002 0.000160837793667034
19.1800000000002 0.000160573984737301
19.1900000000002 0.000160108569878045
19.2000000000002 0.000159443070388051
19.2100000000002 0.000158579253763777
19.2200000000002 0.000157519130473308
19.2300000000002 0.000156264950512201
19.2400000000002 0.00015481919979993
19.2500000000002 0.000153184596528792
19.2600000000002 0.000151364087703279
19.2700000000002 0.000149360846449397
19.2800000000002 0.000147178240141636
19.2900000000002 0.000144819805392293
19.3000000000002 0.000142289332193134
19.3100000000002 0.000139590810633388
19.3200000000002 0.000136728424896922
19.3300000000002 0.000133706547047066
19.3400000000002 0.000130529730607248
19.3500000000002 0.000127202703945865
19.3600000000002 0.000123730358258664
19.3700000000002 0.000120117717365813
19.3800000000002 0.00011637000055443
19.3900000000002 0.000112492571461101
19.4000000000002 0.000108490931501085
19.4100000000002 0.000104370712750969
19.4200000000002 0.000100137670502039
19.4300000000002 9.57976755874988e-05
19.4400000000002 9.13567065396012e-05
19.4500000000002 8.68208416112591e-05
19.4600000000002 8.21962506861712e-05
19.4700000000002 7.74891870960609e-05
19.4800000000002 7.27059793605423e-05
19.4900000000002 6.78530228634217e-05
19.5000000000002 6.29367714782225e-05
19.5100000000003 5.79637291550611e-05
19.5200000000003 5.29404414805778e-05
19.5300000000003 4.78734872223441e-05
19.5400000000003 4.27694698689516e-05
19.5500000000003 3.763500917684e-05
19.5600000000003 3.24767327347129e-05
19.5700000000003 2.73012675563435e-05
19.5800000000003 2.21152317123977e-05
19.5900000000003 1.69252260117349e-05
19.6000000000003 1.17378257426155e-05
19.6100000000003 6.55957248400141e-06
19.6200000000003 1.39696599701296e-06
19.6300000000003 -3.74354379347316e-06
19.6400000000003 -8.85556471740444e-06
19.6500000000003 -1.39327701687081e-05
19.6600000000003 -1.89689067956227e-05
19.6700000000003 -2.3957802063027e-05
19.6800000000003 -2.88933716778071e-05
19.6900000000003 -3.37696268684445e-05
19.7000000000003 -3.85806815103429e-05
19.7100000000003 -4.33207590885622e-05
19.7200000000003 -4.7984199489896e-05
19.7300000000003 -5.25654656165312e-05
19.7400000000003 -5.70591498137036e-05
19.7500000000003 -6.14599801041074e-05
19.7600000000003 -6.57628262220793e-05
19.7700000000003 -6.99627054408245e-05
19.7800000000003 -7.40547881860271e-05
19.7900000000003 -7.80344034307656e-05
19.8000000000003 -8.18970438646185e-05
19.8100000000003 -8.56383708319395e-05
19.8200000000003 -8.92542190343569e-05
19.8300000000003 -9.27406009925156e-05
19.8400000000003 -9.60937112625283e-05
19.8500000000003 -9.93099304029623e-05
19.8600000000003 -0.000102385828688503
19.8700000000003 -0.000105318169566793
19.8800000000003 -0.000108103912855272
19.8900000000003 -0.000110740217675206
19.9000000000003 -0.000113224445120443
19.9100000000003 -0.000115554160658764
19.9200000000003 -0.000117727136264073
19.9300000000003 -0.000119741352278032
19.9400000000003 -0.000121594999000077
19.9500000000003 -0.000123286478005144
19.9600000000003 -0.000124814403188784
19.9700000000003 -0.000126177601553098
19.9800000000003 -0.000127375113687212
19.9900000000003 -0.000128406193994064
20.0000000000003 -0.000129270310658016
20.0100000000003 -0.000129967145335412
20.0200000000003 -0.000130496592578733
20.0300000000003 -0.000130858758996913
20.0400000000003 -0.000131053962154922
20.0500000000003 -0.000131082729216439
20.0600000000003 -0.000130945795334215
20.0700000000003 -0.000130644101793956
20.0800000000003 -0.000130178793919129
20.0900000000003 -0.000129551218746715
20.1000000000003 -0.000128762922488146
20.1100000000003 -0.000127815647797177
20.1200000000003 -0.000126711330880739
20.1300000000003 -0.000125452098518227
20.1400000000003 -0.000124040265120973
20.1500000000004 -0.00012247833013151
20.1600000000004 -0.000120768976567812
20.1700000000004 -0.000118914999710049
20.1800000000004 -0.000116919402216504
20.1900000000004 -0.000114785359502098
20.2000000000004 -0.000112516207363871
20.2100000000004 -0.000110115436998156
20.2200000000004 -0.000107586689847971
20.2300000000004 -0.00010493375228748
20.2400000000004 -0.000102160550150589
20.2500000000004 -9.92711224671312e-05
20.2600000000004 -9.62696500817623e-05
20.2700000000004 -9.31604482110503e-05
20.2800000000004 -8.99479468489516e-05
20.2900000000004 -8.66366851429855e-05
20.3000000000004 -8.32313054372806e-05
20.3100000000004 -7.97365470993919e-05
20.3200000000004 -7.61572401905647e-05
20.3300000000004 -7.24982990141284e-05
20.3400000000004 -6.87647155648821e-05
20.3500000000004 -6.49615528963233e-05
20.3600000000004 -6.10939384193158e-05
20.3700000000004 -5.71670571440433e-05
20.3800000000004 -5.31861448759447e-05
20.3900000000004 -4.91564813757578e-05
20.4000000000004 -4.50833834933321e-05
20.4100000000004 -4.09721982846204e-05
20.4200000000004 -3.68282961210325e-05
20.4300000000004 -3.26570638002265e-05
20.4400000000004 -2.84638976672418e-05
20.4500000000004 -2.42541967547885e-05
20.4600000000004 -2.0033355951413e-05
20.4700000000004 -1.5806759206121e-05
20.4800000000004 -1.15797727779786e-05
20.4900000000004 -7.35773853903639e-06
20.5000000000004 -3.14596733888618e-06
20.5100000000004 1.05026756105539e-06
20.5200000000004 5.2257369754957e-06
20.5300000000004 9.37526219762155e-06
20.5400000000004 1.34937213080996e-05
20.5500000000004 1.75760553832051e-05
20.5600000000004 2.16172745946684e-05
20.5700000000004 2.56124641940092e-05
20.5800000000004 2.9556790374347e-05
20.5900000000004 3.34455060028219e-05
20.6000000000004 3.72739562169823e-05
20.6100000000004 4.10375838786684e-05
20.6200000000004 4.47319348791976e-05
20.6300000000004 4.8352663289794e-05
20.6400000000004 5.18955363514879e-05
20.6500000000004 5.53564392989092e-05
20.6600000000004 5.87313800126096e-05
20.6700000000004 6.2016493494824e-05
20.6800000000004 6.52080461643742e-05
20.6900000000004 6.83024399649858e-05
20.7000000000004 7.12962162836003e-05
20.7100000000004 7.41860596743443e-05
20.7200000000004 7.69688013843807e-05
20.7300000000004 7.96414226780877e-05
20.7400000000004 8.22010579563222e-05
20.7500000000004 8.46449976677753e-05
20.7600000000004 8.69706910097212e-05
20.7700000000004 8.91757484157308e-05
20.7800000000004 9.12579438282193e-05
20.7900000000005 9.32152167539697e-05
20.8000000000005 9.50456741010626e-05
20.8100000000005 9.6747591795947e-05
20.8200000000005 9.83194161796569e-05
20.8300000000005 9.97597651824875e-05
20.8400000000005 0.000101067429276741
20.8500000000005 0.000102241372209384
20.8600000000005 0.000103280731518638
20.8700000000005 0.000104184818799085
20.8800000000005 0.000104953119781424
20.8900000000005 0.000105585294180307
20.9000000000005 0.000106081175324381
20.9100000000005 0.000106440769570435
20.9200000000005 0.000106664255503957
20.9300000000005 0.000106751982928916
20.9400000000005 0.00010670447165018
20.9500000000005 0.000106522410052777
20.9600000000005 0.000106206653483313
20.9700000000005 0.000105758222440515
20.9800000000005 0.000105178300584513
20.9900000000005 0.000104468232579035
21.0000000000005 0.000103629521789005
21.0100000000005 0.000102663827872589
21.0200000000005 0.00010157296434226
21.0300000000005 0.000100358896254364
21.0400000000005 9.90237384184577e-05
21.0500000000005 9.75697303998404e-05
21.0600000000005 9.59992315655516e-05
21.0700000000005 9.43147653378306e-05
21.0800000000005 9.25189875501657e-05
21.0900000000005 9.06146824571283e-05
21.1000000000005 8.86047586031081e-05
21.1100000000005 8.64922445555236e-05
21.1200000000005 8.42802845082531e-05
21.1300000000005 8.19721304245578e-05
21.1400000000005 7.95711227941133e-05
21.1500000000005 7.70807301069743e-05
21.1600000000005 7.45045162887325e-05
21.1700000000005 7.18461363062874e-05
21.1800000000005 6.91093314256162e-05
21.1900000000005 6.62979242580273e-05
21.2000000000005 6.34158136601908e-05
21.2100000000005 6.04669695236705e-05
21.2200000000005 5.74554274761751e-05
21.2300000000005 5.43852835101166e-05
21.2400000000005 5.12606885505568e-05
21.2500000000005 4.80858429727575e-05
21.2600000000005 4.48649910783849e-05
21.2700000000005 4.16024155388188e-05
21.2800000000005 3.83024318135781e-05
21.2900000000005 3.49693825515926e-05
21.3000000000005 3.16076319828969e-05
21.3100000000005 2.82215603081551e-05
21.3200000000005 2.48155580933184e-05
21.3300000000005 2.13940206766564e-05
21.3400000000005 1.79613425952537e-05
21.3500000000005 1.45219120380454e-05
21.3600000000005 1.10801053323242e-05
21.3700000000005 7.64028147058629e-06
21.3800000000005 4.20677668451274e-06
21.3900000000005 7.83899072711053e-07
21.4000000000005 -2.62407671115906e-06
21.4100000000005 -6.01291470347833e-06
21.4200000000005 -9.37842280259691e-06
21.4300000000006 -1.27164578614149e-05
21.4400000000006 -1.60229306945262e-05
21.4500000000006 -1.92938109939844e-05
21.4600000000006 -2.25251321478999e-05
21.4700000000006 -2.57129959562092e-05
21.4800000000006 -2.8853577238142e-05
21.4900000000006 -3.19431283260604e-05
21.5000000000006 -3.49779834405057e-05
21.5100000000006 -3.79545629414855e-05
21.5200000000006 -4.08693774511815e-05
21.5300000000006 -4.37190318434933e-05
21.5400000000006 -4.65002290959582e-05
21.5500000000006 -4.92097739997259e-05
21.5600000000006 -5.18445767240138e-05
21.5700000000006 -5.44016562305932e-05
21.5800000000006 -5.68781435349655e-05
21.5900000000006 -5.92712848108822e-05
21.6000000000006 -6.1578444334948e-05
21.6100000000006 -6.37971072683553e-05
21.6200000000006 -6.59248822729711e-05
21.6300000000006 -6.79595039592719e-05
21.6400000000006 -6.98988351638062e-05
21.6500000000006 -7.17408690541074e-05
21.6600000000006 -7.34837310592297e-05
21.6700000000006 -7.51256806242705e-05
21.6800000000006 -7.66651127875097e-05
21.6900000000006 -7.81005595790251e-05
21.7000000000006 -7.94306912398612e-05
21.7100000000006 -8.06543172610939e-05
21.7200000000006 -8.17703872423507e-05
21.7300000000006 -8.27779915695995e-05
21.7400000000006 -8.36763619180614e-05
21.7500000000006 -8.44648715597963e-05
21.7600000000006 -8.51430355007051e-05
21.7700000000006 -8.57105104416448e-05
21.7800000000006 -8.61670945584891e-05
21.7900000000006 -8.65127271056826e-05
21.8000000000006 -8.6747487844998e-05
21.8100000000006 -8.68715963015787e-05
21.8200000000006 -8.68854108497832e-05
21.8300000000006 -8.6789427631904e-05
21.8400000000006 -8.65842793135861e-05
21.8500000000006 -8.62707336808482e-05
21.8600000000006 -8.58496920852894e-05
21.8700000000006 -8.53221877468368e-05
21.8800000000006 -8.46893839283287e-05
21.8900000000006 -8.3952572005619e-05
21.9000000000006 -8.31131694763035e-05
21.9100000000006 -8.21727179940996e-05
21.9200000000006 -8.11328816277261e-05
21.9300000000006 -7.99954458799011e-05
21.9400000000006 -7.87622680294679e-05
21.9500000000006 -7.74353480965704e-05
21.9600000000006 -7.60167975243658e-05
21.9700000000006 -7.45088337879745e-05
21.9800000000006 -7.29137770859865e-05
21.9900000000006 -7.12340469193063e-05
22.0000000000006 -6.94721585620026e-05
22.0100000000006 -6.76307194289456e-05
22.0200000000006 -6.57124124588267e-05
22.0300000000006 -6.37200136333051e-05
22.0400000000006 -6.1656387362588e-05
22.0500000000006 -5.95244738624397e-05
22.0600000000006 -5.73272854049951e-05
22.0700000000007 -5.5067902357012e-05
22.0800000000007 -5.27494690790553e-05
22.0900000000007 -5.0375189723396e-05
22.1000000000007 -4.79483239527667e-05
22.1100000000007 -4.5472182594667e-05
22.1200000000007 -4.29501232421561e-05
22.1300000000007 -4.03855458100005e-05
22.1400000000007 -3.77818880539324e-05
22.1500000000007 -3.51426210600699e-05
22.1600000000007 -3.24712447111605e-05
22.1700000000007 -2.97712831360385e-05
22.1800000000007 -2.70462801484971e-05
22.1900000000007 -2.42997946816707e-05
22.2000000000007 -2.15353962239033e-05
22.2100000000007 -1.87566602620202e-05
22.2200000000007 -1.59671637378122e-05
22.2300000000007 -1.31704805235129e-05
22.2400000000007 -1.03701769219492e-05
22.2500000000007 -7.56980719699162e-06
22.2600000000007 -4.77290913984015e-06
22.2700000000007 -1.98299967661828e-06
22.2800000000007 7.96429477346643e-07
22.2900000000007 3.56191611125199e-06
22.3000000000007 6.31003175537634e-06
22.3100000000007 9.03738585534724e-06
22.3200000000007 1.17406298791333e-05
22.3300000000007 1.44164613518656e-05
22.3400000000007 1.70616278137331e-05
22.3500000000007 1.96729306962883e-05
22.3600000000007 2.2247229112639e-05
22.3700000000007 2.47814435571564e-05
22.3800000000007 2.72725595104041e-05
22.3900000000007 2.97176309452002e-05
22.4000000000007 3.21137837298374e-05
22.4100000000007 3.44582189246163e-05
22.4200000000007 3.67482159680504e-05
22.4300000000007 3.89811357491804e-05
22.4400000000007 4.11544235626936e-05
22.4500000000007 4.3265611943807e-05
22.4600000000007 4.53123233794902e-05
22.4700000000007 4.72922728935318e-05
22.4800000000007 4.92032705026496e-05
22.4900000000007 5.10432235411639e-05
22.5000000000007 5.28101388519077e-05
22.5100000000007 5.45021248412256e-05
22.5200000000007 5.61173933961176e-05
22.5300000000007 5.76542616617382e-05
22.5400000000007 5.91111536776775e-05
22.5500000000007 6.04866018716173e-05
22.5600000000007 6.1779248409155e-05
22.5700000000007 6.29878463987786e-05
22.5800000000007 6.41112609511596e-05
22.5900000000007 6.51484700921384e-05
22.6000000000007 6.60985655289485e-05
22.6100000000007 6.69607532694482e-05
22.6200000000007 6.77343540952714e-05
22.6300000000007 6.84188038896604e-05
22.6400000000007 6.90136538077484e-05
22.6500000000007 6.95185703142477e-05
22.6600000000007 6.99333350710681e-05
22.6700000000007 7.02578446807672e-05
22.6800000000007 7.049211028709e-05
22.6900000000007 7.06362570341385e-05
22.7000000000007 7.06905233860301e-05
22.7100000000008 7.06552603092986e-05
22.7200000000008 7.0530930320807e-05
22.7300000000008 7.0318106404659e-05
22.7400000000008 7.00174708026647e-05
22.7500000000008 6.96298136846368e-05
22.7600000000008 6.91560317077353e-05
22.7700000000008 6.85971264794991e-05
22.7800000000008 6.79542029499467e-05
22.7900000000008 6.72284677813362e-05
22.8000000000008 6.64212277997778e-05
22.8100000000008 6.55338887855908e-05
22.8200000000008 6.45679371356373e-05
22.8300000000008 6.35249428236355e-05
22.8400000000008 6.24065819481899e-05
22.8500000000008 6.12146180750125e-05
22.8600000000008 5.99508995906215e-05
22.8700000000008 5.86173569624185e-05
22.8800000000008 5.7215999908966e-05
22.8900000000008 5.57489144843476e-05
22.9000000000008 5.42182581751611e-05
22.9100000000008 5.26262471924626e-05
22.9200000000008 5.09751814340625e-05
22.9300000000008 4.92674237989612e-05
22.9400000000008 4.75053972537704e-05
22.9500000000008 4.56915816787013e-05
22.9600000000008 4.38285105783846e-05
22.9700000000008 4.19187676985659e-05
22.9800000000008 3.9964983571303e-05
22.9900000000008 3.79698320028667e-05
23.0000000000008 3.59360265143669e-05
23.0100000000008 3.38663167429597e-05
23.0200000000008 3.17634848102843e-05
23.0300000000008 2.9630341664078e-05
23.0400000000008 2.74697233985279e-05
23.0500000000008 2.52844875586128e-05
23.0600000000008 2.30775094335669e-05
23.0700000000008 2.08516783444451e-05
23.0800000000008 1.86098939306852e-05
23.0900000000008 1.63550624405108e-05
23.1000000000008 1.40900930299253e-05
23.1100000000008 1.1817894075025e-05
23.1200000000008 9.5413695022661e-06
23.1300000000008 7.26341514130013e-06
23.1400000000008 4.9869151049045e-06
23.1500000000008 2.71473820048072e-06
23.1600000000008 4.49734377539264e-07
23.1700000000008 -1.8052687844987e-06
23.1800000000008 -4.04746954392028e-06
23.1900000000008 -6.27409541557026e-06
23.2000000000008 -8.48240653682553e-06
23.2100000000008 -1.06696989766906e-05
23.2200000000008 -1.28333079840908e-05
23.2300000000008 -1.49706111715194e-05
23.2400000000008 -1.70790316303345e-05
23.2500000000008 -1.91560409740758e-05
23.2600000000008 -2.11991623062854e-05
23.2700000000008 -2.32059731094396e-05
23.2800000000008 -2.5174108051692e-05
23.2900000000008 -2.71012617082902e-05
23.3000000000008 -2.89851911945956e-05
23.3100000000008 -3.08237187078059e-05
23.3200000000008 -3.26147339745499e-05
23.3300000000008 -3.43561966018811e-05
23.3400000000008 -3.60461383288947e-05
23.3500000000009 -3.76826651767141e-05
23.3600000000009 -3.92639594946162e-05
23.3700000000009 -4.07882819001838e-05
23.3800000000009 -4.22539731115206e-05
23.3900000000009 -4.36594556697285e-05
23.4000000000009 -4.50032355499906e-05
23.4100000000009 -4.62839036597347e-05
23.4200000000009 -4.75001372225361e-05
23.4300000000009 -4.86507010465422e-05
23.4400000000009 -4.97344486763605e-05
23.4500000000009 -5.07503234275252e-05
23.4600000000009 -5.16973593027822e-05
23.4700000000009 -5.25746817896185e-05
23.4800000000009 -5.33815085385932e-05
23.4900000000009 -5.41171499222069e-05
23.5000000000009 -5.47810094741921e-05
23.5100000000009 -5.53725842116607e-05
23.5200000000009 -5.58914648316465e-05
23.5300000000009 -5.63373357929271e-05
23.5400000000009 -5.67099752803134e-05
23.5500000000009 -5.70092550501327e-05
23.5600000000009 -5.72351401589878e-05
23.5700000000009 -5.73876885769223e-05
23.5800000000009 -5.74670506863723e-05
23.5900000000009 -5.74734686685578e-05
23.6000000000009 -5.74072757793323e-05
23.6100000000009 -5.72688955169893e-05
23.6200000000009 -5.70588406852153e-05
23.6300000000009 -5.67777123554583e-05
23.6400000000009 -5.64261987347511e-05
23.6500000000009 -5.60050739481935e-05
23.6600000000009 -5.55151967513235e-05
23.6700000000009 -5.49575092000764e-05
23.6800000000009 -5.433303533435e-05
23.6900000000009 -5.36428800035188e-05
23.7000000000009 -5.2888228160767e-05
23.7100000000009 -5.20703115523797e-05
23.7200000000009 -5.11904585667974e-05
23.7300000000009 -5.02500696788076e-05
23.7400000000009 -4.92506153356822e-05
23.7500000000009 -4.81936337655498e-05
23.7600000000009 -4.70807287111361e-05
23.7700000000009 -4.59135670920392e-05
23.7800000000009 -4.46938765987689e-05
23.7900000000009 -4.34234354732452e-05
23.8000000000009 -4.21040825266981e-05
23.8100000000009 -4.0737714756838e-05
23.8200000000009 -3.93262791494611e-05
23.8300000000009 -3.78717701834946e-05
23.8400000000009 -3.63762272012127e-05
23.8500000000009 -3.48417316894982e-05
23.8600000000009 -3.32704044958794e-05
23.8700000000009 -3.16644029933761e-05
23.8800000000009 -3.00259182035449e-05
23.8900000000009 -2.83571718847662e-05
23.9000000000009 -2.66604135915467e-05
23.9100000000009 -2.49379177098762e-05
23.9200000000009 -2.31919804732835e-05
23.9300000000009 -2.1424916963959e-05
23.9400000000009 -1.96390581031433e-05
23.9500000000009 -1.78367476348847e-05
23.9600000000009 -1.6020339107177e-05
23.9700000000009 -1.41921928544177e-05
23.9800000000009 -1.23546729850948e-05
23.990000000001 -1.0510144378544e-05
24.000000000001 -8.66096969458021e-06
24.010000000001 -6.80950639976735e-06
24.020000000001 -4.95810381402488e-06
24.030000000001 -3.10910018123704e-06
24.040000000001 -1.26481976747905e-06
24.050000000001 5.72430009598164e-07
24.060000000001 2.40036141672674e-06
24.070000000001 4.21670920288908e-06
24.080000000001 6.01923335573298e-06
24.090000000001 7.80572181323548e-06
24.100000000001 9.57399312739722e-06
24.110000000001 1.132189907682e-05
24.120000000001 1.30473272250925e-05
24.130000000001 1.47482034220203e-05
24.140000000001 1.64224942447814e-05
24.150000000001 1.8068209376227e-05
24.160000000001 1.96834039175909e-05
24.170000000001 2.12661806330039e-05
24.180000000001 2.28146921232977e-05
24.190000000001 2.4327142926672e-05
24.200000000001 2.58017915439031e-05
24.210000000001 2.72369523859189e-05
24.220000000001 2.86309976416704e-05
24.230000000001 2.99823590641898e-05
24.240000000001 3.12895296730726e-05
24.250000000001 3.25510653715969e-05
24.260000000001 3.37655864768355e-05
24.270000000001 3.49317791612324e-05
24.280000000001 3.60483968042477e-05
24.290000000001 3.71142612527847e-05
24.300000000001 3.81282639892321e-05
24.310000000001 3.90893672060951e-05
24.320000000001 3.99966047862906e-05
24.330000000001 4.08490831883243e-05
24.340000000001 4.16459822356859e-05
24.350000000001 4.23865558099246e-05
24.360000000001 4.30701324469974e-05
24.370000000001 4.36961158366087e-05
24.380000000001 4.42639852243909e-05
24.390000000001 4.47732957173233e-05
24.400000000001 4.52236784925099e-05
24.410000000001 4.56148409051148e-05
24.420000000001 4.5946566504878e-05
24.430000000001 4.62187149548527e-05
24.440000000001 4.64312218548949e-05
24.450000000001 4.65840984707396e-05
24.460000000001 4.66774313696793e-05
24.470000000001 4.67113819640651e-05
24.480000000001 4.66861859641064e-05
24.490000000001 4.66021527417716e-05
24.500000000001 4.64596646080491e-05
24.510000000001 4.62591760065057e-05
24.520000000001 4.60012126271668e-05
24.530000000001 4.56863704466005e-05
24.540000000001 4.5315314703515e-05
24.550000000001 4.48887788259755e-05
24.560000000001 4.44075633410578e-05
24.570000000001 4.38725348330962e-05
24.580000000001 4.32846251140101e-05
24.590000000001 4.2644818499937e-05
24.600000000001 4.19541567700415e-05
24.610000000001 4.12137503143591e-05
24.620000000001 4.04247671914477e-05
24.6300000000011 3.95884313776112e-05
24.6400000000011 3.8706020954128e-05
24.6500000000011 3.77788662350745e-05
24.6600000000011 3.68083478383765e-05
24.6700000000011 3.57958937687027e-05
24.6800000000011 3.47429707934154e-05
24.6900000000011 3.36511002670653e-05
24.7000000000011 3.25218450667775e-05
24.7100000000011 3.13568076371308e-05
24.7200000000011 3.01576278964166e-05
24.7300000000011 2.89259810570846e-05
24.7400000000011 2.76635753859483e-05
24.7500000000011 2.63721499183524e-05
24.7600000000011 2.50534721352954e-05
24.7700000000011 2.37093356099272e-05
24.7800000000011 2.23415576284775e-05
24.7900000000011 2.09519767899402e-05
24.8000000000011 1.95424505884029e-05
24.8100000000011 1.81148529816429e-05
24.8200000000011 1.66710719494647e-05
24.8300000000011 1.52130070451409e-05
24.8400000000011 1.37425669432307e-05
24.8500000000011 1.22616669870175e-05
24.8600000000011 1.07722267387385e-05
24.8700000000011 9.27616753575853e-06
24.8800000000011 7.77541005578613e-06
24.8900000000011 6.27187189420824e-06
24.9000000000011 4.76746515656484e-06
24.9100000000011 3.26409406916704e-06
24.9200000000011 1.76365261079925e-06
24.9300000000011 2.68022168419353e-07
24.9400000000011 -1.22093078028911e-06
24.9500000000011 -2.7013569546977e-06
24.9600000000011 -4.17142652658601e-06
24.9700000000011 -5.62933134037525e-06
24.9800000000011 -7.07328709557811e-06
24.9900000000011 -8.50153548889785e-06
25.0000000000011 -9.91234631344581e-06
25.0100000000011 -1.13040195126295e-05
25.0200000000011 -1.26748871863224e-05
25.0300000000011 -1.40233155470005e-05
25.0400000000011 -1.53477068236146e-05
25.0500000000011 -1.6646501111032e-05
25.0600000000011 -1.79181781629623e-05
25.0700000000011 -1.91612591263739e-05
25.0800000000011 -2.03743082154788e-05
25.0900000000011 -2.15559343234336e-05
25.1000000000011 -2.27047925700943e-05
25.1100000000011 -2.38195857840585e-05
25.1200000000011 -2.48990659174869e-05
25.1300000000011 -2.59420353922366e-05
25.1400000000011 -2.69473483759194e-05
25.1500000000011 -2.79139119866069e-05
25.1600000000011 -2.88406874249916e-05
25.1700000000011 -2.97266910329239e-05
25.1800000000011 -3.05709952773303e-05
25.1900000000011 -3.13727296586286e-05
25.2000000000011 -3.21310815428442e-05
25.2100000000011 -3.28452969167514e-05
25.2200000000011 -3.35146810654425e-05
25.2300000000011 -3.41385991718507e-05
25.2400000000011 -3.47164768378421e-05
25.2500000000011 -3.52478005266023e-05
25.2600000000011 -3.57321179261439e-05
25.2700000000012 -3.61690382338701e-05
25.2800000000012 -3.65582323631704e-05
25.2900000000012 -3.68994330687546e-05
25.3000000000012 -3.71924349952321e-05
25.3100000000012 -3.74370946477996e-05
25.3200000000012 -3.76333302848589e-05
25.3300000000012 -3.77811217336001e-05
25.3400000000012 -3.78805101292937e-05
25.3500000000012 -3.79315975791977e-05
25.3600000000012 -3.79345467521581e-05
25.3700000000012 -3.7889580395213e-05
25.3800000000012 -3.77969807788145e-05
25.3900000000012 -3.7657089072715e-05
25.4000000000012 -3.74703046552375e-05
25.4100000000012 -3.72370843597563e-05
25.4200000000012 -3.69579416641802e-05
25.4300000000012 -3.66334458329875e-05
25.4400000000012 -3.62642210291301e-05
25.4500000000012 -3.5850945430804e-05
25.4600000000012 -3.53943504333305e-05
25.4700000000012 -3.48952194366821e-05
25.4800000000012 -3.43543690480715e-05
25.4900000000012 -3.37726781999201e-05
25.5000000000012 -3.31510733068463e-05
25.5100000000012 -3.24905268689816e-05
25.5200000000012 -3.17920560238124e-05
25.5300000000012 -3.10567210486775e-05
25.5400000000012 -3.02856238160648e-05
25.5500000000012 -2.94799062038833e-05
25.5600000000012 -2.86407440002241e-05
25.5700000000012 -2.77693521523786e-05
25.5800000000012 -2.68669838550833e-05
25.5900000000012 -2.59349252111593e-05
25.6000000000012 -2.49744935742196e-05
25.6100000000012 -2.39870358065347e-05
25.6200000000012 -2.29739264803914e-05
25.6300000000012 -2.19365660376989e-05
25.6400000000012 -2.08763789166368e-05
25.6500000000012 -1.97948116513072e-05
25.6600000000012 -1.86933309488888e-05
25.6700000000012 -1.75734217480201e-05
25.6800000000012 -1.6436585261699e-05
25.6900000000012 -1.52843370077106e-05
25.7000000000012 -1.41182048294587e-05
25.7100000000012 -1.29397269099495e-05
25.7200000000012 -1.1750449781627e-05
25.7300000000012 -1.05519263346942e-05
25.7400000000012 -9.34571382651866e-06
25.7500000000012 -8.133371894691e-06
25.7600000000012 -6.91646057627148e-06
25.7700000000012 -5.69653833572308e-06
25.7800000000012 -4.47516010401479e-06
25.7900000000012 -3.25387533133347e-06
25.8000000000012 -2.03422605581394e-06
25.8100000000012 -8.17744990667471e-07
25.8200000000012 3.94046367955083e-07
25.8300000000012 1.59963961006094e-06
25.8400000000012 2.79754125561781e-06
25.8500000000012 3.98627457084615e-06
25.8600000000012 5.16438135482832e-06
25.8700000000012 6.33042369431486e-06
25.8800000000012 7.48298568466035e-06
25.8900000000012 8.6206751148653e-06
25.9000000000012 9.74212511476938e-06
25.9100000000013 1.08459957624848e-05
25.9200000000013 1.19309756502285e-05
25.9300000000013 1.29957834067655e-05
25.9400000000013 1.40391691747517e-05
25.9500000000013 1.50599160413088e-05
25.9600000000013 1.6056841420254e-05
25.9700000000013 1.70287983844547e-05
25.9800000000013 1.79746769468707e-05
25.9900000000013 1.88934052889223e-05
26.0000000000013 1.9783950934827e-05
26.0100000000013 2.06453218707247e-05
26.0200000000013 2.14765676074218e-05
26.0300000000013 2.2276780185678e-05
26.0400000000013 2.30450951230427e-05
26.0500000000013 2.37806923013135e-05
26.0600000000013 2.44827967937846e-05
26.0700000000013 2.51506796315193e-05
26.0800000000013 2.57836585079758e-05
26.0900000000013 2.63810984213863e-05
26.1000000000013 2.69424122543758e-05
26.1100000000013 2.74670612903914e-05
26.1200000000013 2.79545556665953e-05
26.1300000000013 2.84044547629557e-05
26.1400000000013 2.8816367527359e-05
26.1500000000013 2.91899527366482e-05
26.1600000000013 2.95249191937441e-05
26.1700000000013 2.98210258608954e-05
26.1800000000013 3.00780819276096e-05
26.1900000000013 3.02959468168758e-05
26.2000000000013 3.0474530127453e-05
26.2100000000013 3.06137915133648e-05
26.2200000000013 3.07137405011499e-05
26.2300000000013 3.07744362455327e-05
26.2400000000013 3.0795987224309e-05
26.2500000000013 3.07785508734018e-05
26.2600000000013 3.07223331632501e-05
26.2700000000013 3.06275881179716e-05
26.2800000000013 3.04946172791652e-05
26.2900000000013 3.03237691168832e-05
26.3000000000013 3.01154383914424e-05
26.3100000000013 2.98700654718476e-05
26.3200000000013 2.95881356207556e-05
26.3300000000013 2.92701782649307e-05
26.3400000000013 2.89167662918185e-05
26.3500000000013 2.85285154726456e-05
26.3600000000013 2.81060760804231e-05
26.3700000000013 2.76501373360054e-05
26.3800000000013 2.71614328001662e-05
26.3900000000013 2.66407339779479e-05
26.4000000000013 2.6088849163348e-05
26.4100000000013 2.5506622243037e-05
26.4200000000013 2.48949314608663e-05
26.4300000000013 2.42546881449407e-05
26.4400000000013 2.35868350556467e-05
26.4500000000013 2.28923404601736e-05
26.4600000000013 2.21722081157006e-05
26.4700000000013 2.14274690987157e-05
26.4800000000013 2.0659180504169e-05
26.4900000000013 1.98684240583023e-05
26.5000000000013 1.90563046774583e-05
26.5100000000013 1.82239489886046e-05
26.5200000000013 1.73725038203833e-05
26.5300000000013 1.65031346703278e-05
26.5400000000013 1.56170241523128e-05
26.5500000000014 1.47153704274824e-05
26.5600000000014 1.37993856214458e-05
26.5700000000014 1.28702942302641e-05
26.5800000000014 1.19293315176018e-05
26.5900000000014 1.09777419053084e-05
26.6000000000014 1.00167773596339e-05
26.6100000000014 9.04769577523624e-06
26.6200000000014 8.07175935910043e-06
26.6300000000014 7.09023301646246e-06
26.6400000000014 6.10438274080692e-06
26.6500000000014 5.11547400997859e-06
26.6600000000014 4.12477019042827e-06
26.6700000000014 3.13353095158643e-06
26.6800000000014 2.1430106923342e-06
26.6900000000014 1.15445698151154e-06
26.7000000000014 1.6910901437312e-07
26.7100000000014 -8.11803913122744e-07
26.7200000000014 -1.78706391034576e-06
26.7300000000014 -2.75546597653134e-06
26.7400000000014 -3.71581946204956e-06
26.7500000000014 -4.66694950465651e-06
26.7600000000014 -5.60769843904379e-06
26.7700000000014 -6.53692717801497e-06
26.7800000000014 -7.45351656368561e-06
26.7900000000014 -8.35636868713422e-06
26.8000000000014 -9.24440817498779e-06
26.8100000000014 -1.01165834414725e-05
26.8200000000014 -1.09718679045036e-05
26.8300000000014 -1.18092611644557e-05
26.8400000000014 -1.2627790144294e-05
26.8500000000014 -1.34265101898081e-05
26.8600000000014 -1.42045061287436e-05
26.8700000000014 -1.49608932877132e-05
26.8800000000014 -1.56948184657608e-05
26.8900000000014 -1.64054608635747e-05
26.9000000000014 -1.70920329673915e-05
26.9100000000014 -1.7753781386676e-05
26.9200000000014 -1.83899876447478e-05
26.9300000000014 -1.89999689215735e-05
26.9400000000014 -1.95830787480154e-05
26.9500000000014 -2.01387076508887e-05
26.9600000000014 -2.06662837482512e-05
26.9700000000014 -2.11652732944051e-05
26.9800000000014 -2.16351811741683e-05
26.9900000000014 -2.20755513460333e-05
27.0000000000014 -2.2485967233899e-05
27.0100000000014 -2.28660520671317e-05
27.0200000000014 -2.32154691687766e-05
27.0300000000014 -2.35339221918127e-05
27.0400000000014 -2.38211553034093e-05
27.0500000000014 -2.40769533175664e-05
27.0600000000014 -2.43011417749454e-05
27.0700000000014 -2.44935869716638e-05
27.0800000000014 -2.4654195936725e-05
27.0900000000014 -2.47829163581577e-05
27.1000000000014 -2.48797364584253e-05
27.1100000000014 -2.49446848195936e-05
27.1200000000014 -2.49778301588411e-05
27.1300000000014 -2.49792810550132e-05
27.1400000000014 -2.49491856270594e-05
27.1500000000014 -2.48877311653841e-05
27.1600000000014 -2.47951437174025e-05
27.1700000000014 -2.46716876290041e-05
27.1800000000014 -2.45176650442928e-05
27.1900000000015 -2.43334153671579e-05
27.2000000000015 -2.41193146904883e-05
27.2100000000015 -2.38757752035121e-05
27.2200000000015 -2.36032445983735e-05
27.2300000000015 -2.33022055242752e-05
27.2400000000015 -2.29731744897112e-05
27.2500000000015 -2.26166912846042e-05
27.2600000000015 -2.2233335762343e-05
27.2700000000015 -2.18237189859927e-05
27.2800000000015 -2.13884823079161e-05
27.2900000000015 -2.09282964154035e-05
27.3000000000015 -2.04438603437553e-05
27.3100000000015 -1.99359004582609e-05
27.3200000000015 -1.94051694065306e-05
27.3300000000015 -1.88524426047528e-05
27.3400000000015 -1.82785206433517e-05
27.3500000000015 -1.76842292033062e-05
27.3600000000015 -1.7070415590877e-05
27.3700000000015 -1.64379476387726e-05
27.3800000000015 -1.57877125544889e-05
27.3900000000015 -1.5120615733059e-05
27.4000000000015 -1.44375795432568e-05
27.4100000000015 -1.37395420927087e-05
27.4200000000015 -1.30274559756443e-05
27.4300000000015 -1.23022870061436e-05
27.4400000000015 -1.15650129392747e-05
27.4500000000015 -1.08166221822381e-05
27.4600000000015 -1.00581124974811e-05
27.4700000000015 -9.29048969965275e-06
27.4800000000015 -8.51476634820335e-06
27.4900000000015 -7.73196043738795e-06
27.5000000000015 -6.94309408541082e-06
27.5100000000015 -6.14919222440782e-06
27.5200000000015 -5.35128129296185e-06
27.5300000000015 -4.55038793280669e-06
27.5400000000015 -3.74753769137059e-06
27.5500000000015 -2.94375373178341e-06
27.5600000000015 -2.14005555195343e-06
27.5700000000015 -1.33745771429852e-06
27.5800000000015 -5.36968587691055e-07
27.5900000000015 2.60410896844616e-07
27.6000000000015 1.05368847516987e-06
27.6100000000015 1.84188176232228e-06
27.6200000000015 2.6240194466035e-06
27.6300000000015 3.39914246404008e-06
27.6400000000015 4.16630515181866e-06
27.6500000000015 4.92457637934701e-06
27.6600000000015 5.67304065560436e-06
27.6700000000015 6.41079921150312e-06
27.6800000000015 7.13697105600485e-06
27.6900000000015 7.85069400478301e-06
27.7000000000015 8.55112568026135e-06
27.7100000000015 9.23744448189946e-06
27.7200000000015 9.90885052564168e-06
27.7300000000015 1.05645665514835e-05
27.7400000000015 1.12038387981653e-05
27.7500000000015 1.18259378440382e-05
27.7600000000015 1.24301594132156e-05
27.7700000000015 1.30158251461277e-05
27.7800000000015 1.35822833336976e-05
27.7900000000015 1.41289096143786e-05
27.8000000000015 1.46551076333459e-05
27.8100000000015 1.51603096631914e-05
27.8200000000015 1.56439771855211e-05
27.8300000000016 1.61056014329061e-05
27.8400000000016 1.65447038906933e-05
27.8500000000016 1.69608367582321e-05
27.8600000000016 1.73535833691289e-05
27.8700000000016 1.77225585701981e-05
27.8800000000016 1.80674090588274e-05
27.8900000000016 1.83878136785332e-05
27.9000000000016 1.86834836725388e-05
27.9100000000016 1.89541628952564e-05
27.9200000000016 1.91996279816184e-05
27.9300000000016 1.94196884743139e-05
27.9400000000016 1.96141869089758e-05
27.9500000000016 1.97829988568233e-05
27.9600000000016 1.99260329261813e-05
27.9700000000016 2.0043230722147e-05
27.9800000000016 2.01345667649535e-05
27.9900000000016 2.020004836739e-05
28.0000000000016 2.02397154717083e-05
28.0100000000016 2.02536404465296e-05
28.0200000000016 2.02419278443639e-05
28.0300000000016 2.02047141204808e-05
28.0400000000016 2.01421673140413e-05
28.0500000000016 2.00544866926511e-05
28.0600000000016 1.99419023618937e-05
28.0700000000016 1.9804674842076e-05
28.0800000000016 1.96430946156574e-05
28.0900000000016 1.94574816512851e-05
28.1000000000016 1.92481849156642e-05
28.1100000000016 1.90155818972445e-05
28.1200000000016 1.87600782008454e-05
28.1300000000016 1.84821023094367e-05
28.1400000000016 1.81821087455026e-05
28.1500000000016 1.78605806732369e-05
28.1600000000016 1.75180261688308e-05
28.1700000000016 1.71549774601623e-05
28.1800000000016 1.67719901394857e-05
28.1900000000016 1.63696423503065e-05
28.2000000000016 1.59485339496234e-05
28.2100000000016 1.55092856184934e-05
28.2200000000016 1.50525348067703e-05
28.2300000000016 1.45789419390742e-05
28.2400000000016 1.40891853815613e-05
28.2500000000016 1.35839605778203e-05
28.2600000000016 1.30639791317022e-05
28.2700000000016 1.25299678565074e-05
28.2800000000016 1.19826678000445e-05
28.2900000000016 1.14228332509515e-05
28.3000000000016 1.08512307297647e-05
28.3100000000016 1.02686379672893e-05
28.3200000000016 9.67584287233083e-06
28.3300000000016 9.07364249058167e-06
28.3400000000016 8.4628419562895e-06
28.3500000000016 7.84425343825058e-06
28.3600000000016 7.21869508160752e-06
28.3700000000016 6.58698994688723e-06
28.3800000000016 5.9499649476946e-06
28.3900000000016 5.3084497884495e-06
28.4000000000016 4.66327590353887e-06
28.4100000000016 4.01527539924085e-06
28.4200000000016 3.36527999975988e-06
28.4300000000016 2.71411999869907e-06
28.4400000000016 2.0626232172747e-06
28.4500000000016 1.41161397056988e-06
28.4600000000016 7.61912043096371e-07
28.4700000000017 1.1433167492236e-07
28.4800000000017 -5.30319440401301e-07
28.4900000000017 -1.17124114490161e-06
28.5000000000017 -1.8076417910419e-06
28.5100000000017 -2.4387392012108e-06
28.5200000000017 -3.06376161070173e-06
28.5300000000017 -3.68194859306888e-06
28.5400000000017 -4.29255196677034e-06
28.5500000000017 -4.89483668204507e-06
28.5600000000017 -5.48808168699378e-06
28.5700000000017 -6.07158077186596e-06
28.5800000000017 -6.64464339059489e-06
28.5900000000017 -7.20659545864417e-06
28.6000000000017 -7.75678012627321e-06
28.6100000000017 -8.29455852636065e-06
28.6200000000017 -8.81931049596345e-06
28.6300000000017 -9.33043527081726e-06
28.6400000000017 -9.82735215204508e-06
28.6500000000017 -1.03095011443457e-05
28.6600000000017 -1.07763435650009e-05
28.6700000000017 -1.12273626230705e-05
28.6800000000017 -1.16620639681839e-05
28.6900000000017 -1.20799762083806e-05
28.7000000000017 -1.24806513964941e-05
28.7100000000017 -1.28636654846128e-05
28.7200000000017 -1.32286187461973e-05
28.7300000000017 -1.35751361654787e-05
28.7400000000017 -1.39028677937971e-05
28.7500000000017 -1.42114890725955e-05
28.7600000000017 -1.45007011228182e-05
28.7700000000017 -1.47702310005112e-05
28.7800000000017 -1.50198319184676e-05
28.7900000000017 -1.52492834338021e-05
28.8000000000017 -1.54583916013876e-05
28.8100000000017 -1.56469890931277e-05
28.8200000000017 -1.5814935283215e-05
28.8300000000017 -1.59621162989771e-05
28.8400000000017 -1.60884450379929e-05
28.8500000000017 -1.61938611514466e-05
28.8600000000017 -1.62783309938328e-05
28.8700000000017 -1.63418475393314e-05
28.8800000000017 -1.63844302651712e-05
28.8900000000017 -1.6406125002357e-05
28.9000000000017 -1.64070037542105e-05
28.9100000000017 -1.63871644832575e-05
28.9200000000017 -1.63467308671096e-05
28.9300000000017 -1.62858520241418e-05
28.9400000000017 -1.62047022100121e-05
28.9500000000017 -1.61034804864512e-05
28.9600000000017 -1.5982410364441e-05
28.9700000000017 -1.58417394251987e-05
28.9800000000017 -1.56817389250735e-05
28.9900000000017 -1.55027033965651e-05
29.0000000000017 -1.53049502732927e-05
29.0100000000017 -1.50888190933085e-05
29.0200000000017 -1.48546655501193e-05
29.0300000000017 -1.46028709949999e-05
29.0400000000017 -1.43338372469888e-05
29.0500000000017 -1.40479859880605e-05
29.0600000000017 -1.37457581359199e-05
29.0700000000017 -1.34276131953881e-05
29.0800000000017 -1.30940285893439e-05
29.0900000000017 -1.2745498970195e-05
29.1000000000017 -1.23825342680888e-05
29.1100000000018 -1.20056605366497e-05
29.1200000000018 -1.16154202361712e-05
29.1300000000018 -1.12123700108654e-05
29.1400000000018 -1.07970799616819e-05
29.1500000000018 -1.03701328866749e-05
29.1600000000018 -9.93212349919213e-06
29.1700000000018 -9.4836576293286e-06
29.1800000000018 -9.0253514119656e-06
29.1900000000018 -8.55783046370774e-06
29.2000000000018 -8.08172905051269e-06
29.2100000000018 -7.59768924753125e-06
29.2200000000018 -7.10636009252159e-06
29.2300000000018 -6.60839673410639e-06
29.2400000000018 -6.10445957608712e-06
29.2500000000018 -5.59521341898826e-06
29.2600000000018 -5.0813265999855e-06
29.2700000000018 -4.56347013234567e-06
29.2800000000018 -4.04231684550037e-06
29.2900000000018 -3.51854052684794e-06
29.3000000000018 -2.99281506638256e-06
29.3100000000018 -2.46581360522379e-06
29.3200000000018 -1.93820768911304e-06
29.3300000000018 -1.41066642792923e-06
29.3400000000018 -8.83855662263178e-07
29.3500000000018 -3.5843713807131e-07
29.3600000000018 1.64932309580393e-07
29.3700000000018 6.85601562696071e-07
29.3800000000018 1.20292601547604e-06
29.3900000000018 1.71626835744047e-06
29.4000000000018 2.22499934361211e-06
29.4100000000018 2.72849855084858e-06
29.4200000000018 3.22615511943552e-06
29.4300000000018 3.71736847906714e-06
29.4400000000018 4.20154905837761e-06
29.4500000000018 4.67811897719907e-06
29.4600000000018 5.1465127207575e-06
29.4700000000018 5.60617779503839e-06
29.4800000000018 6.05657536258353e-06
29.4900000000018 6.49718085801049e-06
29.5000000000018 6.92748458257256e-06
29.5100000000018 7.34699227710489e-06
29.5200000000018 7.75522567274336e-06
29.5300000000018 8.15172301881958e-06
29.5400000000018 8.53603958737714e-06
29.5500000000018 8.90774815377989e-06
29.5600000000018 9.26643945292497e-06
29.5700000000018 9.61172261059637e-06
29.5800000000018 9.94322554953485e-06
29.5900000000018 1.02605953698318e-05
29.6000000000018 1.05634987032899e-05
29.6100000000018 1.08516220414277e-05
29.6200000000018 1.11246720368404e-05
29.6300000000018 1.13823757776643e-05
29.6400000000018 1.16244810349288e-05
29.6500000000018 1.18507564826142e-05
29.6600000000018 1.20609918902699e-05
29.6700000000018 1.22549982880835e-05
29.6800000000018 1.24326081043286e-05
29.6900000000018 1.25936752751528e-05
29.7000000000018 1.27380753267262e-05
29.7100000000018 1.28657054297943e-05
29.7200000000018 1.29764844264685e-05
29.7300000000018 1.30703528298378e-05
29.7400000000018 1.31472727961965e-05
29.7500000000019 1.32072280701706e-05
29.7600000000019 1.32502239029773e-05
29.7700000000019 1.3276286944093e-05
29.7800000000019 1.32854651066593e-05
29.7900000000019 1.32778274070154e-05
29.8000000000019 1.32534637788214e-05
29.8100000000019 1.32124848623355e-05
29.8200000000019 1.31550217695564e-05
29.8300000000019 1.30812258261666e-05
29.8400000000019 1.29912682915952e-05
29.8500000000019 1.28853400592189e-05
29.8600000000019 1.27636513400917e-05
29.8700000000019 1.26264313365707e-05
29.8800000000019 1.24739279193274e-05
29.8900000000019 1.23064073408208e-05
29.9000000000019 1.2124151143264e-05
29.9100000000019 1.1927458064628e-05
29.9200000000019 1.17166453272812e-05
29.9300000000019 1.14920464716396e-05
29.9400000000019 1.1254010857224e-05
29.9500000000019 1.10029031459803e-05
29.9600000000019 1.07391027686502e-05
29.9700000000019 1.04630033749831e-05
29.9800000000019 1.01750122685791e-05
29.9900000000019 9.87554811574742e-06
30.0000000000019 9.56504405563409e-06
};
\addlegendentry{DDPG};
\end{axis}

\end{tikzpicture}

		\end{figure}
		\begin{figure}\scriptsize
			\hspace{3cm}
			% This file was created by tikzplotlib v0.9.1.
\begin{tikzpicture}

\definecolor{color0}{rgb}{0.12156862745098,0.466666666666667,0.705882352941177}
\definecolor{color1}{rgb}{1,0.498039215686275,0.0549019607843137}

\begin{axis}[
compat=newest,
tick align=outside,
tick pos=left,
x grid style={white!69.0196078431373!black},
xmin=-1.50000000000009, xmax=31.500000000002,
xtick style={color=black},
y grid style={white!69.0196078431373!black},
ymin=-0.01, ymax=0.025,
ytick style={color=black},
%yticklabel style={
%        /pgf/number format/.cd,
%        	fixed,
%        	fixed zerofill,
%         	precision=3,
%        /tikz/.cd
%},
scaled y ticks=true,
scaled y ticks=base 10:3,
width=14cm,
height=10cm,
xlabel=Time (sec),
ylabel=Control Signal
%y label style={at={(-0.2,0.5)}}
]

\addplot [ultra thick, blue!20!gray, dotted]
table {%
0 0
0.01 0
0.02 0
0.03 0
0.04 0
0.05 0
0.06 0
0.07 0
0.08 0
0.09 0
0.1 0
0.11 0
0.12 0
0.13 0
0.14 0
0.15 0
0.16 0
0.17 0
0.18 0
0.19 0
0.2 0
0.21 0
0.22 0
0.23 0
0.24 0
0.25 0
0.26 0
0.27 0
0.28 0
0.29 0
0.3 0
0.31 0
0.32 0
0.33 0
0.34 0
0.35 0
0.36 0
0.37 0
0.38 0
0.39 0
0.4 0
0.41 0
0.42 0
0.43 0
0.44 0
0.45 0
0.46 0
0.47 0
0.48 0
0.49 0
0.5 0
0.51 0
0.52 0
0.53 0
0.54 0
0.55 0
0.56 0
0.57 0
0.58 0
0.59 0
0.6 0
0.61 0
0.62 0
0.63 0
0.64 0
0.65 0
0.66 0
0.67 0
0.68 0
0.69 0
0.7 0
0.71 0
0.72 0
0.73 0
0.74 0
0.75 0
0.76 0
0.77 0
0.78 0
0.79 0
0.8 0
0.81 0
0.820000000000001 0
0.830000000000001 0
0.840000000000001 0
0.850000000000001 0
0.860000000000001 0
0.870000000000001 0
0.880000000000001 0
0.890000000000001 0
0.900000000000001 0
0.910000000000001 0
0.920000000000001 0
0.930000000000001 0
0.940000000000001 0
0.950000000000001 0
0.960000000000001 0
0.970000000000001 0
0.980000000000001 0
0.990000000000001 0
1 0
1.01 0
1.02 0
1.03 0
1.04 0
1.05 0
1.06 0
1.07 0
1.08 0
1.09 0
1.1 0
1.11 0
1.12 0
1.13 0
1.14 0
1.15 0
1.16 0
1.17 0
1.18 0
1.19 0
1.2 0
1.21 0
1.22 0
1.23 0
1.24 0
1.25 0
1.26 0
1.27 0
1.28 0
1.29 0
1.3 0
1.31 0
1.32 0
1.33 0
1.34 0
1.35 0
1.36 0
1.37 0
1.38 0
1.39 0
1.4 0
1.41 0
1.42 0
1.43 0
1.44 0
1.45 0
1.46 0
1.47 0
1.48 0
1.49 0
1.5 0
1.51 0
1.52 0
1.53 0
1.54 0
1.55 0
1.56 0
1.57 0
1.58 0
1.59 0
1.6 0
1.61 0
1.62 0
1.63 0
1.64 0
1.65 0
1.66 0
1.67 0
1.68 0
1.69 0
1.7 0
1.71 0
1.72 0
1.73 0
1.74 0
1.75 0
1.76 0
1.77 0
1.78 0
1.79 0
1.8 0
1.81 0
1.82 0
1.83 0
1.84 0
1.85 0
1.86 0
1.87 0
1.88 0
1.89 0
1.9 0
1.91 0
1.92 0
1.93 0
1.94 0
1.95 0
1.96 0
1.97 0
1.98 0
1.99 0
2 0
2.01 0
2.02 0
2.03 0
2.04 0
2.05 0
2.06 0
2.07 0
2.08 0
2.09 0
2.1 0
2.11 0
2.12 0
2.13 0
2.14 0
2.15 0
2.16 0
2.17 0
2.18 0
2.19 0
2.2 0
2.21 0
2.22 0
2.23 0
2.24 0
2.25 0
2.26 0
2.27 0
2.28 0
2.29 0
2.29999999999999 0
2.30999999999999 0
2.31999999999999 0
2.32999999999999 0
2.33999999999999 0
2.34999999999999 0
2.35999999999999 0
2.36999999999999 0
2.37999999999999 0
2.38999999999999 0
2.39999999999999 0
2.40999999999999 0
2.41999999999999 0
2.42999999999999 0
2.43999999999999 0
2.44999999999999 0
2.45999999999999 0
2.46999999999999 0
2.47999999999999 0
2.48999999999999 0
2.49999999999999 0
2.50999999999999 0
2.51999999999999 0
2.52999999999999 0
2.53999999999999 0
2.54999999999999 0
2.55999999999999 0
2.56999999999999 0
2.57999999999999 0
2.58999999999999 0
2.59999999999999 0
2.60999999999999 0
2.61999999999999 0
2.62999999999999 0
2.63999999999999 0
2.64999999999999 0
2.65999999999999 0
2.66999999999999 0
2.67999999999999 0
2.68999999999999 0
2.69999999999999 0
2.70999999999999 0
2.71999999999999 0
2.72999999999999 0
2.73999999999999 0
2.74999999999999 0
2.75999999999999 0
2.76999999999998 0
2.77999999999998 0
2.78999999999998 0
2.79999999999998 0
2.80999999999998 0
2.81999999999998 0
2.82999999999998 0
2.83999999999998 0
2.84999999999998 0
2.85999999999998 0
2.86999999999998 0
2.87999999999998 0
2.88999999999998 0
2.89999999999998 0
2.90999999999998 0
2.91999999999998 0
2.92999999999998 0
2.93999999999998 0
2.94999999999998 0
2.95999999999998 0
2.96999999999998 0
2.97999999999998 0
2.98999999999998 0
2.99999999999998 0
3.00999999999998 0
3.01999999999998 0
3.02999999999998 0
3.03999999999998 0
3.04999999999998 0
3.05999999999998 0
3.06999999999998 0
3.07999999999998 0
3.08999999999998 0
3.09999999999998 0
3.10999999999998 0
3.11999999999998 0
3.12999999999998 0
3.13999999999998 0
3.14999999999998 0
3.15999999999998 0
3.16999999999998 0
3.17999999999998 0
3.18999999999998 0
3.19999999999998 0
3.20999999999998 0
3.21999999999998 0
3.22999999999998 0
3.23999999999997 0
3.24999999999997 0
3.25999999999997 0
3.26999999999997 0
3.27999999999997 0
3.28999999999997 0
3.29999999999997 0
3.30999999999997 0
3.31999999999997 0
3.32999999999997 0
3.33999999999997 0
3.34999999999997 0
3.35999999999997 0
3.36999999999997 0
3.37999999999997 0
3.38999999999997 0
3.39999999999997 0
3.40999999999997 0
3.41999999999997 0
3.42999999999997 0
3.43999999999997 0
3.44999999999997 0
3.45999999999997 0
3.46999999999997 0
3.47999999999997 0
3.48999999999997 0
3.49999999999997 0
3.50999999999997 0
3.51999999999997 0
3.52999999999997 0
3.53999999999997 0
3.54999999999997 0
3.55999999999997 0
3.56999999999997 0
3.57999999999997 0
3.58999999999997 0
3.59999999999997 0
3.60999999999997 0
3.61999999999997 0
3.62999999999997 0
3.63999999999997 0
3.64999999999997 0
3.65999999999997 0
3.66999999999997 0
3.67999999999997 0
3.68999999999997 0
3.69999999999997 0
3.70999999999996 0
3.71999999999996 0
3.72999999999996 0
3.73999999999996 0
3.74999999999996 0
3.75999999999996 0
3.76999999999996 0
3.77999999999996 0
3.78999999999996 0
3.79999999999996 0
3.80999999999996 0
3.81999999999996 0
3.82999999999996 0
3.83999999999996 0
3.84999999999996 0
3.85999999999996 0
3.86999999999996 0
3.87999999999996 0
3.88999999999996 0
3.89999999999996 0
3.90999999999996 0
3.91999999999996 0
3.92999999999996 0
3.93999999999996 0
3.94999999999996 0
3.95999999999996 0
3.96999999999996 0
3.97999999999996 0
3.98999999999996 0
3.99999999999996 0
4.00999999999996 0
4.01999999999996 0
4.02999999999996 0
4.03999999999996 0
4.04999999999996 0
4.05999999999996 0
4.06999999999996 0
4.07999999999996 0
4.08999999999996 0
4.09999999999996 0
4.10999999999996 0
4.11999999999996 0
4.12999999999996 0
4.13999999999996 0
4.14999999999996 0
4.15999999999996 0
4.16999999999996 0
4.17999999999996 0
4.18999999999996 0
4.19999999999995 0
4.20999999999995 0
4.21999999999995 0
4.22999999999995 0
4.23999999999995 0
4.24999999999995 0
4.25999999999995 0
4.26999999999995 0
4.27999999999995 0
4.28999999999995 0
4.29999999999995 0
4.30999999999995 0
4.31999999999995 0
4.32999999999995 0
4.33999999999995 0
4.34999999999995 0
4.35999999999995 0
4.36999999999995 0
4.37999999999995 0
4.38999999999995 0
4.39999999999995 0
4.40999999999995 0
4.41999999999995 0
4.42999999999995 0
4.43999999999995 0
4.44999999999995 0
4.45999999999995 0
4.46999999999995 0
4.47999999999995 0
4.48999999999995 0
4.49999999999995 0
4.50999999999995 0
4.51999999999995 0
4.52999999999995 0
4.53999999999995 0
4.54999999999995 0
4.55999999999995 0
4.56999999999995 0
4.57999999999995 0
4.58999999999995 0
4.59999999999995 0
4.60999999999995 0
4.61999999999995 0
4.62999999999995 0
4.63999999999995 0
4.64999999999995 0
4.65999999999995 0
4.66999999999994 0
4.67999999999994 0
4.68999999999994 0
4.69999999999994 0
4.70999999999994 0
4.71999999999994 0
4.72999999999994 0
4.73999999999994 0
4.74999999999994 0
4.75999999999994 0
4.76999999999994 0
4.77999999999994 0
4.78999999999994 0
4.79999999999994 0
4.80999999999994 0
4.81999999999994 0
4.82999999999994 0
4.83999999999994 0
4.84999999999994 0
4.85999999999994 0
4.86999999999994 0
4.87999999999994 0
4.88999999999994 0
4.89999999999994 0
4.90999999999994 0
4.91999999999994 0
4.92999999999994 0
4.93999999999994 0
4.94999999999994 0
4.95999999999994 0
4.96999999999994 0
4.97999999999994 0
4.98999999999994 0
4.99999999999994 0
5.00999999999994 0
5.01999999999994 0
5.02999999999994 0
5.03999999999994 0
5.04999999999994 0
5.05999999999994 0
5.06999999999994 0
5.07999999999994 0
5.08999999999994 0
5.09999999999994 0
5.10999999999994 0
5.11999999999994 0
5.12999999999994 0
5.13999999999993 0
5.14999999999993 0
5.15999999999993 0
5.16999999999993 0
5.17999999999993 0
5.18999999999993 0
5.19999999999993 0
5.20999999999993 0
5.21999999999993 0
5.22999999999993 0
5.23999999999993 0
5.24999999999993 0
5.25999999999993 0
5.26999999999993 0
5.27999999999993 0
5.28999999999993 0
5.29999999999993 0
5.30999999999993 0
5.31999999999993 0
5.32999999999993 0
5.33999999999993 0
5.34999999999993 0
5.35999999999993 0
5.36999999999993 0
5.37999999999993 0
5.38999999999993 0
5.39999999999993 0
5.40999999999993 0
5.41999999999993 0
5.42999999999993 0
5.43999999999993 0
5.44999999999993 0
5.45999999999993 0
5.46999999999993 0
5.47999999999993 0
5.48999999999993 0
5.49999999999993 0
5.50999999999993 0
5.51999999999993 0
5.52999999999993 0
5.53999999999993 0
5.54999999999993 0
5.55999999999993 0
5.56999999999993 0
5.57999999999993 0
5.58999999999993 0
5.59999999999993 0
5.60999999999992 0
5.61999999999992 0
5.62999999999992 0
5.63999999999992 0
5.64999999999992 0
5.65999999999992 0
5.66999999999992 0
5.67999999999992 0
5.68999999999992 0
5.69999999999992 0
5.70999999999992 0
5.71999999999992 0
5.72999999999992 0
5.73999999999992 0
5.74999999999992 0
5.75999999999992 0
5.76999999999992 0
5.77999999999992 0
5.78999999999992 0
5.79999999999992 0
5.80999999999992 0
5.81999999999992 0
5.82999999999992 0
5.83999999999992 0
5.84999999999992 0
5.85999999999992 0
5.86999999999992 0
5.87999999999992 0
5.88999999999992 0
5.89999999999992 0
5.90999999999992 0
5.91999999999992 0
5.92999999999992 0
5.93999999999992 0
5.94999999999992 0
5.95999999999992 0
5.96999999999992 0
5.97999999999992 0
5.98999999999992 0
5.99999999999992 0
6.00999999999992 0
6.01999999999992 0
6.02999999999992 0
6.03999999999992 0
6.04999999999992 0
6.05999999999992 0
6.06999999999992 0
6.07999999999991 0
6.08999999999991 0
6.09999999999991 0
6.10999999999991 0
6.11999999999991 0
6.12999999999991 0
6.13999999999991 0
6.14999999999991 0
6.15999999999991 0
6.16999999999991 0
6.17999999999991 0
6.18999999999991 0
6.19999999999991 0
6.20999999999991 0
6.21999999999991 0
6.22999999999991 0
6.23999999999991 0
6.24999999999991 0
6.25999999999991 0
6.26999999999991 0
6.27999999999991 0
6.28999999999991 0
6.29999999999991 0
6.30999999999991 0
6.31999999999991 0
6.32999999999991 0
6.33999999999991 0
6.34999999999991 0
6.35999999999991 0
6.36999999999991 0
6.37999999999991 0
6.38999999999991 0
6.39999999999991 0
6.40999999999991 0
6.41999999999991 0
6.42999999999991 0
6.43999999999991 0
6.44999999999991 0
6.45999999999991 0
6.46999999999991 0
6.47999999999991 0
6.48999999999991 0
6.49999999999991 0
6.50999999999991 0
6.51999999999991 0
6.52999999999991 0
6.53999999999991 0
6.5499999999999 0
6.5599999999999 0
6.5699999999999 0
6.5799999999999 0
6.5899999999999 0
6.5999999999999 0
6.6099999999999 0
6.6199999999999 0
6.6299999999999 0
6.6399999999999 0
6.6499999999999 0
6.6599999999999 0
6.6699999999999 0
6.6799999999999 0
6.6899999999999 0
6.6999999999999 0
6.7099999999999 0
6.7199999999999 0
6.7299999999999 0
6.7399999999999 0
6.7499999999999 0
6.7599999999999 0
6.7699999999999 0
6.7799999999999 0
6.7899999999999 0
6.7999999999999 0
6.8099999999999 0
6.8199999999999 0
6.8299999999999 0
6.8399999999999 0
6.8499999999999 0
6.8599999999999 0
6.8699999999999 0
6.8799999999999 0
6.8899999999999 0
6.8999999999999 0
6.9099999999999 0
6.9199999999999 0
6.9299999999999 0
6.9399999999999 0
6.9499999999999 0
6.9599999999999 0
6.9699999999999 0
6.9799999999999 0
6.9899999999999 0
6.9999999999999 0
7.00999999999989 0
7.01999999999989 0
7.02999999999989 0
7.03999999999989 0
7.04999999999989 0
7.05999999999989 0
7.06999999999989 0
7.07999999999989 0
7.08999999999989 0
7.09999999999989 0
7.10999999999989 0
7.11999999999989 0
7.12999999999989 0
7.13999999999989 0
7.14999999999989 0
7.15999999999989 0
7.16999999999989 0
7.17999999999989 0
7.18999999999989 0
7.19999999999989 0
7.20999999999989 0
7.21999999999989 0
7.22999999999989 0
7.23999999999989 0
7.24999999999989 0
7.25999999999989 0
7.26999999999989 0
7.27999999999989 0
7.28999999999989 0
7.29999999999989 0
7.30999999999989 0
7.31999999999989 0
7.32999999999989 0
7.33999999999989 0
7.34999999999989 0
7.35999999999989 0
7.36999999999989 0
7.37999999999989 0
7.38999999999989 0
7.39999999999989 0
7.40999999999989 0
7.41999999999989 0
7.42999999999989 0
7.43999999999989 0
7.44999999999989 0
7.45999999999989 0
7.46999999999989 0
7.47999999999988 0
7.48999999999988 0
7.49999999999988 0
7.50999999999988 0
7.51999999999988 0
7.52999999999988 0
7.53999999999988 0
7.54999999999988 0
7.55999999999988 0
7.56999999999988 0
7.57999999999988 0
7.58999999999988 0
7.59999999999988 0
7.60999999999988 0
7.61999999999988 0
7.62999999999988 0
7.63999999999988 0
7.64999999999988 0
7.65999999999988 0
7.66999999999988 0
7.67999999999988 0
7.68999999999988 0
7.69999999999988 0
7.70999999999988 0
7.71999999999988 0
7.72999999999988 0
7.73999999999988 0
7.74999999999988 0
7.75999999999988 0
7.76999999999988 0
7.77999999999988 0
7.78999999999988 0
7.79999999999988 0
7.80999999999988 0
7.81999999999988 0
7.82999999999988 0
7.83999999999988 0
7.84999999999988 0
7.85999999999988 0
7.86999999999988 0
7.87999999999988 0
7.88999999999988 0
7.89999999999988 0
7.90999999999988 0
7.91999999999988 0
7.92999999999988 0
7.93999999999988 0
7.94999999999987 0
7.95999999999987 0
7.96999999999987 0
7.97999999999987 0
7.98999999999987 0
7.99999999999987 0
8.00999999999987 0
8.01999999999987 0
8.02999999999987 0
8.03999999999987 0
8.04999999999987 0
8.05999999999987 0
8.06999999999987 0
8.07999999999987 0
8.08999999999987 0
8.09999999999987 0
8.10999999999987 0
8.11999999999987 0
8.12999999999987 0
8.13999999999987 0
8.14999999999987 0
8.15999999999987 0
8.16999999999987 0
8.17999999999987 0
8.18999999999987 0
8.19999999999987 0
8.20999999999987 0
8.21999999999987 0
8.22999999999987 0
8.23999999999987 0
8.24999999999987 0
8.25999999999987 0
8.26999999999987 0
8.27999999999987 0
8.28999999999987 0
8.29999999999987 0
8.30999999999987 0
8.31999999999987 0
8.32999999999987 0
8.33999999999987 0
8.34999999999987 0
8.35999999999987 0
8.36999999999987 0
8.37999999999987 0
8.38999999999987 0
8.39999999999987 0
8.40999999999987 0
8.41999999999986 0
8.42999999999986 0
8.43999999999986 0
8.44999999999986 0
8.45999999999986 0
8.46999999999986 0
8.47999999999986 0
8.48999999999986 0
8.49999999999986 0
8.50999999999986 0
8.51999999999986 0
8.52999999999986 0
8.53999999999986 0
8.54999999999986 0
8.55999999999986 0
8.56999999999986 0
8.57999999999986 0
8.58999999999986 0
8.59999999999986 0
8.60999999999986 0
8.61999999999986 0
8.62999999999986 0
8.63999999999986 0
8.64999999999986 0
8.65999999999986 0
8.66999999999986 0
8.67999999999986 0
8.68999999999986 0
8.69999999999986 0
8.70999999999986 0
8.71999999999986 0
8.72999999999986 0
8.73999999999986 0
8.74999999999986 0
8.75999999999986 0
8.76999999999986 0
8.77999999999986 0
8.78999999999986 0
8.79999999999986 0
8.80999999999986 0
8.81999999999986 0
8.82999999999986 0
8.83999999999986 0
8.84999999999986 0
8.85999999999986 0
8.86999999999986 0
8.87999999999986 0
8.88999999999985 0
8.89999999999985 0
8.90999999999985 0
8.91999999999985 0
8.92999999999985 0
8.93999999999985 0
8.94999999999985 0
8.95999999999985 0
8.96999999999985 0
8.97999999999985 0
8.98999999999985 0
8.99999999999985 0
9.00999999999985 0
9.01999999999985 0
9.02999999999985 0
9.03999999999985 0
9.04999999999985 0
9.05999999999985 0
9.06999999999985 0
9.07999999999985 0
9.08999999999985 0
9.09999999999985 0
9.10999999999985 0
9.11999999999985 0
9.12999999999985 0
9.13999999999985 0
9.14999999999985 0
9.15999999999985 0
9.16999999999985 0
9.17999999999985 0
9.18999999999985 0
9.19999999999985 0
9.20999999999985 0
9.21999999999985 0
9.22999999999985 0
9.23999999999985 0
9.24999999999985 0
9.25999999999985 0
9.26999999999985 0
9.27999999999985 0
9.28999999999985 0
9.29999999999985 0
9.30999999999985 0
9.31999999999985 0
9.32999999999985 0
9.33999999999985 0
9.34999999999985 0
9.35999999999984 0
9.36999999999984 0
9.37999999999984 0
9.38999999999984 0
9.39999999999984 0
9.40999999999984 0
9.41999999999984 0
9.42999999999984 0
9.43999999999984 0
9.44999999999984 0
9.45999999999984 0
9.46999999999984 0
9.47999999999984 0
9.48999999999984 0
9.49999999999984 0
9.50999999999984 0
9.51999999999984 0
9.52999999999984 0
9.53999999999984 0
9.54999999999984 0
9.55999999999984 0
9.56999999999984 0
9.57999999999984 0
9.58999999999984 0
9.59999999999984 0
9.60999999999984 0
9.61999999999984 0
9.62999999999984 0
9.63999999999984 0
9.64999999999984 0
9.65999999999984 0
9.66999999999984 0
9.67999999999984 0
9.68999999999984 0
9.69999999999984 0
9.70999999999984 0
9.71999999999984 0
9.72999999999984 0
9.73999999999984 0
9.74999999999984 0
9.75999999999984 0
9.76999999999984 0
9.77999999999984 0
9.78999999999984 0
9.79999999999984 0
9.80999999999984 0
9.81999999999984 0
9.82999999999983 0
9.83999999999983 0
9.84999999999983 0
9.85999999999983 0
9.86999999999983 0
9.87999999999983 0
9.88999999999983 0
9.89999999999983 0
9.90999999999983 0
9.91999999999983 0
9.92999999999983 0
9.93999999999983 0
9.94999999999983 0
9.95999999999983 0
9.96999999999983 0
9.97999999999983 0
9.98999999999983 0
9.99999999999983 0
10.0099999999998 0
10.0199999999998 0
10.0299999999998 0
10.0399999999998 0
10.0499999999998 0
10.0599999999998 0
10.0699999999998 0
10.0799999999998 0
10.0899999999998 0
10.0999999999998 0
10.1099999999998 0
10.1199999999998 0
10.1299999999998 0
10.1399999999998 0
10.1499999999998 0
10.1599999999998 0
10.1699999999998 0
10.1799999999998 0
10.1899999999998 0
10.1999999999998 0
10.2099999999998 0
10.2199999999998 0
10.2299999999998 0
10.2399999999998 0
10.2499999999998 0
10.2599999999998 0
10.2699999999998 0
10.2799999999998 0
10.2899999999998 0
10.2999999999998 0
10.3099999999998 0
10.3199999999998 0
10.3299999999998 0
10.3399999999998 0
10.3499999999998 0
10.3599999999998 0
10.3699999999998 0
10.3799999999998 0
10.3899999999998 0
10.3999999999998 0
10.4099999999998 0
10.4199999999998 0
10.4299999999998 0
10.4399999999998 0
10.4499999999998 0
10.4599999999998 0
10.4699999999998 0
10.4799999999998 0
10.4899999999998 0
10.4999999999998 0
10.5099999999998 0
10.5199999999998 0
10.5299999999998 0
10.5399999999998 0
10.5499999999998 0
10.5599999999998 0
10.5699999999998 0
10.5799999999998 0
10.5899999999998 0
10.5999999999998 0
10.6099999999998 0
10.6199999999998 0
10.6299999999998 0
10.6399999999998 0
10.6499999999998 0
10.6599999999998 0
10.6699999999998 0
10.6799999999998 0
10.6899999999998 0
10.6999999999998 0
10.7099999999998 0
10.7199999999998 0
10.7299999999998 0
10.7399999999998 0
10.7499999999998 0
10.7599999999998 0
10.7699999999998 0
10.7799999999998 0
10.7899999999998 0
10.7999999999998 0
10.8099999999998 0
10.8199999999998 0
10.8299999999998 0
10.8399999999998 0
10.8499999999998 0
10.8599999999998 0
10.8699999999998 0
10.8799999999998 0
10.8899999999998 0
10.8999999999998 0
10.9099999999998 0
10.9199999999998 0
10.9299999999998 0
10.9399999999998 0
10.9499999999998 0
10.9599999999998 0
10.9699999999998 0
10.9799999999998 0
10.9899999999998 0
10.9999999999998 0
11.0099999999998 0
11.0199999999998 0
11.0299999999998 0
11.0399999999998 0
11.0499999999998 0
11.0599999999998 0
11.0699999999998 0
11.0799999999998 0
11.0899999999998 0
11.0999999999998 0
11.1099999999998 0
11.1199999999998 0
11.1299999999998 0
11.1399999999998 0
11.1499999999998 0
11.1599999999998 0
11.1699999999998 0
11.1799999999998 0
11.1899999999998 0
11.1999999999998 0
11.2099999999998 0
11.2199999999998 0
11.2299999999998 0
11.2399999999998 0
11.2499999999998 0
11.2599999999998 0
11.2699999999998 0
11.2799999999998 0
11.2899999999998 0
11.2999999999998 0
11.3099999999998 0
11.3199999999998 0
11.3299999999998 0
11.3399999999998 0
11.3499999999998 0
11.3599999999998 0
11.3699999999998 0
11.3799999999998 0
11.3899999999998 0
11.3999999999998 0
11.4099999999998 0
11.4199999999998 0
11.4299999999998 0
11.4399999999998 0
11.4499999999998 0
11.4599999999998 0
11.4699999999998 0
11.4799999999998 0
11.4899999999998 0
11.4999999999998 0
11.5099999999998 0
11.5199999999998 0
11.5299999999998 0
11.5399999999998 0
11.5499999999998 0
11.5599999999998 0
11.5699999999998 0
11.5799999999998 0
11.5899999999998 0
11.5999999999998 0
11.6099999999998 0
11.6199999999998 0
11.6299999999998 0
11.6399999999998 0
11.6499999999998 0
11.6599999999998 0
11.6699999999998 0
11.6799999999998 0
11.6899999999998 0
11.6999999999998 0
11.7099999999998 0
11.7199999999998 0
11.7299999999998 0
11.7399999999998 0
11.7499999999998 0
11.7599999999998 0
11.7699999999998 0
11.7799999999998 0
11.7899999999998 0
11.7999999999998 0
11.8099999999998 0
11.8199999999998 0
11.8299999999998 0
11.8399999999998 0
11.8499999999998 0
11.8599999999998 0
11.8699999999998 0
11.8799999999998 0
11.8899999999998 0
11.8999999999998 0
11.9099999999998 0
11.9199999999998 0
11.9299999999998 0
11.9399999999998 0
11.9499999999998 0
11.9599999999998 0
11.9699999999998 0
11.9799999999998 0
11.9899999999998 0
11.9999999999998 0
12.0099999999998 0
12.0199999999998 0
12.0299999999998 0
12.0399999999998 0
12.0499999999998 0
12.0599999999998 0
12.0699999999998 0
12.0799999999998 0
12.0899999999998 0
12.0999999999998 0
12.1099999999998 0
12.1199999999998 0
12.1299999999998 0
12.1399999999998 0
12.1499999999998 0
12.1599999999998 0
12.1699999999998 0
12.1799999999998 0
12.1899999999998 0
12.1999999999998 0
12.2099999999998 0
12.2199999999998 0
12.2299999999998 0
12.2399999999998 0
12.2499999999998 0
12.2599999999998 0
12.2699999999998 0
12.2799999999998 0
12.2899999999998 0
12.2999999999998 0
12.3099999999998 0
12.3199999999998 0
12.3299999999998 0
12.3399999999998 0
12.3499999999998 0
12.3599999999998 0
12.3699999999998 0
12.3799999999998 0
12.3899999999998 0
12.3999999999998 0
12.4099999999998 0
12.4199999999998 0
12.4299999999998 0
12.4399999999998 0
12.4499999999998 0
12.4599999999998 0
12.4699999999998 0
12.4799999999998 0
12.4899999999998 0
12.4999999999998 0
12.5099999999998 0
12.5199999999998 0
12.5299999999998 0
12.5399999999998 0
12.5499999999998 0
12.5599999999998 0
12.5699999999998 0
12.5799999999998 0
12.5899999999998 0
12.5999999999998 0
12.6099999999998 0
12.6199999999998 0
12.6299999999998 0
12.6399999999998 0
12.6499999999998 0
12.6599999999998 0
12.6699999999998 0
12.6799999999998 0
12.6899999999998 0
12.6999999999998 0
12.7099999999998 0
12.7199999999998 0
12.7299999999998 0
12.7399999999998 0
12.7499999999998 0
12.7599999999998 0
12.7699999999998 0
12.7799999999998 0
12.7899999999998 0
12.7999999999998 0
12.8099999999998 0
12.8199999999998 0
12.8299999999998 0
12.8399999999998 0
12.8499999999998 0
12.8599999999998 0
12.8699999999998 0
12.8799999999998 0
12.8899999999998 0
12.8999999999998 0
12.9099999999998 0
12.9199999999998 0
12.9299999999998 0
12.9399999999998 0
12.9499999999998 0
12.9599999999998 0
12.9699999999998 0
12.9799999999998 0
12.9899999999998 0
12.9999999999998 0
13.0099999999998 0
13.0199999999998 0
13.0299999999998 0
13.0399999999998 0
13.0499999999998 0
13.0599999999998 0
13.0699999999998 0
13.0799999999998 0
13.0899999999998 0
13.0999999999998 0
13.1099999999998 0
13.1199999999998 0
13.1299999999998 0
13.1399999999998 0
13.1499999999998 0
13.1599999999998 0
13.1699999999998 0
13.1799999999998 0
13.1899999999998 0
13.1999999999998 0
13.2099999999998 0
13.2199999999998 0
13.2299999999998 0
13.2399999999998 0
13.2499999999998 0
13.2599999999998 0
13.2699999999998 0
13.2799999999998 0
13.2899999999998 0
13.2999999999998 0
13.3099999999998 0
13.3199999999998 0
13.3299999999998 0
13.3399999999998 0
13.3499999999998 0
13.3599999999998 0
13.3699999999998 0
13.3799999999998 0
13.3899999999998 0
13.3999999999998 0
13.4099999999998 0
13.4199999999998 0
13.4299999999998 0
13.4399999999998 0
13.4499999999998 0
13.4599999999998 0
13.4699999999998 0
13.4799999999998 0
13.4899999999998 0
13.4999999999998 0
13.5099999999998 0
13.5199999999998 0
13.5299999999998 0
13.5399999999998 0
13.5499999999998 0
13.5599999999998 0
13.5699999999998 0
13.5799999999998 0
13.5899999999998 0
13.5999999999998 0
13.6099999999998 0
13.6199999999998 0
13.6299999999998 0
13.6399999999998 0
13.6499999999998 0
13.6599999999998 0
13.6699999999998 0
13.6799999999998 0
13.6899999999998 0
13.6999999999998 0
13.7099999999998 0
13.7199999999998 0
13.7299999999998 0
13.7399999999998 0
13.7499999999998 0
13.7599999999998 0
13.7699999999998 0
13.7799999999998 0
13.7899999999998 0
13.7999999999998 0
13.8099999999998 0
13.8199999999997 0
13.8299999999997 0
13.8399999999997 0
13.8499999999997 0
13.8599999999997 0
13.8699999999997 0
13.8799999999997 0
13.8899999999997 0
13.8999999999997 0
13.9099999999997 0
13.9199999999997 0
13.9299999999997 0
13.9399999999997 0
13.9499999999997 0
13.9599999999997 0
13.9699999999997 0
13.9799999999997 0
13.9899999999997 0
13.9999999999997 0
14.0099999999997 0
14.0199999999997 0
14.0299999999997 0
14.0399999999997 0
14.0499999999997 0
14.0599999999997 0
14.0699999999997 0
14.0799999999997 0
14.0899999999997 0
14.0999999999997 0
14.1099999999997 0
14.1199999999997 0
14.1299999999997 0
14.1399999999997 0
14.1499999999997 0
14.1599999999997 0
14.1699999999997 0
14.1799999999997 0
14.1899999999997 0
14.1999999999997 0
14.2099999999997 0
14.2199999999997 0
14.2299999999997 0
14.2399999999997 0
14.2499999999997 0
14.2599999999997 0
14.2699999999997 0
14.2799999999997 0
14.2899999999997 0
14.2999999999997 0
14.3099999999997 0
14.3199999999997 0
14.3299999999997 0
14.3399999999997 0
14.3499999999997 0
14.3599999999997 0
14.3699999999997 0
14.3799999999997 0
14.3899999999997 0
14.3999999999997 0
14.4099999999997 0
14.4199999999997 0
14.4299999999997 0
14.4399999999997 0
14.4499999999997 0
14.4599999999997 0
14.4699999999997 0
14.4799999999997 0
14.4899999999997 0
14.4999999999997 0
14.5099999999997 0
14.5199999999997 0
14.5299999999997 0
14.5399999999997 0
14.5499999999997 0
14.5599999999997 0
14.5699999999997 0
14.5799999999997 0
14.5899999999997 0
14.5999999999997 0
14.6099999999997 0
14.6199999999997 0
14.6299999999997 0
14.6399999999997 0
14.6499999999997 0
14.6599999999997 0
14.6699999999997 0
14.6799999999997 0
14.6899999999997 0
14.6999999999997 0
14.7099999999997 0
14.7199999999997 0
14.7299999999997 0
14.7399999999997 0
14.7499999999997 0
14.7599999999997 0
14.7699999999997 0
14.7799999999997 0
14.7899999999997 0
14.7999999999997 0
14.8099999999997 0
14.8199999999997 0
14.8299999999997 0
14.8399999999997 0
14.8499999999997 0
14.8599999999997 0
14.8699999999997 0
14.8799999999997 0
14.8899999999997 0
14.8999999999997 0
14.9099999999997 0
14.9199999999997 0
14.9299999999997 0
14.9399999999997 0
14.9499999999997 0
14.9599999999997 0
14.9699999999997 0
14.9799999999997 0
14.9899999999997 0
14.9999999999997 1.65143933586237e-09
15.0099999999997 0.000251646628705991
15.0199999999997 0.000504813715529082
15.0299999999997 0.000759409495223724
15.0399999999997 0.0010153171453592
15.0499999999997 0.00127240160730084
15.0599999999997 0.00153051202950275
15.0699999999997 0.0017894842012412
15.0799999999997 0.00204914261052369
15.0899999999997 0.00230930228873294
15.0999999999997 0.00256977047864943
15.1099999999997 0.00283034812071255
15.1199999999997 0.00309083117851801
15.1299999999997 0.00335101182327691
15.1399999999997 0.00361067949426346
15.1499999999997 0.00386962185004272
15.1599999999997 0.00412762562336092
15.1699999999997 0.00438447739093926
15.1799999999997 0.00463996426797917
15.1899999999997 0.0048938745359416
15.1999999999997 0.00514599821107789
15.2099999999997 0.00539612756023699
15.2199999999997 0.00564405756964693
15.2299999999997 0.0058895863716423
15.2399999999997 0.00613251563367615
15.2499999999997 0.00637265091340354
15.2599999999997 0.0066098019831371
15.2699999999997 0.00684378312655668
15.2799999999997 0.00707441341018548
15.2899999999997 0.00730151693182289
15.2999999999997 0.00752492304784595
15.3099999999997 0.0077444665810465
15.3199999999997 0.00795998801045696
15.3299999999997 0.00817133364443574
15.3399999999997 0.0083783557781198
15.3499999999997 0.0085809128362127
15.3599999999997 0.00877886950195709
15.3699999999997 0.00897209683303236
15.3799999999997 0.00916047236503106
15.3899999999997 0.00934388020308567
15.3999999999997 0.0095222111021532
15.4099999999997 0.00969536253640429
15.4199999999997 0.00986323875811496
15.4299999999997 0.0100257508464171
15.4399999999997 0.010182816746761
15.4499999999997 0.0103343612993197
15.4599999999997 0.0104803162594725
15.4699999999997 0.0106206203089849
15.4799999999997 0.0107552190583435
15.4899999999997 0.0108840650409438
15.4999999999997 0.0110071176972975
15.5099999999997 0.011124343352648
15.5199999999997 0.0112357151860114
15.5299999999997 0.0113412131913552
15.5399999999997 0.0114408241310805
15.5499999999997 0.0115345414819743
15.5599999999997 0.0116223653737919
15.5699999999997 0.0117043025206279
15.5799999999997 0.0117803661451226
15.5899999999997 0.0118505758952864
15.5999999999997 0.0119149579143133
15.6099999999997 0.0119735446087437
15.6199999999997 0.0120263743730305
15.6299999999997 0.0120734916350397
15.6399999999997 0.0121149467423113
15.6499999999997 0.0121507958425362
15.6599999999997 0.0121811007584378
15.6699999999997 0.0122059288572524
15.6799999999997 0.0122253529150019
15.6899999999997 0.0122394509757588
15.6999999999997 0.0122483062073349
15.7099999999997 0.0122520067746912
15.7199999999997 0.0122506459870169
15.7299999999997 0.0122443214434493
15.7399999999997 0.0122331352385886
15.7499999999997 0.0122171937960714
15.7599999999997 0.0121966076985984
15.7699999999997 0.0121714915146394
15.7799999999997 0.0121419636227105
15.7899999999997 0.0121081460308243
15.7999999999997 0.0120701641950366
15.8099999999997 0.0120281468355185
15.8199999999997 0.0119822257500982
15.8299999999997 0.0119325356249662
15.8399999999997 0.0118792138450612
15.8499999999997 0.0118224003024289
15.8599999999997 0.0117622372033321
15.8699999999997 0.0116988688743441
15.8799999999997 0.0116324415676611
15.8899999999997 0.0115631032658651
15.8999999999997 0.0114910034863693
15.9099999999997 0.0114162930857789
15.9199999999997 0.0113391240643967
15.9299999999997 0.0112596493713743
15.9399999999997 0.0111780227107287
15.9499999999997 0.0110943983459273
15.9599999999997 0.0110089309081382
15.9699999999997 0.0109217752048475
15.9799999999997 0.0108330860299626
15.9899999999997 0.010743017975616
15.9999999999997 0.0106517252458821
16.0099999999997 0.0105593614726147
16.0199999999997 0.010466079533611
16.0299999999997 0.0103720313733037
16.0399999999997 0.0102773678261791
16.0499999999997 0.0101822384431166
16.0599999999997 0.0100867913207596
16.0699999999997 0.00999117282763884
16.0799999999997 0.00989552761682951
16.0899999999997 0.00979999839634916
16.0999999999997 0.009704725772254
16.1099999999997 0.00960984809572078
16.1199999999997 0.00951550131426175
16.1299999999997 0.00942181882721364
16.1399999999997 0.00932893134564011
16.1499999999997 0.00923696675677972
16.1599999999997 0.00914604999316607
16.1699999999997 0.00905630290654273
16.1799999999997 0.00896784414668765
16.1899999999997 0.00888078904525547
16.1999999999997 0.00879524950474298
16.2099999999997 0.00871133389267151
16.2199999999997 0.00862914694107768
16.2299999999997 0.00854878965139488
16.2399999999997 0.00847035920480175
16.2499999999997 0.00839394887810764
16.2599999999997 0.00831964796523675
16.2699999999997 0.00824754170436735
16.2799999999997 0.008177711210774
16.2899999999997 0.00811023341541516
16.2999999999997 0.00804518100930091
16.3099999999998 0.00798262239366929
16.3199999999998 0.00792262159736296
16.3299999999998 0.00786523821083156
16.3399999999998 0.00781052751081816
16.3499999999998 0.0077585403426363
16.3599999999998 0.00770932304518571
16.3699999999998 0.00766291745392498
16.3799999999998 0.00761936101587842
16.3899999999998 0.0075786866982003
16.3999999999998 0.00754092299039499
16.4099999999998 0.00750609391217645
16.4199999999998 0.00747421902746872
16.4299999999998 0.00744531346441154
16.4399999999998 0.00741938793560937
16.4499999999998 0.00739644877097179
16.4599999999998 0.00737649795300926
16.4699999999998 0.007359533157394
16.4799999999998 0.00734554780005938
16.4899999999998 0.00733453108596858
16.4999999999998 0.00732646806304129
16.5099999999998 0.00732133968568659
16.5199999999998 0.00731912288018435
16.5299999999998 0.00731979021864938
16.5399999999998 0.00732331066143274
16.5499999999998 0.00732964953970922
16.5599999999998 0.00733876845866182
16.5699999999998 0.00735062538578098
16.5799999999998 0.00736517474296724
16.5899999999998 0.00738236750233981
16.5999999999998 0.00740215128563695
16.6099999999998 0.00742447046708126
16.6199999999998 0.00744926627957387
16.6299999999998 0.00747647692407254
16.6399999999998 0.00750603768200355
16.6499999999998 0.00753788102880218
16.6599999999998 0.00757193675519417
16.6699999999998 0.00760813208638035
16.6799999999998 0.0076463918048768
16.6899999999998 0.00768663837567033
16.6999999999998 0.00772879207341752
16.7099999999998 0.00777277111151301
16.7199999999998 0.00781849177285081
16.7299999999998 0.00786586854210142
16.7399999999998 0.00791481423932581
16.7499999999998 0.00796524015474771
16.7599999999998 0.00801705618450379
16.7699999999998 0.00807017096719201
16.7799999999998 0.00812449202103842
16.7899999999998 0.00817992588150534
16.7999999999998 0.00823637823918295
16.8099999999998 0.00829375407769597
16.8199999999998 0.00835195781154313
16.8299999999998 0.00841089342367419
16.8399999999998 0.00847046460261877
16.8499999999998 0.00853057487899296
16.8599999999998 0.00859112776121231
16.8699999999998 0.00865202687024156
16.8799999999998 0.00871317607321355
16.8899999999998 0.00877447961575266
16.8999999999998 0.00883584225283994
16.9099999999998 0.00889716937806082
16.9199999999998 0.00895836715107769
16.9299999999998 0.00901934262317827
16.9399999999998 0.00908000386074412
16.9499999999999 0.00914026006649596
16.9599999999999 0.00920002169835777
16.9699999999999 0.00925920058579342
16.9799999999999 0.00931771004357101
16.9899999999999 0.00937546498272624
16.9999999999999 0.00943238201862552
17.0099999999999 0.00948837957600957
17.0199999999999 0.00954337799089992
17.0299999999999 0.00959729960925543
17.0399999999999 0.00965006888227058
17.0499999999999 0.00970161245821259
17.0599999999999 0.00975185927069889
17.0699999999999 0.00980074062332166
17.0799999999999 0.00984819027053153
17.0899999999999 0.00989414449469739
17.0999999999999 0.00993854217900834
17.1099999999999 0.00998132487436766
17.1199999999999 0.0100224368699089
17.1299999999999 0.0100618252514182
17.1399999999999 0.0100994399579998
17.1499999999999 0.0101352338347464
17.1599999999999 0.010169162681151
17.1699999999999 0.010201185295265
17.1799999999999 0.0102312635136516
17.1899999999999 0.0102593622462963
17.1999999999999 0.0102854495078829
17.2099999999999 0.0103094966656106
17.2199999999999 0.0103314780482327
17.2299999999999 0.0103513711504976
17.2399999999999 0.0103691566538226
17.2499999999999 0.0103848184356419
17.2599999999999 0.0103983435743842
17.2699999999999 0.0104097223501043
17.2799999999999 0.0104189482402235
17.2899999999999 0.0104260179116881
17.2999999999999 0.0104309312106239
17.3099999999999 0.010433691144361
17.3199999999999 0.0104343038608099
17.3299999999999 0.0104327786237182
17.3399999999999 0.0104291277838637
17.3499999999999 0.010423366746246
17.3599999999999 0.0104155139333574
17.3699999999999 0.0104055907445939
17.3799999999999 0.0103936215117882
17.3899999999999 0.0103796334511415
17.3999999999999 0.010363656611495
17.4099999999999 0.0103457238190677
17.4199999999999 0.0103258706187556
17.4299999999999 0.0103041352120879
17.4399999999999 0.0102805583919461
17.4499999999999 0.0102551834741514
17.4599999999999 0.0102280562260332
17.4699999999999 0.0101992247920941
17.4799999999999 0.0101687396168962
17.4899999999999 0.0101366533686173
17.4999999999999 0.0101030208546709
17.5099999999999 0.0100678989305911
17.5199999999999 0.010031346418222
17.5299999999999 0.00999342401676745
17.5399999999999 0.00995419442535841
17.5499999999999 0.00991372222207376
17.5599999999999 0.00987207301579076
17.5699999999999 0.00982931393566132
17.5799999999999 0.00978551352605559
17.59 0.00974074164203425
17.6 0.00969506934481952
17.61 0.00964856879699287
17.62 0.00960131315729651
17.63 0.00955337647500628
17.64 0.00950483358390064
17.65 0.00945575999588796
17.66 0.00940623179437596
17.67 0.00935632552748495
17.68 0.00930611810121651
17.69 0.00925568667269673
17.7 0.00920510854361704
17.71 0.0091544610539973
17.72 0.00910382147640391
17.73 0.0090532669107491
17.74 0.00900287417980236
17.75 0.00895271972554227
17.76 0.00890287950647805
17.77 0.00885342889606734
17.78 0.00880444258235642
17.79 0.00875599446896679
17.8 0.00870815757755021
17.81 0.00866100395183166
17.82 0.00861460456335817
17.83 0.00856902921931836
17.84 0.00852434647181321
17.85 0.00848062352917532
17.86 0.00843792616986219
17.87 0.00839631865839859
17.88 0.00835586366366929
17.89 0.00831662217965859
17.9 0.00827865344872927
17.91 0.00824201488753002
17.92 0.00820676201561655
17.93 0.00817294838686829
17.94 0.00814062552377808
17.95 0.0081098428546887
17.96 0.008080647654046
17.97 0.00805308498573426
17.98 0.00802719764955574
17.99 0.00800302613091186
18 0.00798060855373989
18.01 0.00795998063675453
18.02 0.00794117565304082
18.03 0.00792422439303986
18.04 0.00790915513096667
18.05 0.00789599359469532
18.06 0.007884762744288
18.07 0.00787548300733582
18.08 0.00786817230558084
18.09 0.00786284592711467
18.1 0.00785951651350903
18.11 0.00785819405046554
18.12 0.00785888586198128
18.13 0.00786159660802254
18.14 0.00786632828569432
18.15 0.00787308023388922
18.16 0.0078818491413946
18.17 0.00789262905843304
18.18 0.00790541141160651
18.19 0.00792018502237012
18.2 0.00793693612809392
18.21 0.00795564840723325
18.22 0.00797630300765289
18.2300000000001 0.00799887857802738
18.2400000000001 0.0080233513023219
18.2500000000001 0.00804969493736387
18.2600000000001 0.00807788085343385
18.2700000000001 0.00810787807780682
18.2800000000001 0.00813965334117442
18.2900000000001 0.00817317112687466
18.3000000000001 0.00820839371929159
18.3100000000001 0.00824528126474329
18.3200000000001 0.00828379182688984
18.3300000000001 0.00832388144549953
18.3400000000001 0.00836550419847991
18.3500000000001 0.00840861226620026
18.3600000000001 0.00845315599800352
18.3700000000001 0.00849908398080112
18.3800000000001 0.00854634272925967
18.3900000000001 0.00859487745008576
18.4000000000001 0.00864463196332135
18.4100000000001 0.00869554861500425
18.4200000000001 0.00874756835672299
18.4300000000001 0.00880063082636569
18.4400000000001 0.00885467443003247
18.4500000000001 0.00890963642505956
18.4600000000001 0.00896545300408579
18.4700000000001 0.00902205938008123
18.4800000000001 0.00907938987224893
18.4900000000001 0.00913737799270453
18.5000000000001 0.00919595653383022
18.5100000000001 0.00925505765621255
18.5200000000001 0.00931461297704609
18.5300000000001 0.0093745536588818
18.5400000000001 0.00943481049863445
18.5500000000001 0.00949531401672953
18.5600000000001 0.00955599454627747
18.5700000000001 0.00961678232216488
18.5800000000001 0.00967760756995045
18.5900000000001 0.00973840059445503
18.6000000000001 0.00979909186793499
18.6100000000001 0.00985961211772895
18.6200000000001 0.00991989241326907
18.6300000000001 0.0099798642523487
18.6400000000001 0.0100394596465402
18.6500000000001 0.0100986112056569
18.6600000000001 0.0101572522211559
18.6700000000001 0.0102153167483795
18.6800000000001 0.0102727396875354
18.6900000000001 0.0103294568633162
18.7000000000001 0.0103854051030624
18.7100000000001 0.0104405223133763
18.7200000000001 0.010494747555095
18.7300000000001 0.0105480211165294
18.7400000000001 0.0106002845848855
18.7500000000001 0.0106514809157833
18.7600000000001 0.0107015545039734
18.7700000000001 0.0107504512717014
18.7800000000001 0.0107981186801477
18.7900000000001 0.0108445058164064
18.8000000000001 0.0108895634523672
18.8100000000001 0.0109332441012502
18.8200000000001 0.0109755020717293
18.8300000000001 0.0110162935195867
18.8400000000001 0.0110555764968392
18.8500000000001 0.0110933109981115
18.8600000000001 0.0111294590043825
18.8700000000002 0.0111639845250438
18.8800000000002 0.011196853635944
18.8900000000002 0.0112280345151459
18.9000000000002 0.0112574974758847
18.9100000000002 0.0112852149966938
18.9200000000002 0.0113111617486668
18.9300000000002 0.0113353146198273
18.9400000000002 0.011357652814929
18.9500000000002 0.0113781579625169
18.9600000000002 0.0113968137392299
18.9700000000002 0.0114136061084451
18.9800000000002 0.0114285233303101
18.9900000000002 0.0114415559688302
19.0000000000002 0.0114526968960281
19.0100000000002 0.0114619412931809
19.0200000000002 0.0114692866491428
19.0300000000002 0.0114747327557681
19.0400000000002 0.0114782817004532
19.0500000000002 0.0114799378558183
19.0600000000002 0.0114797078665538
19.0700000000002 0.0114776006334591
19.0800000000002 0.0114736273019303
19.0900000000002 0.0114678012459182
19.1000000000002 0.011460138011133
19.1100000000002 0.0114506553081592
19.1200000000002 0.0114393729828809
19.1300000000002 0.0114263129848933
19.1400000000002 0.0114114993330061
19.1500000000002 0.0113949580779381
19.1600000000002 0.0113767172629327
19.1700000000002 0.0113568068820325
19.1800000000002 0.0113352588360762
19.1900000000002 0.0113121068864855
19.2000000000002 0.0112873866085124
19.2100000000002 0.0112611353394861
19.2200000000002 0.0112333921240472
19.2300000000002 0.011204197661799
19.2400000000002 0.0111735942511785
19.2500000000002 0.0111416257315778
19.2600000000002 0.0111083374238125
19.2700000000002 0.0110737760690405
19.2800000000002 0.0110379902059605
19.2900000000002 0.0110010294952066
19.3000000000002 0.0109629444057806
19.3100000000002 0.0109237865844939
19.3200000000002 0.0108836087831322
19.3300000000002 0.0108424647857294
19.3400000000002 0.0108004093356789
19.3500000000002 0.0107574980625336
19.3600000000002 0.0107137874084253
19.3700000000002 0.0106693345540832
19.3800000000002 0.0106241973444658
19.3900000000002 0.0105784342140428
19.4000000000002 0.010532104111781
19.4100000000002 0.010485266425897
19.4200000000002 0.0104379809084491
19.4300000000002 0.0103903075998455
19.4400000000002 0.0103423067533508
19.4500000000002 0.0102940387596735
19.4600000000002 0.0102455640717224
19.4700000000002 0.0101969431296173
19.4800000000002 0.0101482362860427
19.4900000000002 0.010099503732034
19.5000000000002 0.0100508054232799
19.5100000000003 0.0100022010070337
19.5200000000003 0.00995374974971808
19.5300000000003 0.00990551046530929
19.5400000000003 0.00985754144458774
19.5500000000003 0.00980990038533709
19.5600000000003 0.00976264432357603
19.5700000000003 0.00971582956590308
19.5800000000003 0.009669511623035
19.5900000000003 0.00962374514461584
19.6000000000003 0.00957858385537391
19.6100000000003 0.00953408049270002
19.6200000000003 0.00949028674571989
19.6300000000003 0.0094472531959311
19.6400000000003 0.00940502925947177
19.6500000000003 0.00936366313108792
19.6600000000003 0.00932320172986222
19.6700000000003 0.00928369064676567
19.6800000000003 0.00924517409409102
19.6900000000003 0.00920769485682417
19.7000000000003 0.00917129424600727
19.7100000000003 0.00913601205414524
19.7200000000003 0.00910188651270431
19.7300000000003 0.0090689542517493
19.7400000000003 0.00903725026176343
19.7500000000003 0.00900680785769263
19.7600000000003 0.00897765864525375
19.7700000000003 0.00894983248954452
19.7800000000003 0.00892335748599062
19.7900000000003 0.00889825993366497
19.8000000000003 0.00887456419601257
19.8100000000003 0.00885229264333562
19.8200000000003 0.00883146602239225
19.8300000000003 0.00881210319871706
19.8400000000003 0.00879422114160852
19.8500000000003 0.00877783491155475
19.8600000000003 0.00876295765009684
19.8700000000003 0.00874960057213476
19.8800000000003 0.00873777296067426
19.8900000000003 0.00872748216400975
19.9000000000003 0.00871873359533536
19.9100000000003 0.00871153073477372
19.9200000000003 0.00870587513380845
19.9300000000003 0.00870176642210417
19.9400000000003 0.00869920231669434
19.9500000000003 0.00869817863351469
19.9600000000003 0.00869868930125687
19.9700000000003 0.00870072637751426
19.9800000000003 0.00870428006825619
19.9900000000003 0.00870933874639673
20.0000000000003 0.00871588897526597
20.0100000000003 0.00872391553395347
20.0200000000003 0.00873340144444851
20.0300000000003 0.00874432800077804
20.0400000000003 0.00875667480018836
20.0500000000003 0.0087704197762988
20.0600000000003 0.00878553923417716
20.0700000000003 0.00880200788728699
20.0800000000003 0.00881979889581655
20.0900000000003 0.00883888390666509
20.1000000000003 0.00885923309796183
20.1100000000003 0.00888081522166353
20.1200000000003 0.00890359764889854
20.1300000000003 0.00892754641682059
20.1400000000003 0.0089526262768897
20.1500000000004 0.00897880074450025
20.1600000000004 0.00900603214987272
20.1700000000004 0.00903428123181832
20.1800000000004 0.00906350772736683
20.1900000000004 0.00909367088112138
20.2000000000004 0.009124728939616
20.2100000000004 0.00915663922941042
20.2200000000004 0.00918935822361663
20.2300000000004 0.00922284158940235
20.2400000000004 0.00925704424951913
20.2500000000004 0.00929192044390583
20.2600000000004 0.00932742379144464
20.2700000000004 0.00936350735190004
20.2800000000004 0.00940012368803853
20.2900000000004 0.00943722492790599
20.3000000000004 0.00947476282722477
20.3100000000004 0.00951268883186161
20.3200000000004 0.00955095414031119
20.3300000000004 0.00958950976613354
20.3400000000004 0.00962830660028127
20.3500000000004 0.00966729547324883
20.3600000000004 0.00970642721697542
20.3700000000004 0.00974565272643076
20.3800000000004 0.00978492302081323
20.3900000000004 0.00982418930429049
20.4000000000004 0.00986340302621056
20.4100000000004 0.00990251594071262
20.4200000000004 0.00994148016566892
20.4300000000004 0.0099802482408875
20.4400000000004 0.010018773185508
20.4500000000004 0.0100570085545231
20.4600000000004 0.0100949084943598
20.4700000000004 0.0101324277974554
20.4800000000004 0.010169521955765
20.4900000000004 0.0102061472131388
20.5000000000004 0.0102422606165088
20.5100000000004 0.0102778200658254
20.5200000000004 0.0103127843626889
20.5300000000004 0.0103471132576184
20.5400000000004 0.0103807674959065
20.5500000000004 0.0104137088620082
20.5600000000004 0.010445900222413
20.5700000000004 0.0104773055669554
20.5800000000004 0.010507890048516
20.5900000000004 0.0105376200210705
20.6000000000004 0.0105664630760464
20.6100000000004 0.0105943880769463
20.6200000000004 0.0106213651922018
20.6300000000004 0.0106473659262218
20.6400000000004 0.0106723631486027
20.6500000000004 0.0106963311214683
20.6600000000004 0.010719245524909
20.6700000000004 0.0107410834804927
20.6800000000004 0.0107618235728194
20.6900000000004 0.0107814458690918
20.7000000000004 0.0107999321960732
20.7100000000004 0.0108172657659752
20.7200000000004 0.0108334312185767
20.7300000000004 0.0108484147368904
20.7400000000004 0.0108622040571394
20.7500000000004 0.0108747884767295
20.7600000000004 0.0108861588602154
20.7700000000004 0.0108963076432639
20.7800000000004 0.0109052288346178
20.7900000000005 0.0109129180160701
20.8000000000005 0.0109193723404568
20.8100000000005 0.0109245905276814
20.8200000000005 0.0109285728587861
20.8300000000005 0.0109313211680861
20.8400000000005 0.0109328388333866
20.8500000000005 0.0109331307643044
20.8600000000005 0.0109322033887185
20.8700000000005 0.0109300646373753
20.8800000000005 0.0109267239266783
20.8900000000005 0.0109221921392846
20.9000000000005 0.0109164816036449
20.9100000000005 0.0109096060723629
20.9200000000005 0.0109015806973
20.9300000000005 0.010892422003782
20.9400000000005 0.0108821478632051
20.9500000000005 0.0108707774640819
20.9600000000005 0.0108583312817524
20.9700000000005 0.0108448310466209
20.9800000000005 0.0108302997100256
20.9900000000005 0.0108147614098066
21.0000000000005 0.0107982414343342
21.0100000000005 0.0107807661852536
21.0200000000005 0.0107623631390216
21.0300000000005 0.0107430608072996
21.0400000000005 0.0107228886962702
21.0500000000005 0.010701877375636
21.0600000000005 0.0106800590187101
21.0700000000005 0.010657465696062
21.0800000000005 0.0106341303219268
21.0900000000005 0.0106100866028589
21.1000000000005 0.0105853689871871
21.1100000000005 0.0105600126148461
21.1200000000005 0.010534053267333
21.1300000000005 0.0105075273176349
21.1400000000005 0.0104804716800483
21.1500000000005 0.0104529237598509
21.1600000000005 0.0104249214028172
21.1700000000005 0.0103965028445901
21.1800000000005 0.0103677066599328
21.1900000000005 0.0103385717118986
21.2000000000005 0.0103091371009591
21.2100000000005 0.0102794421141396
21.2200000000005 0.0102495261742126
21.2300000000005 0.0102194287890024
21.2400000000005 0.0101891895008582
21.2500000000005 0.0101588478363509
21.2600000000005 0.0101284432562505
21.2700000000005 0.0100980151058438
21.2800000000005 0.0100676025656472
21.2900000000005 0.0100372446025759
21.3000000000005 0.0100069799216243
21.3100000000005 0.00997684691811625
21.3200000000005 0.00994688363057913
21.3300000000005 0.00991712769430002
21.3400000000005 0.00988761629561607
21.3500000000005 0.00985838612699411
21.3600000000005 0.00982947334295136
21.3700000000005 0.00980091351686879
21.3800000000005 0.00977274159884612
21.3900000000005 0.00974499187441554
21.4000000000005 0.00971769792413127
21.4100000000005 0.00969089258457203
21.4200000000005 0.00966460791034254
21.4300000000006 0.00963887513724786
21.4400000000006 0.00961372464668224
21.4500000000006 0.00958918593127233
21.4600000000006 0.00956528756181334
21.4700000000006 0.00954205715553508
21.4800000000006 0.00951952134573327
21.4900000000006 0.00949770575279986
21.5000000000006 0.00947663495668503
21.5100000000006 0.00945633247082148
21.5200000000006 0.00943682071754127
21.5300000000006 0.00941812100501351
21.5400000000006 0.00940025350573103
21.5500000000006 0.00938323723657347
21.5600000000006 0.00936709003027402
21.5700000000006 0.00935182827974518
21.5800000000006 0.00933746730599168
21.5900000000006 0.00932402130705622
21.6000000000006 0.00931150324737822
21.6100000000006 0.00929992484713247
21.6200000000006 0.00928929657319743
21.6300000000006 0.00927962763175763
21.6400000000006 0.00927092596254231
21.6500000000006 0.00926319823469881
21.6600000000006 0.00925644984429748
21.6700000000006 0.00925068491346289
21.6800000000006 0.00924590629112425
21.6900000000006 0.00924211555537598
21.7000000000006 0.00923931301743724
21.7100000000006 0.00923749772719766
21.7200000000006 0.00923666748033427
21.7300000000006 0.00923681882698279
21.7400000000006 0.00923794708194464
21.7500000000006 0.00924004633640866
21.7600000000006 0.00924310947116468
21.7700000000006 0.00924712817127471
21.7800000000006 0.00925209294288211
21.7900000000006 0.00925799312945789
21.8000000000006 0.00926481692910399
21.8100000000006 0.00927255141586525
21.8200000000006 0.00928118256588767
21.8300000000006 0.00929069527700026
21.8400000000006 0.00930107339277862
21.8500000000006 0.00931229972785384
21.8600000000006 0.00932435609477089
21.8700000000006 0.0093372233315461
21.8800000000006 0.00935088133038266
21.8900000000006 0.00936530906744971
21.9000000000006 0.00938048463366766
21.9100000000006 0.00939638526644649
21.9200000000006 0.00941298738232468
21.9300000000006 0.00943026661045505
21.9400000000006 0.00944819697294348
21.9500000000006 0.00946675267575796
21.9600000000006 0.00948590734758848
21.9700000000006 0.00950563395269109
21.9800000000006 0.00952590483329339
21.9900000000006 0.00954669175126144
22.0000000000006 0.00956796592938478
22.0100000000006 0.00958969809249641
22.0200000000006 0.00961185850855774
22.0300000000006 0.00963441702978116
22.0400000000006 0.00965734313382653
22.0500000000006 0.00968060596508213
22.0600000000006 0.00970417437602359
22.0700000000007 0.00972801696863231
22.0800000000007 0.00975210213584614
22.0900000000007 0.00977639810300917
22.1000000000007 0.00980087296928318
22.1100000000007 0.00982549474897993
22.1200000000007 0.00985023141277126
22.1300000000007 0.00987505092873292
22.1400000000007 0.00989992130317609
22.1500000000007 0.00992481062122068
22.1600000000007 0.00994968708706414
22.1700000000007 0.00997451906389845
22.1800000000007 0.00999927511342933
22.1900000000007 0.010023924034951
22.2000000000007 0.0100484349039309
22.2100000000007 0.0100727771100591
22.2200000000007 0.0100969203947175
22.2300000000007 0.0101208348878243
22.2400000000007 0.0101444911440124
22.2500000000007 0.0101678601780975
22.2600000000007 0.0101909134997953
22.2700000000007 0.0102136231476478
22.2800000000007 0.0102359617220903
22.2900000000007 0.0102579024175407
22.3000000000007 0.0102794190540613
22.3100000000007 0.0103004861075687
22.3200000000007 0.0103210787392397
22.3300000000007 0.0103411728239073
22.3400000000007 0.0103607449774167
22.3500000000007 0.0103797725829088
22.3600000000007 0.0103982338160029
22.3700000000007 0.0104161076688498
22.3800000000007 0.0104333739730291
22.3900000000007 0.0104500134212627
22.4000000000007 0.010466007587923
22.4100000000007 0.0104813389483086
22.4200000000007 0.0104959908966665
22.4300000000007 0.0105099477629376
22.4400000000007 0.0105231948282029
22.4500000000007 0.0105357183388076
22.4600000000007 0.0105475057672667
22.4700000000007 0.0105585454649766
22.4800000000007 0.0105688266732543
22.4900000000007 0.0105783396442515
22.5000000000007 0.0105870756488404
22.5100000000007 0.0105950269831669
22.5200000000007 0.0106021869738682
22.5300000000007 0.0106085499819543
22.5400000000007 0.0106141114053562
22.5500000000007 0.0106188676801448
22.5600000000007 0.0106228162804245
22.5700000000007 0.0106259557169101
22.5800000000007 0.0106282855341939
22.5900000000007 0.0106298063067137
22.6000000000007 0.0106305196334339
22.6100000000007 0.0106304281312506
22.6200000000007 0.0106295354271384
22.6300000000007 0.0106278461490526
22.6400000000007 0.0106253659156062
22.6500000000007 0.0106221013245406
22.6600000000007 0.0106180599400197
22.6700000000007 0.0106132502784488
22.6800000000007 0.0106076817939113
22.6900000000007 0.0106013648620259
22.7000000000007 0.0105943107625377
22.7100000000008 0.010586531661027
22.7200000000008 0.0105780405895876
22.7300000000008 0.010568851426572
22.7400000000008 0.0105589788751494
22.7500000000008 0.0105484384410905
22.7600000000008 0.0105372464096614
22.7700000000008 0.0105254198216339
22.7800000000008 0.0105129764484617
22.7900000000008 0.0104999347666638
22.8000000000008 0.0104863139314332
22.8100000000008 0.0104721337488609
22.8200000000008 0.0104574146482558
22.8300000000008 0.0104421786143048
22.8400000000008 0.0104264470833908
22.8500000000008 0.010410242028045
22.8600000000008 0.0103935859477392
22.8700000000008 0.0103765018322282
22.8800000000008 0.010359013126409
22.8900000000008 0.010341143695736
22.9000000000008 0.0103229177919475
22.9100000000008 0.010304360018956
22.9200000000008 0.0102854952988126
22.9300000000008 0.0102663488376953
22.9400000000008 0.0102469460918975
22.9500000000008 0.0102273127338137
22.9600000000008 0.0102074746179268
22.9700000000008 0.0101874577468156
22.9800000000008 0.0101672882372055
22.9900000000008 0.0101469922860872
23.0000000000008 0.0101265961369372
23.0100000000008 0.0101061260460704
23.0200000000008 0.0100856082491618
23.0300000000008 0.0100650689279711
23.0400000000008 0.0100445341773072
23.0500000000008 0.0100240299722709
23.0600000000008 0.0100035821358116
23.0700000000008 0.00998321630663544
23.0800000000008 0.00996295790750252
23.0900000000008 0.0099428321139495
23.1000000000008 0.00992286382347404
23.1100000000008 0.00990307762521713
23.1200000000008 0.00988349777017835
23.1300000000008 0.00986414814199902
23.1400000000008 0.00984505222834696
23.1500000000008 0.00982623309293612
23.1600000000008 0.0098077133482132
23.1700000000008 0.00978951512888712
23.1800000000008 0.00977166006581299
23.1900000000008 0.00975416926091564
23.2000000000008 0.00973706326288694
23.2100000000008 0.00972036204366223
23.2200000000008 0.0097040849757389
23.2300000000008 0.00968825081036246
23.2400000000008 0.009672877656605
23.2500000000008 0.00965798296135987
23.2600000000008 0.00964358349027506
23.2700000000008 0.00962969530964748
23.2800000000008 0.00961633376929921
23.2900000000008 0.00960351348645585
23.3000000000008 0.00959124833064732
23.3100000000008 0.00957955140965017
23.3200000000008 0.00956843505649135
23.3300000000008 0.00955791081753316
23.3400000000008 0.00954798932208879
23.3500000000009 0.00953868025048325
23.3600000000009 0.00952999270158221
23.3700000000009 0.00952193494863274
23.3800000000009 0.00951451443239214
23.3900000000009 0.00950773775534577
23.4000000000009 0.00950161067701923
23.4100000000009 0.00949613811038464
23.4200000000009 0.00949132411935952
23.4300000000009 0.00948717191739562
23.4400000000009 0.00948368386715375
23.4500000000009 0.00948086148125935
23.4600000000009 0.00947870542413237
23.4700000000009 0.00947721551488367
23.4800000000009 0.00947639073126906
23.4900000000009 0.00947622921469042
23.5000000000009 0.00947672827623284
23.5100000000009 0.00947788440372451
23.5200000000009 0.0094796932698057
23.5300000000009 0.00948214974099146
23.5400000000009 0.00948524788770939
23.5500000000009 0.00948898099529452
23.5600000000009 0.00949334157602143
23.5700000000009 0.0094983213818399
23.5800000000009 0.00950391141811373
23.5900000000009 0.00951010195833621
23.6000000000009 0.00951688255962727
23.6100000000009 0.00952424207905454
23.6200000000009 0.00953216869077859
23.6300000000009 0.00954064990395924
23.6400000000009 0.00954967258148805
23.6500000000009 0.0095592229596207
23.6600000000009 0.00956928666787995
23.6700000000009 0.0095798487498105
23.6800000000009 0.00959089368439685
23.6900000000009 0.00960240540809891
23.7000000000009 0.00961436733748186
23.7100000000009 0.00962676172322793
23.7200000000009 0.00963957079373246
23.7300000000009 0.00965277670592125
23.7400000000009 0.00966636116921041
23.7500000000009 0.00968030547640773
23.7600000000009 0.00969459053342945
23.7700000000009 0.00970919688802113
23.7800000000009 0.00972410475840367
23.7900000000009 0.00973929406187654
23.8000000000009 0.00975474444291464
23.8100000000009 0.00977043530119161
23.8200000000009 0.00978634581956559
23.8300000000009 0.00980245499204364
23.8400000000009 0.00981874165172774
23.8500000000009 0.00983518449873547
23.8600000000009 0.00985176212808181
23.8700000000009 0.00986845305761256
23.8800000000009 0.00988523575592409
23.8900000000009 0.00990208866963825
23.9000000000009 0.00991899025094706
23.9100000000009 0.0099359189849697
23.9200000000009 0.00995285341692
23.9300000000009 0.00996977217905526
23.9400000000009 0.00998665401737665
23.9500000000009 0.0100034778180513
23.9600000000009 0.010020222633526
23.9700000000009 0.0100368677083029
23.9800000000009 0.010053392504348
23.990000000001 0.0100697767261024
24.000000000001 0.010086000345067
24.010000000001 0.0101020436239344
24.020000000001 0.0101178871402372
24.030000000001 0.0101335118094879
24.040000000001 0.0101488989077813
24.050000000001 0.0101640300937772
24.060000000001 0.010178887430173
24.070000000001 0.0101934534046664
24.080000000001 0.0102077109500794
24.090000000001 0.0102216434639045
24.100000000001 0.0102352348271693
24.110000000001 0.0102484694225985
24.120000000001 0.0102613321520522
24.130000000001 0.0102738084532201
24.140000000001 0.0102858843155539
24.150000000001 0.0102975462954169
24.160000000001 0.0103087815304353
24.170000000001 0.0103195777530314
24.180000000001 0.0103299233031235
24.190000000001 0.0103398071399738
24.200000000001 0.0103492188531686
24.210000000001 0.010358148672711
24.220000000001 0.0103665875183114
24.230000000001 0.0103745272609875
24.240000000001 0.0103819601301714
24.250000000001 0.0103888790291817
24.260000000001 0.0103952775412523
24.270000000001 0.0104011499346664
24.280000000001 0.0104064911669935
24.290000000001 0.0104112968884289
24.300000000001 0.0104155634442365
24.310000000001 0.0104192878762982
24.320000000001 0.0104224679237704
24.330000000001 0.0104251020228541
24.340000000001 0.010427189305682
24.350000000001 0.0104287295983295
24.360000000001 0.0104297234179553
24.370000000001 0.0104301719690816
24.380000000001 0.0104300771390216
24.390000000001 0.0104294414924639
24.400000000001 0.0104282682652271
24.410000000001 0.0104265613571937
24.420000000001 0.0104243253244387
24.430000000001 0.0104215653705715
24.440000000001 0.0104182873372875
24.450000000001 0.0104144976941551
24.460000000001 0.0104102035275801
24.470000000001 0.0104054125291724
24.480000000001 0.0104001329833688
24.490000000001 0.010394373754352
24.500000000001 0.0103881442722967
24.510000000001 0.0103814545189671
24.520000000001 0.0103743150134524
24.530000000001 0.0103667367954649
24.540000000001 0.0103587314089789
24.550000000001 0.0103503108858692
24.560000000001 0.0103414877285711
24.570000000001 0.0103322748922107
24.580000000001 0.0103226857662206
24.590000000001 0.0103127343443087
24.600000000001 0.010302435666829
24.610000000001 0.0102918041353853
24.620000000001 0.0102808545371026
24.6300000000011 0.0102696020169903
24.6400000000011 0.0102580620520147
24.6500000000011 0.0102462504262508
24.6600000000011 0.010234183207253
24.6700000000011 0.0102218767213944
24.6800000000011 0.0102093475302428
24.6900000000011 0.0101966124079441
24.7000000000011 0.010183688318032
24.7100000000011 0.0101705923903913
24.7200000000011 0.0101573418982103
24.7300000000011 0.0101439542348442
24.7400000000011 0.010130446890875
24.7500000000011 0.0101168374312327
24.7600000000011 0.0101031434723857
24.7700000000011 0.0100893826596206
24.7800000000011 0.0100755726444276
24.7900000000011 0.0100617310620137
24.8000000000011 0.0100478755089654
24.8100000000011 0.0100340235210835
24.8200000000011 0.0100201925514137
24.8300000000011 0.010006399948497
24.8400000000011 0.00999266293486382
24.8500000000011 0.00997899858579565
24.8600000000011 0.009965423808379
24.8700000000011 0.00995195532087466
24.8800000000011 0.00993860963242639
24.8900000000011 0.00992540302313237
24.9000000000011 0.009912351524502
24.9100000000011 0.00989947090032075
24.9200000000011 0.00988677662794516
24.9300000000011 0.00987428388007082
24.9400000000011 0.00986200750701074
24.9500000000011 0.0098499620192377
24.9600000000011 0.0098381615706969
24.9700000000011 0.00982661994256273
24.9800000000011 0.0098153505275461
24.9900000000011 0.00980436631477041
25.0000000000011 0.00979367987523358
25.0100000000011 0.009783303347873
25.0200000000011 0.00977324842624972
25.0300000000011 0.00976352634586759
25.0400000000011 0.00975414787214288
25.0500000000011 0.00974512328903944
25.0600000000011 0.00973646238838411
25.0700000000011 0.00972817445987749
25.0800000000011 0.00972026828181518
25.0900000000011 0.00971275211253871
25.1000000000011 0.0097056336826479
25.1100000000011 0.00969891989949683
25.1200000000011 0.00969261731440188
25.1300000000011 0.0096867320068308
25.1400000000011 0.00968126950227361
25.1500000000011 0.00967623476768608
25.1600000000011 0.0096716322076681
25.1700000000011 0.00966746566137878
25.1800000000011 0.00966373840018739
25.1900000000011 0.00966045312605865
25.2000000000011 0.00965761197067
25.2100000000011 0.00965521649525789
25.2200000000011 0.00965326769118902
25.2300000000011 0.0096517659812521
25.2400000000011 0.00965071122166443
25.2500000000011 0.00965010270478712
25.2600000000011 0.00964993916254192
25.2700000000012 0.00965021877052159
25.2800000000012 0.00965093915278524
25.2900000000012 0.00965209738732918
25.3000000000012 0.00965369001222283
25.3100000000012 0.00965571303239783
25.3200000000012 0.00965816192707837
25.3300000000012 0.00966103165784997
25.3400000000012 0.00966431667733536
25.3500000000012 0.00966801093847632
25.3600000000012 0.00967210790440386
25.3700000000012 0.00967660055888154
25.3800000000012 0.00968148141730656
25.3900000000012 0.00968674253825242
25.4000000000012 0.00969237553553709
25.4100000000012 0.00969837159080038
25.4200000000012 0.00970472146657451
25.4300000000012 0.00971141551983278
25.4400000000012 0.00971844371600258
25.4500000000012 0.00972579553489451
25.4600000000012 0.00973346027922141
25.4700000000012 0.00974142682088854
25.4800000000012 0.00974968294504866
25.4900000000012 0.00975821676654305
25.5000000000012 0.00976701645285161
25.5100000000012 0.00977606966478282
25.5200000000012 0.00978536390104026
25.5300000000012 0.00979488639652059
25.5400000000012 0.00980462414297269
25.5500000000012 0.00981456390904991
25.5600000000012 0.00982469225996428
25.5700000000012 0.00983499557687757
25.5800000000012 0.00984546007611651
25.5900000000012 0.00985607182826726
25.6000000000012 0.00986681677718213
25.6100000000012 0.00987768075891704
25.6200000000012 0.00988864952060702
25.6300000000012 0.0098997087392798
25.6400000000012 0.00991084404060189
25.6500000000012 0.00992204101754767
25.6600000000012 0.00993328524897895
25.6700000000012 0.00994456231812053
25.6800000000012 0.00995585783091538
25.6900000000012 0.00996715743424159
25.7000000000012 0.00997844683396339
25.7100000000012 0.00998971181283392
25.7200000000012 0.0100009382481934
25.7300000000012 0.0100121121294487
25.7400000000012 0.0100232195753215
25.7500000000012 0.010034246850845
25.7600000000012 0.0100451803840913
25.7700000000012 0.0100560067826094
25.7800000000012 0.0100667128495547
25.7900000000012 0.0100772855994931
25.8000000000012 0.0100877122738604
25.8100000000012 0.0100979803560591
25.8200000000012 0.0101080775861035
25.8300000000012 0.0101179919750524
25.8400000000012 0.0101277118189016
25.8500000000012 0.0101372257120416
25.8600000000012 0.0101465225602927
25.8700000000012 0.0101555915934866
25.8800000000012 0.010164422377564
25.8900000000012 0.0101730048261838
25.9000000000012 0.0101813292118329
25.9100000000013 0.0101893861764204
25.9200000000013 0.0101971667413429
25.9300000000013 0.0102046623170073
25.9400000000013 0.0102118647117994
25.9500000000013 0.0102187661404839
25.9600000000013 0.0102253592320237
25.9700000000013 0.0102316370368027
25.9800000000013 0.0102375930332384
25.9900000000013 0.0102432212363347
26.0000000000013 0.0102485162735125
26.0100000000013 0.0102534729690587
26.0200000000013 0.010258086597836
26.0300000000013 0.0102623528893076
26.0400000000013 0.0102662680309581
26.0500000000013 0.0102698286711099
26.0600000000013 0.0102730319211349
26.0700000000013 0.0102758753570632
26.0800000000013 0.0102783570205911
26.0900000000013 0.0102804754194903
26.1000000000013 0.0102822295274215
26.1100000000013 0.0102836187831565
26.1200000000013 0.0102846430892126
26.1300000000013 0.0102853028099046
26.1400000000013 0.0102855987688195
26.1500000000013 0.0102855322457212
26.1600000000013 0.0102851049728908
26.1700000000013 0.0102843191309112
26.1800000000013 0.0102831773439025
26.1900000000013 0.0102816826742192
26.2000000000013 0.0102798386166198
26.2100000000013 0.010277649091908
26.2200000000013 0.0102751184400707
26.2300000000013 0.0102722514129162
26.2400000000013 0.0102690531662243
26.2500000000013 0.0102655292514217
26.2600000000013 0.0102616856067926
26.2700000000013 0.0102575285482402
26.2800000000013 0.0102530648280868
26.2900000000013 0.0102483014361763
26.3000000000013 0.0102432457645038
26.3100000000013 0.0102379055421013
26.3200000000013 0.010232288823923
26.3300000000013 0.0102264039794137
26.3400000000013 0.0102202596807942
26.3500000000013 0.0102138648910992
26.3600000000013 0.010207229172586
26.3700000000013 0.0102003626594726
26.3800000000013 0.0101932749373773
26.3900000000013 0.010185975855259
26.4000000000013 0.0101784755055269
26.4100000000013 0.0101707842056414
26.4200000000013 0.0101629124806876
26.4300000000013 0.010154870981473
26.4400000000013 0.010146670682978
26.4500000000013 0.0101383226308581
26.4600000000013 0.0101298380138481
26.4700000000013 0.0101212281478995
26.4800000000013 0.0101125044604458
26.4900000000013 0.0101036784747642
26.5000000000013 0.0100947617944163
26.5100000000013 0.0100857660877601
26.5200000000013 0.0100767030725311
26.5300000000013 0.0100675845004961
26.5400000000013 0.0100584221421862
26.5500000000014 0.0100492277717173
26.5600000000014 0.0100400131517112
26.5700000000014 0.0100307900183277
26.5800000000014 0.0100215700664218
26.5900000000014 0.010012364934842
26.6000000000014 0.0100031861918812
26.6100000000014 0.00999404532089858
26.6200000000014 0.00998495370612556
26.6300000000014 0.0099759226186725
26.6400000000014 0.00996696320275091
26.6500000000014 0.00995808646212674
26.6600000000014 0.00994930324682008
26.6700000000014 0.00994062424006604
26.6800000000014 0.00993205994555193
26.6900000000014 0.009923620674945
26.7000000000014 0.00991531653574439
26.7100000000014 0.00990715741945688
26.7200000000014 0.00989915298998437
26.7300000000014 0.0098913126725066
26.7400000000014 0.00988364564266972
26.7500000000014 0.00987616081614628
26.7600000000014 0.00986886683857872
26.7700000000014 0.00986177207591845
26.7800000000014 0.00985488460517193
26.7900000000014 0.00984821220556545
26.8000000000014 0.00984176235013961
26.8100000000014 0.00983554219778476
26.8200000000014 0.00982955858572849
26.8300000000014 0.0098238180224865
26.8400000000014 0.0098183266812887
26.8500000000014 0.0098130903939928
26.8600000000014 0.00980811464549903
26.8700000000014 0.00980340456868126
26.8800000000014 0.00979896463868002
26.8900000000014 0.00979479918091785
26.9000000000014 0.00979091221675561
26.9100000000014 0.0097873073978167
26.9200000000014 0.00978398800294756
26.9300000000014 0.00978095693567513
26.9400000000014 0.00977821672216099
26.9500000000014 0.00977576950965108
26.9600000000014 0.0097736170654192
26.9700000000014 0.00977176077620236
26.9800000000014 0.00977020164812525
26.9900000000014 0.00976894030711107
27.0000000000014 0.00976797699977502
27.0100000000014 0.00976731159479668
27.0200000000014 0.00976694358476669
27.0300000000014 0.00976687208850287
27.0400000000014 0.00976709585383027
27.0500000000014 0.00976761326081924
27.0600000000014 0.00976842232547506
27.0700000000014 0.00976952070387224
27.0800000000014 0.00977090569672527
27.0900000000014 0.00977257421249106
27.1000000000014 0.00977452288961405
27.1100000000014 0.00977674799332987
27.1200000000014 0.00977924545582641
27.1300000000014 0.00978201088284082
27.1400000000014 0.00978503956063095
27.1500000000014 0.00978832646330878
27.1600000000014 0.00979186626052233
27.1700000000014 0.00979565332547132
27.1800000000014 0.00979968174324051
27.1900000000015 0.00980394531943272
27.2000000000015 0.00980843758908115
27.2100000000015 0.00981315182581709
27.2200000000015 0.00981808105126456
27.2300000000015 0.00982321804462606
27.2400000000015 0.00982855535214792
27.2500000000015 0.00983408444009096
27.2600000000015 0.00983979742866301
27.2700000000015 0.00984568639074186
27.2800000000015 0.00985174318919538
27.2900000000015 0.00985795949332604
27.3000000000015 0.00986432679410299
27.3100000000015 0.00987083641858982
27.3200000000015 0.00987747954383122
27.3300000000015 0.00988424721037087
27.3400000000015 0.00989113033551488
27.3500000000015 0.00989811972641662
27.3600000000015 0.00990520609303245
27.3700000000015 0.00991238012432817
27.3800000000015 0.00991963229782435
27.3900000000015 0.00992695311077474
27.4000000000015 0.00993433301394965
27.4100000000015 0.00994176242427784
27.4200000000015 0.00994923173741268
27.4300000000015 0.00995673134021863
27.4400000000015 0.00996425162317151
27.4500000000015 0.00997178299266466
27.4600000000015 0.00997931588321133
27.4700000000015 0.0099868407695331
27.4800000000015 0.00999434817852315
27.4900000000015 0.0100018287010729
27.5000000000015 0.0100092730037503
27.5100000000015 0.0100166718403166
27.5200000000015 0.0100240160630717
27.5300000000015 0.0100312966340128
27.5400000000015 0.0100385046357965
27.5500000000015 0.0100456312824909
27.5600000000015 0.0100526679301065
27.5700000000015 0.0100596060868933
27.5800000000015 0.0100664374233928
27.5900000000015 0.0100731537821855
27.6000000000015 0.0100797471874918
27.6100000000015 0.0100862098544103
27.6200000000015 0.0100925341978618
27.6300000000015 0.0100987128412494
27.6400000000015 0.0101047386248068
27.6500000000015 0.0101106046136271
27.6600000000015 0.0101163041053609
27.6700000000015 0.0101218306375756
27.6800000000015 0.0101271779947642
27.6900000000015 0.010132340214996
27.7000000000015 0.0101373115961986
27.7100000000015 0.0101420867020618
27.7200000000015 0.010146660367552
27.7300000000015 0.0101510276823401
27.7400000000015 0.0101551840646113
27.7500000000015 0.0101591251927502
27.7600000000015 0.0101628471724223
27.7700000000015 0.0101663465062723
27.7800000000015 0.0101696197784313
27.7900000000015 0.0101726638737759
27.8000000000015 0.0101754759805896
27.8100000000015 0.0101780535928202
27.8200000000015 0.0101803945119311
27.8300000000016 0.0101824968483482
27.8400000000016 0.010184359022503
27.8500000000016 0.0101859797654746
27.8600000000016 0.0101873581192307
27.8700000000016 0.0101884934364713
27.8800000000016 0.0101893853800778
27.8900000000016 0.0101900339221694
27.9000000000016 0.0101904393427719
27.9100000000016 0.0101906022281016
27.9200000000016 0.0101905234684705
27.9300000000016 0.010190204255815
27.9400000000016 0.0101896460808566
27.9500000000016 0.0101888507298975
27.9600000000016 0.0101878202812603
27.9700000000016 0.0101865571013776
27.9800000000016 0.0101850638405346
27.9900000000016 0.0101833434282795
28.0000000000016 0.0101813990685066
28.0100000000016 0.0101792342342211
28.0200000000016 0.0101768526619961
28.0300000000016 0.0101742583461311
28.0400000000016 0.010171455532525
28.0500000000016 0.0101684487122752
28.0600000000016 0.0101652426150163
28.0700000000016 0.0101618422020157
28.0800000000016 0.010158252659042
28.0900000000016 0.0101544793890296
28.1000000000016 0.0101505280045648
28.1100000000016 0.0101464043202256
28.1200000000016 0.0101421143448205
28.1300000000016 0.0101376646619812
28.1400000000016 0.0101330621810355
28.1500000000016 0.0101283132694756
28.1600000000016 0.0101234244772278
28.1700000000016 0.0101184025215722
28.1800000000016 0.0101132542735502
28.1900000000016 0.0101079867453585
28.2000000000016 0.0101026070784097
28.2100000000016 0.0100971225318501
28.2200000000016 0.0100915404713945
28.2300000000016 0.0100858683583862
28.2400000000016 0.0100801137390188
28.2500000000016 0.0100742842336786
28.2600000000016 0.0100683875263803
28.2700000000016 0.010062431354279
28.2800000000016 0.0100564234972489
28.2900000000016 0.0100503717675238
28.3000000000016 0.0100442839994003
28.3100000000016 0.0100381680390043
28.3200000000016 0.0100320317341258
28.3300000000016 0.0100258829241294
28.3400000000016 0.0100197294299452
28.3500000000016 0.0100135790441509
28.3600000000016 0.0100074395211515
28.3700000000016 0.0100013185674682
28.3800000000016 0.00999522383214354
28.3900000000016 0.00998916284978715
28.4000000000016 0.00998314317954681
28.4100000000016 0.00997717224092939
28.4200000000016 0.00997125735908177
28.4300000000016 0.00996540575591918
28.4400000000016 0.00995962454143281
28.4500000000016 0.00995392070518602
28.4600000000016 0.00994830110800872
28.4700000000017 0.00994277247391352
28.4800000000017 0.00993734142146279
28.4900000000017 0.0099320143344301
28.5000000000017 0.00992679748037743
28.5100000000017 0.00992169695980021
28.5200000000017 0.0099167186992061
28.5300000000017 0.00991186844445192
28.5400000000017 0.00990715175434695
28.5500000000017 0.00990257399453059
28.5600000000017 0.00989814033163257
28.5700000000017 0.00989385572772342
28.5800000000017 0.00988972493506351
28.5900000000017 0.00988575249115856
28.6000000000017 0.00988194271413014
28.6100000000017 0.00987829969841003
28.6200000000017 0.00987482731076817
28.6300000000017 0.00987152918668468
28.6400000000017 0.00986840872707846
28.6500000000017 0.00986546874568571
28.6600000000017 0.00986271212050616
28.6700000000017 0.00986014150364579
28.6800000000017 0.00985775930056933
28.6900000000017 0.009855567668094
28.7000000000017 0.00985356851271601
28.7100000000017 0.00985176348926924
28.7200000000017 0.00985015399991511
28.7300000000017 0.00984874119346234
28.7400000000017 0.00984752596501489
28.7500000000017 0.00984650895594636
28.7600000000017 0.00984569055419845
28.7700000000017 0.00984507089490093
28.7800000000017 0.00984464986131043
28.7900000000017 0.00984442708606454
28.8000000000017 0.00984440195274784
28.8100000000017 0.00984457359776569
28.8200000000017 0.00984494091252165
28.8300000000017 0.00984550254589356
28.8400000000017 0.00984625690700338
28.8500000000017 0.00984720216827441
28.8600000000017 0.0098483362687713
28.8700000000017 0.0098496569178169
28.8800000000017 0.00985116159887566
28.8900000000017 0.00985284757369909
28.9000000000017 0.00985471188672403
28.9100000000017 0.00985675136971523
28.9200000000017 0.00985896264664217
28.9300000000017 0.00986134213877967
28.9400000000017 0.00986388607001985
28.9500000000017 0.0098665904723818
28.9600000000017 0.0098694511917026
28.9700000000017 0.0098724638934903
28.9800000000017 0.00987562406891471
28.9900000000017 0.00987892704164569
29.0000000000017 0.0098823679750812
29.0100000000017 0.00988594187785231
29.0200000000017 0.00988964268933796
29.0300000000017 0.00989346521927964
29.0400000000017 0.00989740421223726
29.0500000000017 0.00990145426408929
29.0600000000017 0.00990560983716573
29.0700000000017 0.00990986527180383
29.0800000000017 0.00991421479676473
29.0900000000017 0.00991865253909995
29.1000000000017 0.009923172533634
29.1100000000018 0.00992776873217499
29.1200000000018 0.00993243501252876
29.1300000000018 0.00993716518736795
29.1400000000018 0.00994195301299062
29.1500000000018 0.00994679219799125
29.1600000000018 0.00995167641185871
29.1700000000018 0.00995659929350985
29.1800000000018 0.00996155445976273
29.1900000000018 0.00996653551375065
29.2000000000018 0.00997153605327545
29.2100000000018 0.00997654967909687
29.2200000000018 0.00998157000315336
29.2300000000018 0.00998659065670886
29.2400000000018 0.00999160529841916
29.2500000000018 0.00999660762231085
29.2600000000018 0.0100015913656657
29.2700000000018 0.0100065503168028
29.2800000000018 0.0100114783227506
29.2900000000018 0.0100163692968009
29.3000000000018 0.0100212172259371
29.3100000000018 0.010026016178128
29.3200000000018 0.0100307603094803
29.3300000000018 0.0100354438712405
29.3400000000018 0.0100400612166397
29.3500000000018 0.0100446068075724
29.3600000000018 0.010049075221066
29.3700000000018 0.0100534611556674
29.3800000000018 0.0100577594375622
29.3900000000018 0.0100619650265108
29.4000000000018 0.010066073021588
29.4100000000018 0.0100700786667144
29.4200000000018 0.0100739773559709
29.4300000000018 0.0100777646386907
29.4400000000018 0.0100814362243221
29.4500000000018 0.0100849879870549
29.4600000000018 0.0100884159702039
29.4700000000018 0.0100917163903419
29.4800000000018 0.0100948856411766
29.4900000000018 0.010097920297162
29.5000000000018 0.0101008171168369
29.5100000000018 0.0101035730458806
29.5200000000018 0.0101061852198765
29.5300000000018 0.0101086511549967
29.5400000000018 0.010110968568231
29.5500000000018 0.0101131352065278
29.5600000000018 0.010115149017108
29.5700000000018 0.010117008149238
29.5800000000018 0.0101187109557305
29.5900000000018 0.0101202559941718
29.6000000000018 0.0101216420278775
29.6100000000018 0.0101228680265786
29.6200000000018 0.0101239331668366
29.6300000000018 0.0101248368321925
29.6400000000018 0.0101255786130492
29.6500000000018 0.0101261583062906
29.6600000000018 0.0101265759146388
29.6700000000018 0.0101268316457534
29.6800000000018 0.0101269259110743
29.6900000000018 0.0101268593244123
29.7000000000018 0.0101266327002905
29.7100000000018 0.010126247052041
29.7200000000018 0.01012570358966
29.7300000000018 0.0101250037174268
29.7400000000018 0.0101241490312927
29.7500000000019 0.0101231413160405
29.7600000000019 0.0101219825422263
29.7700000000019 0.0101206748629069
29.7800000000019 0.0101192206101598
29.7900000000019 0.0101176222914033
29.8000000000019 0.0101158825855264
29.8100000000019 0.0101140043388358
29.8200000000019 0.010111990560747
29.8300000000019 0.0101098444188529
29.8400000000019 0.0101075692346206
29.8500000000019 0.0101051684787212
29.8600000000019 0.0101026457661442
29.8700000000019 0.0101000048511918
29.8800000000019 0.0100972496223719
29.8900000000019 0.010094384097213
29.9000000000019 0.0100914128997134
29.9100000000019 0.0100883407265318
29.9200000000019 0.0100851717779552
29.9300000000019 0.0100819103842686
29.9400000000019 0.0100785609934611
29.9500000000019 0.0100751281605708
29.9600000000019 0.0100716165381206
29.9700000000019 0.0100680308679005
29.9800000000019 0.0100643759723909
29.9900000000019 0.0100606567461594
30.0000000000019 0.0100568781482715
};
\addlegendentry{PI};
\addplot [ultra thick, blue!20!gray]
table {%
0 0
0.01 -0.000235229022800922
0.02 -0.000333463624119759
0.03 -0.000407897904515266
0.04 -0.000426591299474239
0.05 -0.000441089868545532
0.06 -0.000451741851866245
0.07 -0.000458881184458733
0.08 -0.0004618925973773
0.09 -0.000489976108074188
0.1 -0.000564640872180462
0.11 -0.000636811703443527
0.12 -0.000706040635704994
0.13 -0.000771976262331009
0.14 -0.000834323316812515
0.15 -0.000892852395772934
0.16 -0.00094737209379673
0.17 -0.000997756868600845
0.18 -0.00105596601963043
0.19 -0.00112264700233936
0.2 -0.00118605874478817
0.21 -0.00124645546078682
0.22 -0.00130009293556213
0.23 -0.00134705320000649
0.24 -0.00138752609491348
0.25 -0.00142174765467644
0.26 -0.00145004853606224
0.27 -0.00147280752658844
0.28 -0.00149045318365097
0.29 -0.00150347858667374
0.3 -0.00151240393519402
0.31 -0.00151189193129539
0.32 -0.0015067121386528
0.33 -0.00149866700172424
0.34 -0.00148836016654968
0.35 -0.00147634893655777
0.36 -0.00146320417523384
0.37 -0.00144946813583374
0.38 -0.00143565788865089
0.39 -0.00142226904630661
0.4 -0.00140975132584572
0.41 -0.00139850661158562
0.42 -0.00138896927237511
0.43 -0.00138148918747902
0.44 -0.00137638926506042
0.45 -0.00137395411729813
0.46 -0.00137446254491806
0.47 -0.00137814819812775
0.48 -0.00138518840074539
0.49 -0.00139578104019165
0.5 -0.00141005620360374
0.51 -0.00142812147736549
0.52 -0.00145009726285934
0.53 -0.00147602289915085
0.54 -0.00150381878018379
0.55 -0.00151645809412003
0.56 -0.00152931854128838
0.57 -0.00154244244098663
0.58 -0.00155594438314438
0.59 -0.00156984746456146
0.6 -0.00158096820116043
0.61 -0.00158493727445602
0.62 -0.00158837199211121
0.63 -0.00159135296940804
0.64 -0.00159399151802063
0.65 -0.00159637123346329
0.66 -0.00159020811319351
0.67 -0.00158078759908676
0.68 -0.00157035693526268
0.69 -0.0015590925514698
0.7 -0.00154717206954956
0.71 -0.00153472796082497
0.72 -0.00152193978428841
0.73 -0.00150891095399857
0.74 -0.00149577140808105
0.75 -0.00148262828588486
0.76 -0.00146955519914627
0.77 -0.00145664572715759
0.78 -0.00144398063421249
0.79 -0.00143454566597939
0.8 -0.00142697736620903
0.81 -0.00143760308623314
0.820000000000001 -0.00146538227796555
0.830000000000001 -0.00150063186883926
0.840000000000001 -0.00154299765825272
0.850000000000001 -0.00159201338887215
0.860000000000001 -0.00164715677499771
0.870000000000001 -0.00170788884162903
0.880000000000001 -0.00173109889030457
0.890000000000001 -0.00164038375020027
0.900000000000001 -0.00149962961673737
0.910000000000001 -0.00136783361434937
0.920000000000001 -0.00124604754149914
0.930000000000001 -0.00113405048847198
0.940000000000001 -0.00103105075657368
0.950000000000001 -0.000936014652252197
0.960000000000001 -0.00084961898624897
0.970000000000001 -0.000771529302000999
0.980000000000001 -0.000697738975286484
0.990000000000001 -0.000627612844109535
1 -0.000560706816613674
1.01 -0.000496673658490181
1.02 -0.000435274057090282
1.03 -0.000376371406018734
1.04 -0.000319831520318985
1.05 -0.000265620220452547
1.06 -0.000213718321174383
1.07 -0.000164088495075703
1.08 -9.5225153490901e-05
1.09 -2.6404841337353e-05
1.1 3.85733949951828e-05
1.11 0.000100713418796659
1.12 0.000160468649119139
1.13 0.000218006670475006
1.14 0.000273413900285959
1.15 0.000326652675867081
1.16 0.000377666763961315
1.17 0.00042642954736948
1.18 0.000479342006146908
1.19 0.000531139634549618
1.2 0.000579080507159233
1.21 0.00062370840460062
1.22 0.000663576871156692
1.23 0.000683337524533272
1.24 0.000702638998627663
1.25 0.000720534697175026
1.26 0.00073663517832756
1.27 0.000750698447227478
1.28 0.000762669891119003
1.29 0.000772530362010002
1.3 0.000780249983072281
1.31 0.000785867497324944
1.32 0.000789417997002602
1.33 0.000790902748703957
1.34 0.000795711874961853
1.35 0.000808258652687073
1.36 0.000817397683858872
1.37 0.00083987407386303
1.38 0.000886217206716537
1.39 0.000940242633223534
1.4 0.00100146152079105
1.41 0.00106939926743507
1.42 0.00115267664194107
1.43 0.0012533438205719
1.44 0.00137380287051201
1.45 0.00150189608335495
1.46 0.00162755742669106
1.47 0.00175550699234009
1.48 0.00188974559307098
1.49 0.00202997446060181
1.5 0.00217592388391495
1.51 0.00235153555870056
1.52 0.00254915803670883
1.53 0.00273009568452835
1.54 0.00289028882980347
1.55 0.00303614169359207
1.56 0.00312585324048996
1.57 0.00321423590183258
1.58 0.00332523792982101
1.59 0.00343611717224121
1.6 0.00354682117700577
1.61 0.00363724321126938
1.62 0.00352064937353134
1.63 0.00333268135786057
1.64 0.0031097474694252
1.65 0.00291015326976776
1.66 0.0027290403842926
1.67 0.00256584942340851
1.68 0.00250409454107285
1.69 0.00245376005768776
1.7 0.00239073529839516
1.71 0.00229228496551514
1.72 0.0021931254863739
1.73 0.00209621995687485
1.74 0.00200145840644836
1.75 0.00190872564911842
1.76 0.00183780759572983
1.77 0.00177021250128746
1.78 0.0017019134759903
1.79 0.00163306146860123
1.8 0.00156384363770485
1.81 0.00153644442558289
1.82 0.00154339462518692
1.83 0.00155475407838821
1.84 0.00156769543886185
1.85 0.00157870680093765
1.86 0.00158743143081665
1.87 0.001578379124403
1.88 0.00156733050942421
1.89 0.00155391752719879
1.9 0.00151776477694511
1.91 0.00148520976305008
1.92 0.00145700737833977
1.93 0.00142585143446922
1.94 0.00139071732759476
1.95 0.00132589250802994
1.96 0.00119758397340775
1.97 0.00104353286325932
1.98 0.000873776227235794
1.99 0.00070171631872654
2 0.000528809539973736
2.01 0.000366098508238792
2.02 0.00019080238416791
2.03 1.88958586659282e-05
2.04 -0.000149820623919368
2.05 -0.00031288456171751
2.06 -0.000469043664634228
2.07 -0.000617349669337273
2.08 -0.000756931975483894
2.09 -0.000887014344334602
2.1 -0.00099284216761589
2.11 -0.00105293288826942
2.12 -0.0011602496355772
2.13 -0.0012538181245327
2.14 -0.00131714284420013
2.15 -0.00126465767621994
2.16 -0.00119225412607193
2.17 -0.00110124789178371
2.18 -0.000969101041555405
2.19 -0.000860709846019745
2.2 -0.000802484005689621
2.21 -0.00071638323366642
2.22 -0.000620729587972164
2.23 -0.000516575686633587
2.24 -0.000421811975538731
2.25 -0.000385161116719246
2.26 -0.000360922366380692
2.27 -0.000346146821975708
2.28 -0.000328974835574627
2.29 -0.000325439460575581
2.29999999999999 -0.000316463522613049
2.30999999999999 -0.000283588245511055
2.31999999999999 -0.000256018359214067
2.32999999999999 -0.000226940903812647
2.33999999999999 -0.000194837246090174
2.34999999999999 -0.000159906316548586
2.35999999999999 -0.000122384093701839
2.36999999999999 -8.24852101504803e-05
2.37999999999999 -4.14549699053168e-05
2.38999999999999 -1.6158219659701e-05
2.39999999999999 8.9663261314854e-06
2.40999999999999 5.38204796612263e-05
2.41999999999999 9.88147594034672e-05
2.42999999999999 0.00014377286657691
2.43999999999999 0.000188595429062843
2.44999999999999 0.000233116317540407
2.45999999999999 0.000277273766696453
2.46999999999999 0.000320958346128464
2.47999999999999 0.00036407083272934
2.48999999999999 0.000406551957130432
2.49999999999999 0.000448354445397854
2.50999999999999 0.000489156432449818
2.51999999999999 0.000524625815451145
2.52999999999999 0.000557657144963741
2.53999999999999 0.000588412284851074
2.54999999999999 0.000633680820465088
2.55999999999999 0.000685515552759171
2.56999999999999 0.000737319663167
2.57999999999999 0.000789051353931427
2.58999999999999 0.000840647518634796
2.59999999999999 0.000892071351408958
2.60999999999999 0.000943246558308601
2.61999999999999 0.000994122251868248
2.62999999999999 0.00104465633630753
2.63999999999999 0.00107595533132553
2.64999999999999 0.00107521779835224
2.65999999999999 0.0010697677731514
2.66999999999999 0.00105976887047291
2.67999999999999 0.00104542911052704
2.68999999999999 0.00102694123983383
2.69999999999999 0.00101054489612579
2.70999999999999 0.0010001689940691
2.71999999999999 0.000987390950322151
2.72999999999999 0.00097226694226265
2.73999999999999 0.000954820737242699
2.74999999999999 0.000935120731592178
2.75999999999999 0.00091318666934967
2.76999999999998 0.000889052376151085
2.77999999999998 0.000862767174839974
2.78999999999998 0.000834335759282112
2.79999999999998 0.00080641396343708
2.80999999999998 0.0007993383705616
2.81999999999998 0.000781602188944817
2.82999999999998 0.000757895037531853
2.83999999999998 0.000733425617218018
2.84999999999998 0.000708195567131043
2.85999999999998 0.00068215899169445
2.86999999999998 0.000655302032828331
2.87999999999998 0.000627585798501968
2.88999999999998 0.000598994120955467
2.89999999999998 0.000569498911499977
2.90999999999998 0.000539061352610588
2.91999999999998 0.000510948561131954
2.92999999999998 0.000487970411777496
2.93999999999998 0.000464700274169445
2.94999999999998 0.000441075563430786
2.95999999999998 0.000417055822908878
2.96999999999998 0.00039261482656002
2.97999999999998 0.000367712676525116
2.98999999999998 0.000342304073274136
2.99999999999998 0.000316390208899975
3.00999999999998 0.000289905034005642
3.01999999999998 0.000262866374105215
3.02999999999998 0.000235233791172504
3.03999999999998 0.000206960793584585
3.04999999999998 0.000178052242845297
3.05999999999998 0.000148467672988772
3.06999999999998 0.00011821310967207
3.07999999999998 8.72397888451815e-05
3.08999999999998 5.55597292259336e-05
3.09999999999998 2.31313612312078e-05
3.10999999999998 -1.00421893876046e-05
3.11999999999998 -4.39780252054334e-05
3.12999999999998 -8.30311793833971e-05
3.13999999999998 -0.000139391552656889
3.14999999999998 -0.000199498478323221
3.15999999999998 -0.000263297837227583
3.16999999999998 -0.000330745242536068
3.17999999999998 -0.000397552326321602
3.18999999999998 -0.000459261238574982
3.19999999999998 -0.000523124784231186
3.20999999999998 -0.000589289516210556
3.21999999999998 -0.000648489892482758
3.22999999999998 -0.000700105950236321
3.23999999999997 -0.000751897841691971
3.24999999999997 -0.000804021134972572
3.25999999999997 -0.000856604725122452
3.26999999999997 -0.00090979628264904
3.27999999999997 -0.000963725969195366
3.28999999999997 -0.00101851373910904
3.29999999999997 -0.00107428468763828
3.30999999999997 -0.00113114930689335
3.31999999999997 -0.00118923932313919
3.32999999999997 -0.00124867126345634
3.33999999999997 -0.00130952432751656
3.34999999999997 -0.00137193411588669
3.35999999999997 -0.00143599614500999
3.36999999999997 -0.00150181725621223
3.37999999999997 -0.0015695133805275
3.38999999999997 -0.00163078442215919
3.39999999999997 -0.00163425981998444
3.40999999999997 -0.00163258075714111
3.41999999999997 -0.00163097828626633
3.42999999999997 -0.00162895038723946
3.43999999999997 -0.00162615656852722
3.44999999999997 -0.0016223706305027
3.45999999999997 -0.00161745771765709
3.46999999999997 -0.00161130473017693
3.47999999999997 -0.00160392224788666
3.48999999999997 -0.00159529626369476
3.49999999999997 -0.00158540993928909
3.50999999999997 -0.0015743014216423
3.51999999999997 -0.00155451893806458
3.52999999999997 -0.00145918533205986
3.53999999999997 -0.00138934299349785
3.54999999999997 -0.00133145526051521
3.55999999999997 -0.00127127557992935
3.56999999999997 -0.00119460552930832
3.57999999999997 -0.00112276405096054
3.58999999999997 -0.00105516128242016
3.59999999999997 -0.000991160646080971
3.60999999999997 -0.000930283665657043
3.61999999999997 -0.000871970653533936
3.62999999999997 -0.000815969556570053
3.63999999999997 -0.000761942863464356
3.64999999999997 -0.000709696635603905
3.65999999999997 -0.000659107193350792
3.66999999999997 -0.000610047243535519
3.67999999999997 -0.000562492050230503
3.68999999999997 -0.000516411289572716
3.69999999999997 -0.000471780374646187
3.70999999999996 -0.000416863076388836
3.71999999999996 -0.000357897505164146
3.72999999999996 -0.00030146699398756
3.73999999999996 -0.000246812682598829
3.74999999999996 -0.000193610694259405
3.75999999999996 -0.000141727337613702
3.76999999999996 -9.11575183272362e-05
3.77999999999996 -4.19255485758185e-05
3.78999999999996 5.93483448028564e-06
3.79999999999996 5.23107033222914e-05
3.80999999999996 9.71917621791363e-05
3.81999999999996 0.000140461400151253
3.82999999999996 0.00018206749111414
3.83999999999996 0.000221947263926268
3.84999999999996 0.000260036829859018
3.85999999999996 0.000296278670430183
3.86999999999996 0.000330620370805264
3.87999999999996 0.000362985506653786
3.88999999999996 0.000394580401480198
3.89999999999996 0.000426910631358623
3.90999999999996 0.000456051751971245
3.91999999999996 0.000482348538935184
3.92999999999996 0.000505962856113911
3.93999999999996 0.000526908859610558
3.94999999999996 0.000543501637876034
3.95999999999996 0.000552966296672821
3.96999999999996 0.000561369396746159
3.97999999999996 0.000568341724574566
3.98999999999996 0.000573731549084187
3.99999999999996 0.000577516295015812
4.00999999999996 0.000579641796648502
4.01999999999996 0.000580135509371758
4.02999999999996 0.000579007565975189
4.03999999999996 0.000590442456305027
4.04999999999996 0.000616929866373539
4.05999999999996 0.000655507445335388
4.06999999999996 0.000709182024002075
4.07999999999996 0.000771056711673737
4.08999999999996 0.000840910524129868
4.09999999999996 0.000918498188257217
4.10999999999996 0.00100356444716454
4.11999999999996 0.00109583251178265
4.12999999999996 0.00121042139828205
4.13999999999996 0.00133610740303993
4.14999999999996 0.00146868497133255
4.15999999999996 0.00160153672099113
4.16999999999996 0.00173939049243927
4.17999999999996 0.00190951719880104
4.18999999999996 0.002120770663023
4.19999999999995 0.00234464138746262
4.20999999999995 0.00258101731538773
4.21999999999995 0.00277203410863876
4.22999999999995 0.00286677271127701
4.23999999999995 0.00295543819665909
4.24999999999995 0.00304950892925262
4.25999999999995 0.00315904855728149
4.26999999999995 0.00326779365539551
4.27999999999995 0.00337575614452362
4.28999999999995 0.00348292708396912
4.29999999999995 0.00350861698389053
4.30999999999995 0.00343785017728806
4.31999999999995 0.00327703714370728
4.32999999999995 0.00306260704994202
4.33999999999995 0.00287095725536346
4.34999999999995 0.00269783973693848
4.35999999999995 0.00257694244384766
4.36999999999995 0.00251649647951126
4.37999999999995 0.00246236845850945
4.38999999999995 0.00240445360541344
4.39999999999995 0.00234295547008514
4.40999999999995 0.00226284682750702
4.41999999999995 0.00216756269335747
4.42999999999995 0.00207445621490479
4.43999999999995 0.00198371201753616
4.44999999999995 0.00189552277326584
4.45999999999995 0.00183015257120132
4.46999999999995 0.00176548138260841
4.47999999999995 0.00170017197728157
4.48999999999995 0.00163438782095909
4.49999999999995 0.00163130074739456
4.50999999999995 0.00163542211055756
4.51999999999995 0.00164143189787865
4.52999999999995 0.0016487717628479
4.53999999999995 0.00165693327784538
4.54999999999995 0.00166541039943695
4.55999999999995 0.00166840896010399
4.56999999999995 0.00165469214320183
4.57999999999995 0.00161914050579071
4.58999999999995 0.00156950280070305
4.59999999999995 0.00151636213064194
4.60999999999995 0.00148557007312775
4.61999999999995 0.00145220920443535
4.62999999999995 0.00141619801521301
4.63999999999995 0.00137743577361107
4.64999999999995 0.00129793837666512
4.65999999999995 0.00114489361643791
4.66999999999994 0.000986502021551132
4.67999999999994 0.00082397885620594
4.68999999999994 0.000653143674135208
4.69999999999994 0.000484053716063499
4.70999999999994 0.000317175015807152
4.71999999999994 0.000153734041377902
4.72999999999994 -5.14984130859375e-06
4.73999999999994 -0.000158400200307369
4.74999999999994 -0.000305049605667591
4.75999999999994 -0.000444206260144711
4.76999999999994 -0.000575071573257446
4.77999999999994 -0.000696937069296837
4.78999999999994 -0.000791687741875648
4.79999999999994 -0.000908734500408173
4.80999999999994 -0.0010043527930975
4.81999999999994 -0.00107683733105659
4.82999999999994 -0.00106224089860916
4.83999999999994 -0.00100263401865959
4.84999999999994 -0.000904212668538094
4.85999999999994 -0.000776175037026405
4.86999999999994 -0.000655391290783882
4.87999999999994 -0.000625695213675499
4.88999999999994 -0.000545601136982441
4.89999999999994 -0.00046451497823
4.90999999999994 -0.000375137440860271
4.91999999999994 -0.000278453715145588
4.92999999999994 -0.000235153920948505
4.93999999999994 -0.00020196346566081
4.94999999999994 -0.000165450368076563
4.95999999999994 -0.000132159451022744
4.96999999999994 -0.000115707414224744
4.97999999999994 -0.000115030333399773
4.98999999999994 -0.000126238577067852
4.99999999999994 -0.000126030566170812
5.00999999999994 -0.00010525070130825
5.01999999999994 -8.11351835727692e-05
5.02999999999994 -5.3859818726778e-05
5.03999999999994 -2.3643362801522e-05
5.04999999999994 9.28163470234722e-06
5.05999999999994 4.47093695402145e-05
5.06999999999994 6.73518003895879e-05
5.07999999999994 8.74233618378639e-05
5.08999999999994 0.000106715736910701
5.09999999999994 0.000125197917222977
5.10999999999994 0.000142859173938632
5.11999999999994 0.000159647651016712
5.12999999999994 0.000184198766946793
5.13999999999993 0.000220798216760159
5.14999999999993 0.000257872957736254
5.15999999999993 0.000295232553035021
5.16999999999993 0.000332741439342499
5.17999999999993 0.000370247960090637
5.18999999999993 0.000407630354166031
5.19999999999993 0.000444773994386196
5.20999999999993 0.000478769838809967
5.21999999999993 0.000510041974484921
5.22999999999993 0.000542172230780125
5.23999999999993 0.000588630251586437
5.24999999999993 0.000635500550270081
5.25999999999993 0.000682489424943924
5.26999999999993 0.000729470774531364
5.27999999999993 0.00077639527618885
5.28999999999993 0.000823235064744949
5.29999999999993 0.000869932472705841
5.30999999999993 0.000916450917720795
5.31999999999993 0.000962747111916542
5.32999999999993 0.00100875534117222
5.33999999999993 0.00105444557964802
5.34999999999993 0.00109977953135967
5.35999999999993 0.00112816490232944
5.36999999999993 0.00112930096685886
5.37999999999993 0.00112634859979153
5.38999999999993 0.00111952222883701
5.39999999999993 0.00110897108912468
5.40999999999993 0.00109490253031254
5.41999999999993 0.00107848532497883
5.42999999999993 0.00107042267918587
5.43999999999993 0.00106026336550713
5.44999999999993 0.00104807004332542
5.45999999999993 0.00103389643132687
5.46999999999993 0.00101775459945202
5.47999999999993 0.00099970169365406
5.48999999999993 0.000979775264859199
5.49999999999993 0.000957999303936958
5.50999999999993 0.00093442901968956
5.51999999999993 0.000909059271216392
5.52999999999993 0.000881959646940231
5.53999999999993 0.000853129029273987
5.54999999999993 0.000842853188514709
5.55999999999993 0.000832950621843338
5.56999999999993 0.000818769857287407
5.57999999999993 0.000795671418309212
5.58999999999993 0.000771945863962174
5.59999999999993 0.00074756346642971
5.60999999999992 0.000722497254610062
5.61999999999992 0.000696723982691765
5.62999999999992 0.000670240372419357
5.63999999999992 0.000643020272254944
5.64999999999992 0.000615050494670868
5.65999999999992 0.000586295202374458
5.66999999999992 0.000556735768914223
5.67999999999992 0.000526355542242527
5.68999999999992 0.000499782785773277
5.69999999999992 0.00047563225030899
5.70999999999992 0.000451194979250431
5.71999999999992 0.000426459051668644
5.72999999999992 0.000401366092264652
5.73999999999992 0.000375907868146896
5.74999999999992 0.000350068211555481
5.75999999999992 0.000323791839182377
5.76999999999992 0.000297090020030737
5.77999999999992 0.000269926488399506
5.78999999999992 0.000242293551564217
5.79999999999992 0.000214159656316042
5.80999999999992 0.000185524262487888
5.81999999999992 0.000156358275562525
5.82999999999992 0.000126654598861933
5.83999999999992 9.63859260082245e-05
5.84999999999992 6.55523594468832e-05
5.85999999999992 3.41367395594716e-05
5.86999999999992 6.32941722869873e-06
5.87999999999992 -2.02000071294606e-05
5.88999999999992 -4.67359181493521e-05
5.89999999999992 -7.32990819960833e-05
5.90999999999992 -9.9926358088851e-05
5.91999999999992 -0.000117723178118467
5.92999999999992 -0.000150311142206192
5.93999999999992 -0.000183764286339283
5.94999999999992 -0.000218157451599836
5.95999999999992 -0.000253599993884563
5.96999999999992 -0.000290165450423956
5.97999999999992 -0.000327928476035595
5.98999999999992 -0.000367024466395378
5.99999999999992 -0.000407502874732018
6.00999999999992 -0.000449492931365967
6.01999999999992 -0.000493096895515919
6.02999999999992 -0.000532860979437828
6.03999999999992 -0.000570157617330551
6.04999999999992 -0.000608080327510834
6.05999999999992 -0.000646722093224526
6.06999999999992 -0.000686174631118774
6.07999999999991 -0.000726548060774803
6.08999999999991 -0.000767893418669701
6.09999999999991 -0.000810321792960167
6.10999999999991 -0.000853903070092201
6.11999999999991 -0.000898720547556877
6.12999999999991 -0.000944859459996223
6.13999999999991 -0.000992394387722015
6.14999999999991 -0.00104140900075436
6.15999999999991 -0.00109199032187462
6.16999999999991 -0.00114420033991337
6.17999999999991 -0.00118105053901672
6.18999999999991 -0.00121363863348961
6.19999999999991 -0.00124481290578842
6.20999999999991 -0.0012747848033905
6.21999999999991 -0.00130381345748901
6.22999999999991 -0.00133209139108658
6.23999999999991 -0.00135980397462845
6.24999999999991 -0.00138714641332626
6.25999999999991 -0.00141427114605904
6.26999999999991 -0.00144135683774948
6.27999999999991 -0.00145446509122849
6.28999999999991 -0.00145843982696533
6.29999999999991 -0.00146067321300507
6.30999999999991 -0.00146133452653885
6.31999999999991 -0.00146052494645119
6.32999999999991 -0.00145836994051933
6.33999999999991 -0.00145492613315582
6.34999999999991 -0.00145028337836266
6.35999999999991 -0.00144451424479485
6.36999999999991 -0.00143764540553093
6.37999999999991 -0.00139462113380432
6.38999999999991 -0.00130723848938942
6.39999999999991 -0.00123710654675961
6.40999999999991 -0.00117521524429321
6.41999999999991 -0.00111718766391277
6.42999999999991 -0.00106085777282715
6.43999999999991 -0.00100520186126232
6.44999999999991 -0.00094974972307682
6.45999999999991 -0.000894338339567184
6.46999999999991 -0.000838906019926071
6.47999999999991 -0.000783462524414062
6.48999999999991 -0.000728115364909172
6.49999999999991 -0.000672902017831802
6.50999999999991 -0.000617914758622646
6.51999999999991 -0.000563267506659031
6.52999999999991 -0.000508997477591038
6.53999999999991 -0.000455259419977665
6.5499999999999 -0.000402098931372166
6.5599999999999 -0.000349629670381546
6.5699999999999 -0.000297939497977495
6.5799999999999 -0.000247116629034281
6.5899999999999 -0.000197223592549562
6.5999999999999 -0.00014839556068182
6.6099999999999 -0.000100652920082211
6.6199999999999 -5.41283376514912e-05
6.6299999999999 -8.86559428181499e-06
6.6399999999999 3.50415357388556e-05
6.6499999999999 7.75160500779748e-05
6.6599999999999 0.000118503076955676
6.6699999999999 0.000157934725284576
6.6799999999999 0.000195736587047577
6.6899999999999 0.00023184809833765
6.6999999999999 0.000266197696328163
6.7099999999999 0.000298768412321806
6.7199999999999 0.000329471528530121
6.7299999999999 0.000358274392783642
6.7399999999999 0.00038513608276844
6.7499999999999 0.000410014465451241
6.7599999999999 0.000432863272726536
6.7699999999999 0.000453649424016476
6.7799999999999 0.000472347512841225
6.7899999999999 0.000488931275904179
6.7999999999999 0.000503380671143532
6.8099999999999 0.000515678040683269
6.8199999999999 0.000525798611342907
6.8299999999999 0.00053374607115984
6.8399999999999 0.000539517030119896
6.8499999999999 0.000543093644082546
6.8599999999999 0.000544487349689007
6.8699999999999 0.000553632155060768
6.8799999999999 0.00057790994644165
6.8899999999999 0.000605145692825317
6.8999999999999 0.000640390291810036
6.9099999999999 0.000696742385625839
6.9199999999999 0.000760770961642265
6.9299999999999 0.000832265764474869
6.9399999999999 0.000910973995923996
6.9499999999999 0.000996662229299545
6.9599999999999 0.00109024584293365
6.9699999999999 0.0012073190510273
6.9799999999999 0.00133172586560249
6.9899999999999 0.0014632011950016
6.9999999999999 0.00159835055470467
7.00999999999989 0.00173479542136192
7.01999999999989 0.0019316603243351
7.02999999999989 0.00214354768395424
7.03999999999989 0.00236740708351135
7.04999999999989 0.00260318636894226
7.05999999999989 0.00277219891548157
7.06999999999989 0.00286176949739456
7.07999999999989 0.00294859766960144
7.08999999999989 0.00304676115512848
7.09999999999989 0.00315451472997665
7.10999999999989 0.00326139032840729
7.11999999999989 0.00336739838123322
7.12999999999989 0.00347254872322083
7.13999999999989 0.00350535809993744
7.14999999999989 0.00344235330820084
7.15999999999989 0.00329133003950119
7.16999999999989 0.00307862818241119
7.17999999999989 0.00288873642683029
7.18999999999989 0.00271737575531006
7.19999999999989 0.00259689599275589
7.20999999999989 0.00253970563411713
7.21999999999989 0.00248683363199234
7.22999999999989 0.00243012681603432
7.23999999999989 0.0023698166012764
7.24999999999989 0.00230410858988762
7.25999999999989 0.00220820128917694
7.26999999999989 0.00211457580327988
7.27999999999989 0.00202295079827309
7.28999999999989 0.00193389236927032
7.29999999999989 0.00186460554599762
7.30999999999989 0.00180005297064781
7.31999999999989 0.0017344132065773
7.32999999999989 0.00166785091161728
7.33999999999989 0.00165484130382538
7.34999999999989 0.00165773659944534
7.35999999999989 0.00166249692440033
7.36999999999989 0.00166859954595566
7.37999999999989 0.0016755548119545
7.38999999999989 0.00168286994099617
7.39999999999989 0.0016900746524334
7.40999999999989 0.00167620405554771
7.41999999999989 0.00163861274719238
7.42999999999989 0.00158922612667084
7.43999999999989 0.00153606697916985
7.44999999999989 0.00150060817599297
7.45999999999989 0.00146702140569687
7.46999999999989 0.00143083930015564
7.47999999999988 0.00139202237129211
7.48999999999988 0.00133225336670876
7.49999999999988 0.00118540719151497
7.50999999999988 0.00102871276438236
7.51999999999988 0.000867772921919823
7.52999999999988 0.000696637034416199
7.53999999999988 0.000529168695211411
7.54999999999988 0.000363790206611156
7.55999999999988 0.000201704222708941
7.56999999999988 4.40537231042981e-05
7.57999999999988 -0.000108157536014915
7.58999999999988 -0.000253906920552254
7.59999999999988 -0.000392355658113956
7.60999999999988 -0.000522675700485706
7.61999999999988 -0.000644190311431885
7.62999999999988 -0.000750178843736649
7.63999999999988 -0.000866500958800316
7.64999999999988 -0.000961531028151512
7.65999999999988 -0.00103355050086975
7.66999999999988 -0.00101669691503048
7.67999999999988 -0.000957935675978661
7.68999999999988 -0.000848519802093506
7.69999999999988 -0.000723822936415672
7.70999999999988 -0.000612895451486111
7.71999999999988 -0.000585026107728481
7.72999999999988 -0.000504735819995403
7.73999999999988 -0.000426077134907246
7.74999999999988 -0.00033926747739315
7.75999999999988 -0.000245271548628807
7.76999999999988 -0.000202686991542578
7.77999999999988 -0.000170723218470812
7.78999999999988 -0.000135461408644915
7.79999999999988 -9.6991490572691e-05
7.80999999999988 -8.4619615226984e-05
7.81999999999988 -8.00736434757709e-05
7.82999999999988 -8.94778966903687e-05
7.83999999999988 -9.94852930307388e-05
7.84999999999988 -8.01690109074116e-05
7.85999999999988 -5.74628915637732e-05
7.86999999999988 -3.15737468190491e-05
7.87999999999988 -2.68876552581787e-06
7.88999999999988 2.89571075700223e-05
7.89999999999988 6.3138990662992e-05
7.90999999999988 8.37893318384886e-05
7.91999999999988 0.000102897575125098
7.92999999999988 0.000121330320835114
7.93999999999988 0.000139051219448447
7.94999999999987 0.000156055502593517
7.95999999999987 0.000172301437705755
7.96999999999987 0.000187765806913376
7.97999999999987 0.000211819838732481
7.98999999999987 0.000247058812528849
7.99999999999987 0.000282809361815453
8.00999999999987 0.000318887047469616
8.01999999999987 0.000355154499411583
8.02999999999987 0.000391481593251228
8.03999999999987 0.000427713729441166
8.04999999999987 0.000462201796472073
8.05999999999987 0.000493391156196594
8.06999999999987 0.00052427913993597
8.07999999999987 0.000569750852882862
8.08999999999987 0.000615342892706394
8.09999999999987 0.000661288127303123
8.10999999999987 0.0007072514295578
8.11999999999987 0.000753188803792
8.12999999999987 0.000799059867858887
8.13999999999987 0.000844835042953491
8.14999999999987 0.000890455693006516
8.15999999999987 0.000935873687267303
8.16999999999987 0.000981056690216065
8.17999999999987 0.00102597579360008
8.18999999999987 0.00107055403292179
8.19999999999987 0.00111475564539433
8.20999999999987 0.00114140272140503
8.21999999999987 0.00114278756082058
8.22999999999987 0.00114023290574551
8.23999999999987 0.00113393940031528
8.24999999999987 0.00112405516207218
8.25999999999987 0.00111078284680843
8.26999999999987 0.00109500415623188
8.27999999999987 0.001087396889925
8.28999999999987 0.00107776500284672
8.29999999999987 0.00106616303324699
8.30999999999987 0.00105262316763401
8.31999999999987 0.00103720396757126
8.32999999999987 0.00101995304226875
8.33999999999987 0.00100086934864521
8.34999999999987 0.000980026796460152
8.35999999999987 0.000957445651292801
8.36999999999987 0.000933152139186859
8.37999999999987 0.000907168984413147
8.38999999999987 0.000879544988274574
8.39999999999987 0.000859246328473091
8.40999999999987 0.00084917426109314
8.41999999999986 0.000838613882660866
8.42999999999986 0.000822672694921494
8.43999999999986 0.000799674019217491
8.44999999999986 0.000776032209396362
8.45999999999986 0.000751776397228241
8.46999999999986 0.000726859346032143
8.47999999999986 0.00070126086473465
8.48999999999986 0.000674981325864792
8.49999999999986 0.000647986754775047
8.50999999999986 0.000620272867381573
8.51999999999986 0.000591803230345249
8.52999999999986 0.000562558770179749
8.53999999999986 0.000532527416944504
8.54999999999986 0.00050432600080967
8.55999999999986 0.000480078011751175
8.56999999999986 0.000455565899610519
8.57999999999986 0.000430767051875591
8.58999999999986 0.000405648127198219
8.59999999999986 0.000380171649158001
8.60999999999986 0.000354322679340839
8.61999999999986 0.000328098312020302
8.62999999999986 0.000301460437476635
8.63999999999986 0.00027439720928669
8.64999999999986 0.000246880035847425
8.65999999999986 0.000218899454921484
8.66999999999986 0.000190438237041235
8.67999999999986 0.000161483958363533
8.68999999999986 0.000132017005234957
8.69999999999986 0.000103621743619442
8.70999999999986 7.8943558037281e-05
8.71999999999986 5.43730100616813e-05
8.72999999999986 2.98672681674361e-05
8.73999999999986 5.38289546966553e-06
8.74999999999986 -1.9126528641209e-05
8.75999999999986 -4.3695499189198e-05
8.76999999999986 -6.07291981577873e-05
8.77999999999986 -7.53184594213963e-05
8.78999999999986 -8.97651817649603e-05
8.79999999999986 -0.000114800864830613
8.80999999999986 -0.000143255526199937
8.81999999999986 -0.000172696597874165
8.82999999999986 -0.000203206110745668
8.83999999999986 -0.000234842840582132
8.84999999999986 -0.000267698876559734
8.85999999999986 -0.000301832892000675
8.86999999999986 -0.000337344594299793
8.87999999999986 -0.000374306850135326
8.88999999999985 -0.000412818565964699
8.89999999999985 -0.000452980399131775
8.90999999999985 -0.000489813461899757
8.91999999999985 -0.000524317845702171
8.92999999999985 -0.000559511780738831
8.93999999999985 -0.00059546984732151
8.94999999999985 -0.000632291659712791
8.95999999999985 -0.000670023038983345
8.96999999999985 -0.000708781331777573
8.97999999999985 -0.00074862003326416
8.98999999999985 -0.000789629817008972
8.99999999999985 -0.000831863507628441
9.00999999999985 -0.000875418186187744
9.01999999999985 -0.000920354500412941
9.02999999999985 -0.000966734364628792
9.03999999999985 -0.00101464651525021
9.04999999999985 -0.00105091877281666
9.05999999999985 -0.00108114667236805
9.06999999999985 -0.00111002251505852
9.07999999999985 -0.00113777235150337
9.08999999999985 -0.00116459265351295
9.09999999999985 -0.00119068585336208
9.10999999999985 -0.00121622689068317
9.11999999999985 -0.00124138429760933
9.12999999999985 -0.00126630619168282
9.13999999999985 -0.00129115179181099
9.14999999999985 -0.00131602093577385
9.15999999999985 -0.0013410721719265
9.16999999999985 -0.00136638194322586
9.17999999999985 -0.0013728192448616
9.18999999999985 -0.00137509137392044
9.19999999999985 -0.00137582570314407
9.20999999999985 -0.00137512043118477
9.21999999999985 -0.00137308865785599
9.22999999999985 -0.00136979669332504
9.23999999999985 -0.00136533096432686
9.24999999999985 -0.00135973542928696
9.25999999999985 -0.00135307997465134
9.26999999999985 -0.00134539037942886
9.27999999999985 -0.00129869520664215
9.28999999999985 -0.00121556922793388
9.29999999999985 -0.00114842429757118
9.30999999999985 -0.00108894616365433
9.31999999999985 -0.00103307463228703
9.32999999999985 -0.000978852733969689
9.33999999999985 -0.000925334319472313
9.34999999999985 -0.000872073024511337
9.35999999999984 -0.000818937048316002
9.36999999999984 -0.000765878185629845
9.37999999999984 -0.000712929368019104
9.38999999999984 -0.000660130083560944
9.39999999999984 -0.000607575923204422
9.40999999999984 -0.000555320046842098
9.41999999999984 -0.000503475666046143
9.42999999999984 -0.000452135056257248
9.43999999999984 -0.000401363410055637
9.44999999999984 -0.000351240299642086
9.45999999999984 -0.000301866848021746
9.46999999999984 -0.00025334045290947
9.47999999999984 -0.00020571518689394
9.48999999999984 -0.000159091874957085
9.49999999999984 -0.000113531025126576
9.50999999999984 -6.9128000177443e-05
9.51999999999984 -2.59446958079934e-05
9.52999999999984 1.59335101488978e-05
9.53999999999984 5.64388930797577e-05
9.54999999999984 9.55219753086567e-05
9.55999999999984 0.000133094610646367
9.56999999999984 0.00016911581158638
9.57999999999984 0.000203505884855986
9.58999999999984 0.000236239768564701
9.59999999999984 0.000267227776348591
9.60999999999984 0.00029644999653101
9.61999999999984 0.00032383531332016
9.62999999999984 0.000349374040961266
9.63999999999984 0.000372998975217342
9.64999999999984 0.000394680500030518
9.65999999999984 0.000414399802684784
9.66999999999984 0.00043210219591856
9.67999999999984 0.000447788499295712
9.68999999999984 0.000461430922150612
9.69999999999984 0.00047301173210144
9.70999999999984 0.00048250786960125
9.71999999999984 0.000489921905100346
9.72999999999984 0.000495246723294258
9.73999999999984 0.000498459227383137
9.74999999999984 0.000508820861577988
9.75999999999984 0.000533398240804672
9.76999999999984 0.000560742244124413
9.77999999999984 0.000590634234249592
9.78999999999984 0.000629685372114182
9.79999999999984 0.000689005479216576
9.80999999999984 0.000755564272403717
9.81999999999984 0.000829146057367325
9.82999999999983 0.000909519642591476
9.83999999999983 0.000996450632810593
9.84999999999983 0.00109738647937775
9.85999999999983 0.00121481001377106
9.86999999999983 0.00133914262056351
9.87999999999983 0.00147012546658516
9.88999999999983 0.00160750031471252
9.89999999999983 0.00177517294883728
9.90999999999983 0.00197728231549263
9.91999999999983 0.0021911533176899
9.92999999999983 0.00241651147603989
9.93999999999983 0.00264834761619568
9.94999999999983 0.00277957916259766
9.95999999999983 0.00286750018596649
9.96999999999983 0.00295265883207321
9.97999999999983 0.00305745303630829
9.98999999999983 0.00316382437944412
9.99999999999983 0.00326924324035645
10.0099999999998 0.00337375432252884
10.0199999999998 0.00347732335329056
10.0299999999998 0.00348962187767029
10.0399999999998 0.00342293560504913
10.0499999999998 0.0032686647772789
10.0599999999998 0.00306150108575821
10.0699999999998 0.00287585735321045
10.0799999999998 0.00270812600851059
10.0899999999998 0.00259607911109924
10.0999999999998 0.00254672169685364
10.1099999999998 0.0024944831430912
10.1199999999998 0.00243842765688896
10.1299999999998 0.0023787747323513
10.1399999999998 0.00231577277183533
10.1499999999998 0.00222522303462029
10.1599999999998 0.00213178768754005
10.1699999999998 0.00203993335366249
10.1799999999998 0.00195064276456833
10.1899999999998 0.00187999173998833
10.1999999999998 0.00181527048349381
10.2099999999998 0.00174944028258324
10.2199999999998 0.00168264985084534
10.2299999999998 0.00167260438203812
10.2399999999998 0.00167502328753471
10.2499999999998 0.00167921543121338
10.2599999999998 0.00168469399213791
10.2699999999998 0.00169096171855927
10.2799999999998 0.00169751316308975
10.2899999999998 0.00170394361019135
10.2999999999998 0.00168975502252579
10.3099999999998 0.00164753034710884
10.3199999999998 0.0015979477763176
10.3299999999998 0.00154467090964317
10.3399999999998 0.00150740548968315
10.3499999999998 0.00147332563996315
10.3599999999998 0.00143674060702324
10.3699999999998 0.0013975527882576
10.3799999999998 0.00133215308189392
10.3899999999998 0.00119619302451611
10.3999999999998 0.00104042522609234
10.4099999999998 0.000880472436547279
10.4199999999998 0.000708781331777573
10.4299999999998 0.000542938187718391
10.4399999999998 0.000379222482442856
10.4499999999998 0.000218857135623693
10.4599999999998 6.29101088270545e-05
10.4699999999998 -8.75664129853249e-05
10.4799999999998 -0.000231607332825661
10.4899999999998 -0.000368381813168526
10.4999999999998 -0.000497076660394669
10.5099999999998 -0.000617025122046471
10.5199999999998 -0.000732156559824944
10.5299999999998 -0.000845164731144905
10.5399999999998 -0.000936753302812576
10.5499999999998 -0.0010052827000618
10.5599999999998 -0.000972212180495262
10.5699999999998 -0.000903812348842621
10.5799999999998 -0.000788170620799065
10.5899999999998 -0.000665118917822838
10.5999999999998 -0.000574531406164169
10.6099999999998 -0.000546613074839115
10.6199999999998 -0.000460959672927856
10.6299999999998 -0.00038298811763525
10.6399999999998 -0.000297091212123632
10.6499999999998 -0.00020425146445632
10.6599999999998 -0.000171612463891506
10.6699999999998 -0.000140141937881708
10.6799999999998 -0.000105412220582366
10.6899999999998 -7.09053874015808e-05
10.6999999999998 -5.8076218701899e-05
10.7099999999998 -5.38490898907185e-05
10.7199999999998 -6.06618449091911e-05
10.7299999999998 -7.07170413807035e-05
10.7399999999998 -5.56741701439023e-05
10.7499999999998 -3.36736114695668e-05
10.7599999999998 -8.46326292958111e-06
10.7699999999998 1.97350909002125e-05
10.7799999999998 5.07109286263585e-05
10.7899999999998 8.22205655276775e-05
10.7999999999998 0.000101030804216862
10.8099999999998 0.000119270458817482
10.8199999999998 0.000136919263750315
10.8299999999998 0.000153951589018106
10.8399999999998 0.000170321520417929
10.8499999999998 0.000186013579368591
10.8599999999998 0.000201005134731531
10.8699999999998 0.00021524790674448
10.8799999999998 0.000245579052716494
10.8899999999998 0.000280010886490345
10.8999999999998 0.000314916558563709
10.9099999999998 0.000350151620805264
10.9199999999998 0.00038553349673748
10.9299999999998 0.000420945696532726
10.9399999999998 0.000455044060945511
10.9499999999998 0.000485783033072948
10.9599999999998 0.000518460907042027
10.9699999999998 0.000563117414712906
10.9799999999998 0.000607815310359001
10.9899999999998 0.000652947127819061
10.9999999999998 0.0006980961561203
11.0099999999998 0.000743254721164703
11.0199999999998 0.000788367614150047
11.0299999999998 0.000833374857902527
11.0399999999998 0.000878250524401665
11.0499999999998 0.000922953113913536
11.0599999999998 0.000967441946268082
11.0699999999998 0.00101167686283588
11.0799999999998 0.00105561539530754
11.0899999999998 0.00109919123351574
11.0999999999998 0.00114239901304245
11.1099999999998 0.00115391470491886
11.1199999999998 0.00115413926541805
11.1299999999998 0.0011505950242281
11.1399999999998 0.00114348381757736
11.1499999999998 0.00113296858966351
11.1599999999998 0.00111919961869717
11.1699999999998 0.00110693171620369
11.1799999999998 0.00109899394214153
11.1899999999998 0.00108910657465458
11.1999999999998 0.00107731156051159
11.2099999999998 0.00106364592909813
11.2199999999998 0.0010481895506382
11.2299999999998 0.00103093482553959
11.2399999999998 0.00101195454597473
11.2499999999998 0.000991256982088089
11.2599999999998 0.000968878641724587
11.2699999999998 0.000944874882698059
11.2799999999998 0.000919241681694984
11.2899999999998 0.00089203268289566
11.2999999999998 0.00087043933570385
11.3099999999998 0.000860070884227753
11.3199999999998 0.000849243849515915
11.3299999999998 0.00083794504404068
11.3399999999998 0.00081538163125515
11.3499999999998 0.000792210251092911
11.3599999999998 0.000768413618206978
11.3699999999998 0.000743999928236008
11.3799999999998 0.000718932822346687
11.3899999999998 0.000693207383155823
11.3999999999998 0.000666816979646683
11.4099999999998 0.000639707893133163
11.4199999999998 0.000611909329891205
11.4299999999998 0.00058335680514574
11.4399999999998 0.000554066933691502
11.4499999999998 0.00052398432046175
11.4599999999998 0.000497459657490253
11.4699999999998 0.000473067723214626
11.4799999999998 0.000448408089578152
11.4899999999998 0.000423458702862263
11.4999999999998 0.000398186855018139
11.5099999999998 0.000372585467994213
11.5199999999998 0.000346618741750717
11.5299999999998 0.000320283211767673
11.5399999999998 0.000293523445725441
11.5499999999998 0.000266374684870243
11.5599999999998 0.000238777380436659
11.5699999999998 0.000210752431303263
11.5799999999998 0.000182256419211626
11.5899999999998 0.000157888829708099
11.5999999999998 0.000134695516899228
11.6099999999998 0.000111662177368999
11.6199999999998 8.87257046997547e-05
11.6299999999998 6.58569345250726e-05
11.6399999999998 4.29981295019388e-05
11.6499999999998 2.01350380666554e-05
11.6599999999998 -2.77817249298096e-06
11.6699999999998 -1.65736640337855e-05
11.6799999999998 -2.98040895722806e-05
11.6899999999998 -4.28437534719706e-05
11.6999999999998 -5.56133734062314e-05
11.7099999999998 -6.80700317025185e-05
11.7199999999998 -8.96197557449341e-05
11.7299999999998 -0.000114440266042948
11.7399999999998 -0.000140298092737794
11.7499999999998 -0.00016725329682231
11.7599999999998 -0.000195356346666813
11.7699999999998 -0.000224680993705988
11.7799999999998 -0.000255294293165207
11.7899999999998 -0.000287274029105902
11.7999999999998 -0.00032070629298687
11.8099999999998 -0.000355641320347786
11.8199999999998 -0.000392192378640175
11.8299999999998 -0.000429991371929646
11.8399999999998 -0.000461894981563091
11.8499999999998 -0.000494471751153469
11.8599999999998 -0.000527809448540211
11.8699999999998 -0.000561967939138412
11.8799999999998 -0.000597030110657215
11.8899999999998 -0.000633070543408394
11.8999999999998 -0.000670152306556702
11.9099999999998 -0.000708346217870712
11.9199999999998 -0.000747725367546082
11.9299999999998 -0.000788346752524376
11.9399999999998 -0.000830299183726311
11.9499999999998 -0.000873613208532333
11.9599999999998 -0.000918405428528786
11.9699999999998 -0.000947681367397308
11.9799999999998 -0.000975206196308136
11.9899999999998 -0.00100154891610146
11.9999999999998 -0.00102685146033764
12.0099999999998 -0.00105133809149265
12.0199999999998 -0.00107515506446362
12.0299999999998 -0.00109848201274872
12.0399999999998 -0.00112145207822323
12.0499999999998 -0.00114422053098679
12.0599999999998 -0.00116690963506699
12.0699999999998 -0.00118963487446308
12.0799999999998 -0.00121251121163368
12.0899999999998 -0.00123561561107635
12.0999999999998 -0.00125903010368347
12.1099999999998 -0.00128283932805061
12.1199999999998 -0.00129127040505409
12.1299999999998 -0.00129283741116524
12.1399999999998 -0.00129293948411942
12.1499999999998 -0.0012916561961174
12.1599999999998 -0.00128912582993507
12.1699999999998 -0.00128537178039551
12.1799999999998 -0.0012804901599884
12.1899999999998 -0.00127450317144394
12.1999999999998 -0.00126749619841576
12.2099999999998 -0.0012594573199749
12.2199999999998 -0.00125045925378799
12.2299999999998 -0.00117143504321575
12.2399999999998 -0.00110032685101032
12.2499999999998 -0.00104004420340061
12.2599999999998 -0.00098496325314045
12.2699999999998 -0.000932351276278496
12.2799999999998 -0.000880862101912498
12.2899999999998 -0.000829898193478584
12.2999999999998 -0.000779190063476562
12.3099999999998 -0.000728655904531479
12.3199999999998 -0.000678268373012543
12.3299999999998 -0.000628111883997917
12.3399999999998 -0.000578227080404758
12.3499999999998 -0.000528681464493275
12.3599999999998 -0.000479558259248734
12.3699999999998 -0.000430944599211216
12.3799999999998 -0.000382930934429169
12.3899999999998 -0.000335579328238964
12.3999999999998 -0.000288978666067123
12.4099999999998 -0.000243202373385429
12.4199999999998 -0.000198337491601706
12.4299999999998 -0.000154471304267645
12.4399999999998 -0.000111640719696879
12.4499999999998 -6.9961859844625e-05
12.4599999999998 -2.94673233292997e-05
12.4699999999998 9.75549162831157e-06
12.4799999999998 4.76389238610864e-05
12.4899999999998 8.41207336634397e-05
12.4999999999998 0.000119178080931306
12.5099999999998 0.000152681488543749
12.5199999999998 0.000184636823832989
12.5299999999998 0.000214966014027596
12.5399999999998 0.000243627894669771
12.5499999999998 0.000270576067268848
12.5599999999998 0.000295770056545734
12.5699999999998 0.000319163538515568
12.5799999999998 0.000340726785361767
12.5899999999998 0.000360429584980011
12.5999999999998 0.000378238745033741
12.6099999999998 0.000394141897559166
12.6199999999998 0.000408084355294704
12.6299999999998 0.000420084185898304
12.6399999999998 0.000430101007223129
12.6499999999998 0.000438148602843285
12.6599999999998 0.000444183610379696
12.6699999999998 0.000448246635496616
12.6799999999998 0.000471839345991611
12.6899999999998 0.000498358532786369
12.6999999999998 0.000527304969727993
12.7099999999998 0.000558479651808739
12.7199999999998 0.000591674000024796
12.7299999999998 0.000647215694189072
12.7399999999998 0.000712393000721931
12.7499999999998 0.000784258842468262
12.7599999999998 0.000862590447068214
12.7699999999998 0.000947178229689598
12.7799999999998 0.00104307815432549
12.7899999999998 0.00115686893463135
12.7999999999998 0.00127727955579758
12.8099999999998 0.00140406176447868
12.8199999999998 0.00153695404529572
12.8299999999998 0.00171130314469337
12.8399999999998 0.00191002979874611
12.8499999999998 0.00211952537298203
12.8599999999998 0.00234004601836205
12.8699999999998 0.00257151514291763
12.8799999999998 0.00273725688457489
12.8899999999998 0.00282502233982086
12.8999999999998 0.00290997713804245
12.9099999999998 0.00300936341285706
12.9199999999998 0.00311482220888138
12.9299999999998 0.00321924388408661
12.9399999999998 0.00332267940044403
12.9499999999998 0.00342512279748917
12.9599999999998 0.00350494623184204
12.9699999999998 0.00344672560691833
12.9799999999998 0.00334632217884064
12.9899999999998 0.0031538799405098
12.9999999999998 0.00296116679906845
13.0099999999998 0.00278819650411606
13.0199999999998 0.00265478432178497
13.0299999999998 0.00258117020130157
13.0399999999998 0.00253179669380188
13.0499999999998 0.00247848853468895
13.0599999999998 0.00242146149277687
13.0699999999998 0.00236095398664474
13.0799999999998 0.00229569658637047
13.0899999999998 0.00220088243484497
13.0999999999998 0.00210814595222473
13.1099999999998 0.00201723113656044
13.1199999999998 0.00193172231316566
13.1299999999998 0.00186749488115311
13.1399999999998 0.00180203884840012
13.1499999999998 0.00173551470041275
13.1599999999998 0.00169125646352768
13.1699999999998 0.00169194534420967
13.1799999999998 0.00169462248682976
13.1899999999998 0.00169878423213959
13.1999999999998 0.001703931838274
13.2099999999998 0.00170958489179611
13.2199999999998 0.00171527743339539
13.2299999999998 0.00171464025974274
13.2399999999998 0.0016828279197216
13.2499999999998 0.00163510769605637
13.2599999999998 0.00158369764685631
13.2699999999998 0.00153266653418541
13.2799999999998 0.00149941861629486
13.2899999999998 0.00146377384662628
13.2999999999998 0.00142566710710526
13.3099999999998 0.00137574389576912
13.3199999999998 0.00129126682877541
13.3299999999998 0.00113920092582703
13.3399999999998 0.000982348173856735
13.3499999999998 0.000812834724783897
13.3599999999998 0.00064797542989254
13.3699999999998 0.000484633930027485
13.3799999999998 0.00032401766628027
13.3899999999998 0.000167246144264936
13.3999999999998 1.53934920672327e-05
13.4099999999998 -0.00013055675663054
13.4199999999998 -0.000269718505442142
13.4299999999998 -0.000401292815804482
13.4399999999998 -0.000524518601596355
13.4499999999998 -0.000643884837627411
13.4599999999998 -0.000764613449573517
13.4699999999998 -0.000864774510264397
13.4799999999998 -0.00094267264008522
13.4899999999998 -0.000953929871320725
13.4999999999998 -0.0009011010825634
13.5099999999998 -0.000791362822055817
13.5199999999998 -0.000674007311463356
13.5299999999998 -0.000549674443900585
13.5399999999998 -0.000524349436163902
13.5499999999998 -0.000458326563239098
13.5599999999998 -0.000383981950581074
13.5699999999998 -0.000303388796746731
13.5799999999998 -0.000215520858764648
13.5899999999998 -0.000155271152034402
13.5999999999998 -0.000126173608005047
13.6099999999998 -9.38480161130428e-05
13.6199999999998 -6.26675225794315e-05
13.6299999999998 -3.96161759272218e-05
13.6399999999998 -2.92777805589139e-05
13.6499999999998 -2.4561274331063e-05
13.6599999999998 -3.47470887936652e-05
13.6699999999998 -4.06878627836704e-05
13.6799999999998 -2.13259388692677e-05
13.6899999999998 1.36673450469971e-06
13.6999999999998 2.71713570691645e-05
13.7099999999998 5.58774219825864e-05
13.7199999999998 8.72612465173006e-05
13.7299999999998 0.0001085658185184
13.7399999999998 0.000126142613589764
13.7499999999998 0.000143215591087937
13.7599999999998 0.000159789491444826
13.7699999999998 0.000175794009119272
13.7799999999998 0.000191213022917509
13.7899999999998 0.000206032264977694
13.7999999999998 0.000220196880400181
13.8099999999998 0.000233697388321161
13.8199999999997 0.000258457586169243
13.8299999999997 0.000291737485677004
13.8399999999997 0.000325563997030258
13.8499999999997 0.000359788462519646
13.8599999999997 0.000394232459366322
13.8699999999997 0.000428750962018967
13.8799999999997 0.000460649318993092
13.8899999999997 0.000490235835313797
13.8999999999997 0.000531542859971523
13.9099999999997 0.000575405061244965
13.9199999999997 0.00061933558434248
13.9299999999997 0.000663653090596199
13.9399999999997 0.000708001554012299
13.9499999999997 0.000752319246530533
13.9599999999997 0.000796602144837379
13.9699999999997 0.000840786769986153
13.9799999999997 0.000884831771254539
13.9899999999997 0.000928724184632301
13.9999999999997 0.000972381085157394
14.0099999999997 0.00101579330861568
14.0199999999997 0.00105890884995461
14.0299999999997 0.00110168524086475
14.0399999999997 0.00114409156143665
14.0499999999997 0.00116506546735764
14.0599999999997 0.00116653569042683
14.0699999999997 0.00116432353854179
14.0799999999997 0.00115861013531685
14.0899999999997 0.00114955067634583
14.0999999999997 0.00113732352852821
14.1099999999997 0.00112330056726933
14.1199999999997 0.00111628510057926
14.1299999999997 0.00110737025737762
14.1399999999997 0.00109658606350422
14.1499999999997 0.00108397848904133
14.1599999999997 0.00106961444020271
14.1699999999997 0.00105351477861404
14.1799999999997 0.00103572137653828
14.1899999999997 0.00101626120507717
14.1999999999997 0.000995180606842041
14.2099999999997 0.000972507148981094
14.2199999999997 0.000948261171579361
14.2299999999997 0.000922477915883064
14.2399999999997 0.000895191952586174
14.2499999999997 0.000878242179751396
14.2599999999997 0.000867429673671722
14.2699999999997 0.000856187045574188
14.2799999999997 0.000844475105404854
14.2899999999997 0.00082347959280014
14.2999999999997 0.000800575017929077
14.3099999999997 0.000777085646986961
14.3199999999997 0.000752990618348122
14.3299999999997 0.000728276073932648
14.3399999999997 0.000702918842434883
14.3499999999997 0.000676916241645813
14.3599999999997 0.000650244578719139
14.3699999999997 0.000622895918786526
14.3799999999997 0.000594831481575966
14.3899999999997 0.000566050000488758
14.3999999999997 0.000536527037620544
14.4099999999997 0.000507195889949799
14.4199999999997 0.000482845641672611
14.4299999999997 0.000458265207707882
14.4399999999997 0.000433402732014656
14.4499999999997 0.000408251211047173
14.4599999999997 0.000382782556116581
14.4699999999997 0.000356986746191978
14.4799999999997 0.000330825671553612
14.4899999999997 0.000304303523153067
14.4999999999997 0.000277396533638239
14.5099999999997 0.000250079706311226
14.5199999999997 0.000226400960236788
14.5299999999997 0.000204526204615831
14.5399999999997 0.000182882212102413
14.5499999999997 0.000161401703953743
14.5599999999997 0.000140060875564814
14.5699999999997 0.000118813905864954
14.5799999999997 9.7611965611577e-05
14.5899999999997 7.6422318816185e-05
14.5999999999997 5.52187953144312e-05
14.6099999999997 3.88204562477767e-05
14.6199999999997 2.69222096540034e-05
14.6299999999997 1.52587855700403e-05
14.6399999999997 3.85403633117676e-06
14.6499999999997 -7.20500887837261e-06
14.6599999999997 -1.78956938907504e-05
14.6699999999997 -2.82239727675915e-05
14.6799999999997 -3.81552707403898e-05
14.6899999999997 -5.48563990741968e-05
14.6999999999997 -7.60456221178174e-05
14.7099999999997 -9.82956122606993e-05
14.7199999999997 -0.000121652772650123
14.7299999999997 -0.000146165871992707
14.7399999999997 -0.000171891991049051
14.7499999999997 -0.000198917984962463
14.7599999999997 -0.000227249022573233
14.7699999999997 -0.000256998706609011
14.7799999999997 -0.000288204271346331
14.7899999999997 -0.000320965498685837
14.7999999999997 -0.0003553207218647
14.8099999999997 -0.000388750918209553
14.8199999999997 -0.000418605469167233
14.8299999999997 -0.000449169427156448
14.8399999999997 -0.000480506308376789
14.8499999999997 -0.000512711107730865
14.8599999999997 -0.000545822717249394
14.8699999999997 -0.000579913966357708
14.8799999999997 -0.000615071356296539
14.8899999999997 -0.000651327073574066
14.8999999999997 -0.000688754245638847
14.9099999999997 -0.000727412402629852
14.9199999999997 -0.000767364948987961
14.9299999999997 -0.000805485025048256
14.9399999999997 -0.000831574946641922
14.9499999999997 -0.000856372117996216
14.9599999999997 -0.000880060791969299
14.9699999999997 -0.000902808904647827
14.9799999999997 -0.000924795418977737
14.9899999999997 -0.000946152359247208
14.9999999999997 -0.000967040583491325
15.0099999999997 -0.000987605005502701
15.0199999999997 -0.0010079450160265
15.0299999999997 -0.00100452646613121
15.0399999999997 -0.000305196195840836
15.0499999999997 0.00111739650368691
15.0599999999997 0.00215449094772339
15.0699999999997 0.00296888440847397
15.0799999999997 0.00360273629426956
15.0899999999997 0.0041677787899971
15.0999999999997 0.00457624524831772
15.1099999999997 0.00501188695430756
15.1199999999997 0.00537966430187225
15.1299999999997 0.00560623586177826
15.1399999999997 0.00584659516811371
15.1499999999997 0.00609159767627716
15.1599999999997 0.00633339703083038
15.1699999999997 0.00656776785850525
15.1799999999997 0.00675272226333618
15.1899999999997 0.00688914477825165
15.1999999999997 0.00700097739696503
15.2099999999997 0.00710084557533264
15.2199999999997 0.00720351219177246
15.2299999999997 0.00733181178569794
15.2399999999997 0.00749546706676483
15.2499999999997 0.00758625566959381
15.2599999999997 0.00768635988235474
15.2699999999997 0.00779017448425293
15.2799999999997 0.00789447128772736
15.2899999999997 0.00799869358539581
15.2999999999997 0.00810002267360687
15.3099999999997 0.00819718778133392
15.3199999999997 0.00828935563564301
15.3299999999997 0.00837603986263275
15.3399999999997 0.00845690608024597
15.3499999999997 0.00853177666664123
15.3599999999997 0.00858350574970245
15.3699999999997 0.00861823320388794
15.3799999999997 0.00865230917930603
15.3899999999997 0.00870989441871643
15.3999999999997 0.00876647412776947
15.4099999999997 0.00881593942642212
15.4199999999997 0.00885901868343353
15.4299999999997 0.00880650460720062
15.4399999999997 0.00872707784175873
15.4499999999997 0.00865958571434021
15.4599999999997 0.00862324416637421
15.4699999999997 0.00866562902927399
15.4799999999997 0.00871283710002899
15.4899999999997 0.00875284194946289
15.4999999999997 0.00879500031471252
15.5099999999997 0.0087901097536087
15.5199999999997 0.00875023782253265
15.5299999999997 0.00867277204990387
15.5399999999997 0.00855440616607666
15.5499999999997 0.00845111906528473
15.5599999999997 0.0083563619852066
15.5699999999997 0.00826601564884186
15.5799999999997 0.0081775438785553
15.5899999999997 0.00808946132659912
15.5999999999997 0.00800085663795471
15.6099999999997 0.00794177412986755
15.6199999999997 0.00789152562618256
15.6299999999997 0.00784053921699524
15.6399999999997 0.00778164327144623
15.6499999999997 0.00771770834922791
15.6599999999997 0.00765046834945679
15.6699999999997 0.00758105635643005
15.6799999999997 0.00751012742519379
15.6899999999997 0.0074381685256958
15.6999999999997 0.00736551582813263
15.7099999999997 0.00747515082359314
15.7199999999997 0.00758804738521576
15.7299999999997 0.00767891705036163
15.7399999999997 0.00773127794265747
15.7499999999997 0.00776662766933441
15.7599999999997 0.00757530629634857
15.7699999999997 0.00729478359222412
15.7799999999997 0.00714735150337219
15.7899999999997 0.00681307256221771
15.7999999999997 0.00648105621337891
15.8099999999997 0.00618535816669464
15.8199999999997 0.00597584962844849
15.8299999999997 0.00588376820087433
15.8399999999997 0.00574428975582123
15.8499999999997 0.00558993101119995
15.8599999999997 0.00543163418769836
15.8699999999997 0.00527327477931976
15.8799999999997 0.00511491417884827
15.8899999999997 0.00494496881961822
15.8999999999997 0.00479568332433701
15.9099999999997 0.00465553641319275
15.9199999999997 0.00452105611562729
15.9299999999997 0.00440946012735367
15.9399999999997 0.00433239221572876
15.9499999999997 0.00423536747694016
15.9599999999997 0.00414913982152939
15.9699999999997 0.00405726313591003
15.9799999999997 0.00396608591079712
15.9899999999997 0.00387987315654755
15.9999999999997 0.00386770397424698
16.0099999999997 0.0038410097360611
16.0199999999997 0.00380968421697617
16.0299999999997 0.00379778027534485
16.0399999999997 0.00380133748054504
16.0499999999997 0.00380265176296234
16.0599999999997 0.00380562305450439
16.0699999999997 0.00381221175193787
16.0799999999997 0.00375887274742126
16.0899999999997 0.0036623153090477
16.0999999999997 0.00360171556472778
16.1099999999997 0.00356100022792816
16.1199999999997 0.00353765845298767
16.1299999999997 0.00352794587612152
16.1399999999997 0.0035151818394661
16.1499999999997 0.00350081533193588
16.1599999999997 0.0034980434179306
16.1699999999997 0.00350395411252975
16.1799999999997 0.00351585000753403
16.1899999999997 0.00351855844259262
16.1999999999997 0.0035176146030426
16.2099999999997 0.00352264821529388
16.2199999999997 0.00353392213582993
16.2299999999997 0.00355148434638977
16.2399999999997 0.00357526838779449
16.2499999999997 0.00360522866249084
16.2599999999997 0.00366903394460678
16.2699999999997 0.00373863071203232
16.2799999999997 0.00381074458360672
16.2899999999997 0.00388664245605469
16.2999999999997 0.00393518090248108
16.3099999999998 0.00397254943847656
16.3199999999998 0.00402138501405716
16.3299999999998 0.00407839477062225
16.3399999999998 0.00414191335439682
16.3499999999998 0.00421090990304947
16.3599999999998 0.00428486466407776
16.3699999999998 0.0043633908033371
16.3799999999998 0.00444621831178665
16.3899999999998 0.00453313767910004
16.3999999999998 0.00462390452623367
16.4099999999998 0.00471836298704147
16.4199999999998 0.00481631755828857
16.4299999999998 0.00491761147975922
16.4399999999998 0.00502203643321991
16.4499999999998 0.00512941718101501
16.4599999999998 0.00523957967758179
16.4699999999998 0.00535232961177826
16.4799999999998 0.005467489361763
16.4899999999998 0.00558488070964813
16.4999999999998 0.00570427775382996
16.5099999999998 0.00582554578781128
16.5199999999998 0.00594845950603485
16.5299999999998 0.0060728532075882
16.5399999999998 0.00619852244853973
16.5499999999998 0.00630837976932526
16.5599999999998 0.00642182886600494
16.5699999999998 0.00653778076171875
16.5799999999998 0.00665509343147278
16.5899999999998 0.00676199615001678
16.5999999999998 0.00688324391841888
16.6099999999998 0.00702603936195374
16.6199999999998 0.00716478049755096
16.6299999999998 0.00724586486816406
16.6399999999998 0.00731894791126251
16.6499999999998 0.00739859461784363
16.6599999999998 0.00748032808303833
16.6699999999998 0.00756249189376831
16.6799999999998 0.00764448344707489
16.6899999999998 0.00772606194019318
16.6999999999998 0.00780769765377045
16.7099999999998 0.00791054904460907
16.7199999999998 0.00800636887550354
16.7299999999998 0.00809918761253357
16.7399999999998 0.00819045722484589
16.7499999999998 0.00829125702381134
16.7599999999998 0.00842251062393188
16.7699999999998 0.00854409217834473
16.7799999999998 0.00866186678409576
16.7899999999998 0.00877776741981506
16.7999999999998 0.00890834987163544
16.8099999999998 0.00905006408691406
16.8199999999998 0.00918727517127991
16.8299999999998 0.00932269692420959
16.8399999999998 0.00945678532123566
16.8499999999998 0.00958948016166687
16.8599999999998 0.00972054839134216
16.8699999999998 0.00984973251819611
16.8799999999998 0.00997434139251709
16.8899999999998 0.0100938379764557
16.8999999999998 0.0102095758914948
16.9099999999998 0.0103220820426941
16.9199999999998 0.0104314637184143
16.9299999999998 0.0105376303195953
16.9399999999998 0.0106404888629913
16.9499999999999 0.010739973783493
16.9599999999999 0.0108359801769257
16.9699999999999 0.0109284353256226
16.9799999999999 0.0110172629356384
16.9899999999999 0.0111024284362793
16.9999999999999 0.0111838698387146
17.0099999999999 0.0112615525722504
17.0199999999999 0.0113354468345642
17.0299999999999 0.0114055073261261
17.0399999999999 0.011427149772644
17.0499999999999 0.0114479494094849
17.0599999999999 0.0114967751502991
17.0699999999999 0.0115370678901672
17.0799999999999 0.0115703225135803
17.0899999999999 0.0115976119041443
17.0999999999999 0.0116197621822357
17.1099999999999 0.0116373658180237
17.1199999999999 0.0116508758068085
17.1299999999999 0.0116606390476227
17.1399999999999 0.0116669070720673
17.1499999999999 0.0116698861122131
17.1599999999999 0.0116697537899017
17.1699999999999 0.0116666293144226
17.1799999999999 0.011660623550415
17.1899999999999 0.0116518235206604
17.1999999999999 0.0116403269767761
17.2099999999999 0.011626181602478
17.2199999999999 0.0116094708442688
17.2299999999999 0.01159024477005
17.2399999999999 0.0115685713291168
17.2499999999999 0.0115445005893707
17.2599999999999 0.0115180897712708
17.2699999999999 0.0114893841743469
17.2799999999999 0.0114584469795227
17.2899999999999 0.0114253187179565
17.2999999999999 0.0113900804519653
17.3099999999999 0.0113527512550354
17.3199999999999 0.011313408613205
17.3299999999999 0.0112720990180969
17.3399999999999 0.0112288880348206
17.3499999999999 0.0111838352680206
17.3599999999999 0.0111369752883911
17.3699999999999 0.0110884094238281
17.3799999999999 0.0110381674766541
17.3899999999999 0.0109863352775574
17.3999999999999 0.0109321355819702
17.4099999999999 0.0108742117881775
17.4199999999999 0.0108141934871674
17.4299999999999 0.010752284526825
17.4399999999999 0.0106886827945709
17.4499999999999 0.0106235098838806
17.4599999999999 0.0105569040775299
17.4699999999999 0.0104867839813232
17.4799999999999 0.0104144537448883
17.4899999999999 0.0103408825397491
17.4999999999999 0.0102663266658783
17.5099999999999 0.0101909601688385
17.5199999999999 0.0101149415969849
17.5299999999999 0.0100384664535522
17.5399999999999 0.0099616551399231
17.5499999999999 0.00988466858863831
17.5599999999999 0.00980761349201202
17.5699999999999 0.00973065793514252
17.5799999999999 0.0096538919210434
17.59 0.00957747399806976
17.6 0.00950149953365326
17.61 0.00942607939243317
17.62 0.00935134172439575
17.63 0.00927739858627319
17.64 0.00920433580875397
17.65 0.00913229286670685
17.66 0.00906133592128754
17.67 0.00899160265922546
17.68 0.00892316341400146
17.69 0.0088561350107193
17.7 0.00879059255123139
17.71 0.00872662603855133
17.72 0.00866432070732117
17.73 0.00860376477241516
17.74 0.00854503929615021
17.75 0.00848473012447357
17.76 0.00842554807662964
17.77 0.00836781919002533
17.78 0.00831168174743652
17.79 0.00825722217559815
17.8 0.00820455193519592
17.81 0.00815374851226807
17.82 0.0081048971414566
17.83 0.00805807292461395
17.84 0.0080133330821991
17.85 0.0079707396030426
17.86 0.00793034613132477
17.87 0.00789219617843628
17.88 0.0078563380241394
17.89 0.00782281100749969
17.9 0.00779162466526032
17.91 0.0077628481388092
17.92 0.00773646712303162
17.93 0.007712522149086
17.94 0.00768829941749573
17.95 0.00764640331268311
17.96 0.00760798037052155
17.97 0.00757283926010132
17.98 0.00754084646701813
17.99 0.00751191556453705
18 0.00748599767684937
18.01 0.00746308565139771
18.02 0.00744318127632141
18.03 0.00742625832557678
18.04 0.00741236329078674
18.05 0.00740316569805145
18.06 0.00739852786064148
18.07 0.00739628076553345
18.08 0.00739649653434753
18.09 0.00739926040172577
18.1 0.00740458846092224
18.11 0.00741249084472656
18.12 0.00742298245429993
18.13 0.00743602812290192
18.14 0.00747804641723633
18.15 0.00752891659736633
18.16 0.00755863785743713
18.17 0.00758390784263611
18.18 0.0076094788312912
18.19 0.00763664186000824
18.2 0.00766576588153839
18.21 0.00769690275192261
18.22 0.00773282527923584
18.2300000000001 0.00777136564254761
18.2400000000001 0.00781156182289124
18.2500000000001 0.00785374581813812
18.2600000000001 0.00789790570735931
18.2700000000001 0.00794397830963135
18.2800000000001 0.00799185991287231
18.2900000000001 0.00804145574569702
18.3000000000001 0.00809267282485962
18.3100000000001 0.00814539194107056
18.3200000000001 0.00819948494434357
18.3300000000001 0.00825484454631805
18.3400000000001 0.00831137001514435
18.3500000000001 0.00836893320083618
18.3600000000001 0.00842739343643189
18.3700000000001 0.00848666846752167
18.3800000000001 0.00854661405086517
18.3900000000001 0.00860711216926575
18.4000000000001 0.00866806447505951
18.4100000000001 0.00872931838035584
18.4200000000001 0.00879079878330231
18.4300000000001 0.00885234713554382
18.4400000000001 0.00891387939453125
18.4500000000001 0.00897528886795044
18.4600000000001 0.00903644740581512
18.4700000000001 0.00909723579883575
18.4800000000001 0.00915759265422821
18.4900000000001 0.00921738207340241
18.5000000000001 0.00927651584148407
18.5100000000001 0.00932270050048828
18.5200000000001 0.0093531322479248
18.5300000000001 0.00938919067382813
18.5400000000001 0.00942615926265717
18.5500000000001 0.00946245133876801
18.5600000000001 0.00949749290943146
18.5700000000001 0.00953113496303558
18.5800000000001 0.00956331133842468
18.5900000000001 0.00959402084350586
18.6000000000001 0.00962324380874634
18.6100000000001 0.00965099036693573
18.6200000000001 0.00967726588249207
18.6300000000001 0.00970773458480835
18.6400000000001 0.00973797976970673
18.6500000000001 0.00976545333862305
18.6600000000001 0.00979049265384674
18.6700000000001 0.00981336116790772
18.6800000000001 0.00983426332473755
18.6900000000001 0.00985332250595093
18.7000000000001 0.00987065434455872
18.7100000000001 0.00988633990287781
18.7200000000001 0.0099004465341568
18.7300000000001 0.0099130117893219
18.7400000000001 0.00992407560348511
18.7500000000001 0.00993366599082947
18.7600000000001 0.00994181036949158
18.7700000000001 0.00994852066040039
18.7800000000001 0.00995383620262146
18.7900000000001 0.00995775103569031
18.8000000000001 0.00996029913425446
18.8100000000001 0.00996148943901062
18.8200000000001 0.00996133804321289
18.8300000000001 0.00995987415313721
18.8400000000001 0.00995709717273712
18.8500000000001 0.00995303869247437
18.8600000000001 0.00994771659374237
18.8700000000002 0.00994114875793457
18.8800000000002 0.00993336379528046
18.8900000000002 0.00992437660694122
18.9000000000002 0.00991421520709991
18.9100000000002 0.00990291118621826
18.9200000000002 0.00989047646522522
18.9300000000002 0.00987695157527924
18.9400000000002 0.00986235678195953
18.9500000000002 0.00984672665596008
18.9600000000002 0.0098300963640213
18.9700000000002 0.0098124885559082
18.9800000000002 0.00979393184185028
18.9900000000002 0.00977447211742401
19.0000000000002 0.00975413680076599
19.0100000000002 0.00973296046257019
19.0200000000002 0.00971099257469177
19.0300000000002 0.00968824803829193
19.0400000000002 0.00966477870941162
19.0500000000002 0.00964062213897705
19.0600000000002 0.00961581230163574
19.0700000000002 0.00959038615226746
19.0800000000002 0.0095643937587738
19.0900000000002 0.00953786969184875
19.1000000000002 0.00951085925102234
19.1100000000002 0.00948339402675629
19.1200000000002 0.00945553362369537
19.1300000000002 0.00942730069160462
19.1400000000002 0.00939875364303589
19.1500000000002 0.00936993062496185
19.1600000000002 0.00934086799621582
19.1700000000002 0.00931161105632782
19.1800000000002 0.00928221523761749
19.1900000000002 0.00925269186496735
19.2000000000002 0.00922312021255493
19.2100000000002 0.00919352650642395
19.2200000000002 0.00916395664215088
19.2300000000002 0.00913446366786957
19.2400000000002 0.00910506248474121
19.2500000000002 0.00907581984996796
19.2600000000002 0.0090467780828476
19.2700000000002 0.00901795208454132
19.2800000000002 0.00898940742015839
19.2900000000002 0.00896117925643921
19.3000000000002 0.00893330156803131
19.3100000000002 0.00890580892562866
19.3200000000002 0.00887875139713287
19.3300000000002 0.0088521671295166
19.3400000000002 0.00882607579231262
19.3500000000002 0.00880053043365478
19.3600000000002 0.00877554893493652
19.3700000000002 0.00875117063522339
19.3800000000002 0.00872743666172028
19.3900000000002 0.00870435893535614
19.4000000000002 0.00868198037147522
19.4100000000002 0.00866033732891083
19.4200000000002 0.00863944232463837
19.4300000000002 0.00861932873725891
19.4400000000002 0.00860001027584076
19.4500000000002 0.00858151495456696
19.4600000000002 0.00856387138366699
19.4700000000002 0.00854709327220917
19.4800000000002 0.00853118658065796
19.4900000000002 0.00851621091365814
19.5000000000002 0.00850213527679443
19.5100000000003 0.00848899483680725
19.5200000000003 0.00847679972648621
19.5300000000003 0.00846556544303894
19.5400000000003 0.00845528662204743
19.5500000000003 0.00844598472118378
19.5600000000003 0.00843766272068024
19.5700000000003 0.00843030333518982
19.5800000000003 0.00842395186424255
19.5900000000003 0.00841857135295868
19.6000000000003 0.00841418862342834
19.6100000000003 0.00841078162193298
19.6200000000003 0.00840835809707642
19.6300000000003 0.00840690076351166
19.6400000000003 0.00840641796588898
19.6500000000003 0.00840689659118652
19.6600000000003 0.00840832471847534
19.6700000000003 0.00841069161891937
19.6800000000003 0.00841399431228638
19.6900000000003 0.00841821134090424
19.7000000000003 0.00842330753803253
19.7100000000003 0.00842931509017944
19.7200000000003 0.00843616664409638
19.7300000000003 0.00844388365745545
19.7400000000003 0.00844951629638672
19.7500000000003 0.00845305442810059
19.7600000000003 0.00845786929130554
19.7700000000003 0.00846389532089233
19.7800000000003 0.00847107589244843
19.7900000000003 0.00847937345504761
19.8000000000003 0.00848871827125549
19.8100000000003 0.00849909961223602
19.8200000000003 0.00851049959659576
19.8300000000003 0.00852287232875824
19.8400000000003 0.00853621184825897
19.8500000000003 0.00855048060417175
19.8600000000003 0.00856566965579987
19.8700000000003 0.00858174204826355
19.8800000000003 0.00859866738319397
19.8900000000003 0.00861642956733704
19.9000000000003 0.00863499820232391
19.9100000000003 0.00865433216094971
19.9200000000003 0.00867441415786743
19.9300000000003 0.00869520008563995
19.9400000000003 0.0087166690826416
19.9500000000003 0.00873877227306366
19.9600000000003 0.00876148879528046
19.9700000000003 0.00878477275371552
19.9800000000003 0.00880857527256012
19.9900000000003 0.00883287847042084
20.0000000000003 0.00885763943195343
20.0100000000003 0.0088828045129776
20.0200000000003 0.00890834152698517
20.0300000000003 0.00893421351909638
20.0400000000003 0.00896037220954895
20.0500000000003 0.00898676633834839
20.0600000000003 0.00900772869586945
20.0700000000003 0.00902863800525665
20.0800000000003 0.00904947280883789
20.0900000000003 0.00907018721103668
20.1000000000003 0.00909072518348694
20.1100000000003 0.00911105453968048
20.1200000000003 0.00913113474845886
20.1300000000003 0.00915092468261719
20.1400000000003 0.0091703999042511
20.1500000000004 0.00918954193592071
20.1600000000004 0.0092083203792572
20.1700000000004 0.00922670722007751
20.1800000000004 0.00924469292163849
20.1900000000004 0.00926223576068878
20.2000000000004 0.0092793333530426
20.2100000000004 0.00929597020149231
20.2200000000004 0.00931212544441223
20.2300000000004 0.00932778239250183
20.2400000000004 0.00934292495250702
20.2500000000004 0.00935754358768463
20.2600000000004 0.00937162756919861
20.2700000000004 0.00938514769077301
20.2800000000004 0.00939811646938324
20.2900000000004 0.00941050291061401
20.3000000000004 0.00942231059074402
20.3100000000004 0.00943353354930878
20.3200000000004 0.00944414734840393
20.3300000000004 0.00945416033267975
20.3400000000004 0.00946356117725372
20.3500000000004 0.00947234809398651
20.3600000000004 0.00948051273822785
20.3700000000004 0.00948805510997772
20.3800000000004 0.00949496328830719
20.3900000000004 0.00950124025344849
20.4000000000004 0.00950689256191254
20.4100000000004 0.00951190829277039
20.4200000000004 0.00951629817485809
20.4300000000004 0.00952006220817566
20.4400000000004 0.00952319383621216
20.4500000000004 0.00952570855617523
20.4600000000004 0.00952760398387909
20.4700000000004 0.00952887415885925
20.4800000000004 0.00952954053878784
20.4900000000004 0.00952959656715393
20.5000000000004 0.00952905476093292
20.5100000000004 0.00952792465686798
20.5200000000004 0.00952620446681976
20.5300000000004 0.00952390849590302
20.5400000000004 0.00952105224132538
20.5500000000004 0.00951763153076172
20.5600000000004 0.00951366603374481
20.5700000000004 0.00950915694236755
20.5800000000004 0.00950412452220917
20.5900000000004 0.00949857711791992
20.6000000000004 0.00949252307415009
20.6100000000004 0.00948599219322205
20.6200000000004 0.00947896897792816
20.6300000000004 0.00947149276733398
20.6400000000004 0.00946357131004334
20.6500000000004 0.00945520222187042
20.6600000000004 0.00944642663002014
20.6700000000004 0.00943723320960999
20.6800000000004 0.00942767381668091
20.6900000000004 0.00941773295402527
20.7000000000004 0.00940743446350098
20.7100000000004 0.0093967992067337
20.7200000000004 0.00938583970069885
20.7300000000004 0.00937457084655762
20.7400000000004 0.00936303079128265
20.7500000000004 0.00935120582580566
20.7600000000004 0.00933912932872772
20.7700000000004 0.00932682633399963
20.7800000000004 0.00931430876255035
20.7900000000005 0.00930158197879791
20.8000000000005 0.00928868174552917
20.8100000000005 0.00927562594413757
20.8200000000005 0.00926242113113403
20.8300000000005 0.00924909234046936
20.8400000000005 0.00923565804958344
20.8500000000005 0.00922213971614838
20.8600000000005 0.00920855104923248
20.8700000000005 0.00919490516185761
20.8800000000005 0.00918122470378876
20.8900000000005 0.00916754245758057
20.9000000000005 0.00915385246276855
20.9100000000005 0.00914019107818603
20.9200000000005 0.00912656366825104
20.9300000000005 0.00911299347877502
20.9400000000005 0.00909951210021973
20.9500000000005 0.00908609986305237
20.9600000000005 0.00907280921936035
20.9700000000005 0.00905963480472565
20.9800000000005 0.0090466046333313
20.9900000000005 0.00903373420238495
21.0000000000005 0.00902103066444397
21.0100000000005 0.00900851845741272
21.0200000000005 0.00899620592594147
21.0300000000005 0.00898411214351654
21.0400000000005 0.00897224605083466
21.0500000000005 0.00896062552928925
21.0600000000005 0.00894926249980926
21.0700000000005 0.00893815994262695
21.0800000000005 0.0089273464679718
21.0900000000005 0.00891683161258698
21.1000000000005 0.00890660226345062
21.1100000000005 0.00889671206474304
21.1200000000005 0.00888713002204895
21.1300000000005 0.00887789070606232
21.1400000000005 0.00886898398399353
21.1500000000005 0.00886044204235077
21.1600000000005 0.00885224580764771
21.1700000000005 0.00884443759918213
21.1800000000005 0.00883698880672455
21.1900000000005 0.00882991671562195
21.2000000000005 0.00882323384284973
21.2100000000005 0.00881694436073303
21.2200000000005 0.00881105780601501
21.2300000000005 0.00880555331707001
21.2400000000005 0.00880045592784882
21.2500000000005 0.00879575848579407
21.2600000000005 0.00879148125648499
21.2700000000005 0.00878759920597076
21.2800000000005 0.00878411889076233
21.2900000000005 0.00878106474876404
21.3000000000005 0.00877839684486389
21.3100000000005 0.00877614557743073
21.3200000000005 0.008774294257164
21.3300000000005 0.00877286195755005
21.3400000000005 0.00877181470394135
21.3500000000005 0.00877116203308105
21.3600000000005 0.00877091407775879
21.3700000000005 0.00877105176448822
21.3800000000005 0.00877157509326935
21.3900000000005 0.0087724643945694
21.4000000000005 0.00877374172210693
21.4100000000005 0.00877538204193115
21.4200000000005 0.00877737641334534
21.4300000000006 0.00877972900867462
21.4400000000006 0.00878242909908295
21.4500000000006 0.0087854653596878
21.4600000000006 0.00878882646560669
21.4700000000006 0.0087925124168396
21.4800000000006 0.00879651129245758
21.4900000000006 0.00880081832408905
21.5000000000006 0.00880540013313293
21.5100000000006 0.00881028354167938
21.5200000000006 0.00881543159484863
21.5300000000006 0.00882085084915161
21.5400000000006 0.00882651150226593
21.5500000000006 0.00883242130279541
21.5600000000006 0.00883855402469635
21.5700000000006 0.00884490430355072
21.5800000000006 0.00885145783424377
21.5900000000006 0.00885822236537933
21.6000000000006 0.00886515080928803
21.6100000000006 0.00887225389480591
21.6200000000006 0.00887952268123627
21.6300000000006 0.00888693690299988
21.6400000000006 0.00889449536800384
21.6500000000006 0.00890217125415802
21.6600000000006 0.0089099508523941
21.6700000000006 0.00891783535480499
21.6800000000006 0.00892580687999725
21.6900000000006 0.00893386542797089
21.7000000000006 0.008941969871521
21.7100000000006 0.00895012855529785
21.7200000000006 0.00895833313465118
21.7300000000006 0.00896657407283783
21.7400000000006 0.00897481203079224
21.7500000000006 0.00898306548595429
21.7600000000006 0.00899132132530212
21.7700000000006 0.00899954617023468
21.7800000000006 0.00900775551795959
21.7900000000006 0.00901591658592224
21.8000000000006 0.0090240216255188
21.8100000000006 0.00903208076953888
21.8200000000006 0.0090400630235672
21.8300000000006 0.0090479701757431
21.8400000000006 0.0090557861328125
21.8500000000006 0.00906350076198578
21.8600000000006 0.00907111167907715
21.8700000000006 0.00907860457897186
21.8800000000006 0.00908596992492676
21.8900000000006 0.00909320592880249
21.9000000000006 0.00910029709339142
21.9100000000006 0.00910722255706787
21.9200000000006 0.00911401689052582
21.9300000000006 0.00912063539028168
21.9400000000006 0.00912708342075348
21.9500000000006 0.00913336575031281
21.9600000000006 0.00913945734500885
21.9700000000006 0.00914534568786621
21.9800000000006 0.00915105223655701
21.9900000000006 0.00915656924247742
22.0000000000006 0.00916187763214111
22.0100000000006 0.00916696429252625
22.0200000000006 0.00917184233665466
22.0300000000006 0.00917650997638702
22.0400000000006 0.00918095529079437
22.0500000000006 0.00918517887592316
22.0600000000006 0.00918917953968048
22.0700000000007 0.00919294357299805
22.0800000000007 0.0091964840888977
22.0900000000007 0.00919979512691498
22.1000000000007 0.00920286774635315
22.1100000000007 0.00920570015907288
22.1200000000007 0.00920830309391022
22.1300000000007 0.00921067655086517
22.1400000000007 0.00921279549598694
22.1500000000007 0.00921470105648041
22.1600000000007 0.00921635448932648
22.1700000000007 0.00921778738498688
22.1800000000007 0.00921898245811462
22.1900000000007 0.00921995043754578
22.2000000000007 0.0092206883430481
22.2100000000007 0.00922118782997131
22.2200000000007 0.00922147393226624
22.2300000000007 0.00922153472900391
22.2400000000007 0.00922137558460236
22.2500000000007 0.00922099590301514
22.2600000000007 0.00922041118144989
22.2700000000007 0.00921960592269898
22.2800000000007 0.0092186039686203
22.2900000000007 0.00921740233898163
22.3000000000007 0.00921599984169006
22.3100000000007 0.00921438753604889
22.3200000000007 0.00921258985996246
22.3300000000007 0.00921062409877777
22.3400000000007 0.00920846939086914
22.3500000000007 0.00920614540576935
22.3600000000007 0.00920365273952484
22.3700000000007 0.00920099794864655
22.3800000000007 0.0091981840133667
22.3900000000007 0.0091952246427536
22.4000000000007 0.00919211626052856
22.4100000000007 0.00918887197971344
22.4200000000007 0.00918549478054047
22.4300000000007 0.00918198585510254
22.4400000000007 0.00917836308479309
22.4500000000007 0.00917462885379791
22.4600000000007 0.00917078077793121
22.4700000000007 0.0091668426990509
22.4800000000007 0.00916280210018158
22.4900000000007 0.00915867567062378
22.5000000000007 0.00915447354316712
22.5100000000007 0.00915019869804382
22.5200000000007 0.00914584875106812
22.5300000000007 0.00914144933223724
22.5400000000007 0.00913699090480804
22.5500000000007 0.00913249015808106
22.5600000000007 0.00912794709205627
22.5700000000007 0.0091233628988266
22.5800000000007 0.00911876857280731
22.5900000000007 0.00911413729190826
22.6000000000007 0.0091094970703125
22.6100000000007 0.00910486400127411
22.6200000000007 0.00910021960735321
22.6300000000007 0.00909557819366455
22.6400000000007 0.00909095525741577
22.6500000000007 0.00908636271953583
22.6600000000007 0.0090817803144455
22.6700000000007 0.00907723009586334
22.6800000000007 0.00907271802425385
22.6900000000007 0.00906824707984924
22.7000000000007 0.00906383514404297
22.7100000000008 0.00905946850776672
22.7200000000008 0.00905516028404236
22.7300000000008 0.00905091822147369
22.7400000000008 0.00904674589633942
22.7500000000008 0.00904265403747559
22.7600000000008 0.0090386426448822
22.7700000000008 0.00903472065925598
22.7800000000008 0.00903086841106415
22.7900000000008 0.00902712404727936
22.8000000000008 0.00902347087860107
22.8100000000008 0.00901992559432983
22.8200000000008 0.00901647806167603
22.8300000000008 0.00901314556598663
22.8400000000008 0.00900991797447205
22.8500000000008 0.00900681912899017
22.8600000000008 0.00900382101535797
22.8700000000008 0.00900095403194428
22.8800000000008 0.00899820923805237
22.8900000000008 0.00899558842182159
22.9000000000008 0.00899309813976288
22.9100000000008 0.00899072051048279
22.9200000000008 0.00898849785327911
22.9300000000008 0.00898640096187592
22.9400000000008 0.00898443758487701
22.9500000000008 0.00898261427879334
22.9600000000008 0.0089809238910675
22.9700000000008 0.00897937834262848
22.9800000000008 0.00897795736789703
22.9900000000008 0.00897668182849884
23.0000000000008 0.00897554218769074
23.0100000000008 0.00897453606128693
23.0200000000008 0.00897366881370544
23.0300000000008 0.00897294700145721
23.0400000000008 0.00897234678268433
23.0500000000008 0.00897189199924469
23.0600000000008 0.0089715701341629
23.0700000000008 0.00897137641906738
23.0800000000008 0.00897131860256195
23.0900000000008 0.00897138059139252
23.1000000000008 0.00897158443927765
23.1100000000008 0.00897190153598785
23.1200000000008 0.00897235453128815
23.1300000000008 0.00897291421890259
23.1400000000008 0.00897360563278198
23.1500000000008 0.00897440671920776
23.1600000000008 0.00897532820701599
23.1700000000008 0.00897636294364929
23.1800000000008 0.00897750318050385
23.1900000000008 0.0089787483215332
23.2000000000008 0.00898009896278381
23.2100000000008 0.00898154616355896
23.2200000000008 0.00898309290409088
23.2300000000008 0.00898472368717194
23.2400000000008 0.00898645639419556
23.2500000000008 0.00898826599121094
23.2600000000008 0.00899015247821808
23.2700000000008 0.00899212956428528
23.2800000000008 0.00899417102336884
23.2900000000008 0.00899629890918732
23.3000000000008 0.00899846136569977
23.3100000000008 0.00900071620941162
23.3200000000008 0.00900302052497864
23.3300000000008 0.00900537014007568
23.3400000000008 0.00900778293609619
23.3500000000009 0.00901024281978607
23.3600000000009 0.00901273965835571
23.3700000000009 0.00901527225971222
23.3800000000009 0.00901784241199493
23.3900000000009 0.00902044475078583
23.4000000000009 0.00902306318283081
23.4100000000009 0.00902571976184845
23.4200000000009 0.00902837574481964
23.4300000000009 0.00903105497360229
23.4400000000009 0.00903374373912811
23.4500000000009 0.00903643727302551
23.4600000000009 0.00903913259506226
23.4700000000009 0.00904183208942413
23.4800000000009 0.00904451787471771
23.4900000000009 0.00904719531536102
23.5000000000009 0.00904986143112183
23.5100000000009 0.00905251085758209
23.5200000000009 0.00905514121055603
23.5300000000009 0.00905774891376495
23.5400000000009 0.00906032621860504
23.5500000000009 0.00906287372112274
23.5600000000009 0.00906538486480713
23.5700000000009 0.00906785666942597
23.5800000000009 0.00907029986381531
23.5900000000009 0.00907268345355988
23.6000000000009 0.00907503008842468
23.6100000000009 0.00907733500003815
23.6200000000009 0.00907958388328552
23.6300000000009 0.00908177554607391
23.6400000000009 0.00908391892910004
23.6500000000009 0.00908599078655243
23.6600000000009 0.00908800959587097
23.6700000000009 0.00908996105194092
23.6800000000009 0.00909184873104095
23.6900000000009 0.0090936690568924
23.7000000000009 0.0090954202413559
23.7100000000009 0.00909710466861725
23.7200000000009 0.00909871816635132
23.7300000000009 0.00910025537014008
23.7400000000009 0.00910172343254089
23.7500000000009 0.00910310447216034
23.7600000000009 0.00910442233085632
23.7700000000009 0.00910565197467804
23.7800000000009 0.00910680711269379
23.7900000000009 0.00910787761211395
23.8000000000009 0.00910887598991394
23.8100000000009 0.00910979092121124
23.8200000000009 0.00911061942577362
23.8300000000009 0.0091113817691803
23.8400000000009 0.00911204636096954
23.8500000000009 0.00911263704299927
23.8600000000009 0.00911315202713013
23.8700000000009 0.0091135835647583
23.8800000000009 0.0091139280796051
23.8900000000009 0.00911419749259949
23.9000000000009 0.00911439418792725
23.9100000000009 0.00911450624465942
23.9200000000009 0.0091145384311676
23.9300000000009 0.00911449670791626
23.9400000000009 0.00911437511444092
23.9500000000009 0.00911418676376343
23.9600000000009 0.00911391377449036
23.9700000000009 0.00911357879638672
23.9800000000009 0.00911316692829132
23.990000000001 0.00911268591880798
24.000000000001 0.00911213994026184
24.010000000001 0.00911152541637421
24.020000000001 0.00911084413528442
24.030000000001 0.00911009311676025
24.040000000001 0.00910929024219513
24.050000000001 0.00910842001438141
24.060000000001 0.00910749495029449
24.070000000001 0.00910651504993439
24.080000000001 0.00910548210144043
24.090000000001 0.0091043895483017
24.100000000001 0.00910325825214386
24.110000000001 0.00910207390785217
24.120000000001 0.00910085082054138
24.130000000001 0.00909957408905029
24.140000000001 0.00909826159477234
24.150000000001 0.00909691333770752
24.160000000001 0.00909552276134491
24.170000000001 0.00909409463405609
24.180000000001 0.00909264802932739
24.190000000001 0.00909116685390472
24.200000000001 0.00908965647220612
24.210000000001 0.00908811628818512
24.220000000001 0.00908655822277069
24.230000000001 0.00908499658107758
24.240000000001 0.00908339858055115
24.250000000001 0.00908178806304932
24.260000000001 0.00908016681671143
24.270000000001 0.00907854199409485
24.280000000001 0.00907690644264221
24.290000000001 0.00907525718212128
24.300000000001 0.0090736198425293
24.310000000001 0.00907197415828705
24.320000000001 0.00907033979892731
24.330000000001 0.00906870067119598
24.340000000001 0.00906706511974335
24.350000000001 0.00906543970108032
24.360000000001 0.00906383037567139
24.370000000001 0.00906223237514496
24.380000000001 0.00906064629554749
24.390000000001 0.00905908346176148
24.400000000001 0.00905753374099732
24.410000000001 0.00905600428581238
24.420000000001 0.00905450344085693
24.430000000001 0.00905302584171295
24.440000000001 0.00905157744884491
24.450000000001 0.00905014991760254
24.460000000001 0.00904875695705414
24.470000000001 0.00904739916324616
24.480000000001 0.00904606521129608
24.490000000001 0.00904478311538696
24.500000000001 0.00904352009296417
24.510000000001 0.00904229581356049
24.520000000001 0.00904112160205841
24.530000000001 0.00903997480869293
24.540000000001 0.00903887987136841
24.550000000001 0.00903781950473785
24.560000000001 0.00903681397438049
24.570000000001 0.00903584122657776
24.580000000001 0.00903492271900177
24.590000000001 0.00903404533863068
24.600000000001 0.00903321802616119
24.610000000001 0.00903242468833923
24.620000000001 0.00903168737888336
24.6300000000011 0.00903099536895752
24.6400000000011 0.00903036773204803
24.6500000000011 0.00902977228164673
24.6600000000011 0.00902923285961151
24.6700000000011 0.00902873754501343
24.6800000000011 0.00902829706668854
24.6900000000011 0.00902790784835815
24.7000000000011 0.0090275651216507
24.7100000000011 0.00902727484703064
24.7200000000011 0.00902702629566193
24.7300000000011 0.00902683436870575
24.7400000000011 0.0090267014503479
24.7500000000011 0.00902660489082337
24.7600000000011 0.00902655720710754
24.7700000000011 0.00902656018733978
24.7800000000011 0.00902661383152008
24.7900000000011 0.00902670741081238
24.8000000000011 0.00902684211730957
24.8100000000011 0.00902703166007996
24.8200000000011 0.00902726888656616
24.8300000000011 0.0090275365114212
24.8400000000011 0.00902786433696747
24.8500000000011 0.00902822136878967
24.8600000000011 0.00902862071990967
24.8700000000011 0.00902907073497772
24.8800000000011 0.00902954995632172
24.8900000000011 0.0090300726890564
24.9000000000011 0.00903062343597412
24.9100000000011 0.0090312248468399
24.9200000000011 0.00903184771537781
24.9300000000011 0.00903251171112061
24.9400000000011 0.00903319895267487
24.9500000000011 0.0090339183807373
24.9600000000011 0.00903468072414398
24.9700000000011 0.00903545439243317
24.9800000000011 0.0090362560749054
24.9900000000011 0.0090370899438858
25.0000000000011 0.00903794169425964
25.0100000000011 0.0090388286113739
25.0200000000011 0.00903972446918488
25.0300000000011 0.0090406346321106
25.0400000000011 0.00904156982898712
25.0500000000011 0.00904252409934998
25.0600000000011 0.0090434867143631
25.0700000000011 0.00904446542263031
25.0800000000011 0.00904546022415161
25.0900000000011 0.00904646217823029
25.1000000000011 0.00904747247695923
25.1100000000011 0.00904849052429199
25.1200000000011 0.00904951810836792
25.1300000000011 0.00905054807662964
25.1400000000011 0.00905157327651978
25.1500000000011 0.00905260443687439
25.1600000000011 0.00905364155769348
25.1700000000011 0.0090546727180481
25.1800000000011 0.00905570387840271
25.1900000000011 0.0090567272901535
25.2000000000011 0.00905773460865021
25.2100000000011 0.00905875742435455
25.2200000000011 0.00905975580215454
25.2300000000011 0.00906075239181519
25.2400000000011 0.00906172871589661
25.2500000000011 0.00906270027160645
25.2600000000011 0.00906366646289825
25.2700000000012 0.00906461179256439
25.2800000000012 0.00906553864479065
25.2900000000012 0.00906646013259888
25.3000000000012 0.00906735301017761
25.3100000000012 0.00906822800636291
25.3200000000012 0.00906908333301544
25.3300000000012 0.0090699177980423
25.3400000000012 0.00907073378562927
25.3500000000012 0.00907152712345123
25.3600000000012 0.00907230079174042
25.3700000000012 0.00907303810119629
25.3800000000012 0.00907376170158386
25.3900000000012 0.00907445728778839
25.4000000000012 0.00907512187957764
25.4100000000012 0.00907577216625214
25.4200000000012 0.00907638549804688
25.4300000000012 0.00907697260379791
25.4400000000012 0.00907753586769104
25.4500000000012 0.00907805919647217
25.4600000000012 0.00907856106758118
25.4700000000012 0.00907903969287872
25.4800000000012 0.0090794712305069
25.4900000000012 0.00907989919185638
25.5000000000012 0.00908027470111847
25.5100000000012 0.00908062756061554
25.5200000000012 0.00908094167709351
25.5300000000012 0.00908124506473541
25.5400000000012 0.00908150374889374
25.5500000000012 0.00908173561096191
25.5600000000012 0.00908192038536072
25.5700000000012 0.00908209621906281
25.5800000000012 0.00908222675323486
25.5900000000012 0.00908233880996704
25.6000000000012 0.00908241629600525
25.6100000000012 0.00908246338367462
25.6200000000012 0.00908248424530029
25.6300000000012 0.00908246636390686
25.6400000000012 0.00908242762088776
25.6500000000012 0.00908236026763916
25.6600000000012 0.00908225655555725
25.6700000000012 0.00908213436603546
25.6800000000012 0.0090819787979126
25.6900000000012 0.00908180475234985
25.7000000000012 0.00908159852027893
25.7100000000012 0.00908137559890747
25.7200000000012 0.0090811163187027
25.7300000000012 0.00908083200454712
25.7400000000012 0.009080531001091
25.7500000000012 0.00908020734786987
25.7600000000012 0.00907985150814056
25.7700000000012 0.0090794837474823
25.7800000000012 0.00907909870147705
25.7900000000012 0.0090786874294281
25.8000000000012 0.00907826006412506
25.8100000000012 0.00907780170440674
25.8200000000012 0.00907734572887421
25.8300000000012 0.00907685518264771
25.8400000000012 0.0090763533115387
25.8500000000012 0.00907583653926849
25.8600000000012 0.00907531201839447
25.8700000000012 0.00907477915287018
25.8800000000012 0.00907421946525574
25.8900000000012 0.0090736573934555
25.9000000000012 0.00907307982444763
25.9100000000013 0.00907248318195343
25.9200000000013 0.00907189905643463
25.9300000000013 0.0090712982416153
25.9400000000013 0.00907068073749542
25.9500000000013 0.00907006800174713
25.9600000000013 0.00906945526599884
25.9700000000013 0.00906882166862488
25.9800000000013 0.00906818807125092
25.9900000000013 0.00906756639480591
26.0000000000013 0.00906692862510681
26.0100000000013 0.0090662944316864
26.0200000000013 0.00906566321849823
26.0300000000013 0.00906502306461334
26.0400000000013 0.00906439125537872
26.0500000000013 0.00906376779079437
26.0600000000013 0.00906314492225647
26.0700000000013 0.00906252443790436
26.0800000000013 0.00906191051006317
26.0900000000013 0.00906130015850067
26.1000000000013 0.00906069219112396
26.1100000000013 0.00906010448932648
26.1200000000013 0.00905952274799347
26.1300000000013 0.00905893921852112
26.1400000000013 0.00905837416648865
26.1500000000013 0.00905780851840973
26.1600000000013 0.00905725836753845
26.1700000000013 0.00905672192573547
26.1800000000013 0.00905619561672211
26.1900000000013 0.00905568420886993
26.2000000000013 0.00905518651008606
26.2100000000013 0.00905470788478851
26.2200000000013 0.00905423402786255
26.2300000000013 0.00905377924442291
26.2400000000013 0.00905333459377289
26.2500000000013 0.00905291020870209
26.2600000000013 0.00905249297618866
26.2700000000013 0.00905210137367249
26.2800000000013 0.00905172824859619
26.2900000000013 0.00905136287212372
26.3000000000013 0.00905102372169495
26.3100000000013 0.00905069291591644
26.3200000000013 0.00905038475990295
26.3300000000013 0.00905009388923645
26.3400000000013 0.00904983222484589
26.3500000000013 0.00904957115650177
26.3600000000013 0.00904934108257294
26.3700000000013 0.00904912412166596
26.3800000000013 0.00904892027378082
26.3900000000013 0.00904873609542847
26.4000000000013 0.00904857754707336
26.4100000000013 0.00904843747615814
26.4200000000013 0.00904831349849701
26.4300000000013 0.00904820442199707
26.4400000000013 0.00904812097549438
26.4500000000013 0.00904805421829224
26.4600000000013 0.00904800117015839
26.4700000000013 0.00904795944690704
26.4800000000013 0.00904794275760651
26.4900000000013 0.00904795944690704
26.5000000000013 0.00904797554016113
26.5100000000013 0.00904801070690155
26.5200000000013 0.0090480637550354
26.5300000000013 0.00904814064502716
26.5400000000013 0.00904822528362274
26.5500000000014 0.0090483295917511
26.5600000000014 0.00904845952987671
26.5700000000014 0.00904858350753784
26.5800000000014 0.00904874265193939
26.5900000000014 0.0090489000082016
26.6000000000014 0.00904908239841461
26.6100000000014 0.00904927492141724
26.6200000000014 0.00904948949813843
26.6300000000014 0.00904970586299896
26.6400000000014 0.00904992997646332
26.6500000000014 0.00905017733573914
26.6600000000014 0.00905043423175812
26.6700000000014 0.00905069947242737
26.6800000000014 0.00905097603797913
26.6900000000014 0.0090512627363205
26.7000000000014 0.00905155658721924
26.7100000000014 0.00905185878276825
26.7200000000014 0.00905218005180359
26.7300000000014 0.00905249953269958
26.7400000000014 0.00905283331871033
26.7500000000014 0.00905316114425659
26.7600000000014 0.00905350625514984
26.7700000000014 0.0090538614988327
26.7800000000014 0.00905421316623688
26.7900000000014 0.00905456960201263
26.8000000000014 0.00905494213104248
26.8100000000014 0.00905529737472534
26.8200000000014 0.00905567467212677
26.8300000000014 0.00905604064464569
26.8400000000014 0.00905641496181488
26.8500000000014 0.00905679106712341
26.8600000000014 0.0090571653842926
26.8700000000014 0.00905754446983337
26.8800000000014 0.00905792117118836
26.8900000000014 0.00905829131603241
26.9000000000014 0.00905866622924805
26.9100000000014 0.00905904829502106
26.9200000000014 0.00905940890312195
26.9300000000014 0.00905978322029114
26.9400000000014 0.00906014084815979
26.9500000000014 0.00906050801277161
26.9600000000014 0.00906086087226868
26.9700000000014 0.00906121015548706
26.9800000000014 0.00906156718730926
26.9900000000014 0.00906190574169159
27.0000000000014 0.00906224310398102
27.0100000000014 0.00906257331371307
27.0200000000014 0.0090628981590271
27.0300000000014 0.00906322300434113
27.0400000000014 0.009063521027565
27.0500000000014 0.00906382977962494
27.0600000000014 0.00906412482261658
27.0700000000014 0.00906440913677216
27.0800000000014 0.00906468868255615
27.0900000000014 0.00906496047973633
27.1000000000014 0.00906522572040558
27.1100000000014 0.0090654718875885
27.1200000000014 0.00906571269035339
27.1300000000014 0.00906594574451447
27.1400000000014 0.00906617105007172
27.1500000000014 0.00906638026237488
27.1600000000014 0.00906658470630646
27.1700000000014 0.00906677424907684
27.1800000000014 0.00906695306301117
27.1900000000015 0.00906712591648102
27.2000000000015 0.00906728744506836
27.2100000000015 0.00906743049621582
27.2200000000015 0.00906756520271301
27.2300000000015 0.00906769752502441
27.2400000000015 0.00906781494617462
27.2500000000015 0.00906792104244232
27.2600000000015 0.00906801402568817
27.2700000000015 0.00906809628009796
27.2800000000015 0.00906816184520721
27.2900000000015 0.00906822800636291
27.3000000000015 0.00906827926635742
27.3100000000015 0.00906831920146942
27.3200000000015 0.00906834542751312
27.3300000000015 0.00906835973262787
27.3400000000015 0.00906836807727814
27.3500000000015 0.00906837224960327
27.3600000000015 0.00906835854053497
27.3700000000015 0.00906833052635193
27.3800000000015 0.00906829237937927
27.3900000000015 0.00906824707984924
27.4000000000015 0.00906820178031921
27.4100000000015 0.00906813502311707
27.4200000000015 0.00906806111335754
27.4300000000015 0.00906798779964447
27.4400000000015 0.00906788408756256
27.4500000000015 0.00906779408454895
27.4600000000015 0.00906767666339874
27.4700000000015 0.00906757950782776
27.4800000000015 0.00906744420528412
27.4900000000015 0.00906731724739075
27.5000000000015 0.00906718611717224
27.5100000000015 0.00906703293323517
27.5200000000015 0.00906688451766968
27.5300000000015 0.00906672060489655
27.5400000000015 0.00906655848026276
27.5500000000015 0.00906639158725739
27.5600000000015 0.0090662145614624
27.5700000000015 0.00906603872776032
27.5800000000015 0.00906584918498993
27.5900000000015 0.00906566619873047
27.6000000000015 0.00906545996665955
27.6100000000015 0.00906526863574982
27.6200000000015 0.0090650600194931
27.6300000000015 0.00906486213207245
27.6400000000015 0.00906465351581573
27.6500000000015 0.0090644359588623
27.6600000000015 0.00906423091888428
27.6700000000015 0.00906401693820953
27.6800000000015 0.00906379997730255
27.6900000000015 0.0090635758638382
27.7000000000015 0.00906336188316345
27.7100000000015 0.00906314194202423
27.7200000000015 0.00906291365623474
27.7300000000015 0.00906269669532776
27.7400000000015 0.00906247854232788
27.7500000000015 0.00906225919723511
27.7600000000015 0.00906204462051392
27.7700000000015 0.00906182706356049
27.7800000000015 0.00906160831451416
27.7900000000015 0.00906139135360718
27.8000000000015 0.00906118810176849
27.8100000000015 0.00906097412109375
27.8200000000015 0.00906076490879059
27.8300000000016 0.00906055927276611
27.8400000000016 0.00906035959720612
27.8500000000016 0.00906015753746033
27.8600000000016 0.00905996978282929
27.8700000000016 0.00905977129936218
27.8800000000016 0.0090595817565918
27.8900000000016 0.00905940175056458
27.9000000000016 0.00905921459197998
27.9100000000016 0.00905904531478882
27.9200000000016 0.00905887305736542
27.9300000000016 0.00905870795249939
27.9400000000016 0.00905854761600494
27.9500000000016 0.00905839443206787
27.9600000000016 0.00905825197696686
27.9700000000016 0.00905809938907623
27.9800000000016 0.00905796110630035
27.9900000000016 0.0090578305721283
28.0000000000016 0.00905770719051361
28.0100000000016 0.00905758798122406
28.0200000000016 0.0090574723482132
28.0300000000016 0.0090573662519455
28.0400000000016 0.0090572601556778
28.0500000000016 0.00905716419219971
28.0600000000016 0.00905707597732544
28.0700000000016 0.00905698835849762
28.0800000000016 0.00905692279338837
28.0900000000016 0.00905684769153595
28.1000000000016 0.00905677855014801
28.1100000000016 0.00905672550201416
28.1200000000016 0.00905667841434479
28.1300000000016 0.00905663549900055
28.1400000000016 0.0090565949678421
28.1500000000016 0.00905656933784485
28.1600000000016 0.00905653595924377
28.1700000000016 0.00905652284622192
28.1800000000016 0.0090565150976181
28.1900000000016 0.00905651152133942
28.2000000000016 0.00905651152133942
28.2100000000016 0.00905651926994324
28.2200000000016 0.00905653059482574
28.2300000000016 0.0090565437078476
28.2400000000016 0.00905658066272736
28.2500000000016 0.00905660808086395
28.2600000000016 0.00905663669109344
28.2700000000016 0.00905668377876282
28.2800000000016 0.00905673146247864
28.2900000000016 0.00905678629875183
28.3000000000016 0.00905683755874634
28.3100000000016 0.00905690550804138
28.3200000000016 0.00905697226524353
28.3300000000016 0.00905704021453857
28.3400000000016 0.00905711889266968
28.3500000000016 0.00905719578266144
28.3600000000016 0.00905728876590729
28.3700000000016 0.00905737400054932
28.3800000000016 0.00905746877193451
28.3900000000016 0.0090575635433197
28.4000000000016 0.00905766367912293
28.4100000000016 0.0090577757358551
28.4200000000016 0.0090578705072403
28.4300000000016 0.00905798614025116
28.4400000000016 0.0090580940246582
28.4500000000016 0.00905821800231934
28.4600000000016 0.0090583348274231
28.4700000000017 0.00905844807624817
28.4800000000017 0.00905857563018799
28.4900000000017 0.00905869841575623
28.5000000000017 0.00905882716178894
28.5100000000017 0.0090589439868927
28.5200000000017 0.009059077501297
28.5300000000017 0.0090592098236084
28.5400000000017 0.00905933260917664
28.5500000000017 0.0090594619512558
28.5600000000017 0.00905959188938141
28.5700000000017 0.00905972361564636
28.5800000000017 0.00905985057353973
28.5900000000017 0.00905998349189758
28.6000000000017 0.00906011581420898
28.6100000000017 0.0090602445602417
28.6200000000017 0.00906037330627441
28.6300000000017 0.00906050324440002
28.6400000000017 0.00906063199043274
28.6500000000017 0.00906075537204742
28.6600000000017 0.00906088590621948
28.6700000000017 0.00906101286411285
28.6800000000017 0.00906112670898438
28.6900000000017 0.00906124711036682
28.7000000000017 0.00906136870384216
28.7100000000017 0.00906148612499237
28.7200000000017 0.009061598777771
28.7300000000017 0.00906170666217804
28.7400000000017 0.00906181931495667
28.7500000000017 0.00906192719936371
28.7600000000017 0.00906203091144562
28.7700000000017 0.00906212627887726
28.7800000000017 0.00906222760677338
28.7900000000017 0.00906231760978699
28.8000000000017 0.00906241834163666
28.8100000000017 0.00906250417232513
28.8200000000017 0.00906258404254913
28.8300000000017 0.00906267881393433
28.8400000000017 0.00906274974346161
28.8500000000017 0.00906282007694244
28.8600000000017 0.00906289458274841
28.8700000000017 0.00906296670436859
28.8800000000017 0.00906302034854889
28.8900000000017 0.00906307816505432
28.9000000000017 0.00906314373016357
28.9100000000017 0.00906319856643677
28.9200000000017 0.00906324207782745
28.9300000000017 0.00906328976154327
28.9400000000017 0.00906332731246948
28.9500000000017 0.00906337022781372
28.9600000000017 0.00906340301036835
28.9700000000017 0.00906343162059784
28.9800000000017 0.00906345069408417
28.9900000000017 0.00906348645687103
29.0000000000017 0.00906349062919617
29.0100000000017 0.00906350553035736
29.0200000000017 0.00906352519989014
29.0300000000017 0.00906352818012237
29.0400000000017 0.00906352818012237
29.0500000000017 0.009063521027565
29.0600000000017 0.00906352460384369
29.0700000000017 0.00906351327896118
29.0800000000017 0.00906350195407868
29.0900000000017 0.00906348943710327
29.1000000000017 0.0090634697675705
29.1100000000018 0.009063441157341
29.1200000000018 0.00906341552734375
29.1300000000018 0.00906338632106781
29.1400000000018 0.00906335592269897
29.1500000000018 0.00906332552433014
29.1600000000018 0.00906328201293945
29.1700000000018 0.00906323671340942
29.1800000000018 0.00906319856643677
29.1900000000018 0.0090631502866745
29.2000000000018 0.00906310379505157
29.2100000000018 0.00906305730342865
29.2200000000018 0.00906300246715546
29.2300000000018 0.00906293511390686
29.2400000000018 0.00906287729740143
29.2500000000018 0.00906281471252442
29.2600000000018 0.00906275928020477
29.2700000000018 0.00906269073486328
29.2800000000018 0.00906262814998627
29.2900000000018 0.0090625536441803
29.3000000000018 0.00906248807907105
29.3100000000018 0.00906241595745087
29.3200000000018 0.00906234204769134
29.3300000000018 0.00906227052211762
29.3400000000018 0.00906219720840454
29.3500000000018 0.00906211674213409
29.3600000000018 0.00906204402446747
29.3700000000018 0.00906196236610413
29.3800000000018 0.00906189143657684
29.3900000000018 0.00906180620193481
29.4000000000018 0.00906173229217529
29.4100000000018 0.00906165182590485
29.4200000000018 0.00906157195568085
29.4300000000018 0.00906149446964264
29.4400000000018 0.00906140744686127
29.4500000000018 0.00906133234500885
29.4600000000018 0.00906125545501709
29.4700000000018 0.00906117677688599
29.4800000000018 0.00906110048294067
29.4900000000018 0.00906102001667023
29.5000000000018 0.00906094789505005
29.5100000000018 0.00906087219715118
29.5200000000018 0.00906079888343811
29.5300000000018 0.00906072616577148
29.5400000000018 0.00906064569950104
29.5500000000018 0.00906057238578796
29.5600000000018 0.00906050622463226
29.5700000000018 0.00906043887138367
29.5800000000018 0.0090603643655777
29.5900000000018 0.00906030714511871
29.6000000000018 0.00906023144721985
29.6100000000018 0.00906017482280731
29.6200000000018 0.00906011521816254
29.6300000000018 0.00906004369258881
29.6400000000018 0.00905999779701233
29.6500000000018 0.00905993223190308
29.6600000000018 0.00905988395214081
29.6700000000018 0.00905983150005341
29.6800000000018 0.00905978620052338
29.6900000000018 0.00905973792076111
29.7000000000018 0.00905969142913818
29.7100000000018 0.00905964612960816
29.7200000000018 0.00905960440635681
29.7300000000018 0.00905957043170929
29.7400000000018 0.00905953586101532
29.7500000000019 0.00905949532985687
29.7600000000019 0.00905946671962738
29.7700000000019 0.00905943393707275
29.7800000000019 0.00905940532684326
29.7900000000019 0.00905937492847443
29.8000000000019 0.00905935049057007
29.8100000000019 0.00905933678150177
29.8200000000019 0.00905932068824768
29.8300000000019 0.00905929803848267
29.8400000000019 0.00905929148197174
29.8500000000019 0.00905927360057831
29.8600000000019 0.00905926525592804
29.8700000000019 0.0090592622756958
29.8800000000019 0.0090592610836029
29.8900000000019 0.00905925869941712
29.9000000000019 0.0090592622756958
29.9100000000019 0.00905925869941712
29.9200000000019 0.0090592634677887
29.9300000000019 0.00905927360057831
29.9400000000019 0.00905928194522858
29.9500000000019 0.00905929148197174
29.9600000000019 0.00905929923057556
29.9700000000019 0.00905932009220123
29.9800000000019 0.00905933499336243
29.9900000000019 0.00905935406684875
30.0000000000019 0.00905937612056732
};
\addlegendentry{DDPG};
\end{axis}

\end{tikzpicture}

		\end{figure}
	\end{minipage}
	\begin{minipage}{0.5\textwidth}
		\begin{figure}\scriptsize
			\hspace{3cm}
			% This file was created by tikzplotlib v0.9.1.
\begin{tikzpicture}[trim axis left]

\definecolor{color0}{rgb}{0.12156862745098,0.466666666666667,0.705882352941177}
\definecolor{color1}{rgb}{1,0.498039215686275,0.0549019607843137}

\pgfplotsset{scaled y ticks=false}


\begin{axis}[
tick align=outside,
tick pos=left,
x grid style={white!69.0196078431373!black},
xmin=-1.50000000000009, xmax=31.500000000002,
xtick style={color=black},
y grid style={white!69.0196078431373!black},
ymin=-0.0224612300423489, ymax=0.00730709319820797,
ytick style={color=black},
yticklabel style={
        /pgf/number format/fixed,
        /pgf/number format/precision=5
},
width=25cm,
height=10cm,
xlabel=time (seconds),
ylabel=Frequency (Hz)
]
\addplot [line width=3pt, green!20!gray]
table {%
0 0
0.01 0
0.02 0
0.03 0
0.04 0
0.05 0
0.06 0
0.07 0
0.08 0
0.09 0
0.1 0
0.11 0
0.12 0
0.13 0
0.14 0
0.15 0
0.16 0
0.17 0
0.18 0
0.19 0
0.2 0
0.21 0
0.22 0
0.23 0
0.24 0
0.25 0
0.26 0
0.27 0
0.28 0
0.29 0
0.3 0
0.31 0
0.32 0
0.33 0
0.34 0
0.35 0
0.36 0
0.37 0
0.38 0
0.39 0
0.4 0
0.41 0
0.42 0
0.43 0
0.44 0
0.45 0
0.46 0
0.47 0
0.48 0
0.49 0
0.5 0
0.51 0
0.52 0
0.53 0
0.54 0
0.55 0
0.56 0
0.57 0
0.58 0
0.59 0
0.6 0
0.61 0
0.62 0
0.63 0
0.64 0
0.65 0
0.66 0
0.67 0
0.68 0
0.69 0
0.7 0
0.71 0
0.72 0
0.73 0
0.74 0
0.75 0
0.76 0
0.77 0
0.78 0
0.79 0
0.8 0
0.81 0
0.820000000000001 0
0.830000000000001 0
0.840000000000001 0
0.850000000000001 0
0.860000000000001 0
0.870000000000001 0
0.880000000000001 0
0.890000000000001 0
0.900000000000001 0
0.910000000000001 0
0.920000000000001 0
0.930000000000001 0
0.940000000000001 0
0.950000000000001 0
0.960000000000001 0
0.970000000000001 0
0.980000000000001 0
0.990000000000001 0
1 -6.20824004696096e-10
1.01 -0.000599804935089449
1.02 -0.00119907507020158
1.03 -0.00179752917333065
1.04 -0.00239483168752664
1.05 -0.00299060965330246
1.06 -0.0035844546477612
1.07 -0.00417592916044445
1.08 -0.00476457203700884
1.09 -0.00534990321841046
1.1 -0.00593142791365752
1.11 -0.0065086403093356
1.12 -0.00708102687294505
1.13 -0.00764806929888523
1.14 -0.00820924713944498
1.15 -0.00876404015770972
1.16 -0.00931193043460291
1.17 -0.00985240425820529
1.18 -0.010384953819942
1.19 -0.0109090787391257
1.2 -0.0114242874346331
1.21 -0.0119300983601258
1.22 -0.0124260411171524
1.23 -0.0129116574586643
1.24 -0.0133865021938943
1.25 -0.0138501440041666
1.26 -0.0143021661780033
1.27 -0.014742167272839
1.28 -0.0151697617097358
1.29 -0.0155845803066927
1.3 -0.0159862707554451
1.31 -0.0163744980460401
1.32 -0.0167489448429481
1.33 -0.0171093118160071
1.34 -0.0174553179291005
1.35 -0.0177867006891191
1.36 -0.0181032163574599
1.37 -0.0184046401260522
1.38 -0.0186907662596788
1.39 -0.0189614082061646
1.4 -0.0192163986758395
1.41 -0.0194555896915384
1.42 -0.0196788526102796
1.43 -0.0198860781176575
1.44 -0.0200771761963022
1.45 -0.0202520760677041
1.46 -0.0204107261104341
1.47 -0.0205530937542481
1.48 -0.0206791653509738
1.49 -0.0207889460232166
1.5 -0.0208824594900378
1.51 -0.0209597478726123
1.52 -0.0210208714788517
1.53 -0.0210659085679938
1.54 -0.0210949550957385
1.55 -0.0211081244405054
1.56 -0.0211055471113838
1.57 -0.0210873704383387
1.58 -0.021053758245239
1.59 -0.0210048905054337
1.6 -0.0209409629785557
1.61 -0.0208621868369109
1.62 -0.0207687882752429
1.63 -0.0206610081056492
1.64 -0.0205391013384232
1.65 -0.0204033367492368
1.66 -0.0202539963849012
1.67 -0.0200913752094865
1.68 -0.0199157806176659
1.69 -0.0197275319393628
1.7 -0.0195269599529585
1.71 -0.0193144063891981
1.72 -0.0190902234264388
1.73 -0.0188547731778899
1.74 -0.0186084271714953
1.75 -0.0183515658231156
1.76 -0.0180845779036665
1.77 -0.0178078600008771
1.78 -0.0175218159769454
1.79 -0.0172268564203725
1.8 -0.0169233980959289
1.81 -0.0166118633916627
1.82 -0.0162926797637
1.83 -0.0159662791796423
1.84 -0.0156330975612198
1.85 -0.0152935742268575
1.86 -0.0149481513348054
1.87 -0.0145972733274811
1.88 -0.0142413863776684
1.89 -0.013880937837207
1.9 -0.0135163756888059
1.91 -0.0131481480016023
1.92 -0.0127767023910808
1.93 -0.0124024854839615
1.94 -0.0120259423886515
1.95 -0.0116475161718502
1.96 -0.0112676473418824
1.97 -0.0108867733393261
1.98 -0.0105053280354897
1.99 -0.0101237412392771
2 -0.00974243821297125
2.01 -0.00936183919745007
2.02 -0.00898235894733461
2.03 -0.0086044062765558
2.04 -0.0082283836148098
2.05 -0.00785468657535736
2.06 -0.00748370353446157
2.07 -0.00711581519578193
2.08 -0.00675139425808641
2.09 -0.00639080499805628
2.1 -0.00603440290085287
2.11 -0.00568253430445305
2.12 -0.00533553605807665
2.13 -0.0049937351950123
2.14 -0.00465744862013147
2.15 -0.00432698281236135
2.16 -0.00400263354236993
2.17 -0.00368468560569651
2.18 -0.00337341257154243
2.19 -0.00306907654741639
2.2 -0.00277192795980941
2.21 -0.0024822053510542
2.22 -0.00220013519250341
2.23 -0.00192593171414118
2.24 -0.00165979675072184
2.25 -0.00140191960450959
2.26 -0.00115247692467226
2.27 -0.000911632603362458
2.28 -0.000679537688498825
2.29 -0.000456330313240723
2.29999999999999 -0.000242135642129242
2.30999999999999 -3.70658338486513e-05
2.31999999999999 0.000158779979457218
2.32999999999999 0.000345315696388275
2.33999999999999 0.000522468234150686
2.34999999999999 0.000690177500144704
2.35999999999999 0.00084839634744394
2.36999999999999 0.000997090515621051
2.37999999999999 0.00113623855329664
2.38999999999999 0.0012658317247547
2.39999999999999 0.00138587390394453
2.40999999999999 0.00149638145190855
2.41999999999999 0.00159738307915531
2.42999999999999 0.00168891969315537
2.43999999999999 0.00177104423111326
2.44999999999999 0.00184382147812196
2.45999999999999 0.00190732792183416
2.46999999999999 0.00196165178622707
2.47999999999999 0.00200689229538038
2.48999999999999 0.0020431596715634
2.49999999999999 0.0020705748956772
2.50999999999999 0.00208926945455804
2.51999999999999 0.00209938507597944
2.52999999999999 0.00210107345204816
2.53999999999999 0.002094495951605
2.54999999999999 0.00207982332219134
2.55999999999999 0.00205723538211287
2.56999999999999 0.00202692070311458
2.57999999999999 0.00198907628417225
2.58999999999999 0.00194390721690139
2.59999999999999 0.00189162634308319
2.60999999999999 0.00183245390480826
2.61999999999999 0.00176661718774031
2.62999999999999 0.00169435015800467
2.63999999999999 0.00161589309320912
2.64999999999999 0.00153149220810705
2.65999999999999 0.00144139927541541
2.66999999999999 0.00134587124230211
2.67999999999999 0.00124516984305894
2.68999999999999 0.00113956120847756
2.69999999999999 0.00102931547244663
2.70999999999999 0.000914706376288181
2.71999999999999 0.000796010871351132
2.72999999999999 0.000673508720378468
2.73999999999999 0.00054748209816333
2.74999999999999 0.000418215192006679
2.75999999999999 0.000285993802486533
2.76999999999998 0.000151104945045282
2.77999999999998 1.38364528974718e-05
2.78999999999998 -0.000125523418244083
2.79999999999998 -0.000266686383131384
2.80999999999998 -0.000409364517146857
2.81999999999998 -0.000553270643191168
2.82999999999998 -0.000698118714819796
2.83999999999998 -0.000843624195162716
2.84999999999998 -0.000989504431169413
2.85999999999998 -0.00113547902272993
2.86999999999998 -0.00128127018623205
2.87999999999998 -0.00142660311212407
2.88999999999998 -0.00157120631606294
2.89999999999998 -0.00171481198323777
2.90999999999998 -0.00185715630546675
2.91999999999998 -0.0019979798106756
2.92999999999998 -0.00213702768442753
2.93999999999998 -0.002274050083068
2.94999999999998 -0.00240880243818
2.95999999999998 -0.00254104575200257
2.96999999999998 -0.00267054688348841
2.97999999999998 -0.00279707882469062
2.98999999999998 -0.00292042096718226
2.99999999999998 -0.00304035935822703
3.00999999999998 -0.00315668694643378
3.01999999999998 -0.00326920381664274
3.02999999999998 -0.00337771741380613
3.03999999999998 -0.00348204275564128
3.04999999999998 -0.00358200263385007
3.05999999999998 -0.0036774278037138
3.06999999999998 -0.00376815716188857
3.07999999999998 -0.00385403791224242
3.08999999999998 -0.00393492571959095
3.09999999999998 -0.00401068485116643
3.10999999999998 -0.00408118830584149
3.11999999999998 -0.00414631793090938
3.12999999999998 -0.0042059645262336
3.13999999999998 -0.00426002793589267
3.14999999999998 -0.00430841712720164
3.15999999999998 -0.00435105025707846
3.16999999999998 -0.00438785472595898
3.17999999999998 -0.00441876721961255
3.18999999999998 -0.00444373373514415
3.19999999999998 -0.00446270959802892
3.20999999999998 -0.00447565946523311
3.21999999999998 -0.00448255731577869
3.22999999999998 -0.00448338642880349
3.23999999999997 -0.00447813934917143
3.24999999999997 -0.00446681784068277
3.25999999999997 -0.00444943282692012
3.26999999999997 -0.00442600431973534
3.27999999999997 -0.00439656133532557
3.28999999999997 -0.00436114179774506
3.29999999999997 -0.00431979242952093
3.30999999999997 -0.00427256998542847
3.31999999999997 -0.00421953778052145
3.32999999999997 -0.00416076834767293
3.33999999999997 -0.00409634251186089
3.34999999999997 -0.00402634923381546
3.35999999999997 -0.00395088544581628
3.36999999999997 -0.00387005588188806
3.37999999999997 -0.00378397290724588
3.38999999999997 -0.00369275637567997
3.39999999999997 -0.00359653268860509
3.40999999999997 -0.00349543557059216
3.41999999999997 -0.00338960562032342
3.42999999999997 -0.00327918997186837
3.43999999999997 -0.00316434205731013
3.44999999999997 -0.00304522136228964
3.45999999999997 -0.00292199317481146
3.46999999999997 -0.00279482832766527
3.47999999999997 -0.00266390281117687
3.48999999999997 -0.00252939748714201
3.49999999999997 -0.00239149828907784
3.50999999999997 -0.0022503955815361
3.51999999999997 -0.00210628389052851
3.52999999999997 -0.00195936162176884
3.53999999999997 -0.00180983077191651
3.54999999999997 -0.0016578966350299
3.55999999999997 -0.00150376750539006
3.56999999999997 -0.00134765437744442
3.57999999999997 -0.00118977064344378
3.58999999999997 -0.00103032891787928
3.59999999999997 -0.000869549872670158
3.60999999999997 -0.000707652187363879
3.61999999999997 -0.000544855728184272
3.62999999999997 -0.000381381238593514
3.63999999999997 -0.000217450030708302
3.64999999999997 -5.32836778285742e-05
3.65999999999997 0.000110896291623577
3.66999999999997 0.000274868698432642
3.67999999999997 0.000438413014848592
3.68999999999997 0.000601309664426964
3.69999999999997 0.000763340319099222
3.70999999999996 0.000924288193135567
3.71999999999996 0.00108393833362723
3.72999999999996 0.0012420779071484
3.73999999999996 0.00139849648225157
3.74999999999996 0.00155298630745587
3.75999999999996 0.00170534258439347
3.76999999999996 0.00185536373578626
3.77999999999996 0.00200285166793178
3.78999999999996 0.00214761202738551
3.79999999999996 0.0022894544515349
3.80999999999996 0.00242819281276937
3.81999999999996 0.00256364545595971
3.82999999999996 0.00269563542904389
3.83999999999996 0.00282399070619741
3.84999999999996 0.00294854440359831
3.85999999999996 0.00306913498753567
3.86999999999996 0.00318560647448808
3.87999999999996 0.00329780862300728
3.88999999999996 0.00340559711719455
3.89999999999996 0.00350883374156981
3.90999999999996 0.00360738654714454
3.91999999999996 0.0037011300085222
3.92999999999996 0.00378994517186203
3.93999999999996 0.00387371979355395
3.94999999999996 0.00395234846946565
3.95999999999996 0.00402573275463446
3.96999999999996 0.0040937812732901
3.97999999999996 0.00415640981910618
3.98999999999996 0.00421354144559161
3.99999999999996 0.00426510654654482
4.00999999999996 0.00431104292650664
4.01999999999996 0.00435129586115921
4.02999999999996 0.00438581814763041
4.03999999999996 0.00441457014467423
4.04999999999996 0.00443751980270837
4.05999999999996 0.00445464268377633
4.06999999999996 0.00446592197322277
4.07999999999996 0.00447134847498853
4.08999999999996 0.00447092060329592
4.09999999999996 0.00446464464673244
4.10999999999996 0.00445253418648235
4.11999999999996 0.00443461034586673
4.12999999999996 0.00441090172501595
4.13999999999996 0.00438144434208721
4.14999999999996 0.00434628156505436
4.15999999999996 0.00430546403421719
4.16999999999996 0.00425904957560252
4.17999999999996 0.00420710310546587
4.18999999999996 0.00414969652616012
4.19999999999995 0.00408690861374305
4.20999999999995 0.00401882489790146
4.21999999999995 0.00394553753698013
4.22999999999995 0.00386714518338983
4.23999999999995 0.00378375285389456
4.24999999999995 0.00369547181217314
4.25999999999995 0.00360241856259226
4.26999999999995 0.00350471595296653
4.27999999999995 0.00340249269161272
4.28999999999995 0.00329588295552423
4.29999999999995 0.00318502621239112
4.30999999999995 0.00307006703098643
4.31999999999995 0.00295115488310913
4.32999999999995 0.00282844393862768
4.33999999999995 0.00270209285450631
4.34999999999995 0.00257226455840815
4.35999999999995 0.00243912602733374
4.36999999999995 0.00230284806168909
4.37999999999995 0.00216360505514386
4.38999999999995 0.0020215747606239
4.39999999999995 0.00187693805277433
4.40999999999995 0.00172987868722497
4.41999999999995 0.00158058305698879
4.42999999999995 0.00142923994632348
4.43999999999995 0.00127604028238626
4.44999999999995 0.00112117688501278
4.45999999999995 0.000964844214950388
4.46999999999995 0.000807235461569877
4.47999999999995 0.000648550615547445
4.48999999999995 0.000488987496423043
4.49999999999995 0.000328744554208681
4.50999999999995 0.000168020614053058
4.51999999999995 7.01462139128355e-06
4.52999999999995 -0.000154074611784431
4.53999999999995 -0.000315048659654247
4.54999999999995 -0.000475709733840413
4.55999999999995 -0.000635860932909991
4.56999999999995 -0.000795306489805999
4.57999999999995 -0.000953852016904347
4.58999999999995 -0.00111130474839825
4.59999999999995 -0.00126747377971501
4.60999999999995 -0.00142217030367456
4.61999999999995 -0.00157520784310449
4.62999999999995 -0.00172640247963031
4.63999999999995 -0.00187557307836667
4.64999999999995 -0.00202254150824036
4.65999999999995 -0.00216713285768297
4.66999999999994 -0.00230917564543689
4.67999999999994 -0.00244850202622659
4.68999999999994 -0.00258494799105343
4.69999999999994 -0.00271835356186927
4.70999999999994 -0.00284856298042708
4.71999999999994 -0.00297542489110627
4.72999999999994 -0.00309879251742872
4.73999999999994 -0.00321852383213038
4.74999999999994 -0.00333448172057924
4.75999999999994 -0.00344653413580643
4.76999999999994 -0.00355455423942545
4.77999999999994 -0.00365842056840961
4.78999999999994 -0.00375801715584102
4.79999999999994 -0.003853233658577
4.80999999999994 -0.00394396547730996
4.81999999999994 -0.00403011386889854
4.82999999999994 -0.00411158605085688
4.83999999999994 -0.00418829529789924
4.84999999999994 -0.0042601610304998
4.85999999999994 -0.00432710889525932
4.86999999999994 -0.00438907083693039
4.87999999999994 -0.00444598516249748
4.88999999999994 -0.0044977959748439
4.89999999999994 -0.0045444553353715
4.90999999999994 -0.00458592093713267
4.91999999999994 -0.00462215701171856
4.92999999999994 -0.00465313435135912
4.93999999999994 -0.00467883032269187
4.94999999999994 -0.00469922887221158
4.95999999999994 -0.00471432052342417
4.96999999999994 -0.00472410236800312
4.97999999999994 -0.00472857804180838
4.98999999999994 -0.00472775769965577
4.99999999999994 -0.00472165797986968
5.00999999999994 -0.00471030196070779
5.01999999999994 -0.00469371910887444
5.02999999999994 -0.00467194522022368
5.03999999999994 -0.00464502235277793
5.04999999999994 -0.00461299875220802
5.05999999999994 -0.00457592876994707
5.06999999999994 -0.00453387277415419
5.07999999999994 -0.00448689705381486
5.08999999999994 -0.00443507371736574
5.09999999999994 -0.00437848058417031
5.10999999999994 -0.00431720106968731
5.11999999999994 -0.00425132407180661
5.12999999999994 -0.00418094385854662
5.13999999999993 -0.00410615997131388
5.14999999999993 -0.00402707718060641
5.15999999999993 -0.003943804157296
5.16999999999993 -0.00385645460940342
5.17999999999993 -0.00376514718744777
5.18999999999993 -0.00367000475357741
5.19999999999993 -0.00357115429074456
5.20999999999993 -0.00346872743401102
5.21999999999993 -0.00336285961822173
5.22999999999993 -0.00325368994875038
5.23999999999993 -0.00314136103808894
5.24999999999993 -0.00302601882666007
5.25999999999993 -0.00290781239478033
5.26999999999993 -0.00278689376880001
5.27999999999993 -0.00266341772293999
5.28999999999993 -0.00253754157770604
5.29999999999993 -0.00240942499546831
5.30999999999993 -0.00227922977364922
5.31999999999993 -0.00214711963589068
5.32999999999993 -0.00201326002153099
5.33999999999993 -0.00187781787369954
5.34999999999993 -0.00174096142632524
5.35999999999993 -0.00160285999034549
5.36999999999993 -0.0014636837393984
5.37999999999993 -0.00132360349527647
5.38999999999993 -0.00118279051341722
5.39999999999993 -0.00104141626870399
5.40999999999993 -0.000899652241847281
5.41999999999993 -0.000757669706614728
5.42999999999993 -0.000615639518175104
5.43999999999993 -0.000473731902818848
5.44999999999993 -0.000332116249314907
5.45999999999993 -0.000190960902159837
5.46999999999993 -5.04329569720495e-05
5.47999999999993 8.93019417202205e-05
5.48999999999993 0.000228079800041996
5.49999999999993 0.000365738471476317
5.50999999999993 0.000502117850449643
5.51999999999993 0.00063706006256323
5.52999999999993 0.000770409651257955
5.53999999999993 0.000902013760688953
5.54999999999993 0.00103172231459537
5.55999999999993 0.00115938819095742
5.56999999999993 0.00128486739223842
5.57999999999993 0.00140801921101611
5.58999999999993 0.00152870639081464
5.59999999999993 0.00164679528195464
5.60999999999992 0.00176215599224679
5.61999999999992 0.00187466253236073
5.62999999999992 0.00198419295570918
5.63999999999992 0.00209062949269439
5.64999999999992 0.00219385867917211
5.65999999999992 0.00229377147899574
5.66999999999992 0.00239026340051196
5.67999999999992 0.00248323460688686
5.68999999999992 0.00257259002015007
5.69999999999992 0.00265823941885293
5.70999999999992 0.00274009752924485
5.71999999999992 0.00281808410988128
5.72999999999992 0.00289212402958467
5.73999999999992 0.0029621473386891
5.74999999999992 0.00302808933350765
5.75999999999992 0.00308989061397087
5.76999999999992 0.00314749713439325
5.77999999999992 0.0032008602473337
5.78999999999992 0.00324993674052499
5.79999999999992 0.00329468886685635
5.80999999999992 0.00333508436740206
5.81999999999992 0.00337109648937566
5.82999999999992 0.00340270399230959
5.83999999999992 0.00342989115028734
5.84999999999992 0.00345264774830502
5.85999999999992 0.00347096907076114
5.86999999999992 0.00348485588315292
5.87999999999992 0.00349431440704464
5.88999999999992 0.00349935628838909
5.89999999999992 0.00349999855930359
5.90999999999992 0.00349626359343003
5.91999999999992 0.0034881790550528
5.92999999999992 0.00347577784222114
5.93999999999992 0.00345909802425477
5.94999999999992 0.00343818277426739
5.95999999999992 0.00341308029787488
5.96999999999992 0.00338384376045218
5.97999999999992 0.00335053121826293
5.98999999999992 0.00331320556689449
5.99999999999992 0.00327193453511134
6.00999999999992 0.00322679047194985
6.01999999999992 0.00317784838969669
6.02999999999992 0.0031251889301446
6.03999999999992 0.00306889694808106
6.04999999999992 0.00300906138554603
6.05999999999992 0.00294577514168215
6.06999999999992 0.00287913493811646
6.07999999999991 0.00280924092472588
6.08999999999991 0.00273619668547566
6.09999999999991 0.00266010981217613
6.10999999999991 0.00258109103627024
6.11999999999991 0.00249925412753527
6.12999999999991 0.00241471576218468
6.13999999999991 0.00232759537632771
6.14999999999991 0.00223801501117397
6.15999999999991 0.00214609915289263
6.16999999999991 0.00205197456862247
6.17999999999991 0.00195577013950163
6.18999999999991 0.00185761669128764
6.19999999999991 0.00175764681976139
6.20999999999991 0.00165599472178684
6.21999999999991 0.00155279602337261
6.22999999999991 0.00144818760072067
6.23999999999991 0.00134230740398042
6.24999999999991 0.00123529428039433
6.25999999999991 0.00112728779707863
6.26999999999991 0.00101842806367563
6.27999999999991 0.000908855555109472
6.28999999999991 0.00079871093467343
6.29999999999991 0.000688134877673387
6.30999999999991 0.000577267895849757
6.31999999999991 0.000466250162796697
6.32999999999991 0.000355221340595178
6.33999999999991 0.000244320407873236
6.34999999999991 0.000133685489503834
6.35999999999991 2.34536881477034e-05
6.36999999999991 -8.62390821554827e-05
6.37999999999991 -0.000195258260139037
6.38999999999991 -0.000303470799033551
6.39999999999991 -0.000410745325644314
6.40999999999991 -0.000516952296601492
6.41999999999991 -0.000621964151602294
6.42999999999991 -0.000725660593380797
6.43999999999991 -0.000827913571499038
6.44999999999991 -0.000928602358815346
6.45999999999991 -0.00102760878570075
6.46999999999991 -0.00112481737599721
6.47999999999991 -0.00122011547893183
6.48999999999991 -0.00131339339684441
6.49999999999991 -0.00140454450859067
6.50999999999991 -0.0014934653884889
6.51999999999991 -0.0015800559206836
6.52999999999991 -0.00166421940880534
6.53999999999991 -0.00174586268081171
6.5499999999999 -0.00182489618890096
6.5599999999999 -0.00190123410439557
6.5699999999999 -0.00197478730522858
6.5799999999999 -0.00204548490615167
6.5899999999999 -0.00211325279471569
6.5999999999999 -0.00217802090857016
6.6099999999999 -0.00223972330370486
6.6199999999999 -0.00229829821722321
6.6299999999999 -0.00235368812459102
6.6399999999999 -0.00240583979131233
6.6499999999999 -0.00245470431899126
6.6599999999999 -0.00250023718574636
6.6699999999999 -0.00254239828095193
6.6799999999999 -0.00258115193428784
6.6899999999999 -0.00261646693908733
6.6999999999999 -0.00264831656997999
6.7099999999999 -0.0026766785964438
6.7199999999999 -0.00270153528659064
6.7299999999999 -0.00272287340762525
6.7399999999999 -0.00274068422175761
6.7499999999999 -0.00275496347556291
6.7599999999999 -0.00276571138379215
6.7699999999999 -0.00277293260769355
6.7799999999999 -0.00277663622791939
6.7899999999999 -0.00277683571211181
6.7999999999999 -0.00277354887728733
6.8099999999999 -0.00276679784718096
6.8199999999999 -0.00275660900477616
6.8299999999999 -0.00274301294036844
6.8399999999999 -0.00272604439572029
6.8499999999999 -0.00270574220530371
6.8599999999999 -0.00268214923655104
6.8699999999999 -0.00265531233316555
6.8799999999999 -0.00262528226707587
6.8899999999999 -0.00259211362150006
6.8999999999999 -0.00255586474066894
6.9099999999999 -0.0025165967870219
6.9199999999999 -0.00247437445991288
6.9299999999999 -0.00242926630670533
6.9399999999999 -0.00238134418285538
6.9499999999999 -0.00233068314459918
6.9599999999999 -0.0022773613382474
6.9699999999999 -0.00222145940927249
6.9799999999999 -0.00216306121370674
6.9899999999999 -0.00210225362588827
6.9999999999999 -0.00203912613635554
7.00999999999989 -0.00197377067024289
7.01999999999989 -0.00190628148247331
7.02999999999989 -0.00183675504006929
7.03999999999989 -0.00176528989673935
7.04999999999989 -0.00169198656258793
7.05999999999989 -0.00161694737041245
7.06999999999989 -0.00154027633942891
7.07999999999989 -0.00146207903696684
7.08999999999989 -0.00138246243852087
7.09999999999989 -0.00130153478646362
7.10999999999989 -0.00121940544767849
7.11999999999989 -0.00113618477034365
7.12999999999989 -0.00105198394008244
7.13999999999989 -0.000966914835684872
7.14999999999989 -0.000881089884597699
7.15999999999989 -0.000794621918376466
7.16999999999989 -0.000707624028288304
7.17999999999989 -0.000620209421252096
7.18999999999989 -0.000532491276299447
7.19999999999989 -0.000444582601737799
7.20999999999989 -0.000356596093194438
7.21999999999989 -0.00026864399271485
7.22999999999989 -0.000180837949082838
7.23999999999989 -9.32888795676861e-05
7.24999999999989 -6.10683323587312e-06
7.25999999999989 8.05991440051282e-05
7.26999999999989 0.000166721142457096
7.27999999999989 0.000252156637665218
7.28999999999989 0.000336796596390564
7.29999999999989 0.000420537295889735
7.30999999999989 0.000503276654801194
7.31999999999989 0.000584914354700945
7.32999999999989 0.000665351958822972
7.33999999999989 0.000744493027812859
7.34999999999989 0.000822243232385642
7.35999999999989 0.000898510462763617
7.36999999999989 0.000973204934773446
7.37999999999989 0.00104623929225828
7.38999999999989 0.00111752870621502
7.39999999999989 0.00118699097066564
7.40999999999989 0.00125454659396195
7.41999999999989 0.00132011888652417
7.42999999999989 0.00138363404461002
7.43999999999989 0.00144502123002631
7.44999999999989 0.00150421264570028
7.45999999999989 0.00156113646160955
7.46999999999989 0.00161573844811654
7.47999999999988 0.00166796039515402
7.48999999999988 0.00171774739637853
7.49999999999988 0.00176504790293984
7.50999999999988 0.0018098137727652
7.51999999999988 0.00185200031531569
7.52999999999988 0.00189156633177879
7.53999999999988 0.00192847415066709
7.54999999999988 0.00196268965879968
7.55999999999988 0.00199418232793785
7.56999999999988 0.00202292523586171
7.57999999999988 0.00204889508358017
7.58999999999988 0.00207207220966843
7.59999999999988 0.00209244059477779
7.60999999999988 0.00210998786576541
7.61999999999988 0.00212470529468979
7.62999999999988 0.00213658779278543
7.63999999999988 0.00214563389991501
7.64999999999988 0.00215184576955495
7.65999999999988 0.00215522914938708
7.66999999999988 0.00215579335759259
7.67999999999988 0.00215355125497954
7.68999999999988 0.00214851921313195
7.69999999999988 0.00214071707886493
7.70999999999988 0.00213016813544367
7.71999999999988 0.00211689906135571
7.72999999999988 0.00210093988810687
7.73999999999988 0.00208232394654373
7.74999999999988 0.00206108779547985
7.75999999999988 0.00203727118366369
7.76999999999988 0.00201091700812386
7.77999999999988 0.00198207128432461
7.78999999999988 0.001950781828365
7.79999999999988 0.00191710019292825
7.80999999999988 0.0018810807356852
7.81999999999988 0.00184278053642522
7.82999999999988 0.00180225931126111
7.83999999999988 0.00175957932400581
7.84999999999988 0.00171480529482668
7.85999999999988 0.00166800373718787
7.86999999999988 0.00161924411007678
7.87999999999988 0.00156859801312588
7.88999999999988 0.00151613909179281
7.89999999999988 0.00146194297054566
7.90999999999988 0.00140608712173824
7.91999999999988 0.00134865074843961
7.92999999999988 0.00128971468263182
7.93999999999988 0.00122936127917223
7.94999999999987 0.00116767430682384
7.95999999999987 0.00110473883711284
7.96999999999987 0.00104064113150293
7.97999999999987 0.000975468527233935
7.98999999999987 0.00090930932209475
7.99999999999987 0.000842252658356019
8.00999999999987 0.000774388406061043
8.01999999999987 0.000705807045857621
8.02999999999987 0.000636599551542444
8.03999999999987 0.000566857272482888
8.04999999999987 0.000496671816075962
8.05999999999987 0.000426134930400468
8.06999999999987 0.000355338387215376
8.07999999999987 0.000284373865454576
8.08999999999987 0.000213332835366784
8.09999999999987 0.000142306443445659
8.10999999999987 7.13853982943479e-05
8.11999999999987 6.59857565842528e-07
8.12999999999987 -6.97806838820047e-05
8.13999999999987 -0.000139850805210543
8.14999999999987 -0.000209459774956198
8.15999999999987 -0.000278520749688337
8.16999999999987 -0.00034694797627391
8.17999999999987 -0.000414656895971967
8.18999999999987 -0.000481564246497663
8.19999999999987 -0.000547588161941186
8.20999999999987 -0.000612648270429573
8.21999999999987 -0.000676665789421994
8.22999999999987 -0.000739563618531636
8.23999999999987 -0.000801266429770983
8.24999999999987 -0.000861700755119758
8.25999999999987 -0.000920795071318704
8.26999999999987 -0.000978479881796109
8.27999999999987 -0.0010346877955724
8.28999999999987 -0.00108935360284962
8.29999999999987 -0.00114241434862255
8.30999999999987 -0.00119380940187249
8.31999999999987 -0.00124348052188914
8.32999999999987 -0.00129137192124582
8.33999999999987 -0.00133743032536279
8.34999999999987 -0.00138160502859717
8.35999999999987 -0.00142384794680209
8.36999999999987 -0.00146411366630185
8.37999999999987 -0.00150235219478862
8.38999999999987 -0.00153853113148808
8.39999999999987 -0.00157261339737164
8.40999999999987 -0.00160456474580239
8.41999999999986 -0.00163435379307395
8.42999999999986 -0.00166195204518922
8.43999999999986 -0.00168733392086326
8.44999999999986 -0.00171047677073958
8.45999999999986 -0.00173136089281494
8.46999999999986 -0.00174996954487427
8.47999999999986 -0.00176628895173926
8.48999999999986 -0.00178030830796584
8.49999999999986 -0.00179201977931951
8.50999999999986 -0.00180141849937683
8.51999999999986 -0.00180850256243773
8.52999999999986 -0.00181327301279681
8.53999999999986 -0.0018157338304369
8.54999999999986 -0.00181589191323062
8.55999999999986 -0.00181375705576794
8.56999999999986 -0.00180934192498063
8.57999999999986 -0.00180266203282225
8.58999999999986 -0.0017937357064186
8.59999999999986 -0.00178258405096769
8.60999999999986 -0.00176923090555191
8.61999999999986 -0.00175370280725719
8.62999999999986 -0.00173602894993056
8.63999999999986 -0.00171624114289856
8.64999999999986 -0.00169437377320341
8.65999999999986 -0.00167046378140815
8.66999999999986 -0.00164455006632706
8.67999999999986 -0.00161667379237684
8.68999999999986 -0.00158687879244762
8.69999999999986 -0.0015552111428222
8.70999999999986 -0.00152171909405082
8.71999999999986 -0.00148645299948285
8.72999999999986 -0.00144946524153963
8.73999999999986 -0.00141081015581826
8.74999999999986 -0.00137054351940971
8.75999999999986 -0.00132872339734451
8.76999999999986 -0.00128540957814996
8.77999999999986 -0.00124066346295995
8.78999999999986 -0.00119454798648718
8.79999999999986 -0.0011471275337895
8.80999999999986 -0.00109846785422423
8.81999999999986 -0.00104863597940121
8.82999999999986 -0.00099770013545651
8.83999999999986 -0.000945729639301679
8.84999999999986 -0.000892794806543536
8.85999999999986 -0.000838966859432882
8.86999999999986 -0.000784317833618063
8.87999999999986 -0.000728920483983416
8.88999999999985 -0.000672848189794361
8.89999999999985 -0.000616174859335745
8.90999999999985 -0.000558974834208195
8.91999999999985 -0.000501322793433962
8.92999999999985 -0.000443293657513949
8.93999999999985 -0.000384962492571719
8.94999999999985 -0.000326404414715169
8.95999999999985 -0.000267694494743351
8.96999999999985 -0.000208907663322885
8.97999999999985 -0.000150118616755732
8.98999999999985 -9.14017234582041e-05
8.99999999999985 -3.28309312683047e-05
9.00999999999985 2.55203243028524e-05
9.01999999999985 8.35792107619375e-05
9.02999999999985 0.00014127677580298
9.03999999999985 0.000198538903520057
9.04999999999985 0.000255295104574844
9.05999999999985 0.000311475837877357
9.06999999999985 0.000367012594672128
9.07999999999985 0.000421837980903006
9.08999999999985 0.000475885797764912
9.09999999999985 0.000529091120353115
9.10999999999985 0.000581390374322429
9.11999999999985 0.000632721410471155
9.12999999999985 0.000683023577166984
9.13999999999985 0.00073223779053474
9.14999999999985 0.000780306602328778
9.15999999999985 0.000827174265415604
9.16999999999985 0.00087278679646283
9.17999999999985 0.000917092036956854
9.18999999999985 0.000960039710998503
9.19999999999985 0.00100158148045361
9.20999999999985 0.00104167099747818
9.21999999999985 0.00108026395427324
9.22999999999985 0.00111731813001731
9.23999999999985 0.00115279343492791
9.24999999999985 0.00118665195140671
9.25999999999985 0.00121885088829974
9.26999999999985 0.0012493640696118
9.27999999999985 0.0012781603788664
9.28999999999985 0.00130521100880888
9.29999999999985 0.00133048948757102
9.30999999999985 0.00135397170176536
9.31999999999985 0.00137563591649521
9.32999999999985 0.00139546279227105
9.33999999999985 0.00141343539882871
9.34999999999985 0.00142953922598494
9.35999999999984 0.00144376219237113
9.36999999999984 0.00145609464721752
9.37999999999984 0.00146652937332167
9.38999999999984 0.00147506158548325
9.39999999999984 0.00148168892592673
9.40999999999984 0.00148641145675435
9.41999999999984 0.00148923164948628
9.42999999999984 0.00149015437176504
9.43999999999984 0.00148918687129692
9.44999999999984 0.00148633875496134
9.45999999999984 0.00148162196563481
9.46999999999984 0.00147505075766903
9.47999999999984 0.00146664166942775
9.48999999999984 0.00145641349345246
9.49999999999984 0.00144438724455574
9.50999999999984 0.00143058612635071
9.51999999999984 0.0014150354971814
9.52999999999984 0.00139776283748148
9.53999999999984 0.00137879772374261
9.54999999999984 0.00135817169861324
9.55999999999984 0.00133591761490014
9.56999999999984 0.00131207077615943
9.57999999999984 0.00128666834445051
9.58999999999984 0.00125974928451275
9.59999999999984 0.00123135430599897
9.60999999999984 0.00120152580383278
9.61999999999984 0.00117030779676157
9.62999999999984 0.00113774560945676
9.63999999999984 0.00110388623574552
9.64999999999984 0.00106877824485185
9.65999999999984 0.0010324715504204
9.66999999999984 0.000995017347696172
9.67999999999984 0.000956468046548749
9.68999999999984 0.000916877201846541
9.69999999999984 0.000876299441927932
9.70999999999984 0.000834790395593225
9.71999999999984 0.000792406617890752
9.72999999999984 0.000749205514895125
9.73999999999984 0.00070524526783527
9.74999999999984 0.000660584760629904
9.75999999999984 0.000615283498232654
9.76999999999984 0.000569401526763872
9.77999999999984 0.000522999355723668
9.78999999999984 0.000476137879843199
9.79999999999984 0.000428878300708699
9.80999999999984 0.000381282048282437
9.81999999999984 0.000333410702436891
9.82999999999983 0.000285325914613641
9.83999999999983 0.00023708932971419
9.84999999999983 0.000188762508327309
9.85999999999983 0.000140406849394422
9.86999999999983 9.20835132091202e-05
9.87999999999983 4.38533453939232e-05
9.88999999999983 -4.22319810482498e-06
9.89999999999983 -5.20861259955194e-05
9.90999999999983 -9.96759853255754e-05
9.91999999999983 -0.000146933934163415
9.92999999999983 -0.000193805135612527
9.93999999999983 -0.000240229292625882
9.94999999999983 -0.000286149815350018
9.95999999999983 -0.000331510999699932
9.96999999999983 -0.000376258093959556
9.97999999999983 -0.000420337363852729
9.98999999999983 -0.000463696156012699
9.99999999999983 -0.000506282959779911
10.0099999999998 -0.000548047467260037
10.0199999999998 -0.000588940631575461
10.0299999999998 -0.000628914723246592
10.0399999999998 -0.000667923384640857
10.0499999999998 -0.000705921682296621
10.0599999999998 -0.000742866157362893
10.0699999999998 -0.000778714874204813
10.0799999999998 -0.000813427466358761
10.0899999999998 -0.000846965180478304
10.0999999999998 -0.000879290918029244
10.1099999999998 -0.000910369274689687
10.1199999999998 -0.0009401598003089
10.1299999999998 -0.000968637432383786
10.1399999999998 -0.000995772124711121
10.1499999999998 -0.00102153566112673
10.1599999999998 -0.0010459016829772
10.1699999999998 -0.00106884571411795
10.1799999999998 -0.00109034518341877
10.1899999999998 -0.00111037944476166
10.1999999999998 -0.00112892979451891
10.2099999999998 -0.00114597948650334
10.2199999999998 -0.00116151374438606
10.2299999999998 -0.00117551977158153
10.2399999999998 -0.00118798675942843
10.2499999999998 -0.00119890589067875
10.2599999999998 -0.00120827034192755
10.2699999999998 -0.00121607528386175
10.2799999999998 -0.00122231787860177
10.2899999999998 -0.00122699727395598
10.2999999999998 -0.00123011459541521
10.3099999999998 -0.00123167293598113
10.3199999999998 -0.00123167734359579
10.3299999999998 -0.00123013480621586
10.3399999999998 -0.00122705423458603
10.3499999999998 -0.00122244644278191
10.3599999999998 -0.001216324126617
10.3699999999998 -0.00120870184004906
10.3799999999998 -0.00119959596979458
10.3899999999998 -0.00118902470849569
10.3999999999998 -0.0011770080270711
10.4099999999998 -0.00116356764752744
10.4199999999998 -0.00114872701915179
10.4299999999998 -0.00113251130674797
10.4399999999998 -0.00111494677520048
10.4499999999998 -0.00109606151968201
10.4599999999998 -0.00107588529014164
10.4699999999998 -0.00105444933396529
10.4799999999998 -0.00103178634936328
10.4899999999998 -0.00100793043719807
10.4999999999998 -0.000982917051309483
10.5099999999998 -0.000956782947398294
10.5199999999998 -0.00092956581235388
10.5299999999998 -0.000901304910020429
10.5399999999998 -0.000872040628589724
10.5499999999998 -0.000841814428852033
10.5599999999998 -0.000810668791075025
10.5699999999998 -0.000778647159077296
10.5799999999998 -0.000745793882423534
10.5899999999998 -0.000712154157227079
10.5999999999998 -0.000677773965849966
10.6099999999998 -0.000642700015696486
10.6199999999998 -0.00060697967724797
10.6299999999998 -0.000570660921460712
10.6399999999998 -0.000533792256633873
10.6499999999998 -0.000496422664845904
10.6599999999998 -0.000458601537760285
10.6699999999998 -0.000420378612823202
10.6799999999998 -0.000381803909214266
10.6899999999998 -0.000342927663125138
10.6999999999998 -0.000303800263471485
10.7099999999998 -0.000264472188679314
10.7199999999998 -0.000224993941329008
10.7299999999998 -0.000185415983906064
10.7399999999998 -0.000145788675373281
10.7499999999998 -0.000106162208139414
10.7599999999998 -6.65865455067837e-05
10.7699999999998 -2.71113596776424e-05
10.7799999999998 1.22140296020315e-05
10.7899999999998 5.13407156844729e-05
10.7999999999998 9.02202648730018e-05
10.8099999999998 0.000128804775414271
10.8199999999998 0.00016704693555465
10.8299999999998 0.000204900080593005
10.8399999999998 0.00024232190573987
10.8499999999998 0.000279263943058864
10.8599999999998 0.000315681785078567
10.8699999999998 0.000351531884760017
10.8799999999998 0.000386771606715114
10.8899999999998 0.000421359277024456
10.8999999999998 0.000455254231599564
10.9099999999998 0.000488416863036509
10.9199999999998 0.000520808665909417
10.9299999999998 0.000552392280409047
10.9399999999998 0.000583131534214497
10.9499999999998 0.000612991483231111
10.9599999999998 0.00064193844995652
10.9699999999998 0.000669933998864622
10.9799999999998 0.000696953082136159
10.9899999999998 0.000722966088776144
10.9999999999998 0.00074794481225959
11.0099999999998 0.000771862479840828
11.0199999999998 0.000794693779914225
11.0299999999998 0.000816414887401842
11.0399999999998 0.000837003487146393
11.0499999999998 0.000856438795290178
11.0599999999998 0.00087470157862381
11.0699999999998 0.000891774171891166
11.0799999999998 0.000907640493234607
11.0899999999998 0.000922286057344735
11.0999999999998 0.000935697986221989
11.1099999999998 0.000947865018170779
11.1199999999998 0.000958777514952468
11.1299999999998 0.000968427466799934
11.1399999999998 0.000976808494020223
11.1499999999998 0.000983915848419613
11.1599999999998 0.000989746411946237
11.1699999999998 0.000994298693292091
11.1799999999998 0.000997572822472357
11.1899999999998 0.000999570543404075
11.1999999999998 0.00100029520451101
11.2099999999998 0.000999751747387591
11.2199999999998 0.00099794669356283
11.2299999999998 0.000994888129377038
11.2399999999998 0.000990585688929095
11.2499999999998 0.000985050535831495
11.2599999999998 0.00097829534299218
11.2699999999998 0.000970334271192795
11.2799999999998 0.000961182946747945
11.2899999999998 0.00095085843901779
11.2999999999998 0.000939379239445999
11.3099999999998 0.000926765246471214
11.3199999999998 0.00091303753131548
11.3299999999998 0.000898218263948649
11.3399999999998 0.000882331161393508
11.3499999999998 0.000865401182911744
11.3599999999998 0.000847454492481596
11.3699999999998 0.000828518419978735
11.3799999999998 0.000808621421106119
11.3899999999998 0.000787793036121549
11.3999999999998 0.000766063698828781
11.4099999999998 0.000743464905264325
11.4199999999998 0.000720029235848534
11.4299999999998 0.000695790167285268
11.4399999999998 0.000670782030727629
11.4499999999998 0.00064503996700993
11.4599999999998 0.00061859988003931
11.4699999999998 0.000591498388885998
11.4799999999998 0.00056377277886341
11.4899999999998 0.000535460951804529
11.4999999999998 0.000506601375662026
11.5099999999998 0.000477233033542929
11.5199999999998 0.000447395372270723
11.5299999999998 0.000417128250558328
11.5399999999998 0.00038647188686994
11.5499999999998 0.000355466807046172
11.5599999999998 0.000324153791764852
11.5699999999998 0.000292573823907893
11.5799999999998 0.000260768035903815
11.5899999999998 0.000228777657114424
11.5999999999998 0.000196643961333088
11.6099999999998 0.000164408214461729
11.6199999999998 0.000132111622432408
11.6299999999998 9.97952794389247e-05
11.6399999999998 6.75001165428969e-05
11.6499999999998 3.52668507179086e-05
11.6599999999998 3.13593439449729e-06
11.6699999999998 -2.88524944323342e-05
11.6799999999998 -6.06586614716403e-05
11.6899999999998 -9.22432047239924e-05
11.6999999999998 -0.000123567222326379
11.7099999999998 -0.000154592319598845
11.7199999999998 -0.00018528065523672
11.7299999999998 -0.000215594986593878
11.7399999999998 -0.000245498714003796
11.7499999999998 -0.000274955924086829
11.7599999999998 -0.00030393143199341
11.7699999999998 -0.000332390822534617
11.7799999999998 -0.000360300490153205
11.7899999999998 -0.000387627677689754
11.7999999999998 -0.000414340513900453
11.8099999999998 -0.000440408049684799
11.8199999999998 -0.000465800292967466
11.8299999999998 -0.000490488242259983
11.8399999999998 -0.000514443918772006
11.8499999999998 -0.000537640397095405
11.8599999999998 -0.000560051834420414
11.8699999999998 -0.000581653498250909
11.8799999999998 -0.000602421792590642
11.8899999999998 -0.000622334282574405
11.8999999999998 -0.000641369717520064
11.9099999999998 -0.000659508052379704
11.9199999999998 -0.000676730467570066
11.9299999999998 -0.000693019387164815
11.9399999999998 -0.000708358495433186
11.9499999999998 -0.000722732751711902
11.9599999999998 -0.00073612840359936
11.9699999999998 -0.000748532998463381
11.9799999999998 -0.000759935393256034
11.9899999999998 -0.000770325762631275
11.9999999999998 -0.000779695605363544
12.0099999999998 -0.000788037749619298
12.0199999999998 -0.000795346354954356
12.0299999999998 -0.000801616913978034
12.0399999999998 -0.000806846252139257
12.0499999999998 -0.000811032525705427
12.0599999999998 -0.000814175218081797
12.0699999999998 -0.000816275134487731
12.0799999999998 -0.000817334395009869
12.0899999999998 -0.000817356426056656
12.0999999999998 -0.000816345950244471
12.1099999999998 -0.000814308974753389
12.1199999999998 -0.000811252778202241
12.1299999999998 -0.000807185896110798
12.1399999999998 -0.000802118105047445
12.1499999999998 -0.000796060405616482
12.1599999999998 -0.000789025004463191
12.1699999999998 -0.00078102529493584
12.1799999999998 -0.000772075840678562
12.1899999999998 -0.000762192359837235
12.1999999999998 -0.000751391719159772
12.2099999999998 -0.000739691421513665
12.2199999999998 -0.000727110343343211
12.2299999999998 -0.00071366840212824
12.2399999999998 -0.000699386513873565
12.2499999999998 -0.000684286561870729
12.2599999999998 -0.00066839136441894
12.2699999999998 -0.000651724641544197
12.2799999999998 -0.000634310965685707
12.2899999999998 -0.000616175557589494
12.2999999999998 -0.000597344677470745
12.3099999999998 -0.000577845332120829
12.3199999999998 -0.00055770524243756
12.3299999999998 -0.000536952807806668
12.3399999999998 -0.000515617068643571
12.3499999999998 -0.000493727667697945
12.3599999999998 -0.000471314810438691
12.3699999999998 -0.000448409224710301
12.3799999999998 -0.000425042119790242
12.3899999999998 -0.000401245144945481
12.3999999999998 -0.000377050347569306
12.4099999999998 -0.000352490130969656
12.4199999999998 -0.000327597211874684
12.4299999999998 -0.000302404577717638
12.4399999999998 -0.000276945443760822
12.4499999999998 -0.000251253210116984
12.4599999999998 -0.000225361418725329
12.4699999999998 -0.000199303710338389
12.4799999999998 -0.000173113781575436
12.4899999999998 -0.000146825342097325
12.4999999999998 -0.000120472071957043
12.5099999999998 -9.40875791796826e-05
12.5199999999998 -6.77053576248382e-05
12.5299999999998 -4.13587451837116e-05
12.5399999999998 -1.5080882362538e-05
12.5499999999998 1.10953286968424e-05
12.5599999999998 3.71372647090632e-05
12.5699999999998 6.30126212014245e-05
12.5799999999998 8.86894518858158e-05
12.5899999999998 0.000114136207398641
12.5999999999998 0.000139321773363041
12.6099999999998 0.000164215507728238
12.6199999999998 0.000188787277341968
12.6299999999998 0.000213007493713324
12.6399999999998 0.000236847147924303
12.6499999999998 0.000260277844649916
12.6599999999998 0.000283271835248135
12.6699999999998 0.000305802049881257
12.6799999999998 0.000327842128632508
12.6899999999998 0.00034936645158463
12.6999999999998 0.000370350167821034
12.7099999999998 0.000390769223330445
12.7199999999998 0.000410600387784527
12.7299999999998 0.000429821280133416
12.7399999999998 0.000448410393019949
12.7499999999998 0.000466347115978384
12.7599999999998 0.000483611757393901
12.7699999999998 0.000500185565200803
12.7799999999998 0.000516050746299041
12.7899999999998 0.00053119048467044
12.7999999999998 0.000545588958177727
12.8099999999998 0.000559231354031266
12.8199999999998 0.000572103882910077
12.8299999999998 0.000584193791725642
12.8399999999998 0.000595489375018711
12.8499999999998 0.000605979984981143
12.8599999999998 0.000615656040096698
12.8699999999998 0.000624509032396473
12.8799999999998 0.000632531533326602
12.8899999999998 0.00063971719841113
12.8999999999998 0.000646060770268486
12.9099999999998 0.000651558079876441
12.9199999999998 0.000656206047233107
12.9299999999998 0.000660002680317999
12.9399999999998 0.000662947072692092
12.9499999999998 0.000665039399749036
12.9599999999998 0.000666280913632477
12.9699999999998 0.000666673936837708
12.9799999999998 0.000666221854520026
12.9899999999998 0.000664929105537636
12.9999999999998 0.000662801172264773
13.0099999999998 0.000659844569222594
13.0199999999998 0.0006560668305948
13.0299999999998 0.000651476496728856
13.0399999999998 0.000646083099786722
13.0499999999998 0.000639897148836309
13.0599999999998 0.000632930114955381
13.0699999999998 0.000625194417606967
13.0799999999998 0.000616703415624639
13.0899999999998 0.0006074711952053
13.0999999999998 0.000597512654842556
13.1099999999998 0.000586843691385702
13.1199999999998 0.000575481028421658
13.1299999999998 0.000563442191152152
13.1399999999998 0.000550745480405771
13.1499999999998 0.000537409945816143
13.1599999999998 0.000523455358199445
13.1699999999998 0.000508902078121328
13.1799999999998 0.000493771180264855
13.1899999999998 0.000478084455917094
13.1999999999998 0.00046186429169699
13.2099999999998 0.000445133641525074
13.2199999999998 0.000427915996667109
13.2299999999998 0.000410235354564144
13.2399999999998 0.000392116186800873
13.2499999999998 0.000373583406410736
13.2599999999998 0.000354662334644755
13.2699999999998 0.000335378667295329
13.2799999999998 0.000315758440647355
13.2899999999998 0.000295827997118484
13.2999999999998 0.000275613950644262
13.3099999999998 0.000255143151860262
13.3199999999998 0.000234442653130861
13.3299999999998 0.000213539673472963
13.3399999999998 0.000192461563421772
13.3499999999998 0.000171235769884984
13.3599999999998 0.000149889801031104
13.3699999999998 0.00012845119125702
13.3799999999998 0.00010694746627941
13.3899999999998 8.54061083941295e-05
13.3999999999998 6.38545219471043e-05
13.4099999999998 4.23199990596893e-05
13.4199999999998 2.08296856510152e-05
13.4299999999998 -5.89452200967218e-07
13.4399999999998 -2.19106615185914e-05
13.4499999999998 -4.31074352629305e-05
13.4599999999998 -6.41535447129527e-05
13.4699999999998 -8.50230713451762e-05
13.4799999999998 -0.000105690438176402
13.4899999999998 -0.000126130440532139
13.4999999999998 -0.000146318276204421
13.5099999999998 -0.000166229574963604
13.5199999999998 -0.000185840427389775
13.5299999999998 -0.000205127412990355
13.5399999999998 -0.000224067627571641
13.5499999999998 -0.000242638709832977
13.5599999999998 -0.000260818867153567
13.5699999999998 -0.000278586900542935
13.5799999999998 -0.000295922228727318
13.5899999999998 -0.000312804911338413
13.5999999999998 -0.000329215671208092
13.6099999999998 -0.000345135915701828
13.6199999999998 -0.000360547757095391
13.6299999999998 -0.000375434031968691
13.6399999999998 -0.000389778319595773
13.6499999999998 -0.000403564959312329
13.6599999999998 -0.000416779066843463
13.6699999999998 -0.000429406549575891
13.6799999999998 -0.000441434120760122
13.6899999999998 -0.000452849312629623
13.6999999999998 -0.00046364048842544
13.7099999999998 -0.000473796853316164
13.7199999999998 -0.000483308464204601
13.7299999999998 -0.000492166238414038
13.7399999999998 -0.000500361961248407
13.7499999999998 -0.00050788829242221
13.7599999999998 -0.000514738771357551
13.7699999999998 -0.000520907821347156
13.7799999999998 -0.000526390752899637
13.7899999999998 -0.000531183765010454
13.7999999999998 -0.000535283946203285
13.8099999999998 -0.00053868927421298
13.8199999999997 -0.000541398614518163
13.8299999999997 -0.000543411717767361
13.8399999999997 -0.000544729216109793
13.8499999999997 -0.000545352618444436
13.8599999999997 -0.000545284304603994
13.8699999999997 -0.000544527518494305
13.8799999999997 -0.000543086360215051
13.8899999999997 -0.000540965777195614
13.8999999999997 -0.000538171554392373
13.9099999999997 -0.000534710303614867
13.9199999999997 -0.000530589452086588
13.9299999999997 -0.000525817230419879
13.9399999999997 -0.000520402660340968
13.9499999999997 -0.000514355542863477
13.9599999999997 -0.00050768644856149
13.9699999999997 -0.000500406693323839
13.9799999999997 -0.000492528031245791
13.9899999999997 -0.000484063140691113
13.9999999999997 -0.000475025384191528
14.0099999999997 -0.000465428788254214
14.0199999999997 -0.00045528802245063
14.0299999999997 -0.000444618377811752
14.0399999999997 -0.000433435744556406
14.0499999999997 -0.000421756581277887
14.0599999999997 -0.000409597772127079
14.0699999999997 -0.000396976894446732
14.0799999999997 -0.00038391202101897
14.0899999999997 -0.000370421698332026
14.0999999999997 -0.000356524922762475
14.1099999999997 -0.000342241115536871
14.1199999999997 -0.000327590096872079
14.1299999999997 -0.000312592059505547
14.1399999999997 -0.000297267541742847
14.1499999999997 -0.000281637400109121
14.1599999999997 -0.00026572278166998
14.1699999999997 -0.0002495450960761
14.1799999999997 -0.000233125987379256
14.1899999999997 -0.000216487305663272
14.1999999999997 -0.000199651078532081
14.2099999999997 -0.000182639482494276
14.2199999999997 -0.000165474814283279
14.2299999999997 -0.000148179462151241
14.2399999999997 -0.000130775877174258
14.2499999999997 -0.000113286544606004
14.2599999999997 -9.57339553164048e-05
14.2699999999997 -7.81405773515591e-05
14.2799999999997 -6.05288276506781e-05
14.2899999999997 -4.29210439553692e-05
14.2999999999997 -2.533945694616e-05
14.3099999999997 -7.80616264058814e-06
14.3199999999997 9.65690491326938e-06
14.3299999999997 2.70280006144464e-05
14.3399999999997 4.42855949254695e-05
14.3499999999997 6.14084000952201e-05
14.3599999999997 7.83753959568158e-05
14.3699999999997 9.51658552698273e-05
14.3799999999997 0.000111759368576867
14.3899999999997 0.000128135868545233
14.3999999999997 0.000144275653765281
14.4099999999997 0.000160159411977712
14.4199999999997 0.000175768242703185
14.4299999999997 0.000191083679248219
14.4399999999997 0.000206087710062477
14.4499999999997 0.000220762799423354
14.4599999999997 0.000235091907424808
14.4699999999997 0.000249058509245985
14.4799999999997 0.000262646613685459
14.4899999999997 0.000275840780937825
14.4999999999997 0.000288626139585234
14.5099999999997 0.000300988402796541
14.5199999999997 0.000312913883713742
14.5299999999997 0.000324389510009941
14.5399999999997 0.000335402837604328
14.5499999999997 0.000345942063520651
14.5599999999997 0.000355996037876934
14.5699999999997 0.000365554274995271
14.5799999999997 0.000374606963621784
14.5899999999997 0.000383144976247927
14.5999999999997 0.000391159877525616
14.6099999999997 0.000398643931769751
14.6199999999997 0.000405590109543024
14.6299999999997 0.000411992093319039
14.6399999999997 0.000417844282221053
14.6499999999997 0.000423141795834873
14.6599999999997 0.00042788047720598
14.6699999999997 0.000432056894704426
14.6799999999997 0.000435668342904144
14.6899999999997 0.000438712842846736
14.6999999999997 0.000441189141248609
14.7099999999997 0.00044309670880101
14.7199999999997 0.000444435737571276
14.7299999999997 0.000445207137515471
14.7399999999997 0.000445412532114805
14.7499999999997 0.000445054253151072
14.7599999999997 0.000444135334640054
14.7699999999997 0.000442659505947263
14.7799999999997 0.000440631184118576
14.7899999999997 0.000438055465471802
14.7999999999997 0.000434938116518866
14.8099999999997 0.000431285564332488
14.8199999999997 0.000427104886560968
14.8299999999997 0.000422403801493448
14.8399999999997 0.000417190659068974
14.8499999999997 0.000411474435207186
14.8599999999997 0.000405264563042907
14.8699999999997 0.000398571058123147
14.8799999999997 0.000391404583239316
14.8899999999997 0.000383776350883937
14.8999999999997 0.00037569810643849
14.9099999999997 0.000367182110784675
14.9199999999997 0.000358241122360537
14.9299999999997 0.000348888378683989
14.9399999999997 0.000339137507499895
14.9499999999997 0.000329002614071706
14.9599999999997 0.000318498277182189
14.9699999999997 0.000307639470659716
14.9799999999997 0.000296441544594481
14.9899999999997 0.000284920205298484
14.9999999999997 0.000273091494469516
15.0099999999997 0.000260971767788228
15.0199999999997 0.000248577673078225
15.0299999999997 0.000235926128112835
15.0399999999997 0.000223034298128913
15.0499999999997 0.000209919573095712
15.0599999999997 0.000196599544779947
15.0699999999997 0.000183091983644134
15.0799999999997 0.00016941481561294
15.0899999999997 0.000155586098740675
15.0999999999997 0.000141623999812085
15.1099999999997 0.000127546770907879
15.1199999999997 0.000113372725965858
15.1299999999997 9.91202173681626e-05
15.1399999999997 8.48076125846582e-05
15.1499999999997 7.04532709022185e-05
15.1599999999997 5.60755202692905e-05
15.1699999999997 4.16926342846892e-05
15.1799999999997 2.73228093593485e-05
15.1899999999997 1.29841420792082e-05
15.1999999999997 -1.30539320290625e-06
15.2099999999997 -1.55279665190122e-05
15.2199999999997 -2.96659141543571e-05
15.2299999999997 -4.37017602062659e-05
15.2399999999997 -5.76182378072207e-05
15.2499999999997 -7.13983099870508e-05
15.2599999999997 -8.50251901493664e-05
15.2699999999997 -9.84823621381693e-05
15.2799999999997 -0.000111753599871037
15.2899999999997 -0.000124822986516083
15.2999999999997 -0.000137674933190478
15.3099999999997 -0.000150294197159113
15.3199999999997 -0.000162665899512608
15.3299999999997 -0.000174775542304769
15.3399999999997 -0.000186609025130238
15.3499999999997 -0.000198152661123973
15.3599999999997 -0.00020939319236154
15.3699999999997 -0.000220317804657804
15.3799999999997 -0.000230914141725714
15.3899999999997 -0.000241170318695671
15.3999999999997 -0.000251074934977331
15.4099999999997 -0.000260617086450748
15.4199999999997 -0.000269786376974368
15.4299999999997 -0.000278572929198529
15.4399999999997 -0.000286967394674049
15.4499999999997 -0.000294960963246333
15.4599999999997 -0.000302545371726477
15.4699999999997 -0.000309712911831794
15.4799999999997 -0.000316456437389093
15.4899999999997 -0.000322769370795138
15.4999999999997 -0.000328645708729586
15.5099999999997 -0.000334080027116765
15.5199999999997 -0.00033906748533363
15.5299999999997 -0.000343603829662253
15.5399999999997 -0.000347685395986213
15.5499999999997 -0.000351309111895355
15.5599999999997 -0.000354472497532847
15.5699999999997 -0.000357173666199675
15.5799999999997 -0.000359411324045121
15.5899999999997 -0.000361184769010774
15.5999999999997 -0.000362493889037553
15.6099999999997 -0.000363339159543347
15.6199999999997 -0.000363721640180515
15.6299999999997 -0.000363642970884572
15.6399999999997 -0.000363105367228031
15.6499999999997 -0.000362111615097073
15.6599999999997 -0.000360665064714189
15.6699999999997 -0.000358769624038643
15.6799999999997 -0.000356429751591396
15.6899999999997 -0.000353650448778008
15.6999999999997 -0.000350437251835442
15.7099999999997 -0.000346796223640152
15.7199999999997 -0.000342733945875639
15.7299999999997 -0.000338257512751145
15.7399999999997 -0.000333374493598173
15.7499999999997 -0.000328092782276014
15.7599999999997 -0.000322420873920144
15.7699999999997 -0.000316367718459572
15.7799999999997 -0.000309942707114203
15.7899999999997 -0.000303155658413078
15.7999999999997 -0.000296016803750748
15.8099999999997 -0.000288536772500005
15.8199999999997 -0.000280726567718945
15.8299999999997 -0.000272597483929582
15.8399999999997 -0.0002641612700703
15.8499999999997 -0.000255430003952908
15.8599999999997 -0.000246416077644175
15.8699999999997 -0.00023713218151369
15.8799999999997 -0.000227591287492417
15.8899999999997 -0.00021780663179615
15.8999999999997 -0.000207791697249802
15.9099999999997 -0.000197560195295049
15.9199999999997 -0.000187126047737974
15.9299999999997 -0.000176503368279847
15.9399999999997 -0.000165706443866856
15.9499999999997 -0.000154749715890412
15.9599999999997 -0.000143647761267174
15.9699999999997 -0.000132415273426343
15.9799999999997 -0.000121067043230868
15.9899999999997 -0.000109617939858404
15.9999999999997 -9.80828916674827e-05
16.0099999999997 -8.64768670739061e-05
16.0199999999997 -7.48148554620373e-05
16.0299999999997 -6.31118481553827e-05
16.0399999999997 -5.13828194705991e-05
16.0499999999997 -3.96427078787278e-05
16.0599999999997 -2.79063972971126e-05
16.0699999999997 -1.61886985352924e-05
16.0799999999997 -4.50433091771214e-06
16.0899999999997 7.13209589430602e-06
16.0999999999997 1.87060998590524e-05
16.1099999999997 3.02033442677395e-05
16.1199999999997 4.16096551933497e-05
16.1299999999997 5.29110386539296e-05
16.1399999999997 6.40936974701323e-05
16.1499999999997 7.51440477968834e-05
16.1599999999997 8.60487353099104e-05
16.1699999999997 9.67946510280981e-05
16.1799999999997 0.000107368946753401
16.1899999999997 0.000117759050110515
16.1999999999997 0.000127952679169126
16.2099999999997 0.000137937856632119
16.2199999999997 0.000147702923573879
16.2299999999997 0.000157236552713283
16.2399999999997 0.000166527761205562
16.2499999999997 0.000175565922943215
16.2599999999997 0.000184340780348866
16.2699999999997 0.000192842455646326
16.2799999999997 0.000201061461601346
16.2899999999997 0.000208988711719658
16.2999999999997 0.000216615529891897
16.3099999999998 0.000223933659475811
16.3199999999998 0.000230935271806838
16.3299999999998 0.000237612974128999
16.3399999999998 0.000243959816938712
16.3499999999998 0.000249969300735035
16.3599999999998 0.000255635382170574
16.3699999999998 0.000260952479598085
16.3799999999998 0.000265915478008614
16.3899999999998 0.000270519733357844
16.3999999999998 0.000274761076278103
16.4099999999998 0.000278635815174317
16.4199999999998 0.000282140738703044
16.4299999999998 0.000285273117693726
16.4399999999998 0.000288030706325334
16.4499999999998 0.000290411742707737
16.4599999999998 0.000292414948959713
16.4699999999998 0.000294039530620252
16.4799999999998 0.000295285175456004
16.4899999999998 0.000296152051670527
16.4999999999998 0.000296640805522219
16.5099999999998 0.000296752558359368
16.5199999999998 0.000296488903082638
16.5299999999998 0.000295851900047881
16.5399999999998 0.000294844072425853
16.5499999999998 0.000293468401041079
16.5599999999998 0.000291728318721462
16.5699999999998 0.000289627704206707
16.5799999999998 0.000287170875694597
16.5899999999998 0.000284362584167559
16.5999999999998 0.000281208006783189
16.6099999999998 0.00027771274096503
16.6199999999998 0.000273882800895311
16.6299999999998 0.000269724484597117
16.6399999999998 0.000265244510054452
16.6499999999998 0.000260450011543294
16.6599999999998 0.000255348488485641
16.6699999999998 0.000249947794218776
16.6799999999998 0.000244256124380495
16.6899999999998 0.000238282004925012
16.6999999999998 0.000232034279784848
16.7099999999998 0.000225522049737429
16.7199999999998 0.000218754737872679
16.7299999999998 0.000211742077580092
16.7399999999998 0.000204494064267539
16.7499999999998 0.000197020942769095
16.7599999999998 0.000189333193946225
16.7699999999998 0.000181441520773213
16.7799999999998 0.000173356834052937
16.7899999999998 0.000165090237846757
16.7999999999998 0.000156653014672782
16.8099999999998 0.000148056610512119
16.8199999999998 0.000139312619654597
16.8299999999998 0.000130432769411226
16.8399999999998 0.000121428904717988
16.8499999999998 0.000112312972653921
16.8599999999998 0.000103097006895617
16.8699999999998 9.37931121294413e-05
16.8799999999998 8.4413448442461e-05
16.8899999999998 7.49702157125085e-05
16.8999999999998 6.54756380177869e-05
16.9099999999998 5.59419480859542e-05
16.9199999999998 4.6381371802455e-05
16.9299999999998 3.68061127976659e-05
16.9399999999998 2.72283371321506e-05
16.9499999999999 1.76601580990443e-05
16.9599999999999 8.11362116237096e-06
16.9699999999999 -1.39931095019847e-06
16.9799999999999 -1.0866772981961e-05
16.9899999999999 -2.02770116912607e-05
16.9999999999999 -2.96184001875138e-05
17.0099999999999 -3.88794520419439e-05
17.0199999999999 -4.80488351561628e-05
17.0299999999999 -5.71153853722229e-05
17.0399999999999 -6.60681198080466e-05
17.0499999999999 -7.48962499026365e-05
17.0599999999999 -8.35891941559153e-05
17.0699999999999 -9.2136590548448e-05
17.0799999999999 -0.000100528308626836
17.0899999999999 -0.000108754461241016
17.0999999999999 -0.000116805415920221
17.1099999999999 -0.000124671805874917
17.1199999999999 -0.000132344540612333
17.1299999999999 -0.00013981481615317
17.1399999999999 -0.00014707412484326
17.1499999999999 -0.000154114264741263
17.1599999999999 -0.000160927348578547
17.1699999999999 -0.000167505812280361
17.1799999999999 -0.000173842423039455
17.1899999999999 -0.000179930286934075
17.1999999999999 -0.000185762856082805
17.2099999999999 -0.000191333935329344
17.2199999999999 -0.000196637688450985
17.2299999999999 -0.00020166864388513
17.2399999999999 -0.000206421699968916
17.2499999999999 -0.000210892129687566
17.2599999999999 -0.000215075584927804
17.2699999999999 -0.000218968100233304
17.2799999999999 -0.00022256609605983
17.2899999999999 -0.000225866381528326
17.2999999999999 -0.000228866156674972
17.3099999999999 -0.000231563014201154
17.3199999999999 -0.000233954940782722
17.3299999999999 -0.000236040317663593
17.3399999999999 -0.000237817920975066
17.3499999999999 -0.000239286921479628
17.3599999999999 -0.000240446883820352
17.3699999999999 -0.000241297765280081
17.3799999999999 -0.000241839914055525
17.3899999999999 -0.000242074067052544
17.3999999999999 -0.000242001347210246
17.4099999999999 -0.000241623260363368
17.4199999999999 -0.000240941691654862
17.4299999999999 -0.000239958901514395
17.4399999999999 -0.000238677521224399
17.4499999999999 -0.000237100548105484
17.4599999999999 -0.000235231340371574
17.4699999999999 -0.00023307361174158
17.4799999999999 -0.000230631425972179
17.4899999999999 -0.000227909191659914
17.4999999999999 -0.000224911658153594
17.5099999999999 -0.000221643878150976
17.5199999999999 -0.000218111140552256
17.5299999999999 -0.000214319124834861
17.5399999999999 -0.000210273812371904
17.5499999999999 -0.000205981477420634
17.5599999999999 -0.000201448677791736
17.5699999999999 -0.000196682245211353
17.5799999999999 -0.000191689275388244
17.59 -0.000186477110300877
17.6 -0.000181053289650679
17.61 -0.000175425651070457
17.62 -0.000169602250227349
17.63 -0.000163591351011916
17.64 -0.000157401414873392
17.65 -0.000151041089644756
17.66 -0.000144519198019692
17.67 -0.000137844725768659
17.68 -0.000131026809747648
17.69 -0.000124074725736505
17.7 -0.000116997876135177
17.71 -0.000109805777541521
17.72 -0.000102508048231545
17.73 -9.51143955614539e-05
17.74 -8.7634603309762e-05
17.75 -8.00785189771546e-05
17.76 -7.24560410613114e-05
17.77 -6.47771063236098e-05
17.78 -5.70516770642776e-05
17.79 -4.92897284224754e-05
17.8 -4.15012357174812e-05
17.81 -3.36961618470138e-05
17.82 -2.58844447585475e-05
17.83 -1.80759850091712e-05
17.84 -1.02806334295148e-05
17.85 -2.50817890683541e-06
17.86 5.2316636977175e-06
17.87 1.29292654829582e-05
17.88 2.05750952782383e-05
17.89 2.81597312365408e-05
17.9 3.56738722362292e-05
17.91 4.31083490779168e-05
17.92 5.04541354631978e-05
17.93 5.77023587423983e-05
17.94 6.48443104187461e-05
17.95 7.18714563968683e-05
17.96 7.87754469637392e-05
17.97 8.55481264907982e-05
17.98 9.21815428461363e-05
17.99 9.86679565062817e-05
18 0.000104999849357396
18.01 0.000111169933175586
18.02 0.000117171157779279
18.03 0.000122996718842391
18.04 0.000128640065360398
18.05 0.000134094906762366
18.06 0.000139355219661191
18.07 0.000144415254235208
18.08 0.000149269540234762
18.09 0.000153912892607935
18.1 0.000158340416740075
18.11 0.000162547513302315
18.12 0.000166529882704775
18.13 0.000170283529150675
18.14 0.000173804764288138
18.15 0.000177090210456956
18.16 0.000180136803528144
18.17 0.000182941795334686
18.18 0.000185502755692345
18.19 0.000187817574010038
18.2 0.000189884460518959
18.21 0.000191701947022991
18.22 0.000193268887267292
18.2300000000001 0.000194584456935689
18.2400000000001 0.000195648153220903
18.2500000000001 0.000196459793993374
18.2600000000001 0.000197019516572497
18.2700000000001 0.000197327776104922
18.2800000000001 0.000197385343555606
18.2900000000001 0.000197193303318535
18.3000000000001 0.000196753050455815
18.3100000000001 0.000196066287576281
18.3200000000001 0.000195135021368621
18.3300000000001 0.000193961558810377
18.3400000000001 0.000192548503085377
18.3500000000001 0.000190898749263421
18.3600000000001 0.000189015479839598
18.3700000000001 0.000186902160328373
18.3800000000001 0.000184562535353403
18.3900000000001 0.000182000626415307
18.4000000000001 0.000179220632870552
18.4100000000001 0.000176227051055461
18.4200000000001 0.000173024645503874
18.4300000000001 0.000169618422931102
18.4400000000001 0.000166013624745829
18.4500000000001 0.000162215719306537
18.4600000000001 0.000158230393932447
18.4700000000001 0.000154063546679445
18.4800000000001 0.00014972124582667
18.4900000000001 0.00014520977420943
18.5000000000001 0.000140535618842571
18.5100000000001 0.000135705440272576
18.5200000000001 0.000130726064155908
18.5300000000001 0.000125604472317889
18.5400000000001 0.000120347793476951
18.5500000000001 0.00011496329372791
18.5600000000001 0.000109458366838247
18.5700000000001 0.000103840524392782
18.5800000000001 9.81173858125417e-05
18.5900000000001 9.22966682686517e-05
18.6000000000001 8.63861765091416e-05
18.6100000000001 8.03937926150145e-05
18.6200000000001 7.43274657007786e-05
18.6300000000001 6.81952015741017e-05
18.6400000000001 6.20050523687606e-05
18.6500000000001 5.57651061647793e-05
18.6600000000001 4.94834766094033e-05
18.6700000000001 4.31682925523729e-05
18.6800000000001 3.68276877087872e-05
18.6900000000001 3.04697903626926e-05
18.7000000000001 2.41027131243682e-05
18.7100000000001 1.77345427541268e-05
18.7200000000001 1.13733300652775e-05
18.7300000000001 5.02707991871055e-06
18.7400000000001 -1.29625867859369e-06
18.7500000000001 -7.58880235908147e-06
18.7600000000001 -1.38427430522601e-05
18.7700000000001 -2.00503575028455e-05
18.7800000000001 -2.6204016637964e-05
18.7900000000001 -3.2296194772269e-05
18.8000000000001 -3.83194786400729e-05
18.8100000000001 -4.42665762438365e-05
18.8200000000001 -5.01303255086962e-05
18.8300000000001 -5.59037027329634e-05
18.8400000000001 -6.15798308248272e-05
18.8500000000001 -6.71519873158649e-05
18.8600000000001 -7.26136121422397e-05
18.8700000000002 -7.79583151848039e-05
18.8800000000002 -8.31798835597204e-05
18.8900000000002 -8.82722886514114e-05
18.9000000000002 -9.32296928800033e-05
18.9100000000002 -9.80464561976024e-05
18.9200000000002 -0.000102717142303262
18.9300000000002 -0.000107236524572534
18.9400000000002 -0.000111599591694729
18.9500000000002 -0.000115801553012139
18.9600000000002 -0.000119837843555837
18.9700000000002 -0.000123704128773143
18.9800000000002 -0.000127396308942164
18.9900000000002 -0.000130910523269343
19.0000000000002 -0.000134243153666284
19.0100000000002 -0.000137390828202604
19.0200000000002 -0.000140350424231978
19.0300000000002 -0.000143119071188961
19.0400000000002 -0.000145694153054638
19.0500000000002 -0.000148073310489561
19.0600000000002 -0.000150254442632907
19.0700000000002 -0.000152235708567194
19.0800000000002 -0.000154015528451632
19.0900000000002 -0.000155592584340834
19.1000000000002 -0.000156965820588485
19.1100000000002 -0.000158134444009021
19.1200000000002 -0.000159097923676843
19.1300000000002 -0.000159855990397105
19.1400000000002 -0.000160408635850924
19.1500000000002 -0.000160756111418443
19.1600000000002 -0.000160898926683977
19.1700000000002 -0.00016083784762836
19.1800000000002 -0.000160573894514819
19.1900000000002 -0.000160108339476392
19.2000000000002 -0.000159442703815391
19.2100000000002 -0.000158578755029436
19.2200000000002 -0.00015751850358545
19.2300000000002 -0.000156264199475532
19.2400000000002 -0.00015481832861344
19.2500000000002 -0.000153183609183532
19.2600000000002 -0.000151362988180156
19.2700000000002 -0.000149359638716983
19.2800000000002 -0.000147176928154368
19.2900000000002 -0.000144818393089039
19.3000000000002 -0.00014228782349463
19.3100000000002 -0.000139589209440275
19.3200000000002 -0.000136726735087831
19.3300000000002 -0.000133704772476738
19.3400000000002 -0.000130527875104718
19.3500000000002 -0.000127200771312677
19.3600000000002 -0.000123728352267202
19.3700000000002 -0.000120115641757937
19.3800000000002 -0.000116367859039402
19.3900000000002 -0.000112490367714009
19.4000000000002 -0.000108488669161309
19.4100000000002 -0.000104368395420686
19.4200000000002 -0.000100135301744775
19.4300000000002 -9.57952589267316e-05
19.4400000000002 -9.13542454574073e-05
19.4500000000002 -8.68183395470082e-05
19.4600000000002 -8.21937110352689e-05
19.4700000000002 -7.74866132087374e-05
19.4800000000002 -7.27033745406893e-05
19.4900000000002 -6.78503903674753e-05
19.5000000000002 -6.29341145140939e-05
19.5100000000003 -5.79610508811128e-05
19.5200000000003 -5.29377450046468e-05
19.5300000000003 -4.78707756008114e-05
19.5400000000003 -4.27667461058578e-05
19.5500000000003 -3.76322762230453e-05
19.5600000000003 -3.24739934871047e-05
19.5700000000003 -2.72985248570876e-05
19.5800000000003 -2.2112488348235e-05
19.5900000000003 -1.69224847133218e-05
19.6000000000003 -1.17350891839099e-05
19.6100000000003 -6.55684328169103e-06
19.6200000000003 -1.39424670998765e-06
19.6300000000003 3.74625066460602e-06
19.6400000000003 8.85825673076206e-06
19.6500000000003 1.39354449415323e-05
19.6600000000003 1.89715620046384e-05
19.6700000000003 2.39604354447592e-05
19.6800000000003 2.88959810288575e-05
19.6900000000003 3.37722100457278e-05
19.7000000000003 3.8583236431284e-05
19.7100000000003 4.33232837312533e-05
19.7200000000003 4.79866918932162e-05
19.7300000000003 5.25679238802275e-05
19.7400000000003 5.70615720984346e-05
19.7500000000003 6.14623646314506e-05
19.7600000000003 6.57651712745016e-05
19.7700000000003 6.99650093616179e-05
19.7800000000003 7.40570493792086e-05
19.7900000000003 7.80366203609441e-05
19.8000000000003 8.18992150568266e-05
19.8100000000003 8.56404948714334e-05
19.8200000000003 8.92562945663838e-05
19.8300000000003 9.27426267220491e-05
19.8400000000003 9.60956859539733e-05
19.8500000000003 9.93118528798297e-05
19.8600000000003 0.000102387697833055
19.8700000000003 0.000105319984319658
19.8800000000003 0.000108105672215033
19.8900000000003 0.000110741920697964
19.9000000000003 0.000113226090919349
19.9100000000003 0.000115555748403524
19.9200000000003 0.000117728665180436
19.9300000000003 0.000119742821647247
19.9400000000003 0.000121596408158326
19.9500000000003 0.000123287826342956
19.9600000000003 0.000124815690150421
19.9700000000003 0.000126178826635927
19.9800000000003 0.000127376276441049
19.9900000000003 0.000128407294020502
20.0000000000003 0.000129271347609733
20.0100000000003 0.000129968118915459
20.0200000000003 0.000130497502539809
20.0300000000003 0.000130859605140613
20.0400000000003 0.000131054744330981
20.0500000000003 0.000131083447321949
20.0600000000003 0.000130946449312838
20.0700000000003 0.00013064469163511
20.0800000000003 0.000130179319657174
20.0900000000003 0.000129551680460116
20.1000000000003 0.000128763320298632
20.1100000000003 0.000127815981868879
20.1200000000003 0.00012671160141933
20.1300000000003 0.00012545230577004
20.1400000000003 0.000124040409372119
20.1500000000004 0.000122478411706982
20.1600000000004 0.00012076899583059
20.1700000000004 0.000118914957059857
20.1800000000004 0.000116919298089228
20.1900000000004 0.000114785194368903
20.2000000000004 0.000112515981730273
20.2100000000004 0.00011011515140308
20.2200000000004 0.000107586344862811
20.2300000000004 0.000104933348515155
20.2400000000004 0.000102160088224597
20.2500000000004 9.92706030505466e-05
20.2600000000004 9.62690738663514e-05
20.2700000000004 9.31598159163614e-05
20.2800000000004 8.99472592213503e-05
20.2900000000004 8.66359429546961e-05
20.3000000000004 8.32305094854302e-05
20.3100000000004 7.97356982050602e-05
20.3200000000004 7.61563391978357e-05
20.3300000000004 7.24973467891465e-05
20.3400000000004 6.8763712994912e-05
20.3500000000004 6.49605008888137e-05
20.3600000000004 6.10928379009685e-05
20.3700000000004 5.71659090598865e-05
20.3800000000004 5.31849501884116e-05
20.3900000000004 4.9155241063771e-05
20.4000000000004 4.50820985513942e-05
20.4100000000004 4.09708697219106e-05
20.4200000000004 3.68269249605107e-05
20.4300000000004 3.26556510777443e-05
20.4400000000004 2.84624444306614e-05
20.4500000000004 2.42527040631095e-05
20.4600000000004 2.00318248739074e-05
20.4700000000004 1.58051908214778e-05
20.4800000000004 1.15781681734561e-05
20.4900000000004 7.35609880962562e-06
20.5000000000004 3.14429358648295e-06
20.5100000000004 -1.05197422846776e-06
20.5200000000004 -5.22747544465327e-06
20.5300000000004 -9.37703135077591e-06
20.5400000000004 -1.34955200238029e-05
20.5500000000004 -1.75778825370837e-05
20.5600000000004 -2.16191290601854e-05
20.5700000000004 -2.56143448432141e-05
20.5800000000004 -2.95586960786142e-05
20.5900000000004 -3.34474356335764e-05
20.6000000000004 -3.72759086464126e-05
20.6100000000004 -4.10395579804268e-05
20.6200000000004 -4.47339295290881e-05
20.6300000000004 -4.83546773664461e-05
20.6400000000004 -5.18975687370174e-05
20.6500000000004 -5.53584888795656e-05
20.6600000000004 -5.87334456794098e-05
20.6700000000004 -6.20185741441715e-05
20.6800000000004 -6.5210140698665e-05
20.6900000000004 -6.83045472932003e-05
20.7000000000004 -7.12983353218802e-05
20.7100000000004 -7.41881893465561e-05
20.7200000000004 -7.69709406226639e-05
20.7300000000004 -7.96435704233895e-05
20.7400000000004 -8.22032131589164e-05
20.7500000000004 -8.46471592877675e-05
20.7600000000004 -8.69728580175386e-05
20.7700000000004 -8.91779197925979e-05
20.7800000000004 -9.12601185666183e-05
20.7900000000005 -9.32173938580867e-05
20.8000000000005 -9.50478525872183e-05
20.8100000000005 -9.67497706930128e-05
20.8200000000005 -9.83215945294566e-05
20.8300000000005 -9.97619420401835e-05
20.8400000000005 -0.000101069603711206
20.8500000000005 -0.000102243543303558
20.8600000000005 -0.000103282898369871
20.8700000000005 -0.000104186980519464
20.8800000000005 -0.000104955275498085
20.8900000000005 -0.000105587443035735
20.9000000000005 -0.000106083316476693
20.9100000000005 -0.000106442902193651
20.9200000000005 -0.000106666378788252
20.9300000000005 -0.000106754096080865
20.9400000000005 -0.000106706573892981
20.9500000000005 -0.000106524500626467
20.9600000000005 -0.000106208731644965
20.9700000000005 -0.000105760287464424
20.9800000000005 -0.00010518035176237
20.9900000000005 -0.000104470269220083
21.0000000000005 -0.000103631543220188
21.0100000000005 -0.000102665833438682
21.0200000000005 -0.000101574953405998
21.0300000000005 -0.000100360868196547
21.0400000000005 -9.90256926380645e-05
21.0500000000005 -9.75716663139729e-05
21.0600000000005 -9.60011486093959e-05
21.0700000000005 -9.43166629649124e-05
21.0800000000005 -9.2520865232392e-05
21.0900000000005 -9.06165396848177e-05
21.1000000000005 -8.86065948850087e-05
21.1100000000005 -8.64940594188199e-05
21.1200000000005 -8.4282077498561e-05
21.1300000000005 -8.19739011058929e-05
21.1400000000005 -7.95728707487818e-05
21.1500000000005 -7.70824549356678e-05
21.1600000000005 -7.45062176104688e-05
21.1700000000005 -7.18478137583454e-05
21.1800000000005 -6.91109846634631e-05
21.1900000000005 -6.62995529552357e-05
21.2000000000005 -6.34174175083484e-05
21.2100000000005 -6.04685482322799e-05
21.2200000000005 -5.74569807725449e-05
21.2300000000005 -5.43868111392434e-05
21.2400000000005 -5.12621902749997e-05
21.2500000000005 -4.80873185725023e-05
21.2600000000005 -4.48664403507014e-05
21.2700000000005 -4.160383829811e-05
21.2800000000005 -3.83038278912207e-05
21.2900000000005 -3.49707517957707e-05
21.3000000000005 -3.16089742584276e-05
21.3100000000005 -2.82228754963073e-05
21.3200000000005 -2.48168460916246e-05
21.3300000000005 -2.13952813987178e-05
21.3400000000005 -1.79625759705383e-05
21.3500000000005 -1.45231180116801e-05
21.3600000000005 -1.10812838648813e-05
21.3700000000005 -7.64143253786369e-06
21.3800000000005 -4.20790027730833e-06
21.3900000000005 -7.84995196592352e-07
21.4000000000005 2.62300803609103e-06
21.4100000000005 6.01187344282973e-06
21.4200000000005 9.37740890792652e-06
21.4300000000006 1.2715471270484e-05
21.4400000000006 1.60219713315524e-05
21.4500000000006 1.92928787699004e-05
21.4600000000006 2.25242269606146e-05
21.4700000000006 2.57121176908737e-05
21.4800000000006 2.88527257674183e-05
21.4900000000006 3.19423035103932e-05
21.5000000000006 3.49771851283972e-05
21.5100000000006 3.79537909697733e-05
21.5200000000006 4.08686316453183e-05
21.5300000000006 4.37183120178292e-05
21.5400000000006 4.64995350540249e-05
21.5500000000006 4.92091055345228e-05
21.5600000000006 5.18439336182959e-05
21.5700000000006 5.44010382571603e-05
21.5800000000006 5.68775504569524e-05
21.5900000000006 5.92707163820504e-05
21.6000000000006 6.15779002999784e-05
21.6100000000006 6.37965873631407e-05
21.6200000000006 6.59243862249097e-05
21.6300000000006 6.79590314875613e-05
21.6400000000006 6.98983859797373e-05
21.6500000000006 7.17404428613579e-05
21.6600000000006 7.34833275541578e-05
21.6700000000006 7.51252994962079e-05
21.6800000000006 7.66647537190542e-05
21.6900000000006 7.81002222463325e-05
21.7000000000006 7.94303753129359e-05
21.7100000000006 8.06540224040784e-05
21.7200000000006 8.17701131138148e-05
21.7300000000006 8.27777378228264e-05
21.7400000000006 8.36761282013339e-05
21.7500000000006 8.44646575166804e-05
21.7600000000006 8.51428407703321e-05
21.7700000000006 8.57103346589912e-05
21.7800000000006 8.61669373546549e-05
21.7900000000006 8.65125881081667e-05
21.8000000000006 8.67473666779714e-05
21.8100000000006 8.68714925861559e-05
21.8200000000006 8.68853242042899e-05
21.8300000000006 8.67893576721433e-05
21.8400000000006 8.65842256531007e-05
21.8500000000006 8.62706959311807e-05
21.8600000000006 8.58496698562385e-05
21.8700000000006 8.53221806467123e-05
21.8800000000006 8.46893915642014e-05
21.8900000000006 8.39525939835697e-05
21.9000000000006 8.31132054016673e-05
21.9100000000006 8.21727674717092e-05
21.9200000000006 8.11329442621533e-05
21.9300000000006 7.99955212757008e-05
21.9400000000006 7.87623557912966e-05
21.9500000000006 7.74354478294771e-05
21.9600000000006 7.60169088340193e-05
21.9700000000006 7.45089562808815e-05
21.9800000000006 7.29139103697051e-05
21.9900000000006 7.12341906026556e-05
22.0000000000006 6.94723122552675e-05
22.0100000000006 6.76308827440781e-05
22.0200000000006 6.57125850096143e-05
22.0300000000006 6.37201950355791e-05
22.0400000000006 6.16565772344265e-05
22.0500000000006 5.95246718243497e-05
22.0600000000006 5.73274910800903e-05
22.0700000000007 5.50681153711864e-05
22.0800000000007 5.27496890611523e-05
22.0900000000007 5.03754163053735e-05
22.1000000000007 4.79485567698565e-05
22.1100000000007 4.54724212855308e-05
22.1200000000007 4.29503674490356e-05
22.1300000000007 4.0385795178864e-05
22.1400000000007 3.77821422346161e-05
22.1500000000007 3.51428797064144e-05
22.1600000000007 3.2471507481143e-05
22.1700000000007 2.97715496919006e-05
22.1800000000007 2.70465501568674e-05
22.1900000000007 2.43000678136831e-05
22.2000000000007 2.15356721553101e-05
22.2100000000007 1.87569386733013e-05
22.2200000000007 1.59674443142796e-05
22.2300000000007 1.31707629554105e-05
22.2400000000007 1.0370460904547e-05
22.2500000000007 7.57009243067744e-06
22.2600000000007 4.7731953302047e-06
22.2700000000007 1.98328653453727e-06
22.2800000000007 -7.96142235635814e-07
22.2900000000007 -3.56162876407744e-06
22.3000000000007 -6.30974457556375e-06
22.3100000000007 -9.03709911015586e-06
22.3200000000007 -1.1740343830197e-05
22.3300000000007 -1.44161762551376e-05
22.3400000000007 -1.70613439194349e-05
22.3500000000007 -1.96726482488635e-05
22.3600000000007 -2.22469483507113e-05
22.3700000000007 -2.47811647134905e-05
22.3800000000007 -2.72722828118722e-05
22.3900000000007 -2.97173566127519e-05
22.4000000000007 -3.21135119784738e-05
22.4100000000007 -3.44579499633678e-05
22.4200000000007 -3.67479499999587e-05
22.4300000000007 -3.89808729712841e-05
22.4400000000007 -4.11541641660179e-05
22.4500000000007 -4.32653561133567e-05
22.4600000000007 -4.53120712942467e-05
22.4700000000007 -4.72920247264539e-05
22.4800000000007 -4.9203026420677e-05
22.4900000000007 -5.10429837052253e-05
22.5000000000007 -5.28099034169316e-05
22.5100000000007 -5.45018939561547e-05
22.5200000000007 -5.61171672039267e-05
22.5300000000007 -5.76540402994549e-05
22.5400000000007 -5.91109372764064e-05
22.5500000000007 -6.04863905565673e-05
22.5600000000007 -6.17790422996695e-05
22.5700000000007 -6.29876456083685e-05
22.5800000000007 -6.411106558754e-05
22.5900000000007 -6.51482802572667e-05
22.6000000000007 -6.60983813190667e-05
22.6100000000007 -6.6960574775127e-05
22.6200000000007 -6.77341814014578e-05
22.6300000000007 -6.84186370757264e-05
22.6400000000007 -6.90134929475434e-05
22.6500000000007 -6.95184154761529e-05
22.6600000000007 -6.99331863180533e-05
22.6700000000007 -7.02577020704497e-05
22.6800000000007 -7.04919738717959e-05
22.6900000000007 -7.06361268609659e-05
22.7000000000007 -7.06903994969143e-05
22.7100000000008 -7.06551427410792e-05
22.7200000000008 -7.05308191052976e-05
22.7300000000008 -7.03180015687175e-05
22.7400000000008 -7.00173723682665e-05
22.7500000000008 -6.96297216689489e-05
22.7600000000008 -6.91559461231921e-05
22.7700000000008 -6.85970473338801e-05
22.7800000000008 -6.79541302464549e-05
22.7900000000008 -6.72284015186788e-05
22.8000000000008 -6.64211679722474e-05
22.8100000000008 -6.5533835383148e-05
22.8200000000008 -6.45678901440114e-05
22.8300000000008 -6.35249022244104e-05
22.8400000000008 -6.24065477188689e-05
22.8500000000008 -6.1214590189106e-05
22.8600000000008 -5.99508780177353e-05
22.8700000000008 -5.86173416683436e-05
22.8800000000008 -5.72159908557677e-05
22.8900000000008 -5.57489116304568e-05
22.9000000000008 -5.42182614754661e-05
22.9100000000008 -5.26262565984069e-05
22.9200000000008 -5.0975196893725e-05
22.9300000000008 -4.926744525715e-05
22.9400000000008 -4.7505424652117e-05
22.9500000000008 -4.56916149557544e-05
22.9600000000008 -4.38285496697039e-05
22.9700000000008 -4.19188125368166e-05
22.9800000000008 -3.99650340863492e-05
22.9900000000008 -3.79698881218659e-05
23.0000000000008 -3.59360881618635e-05
23.0100000000008 -3.3866383840979e-05
23.0200000000008 -3.17635572784265e-05
23.0300000000008 -2.96304194196112e-05
23.0400000000008 -2.74698063564816e-05
23.0500000000008 -2.52845756318707e-05
23.0600000000008 -2.30776025329597e-05
23.0700000000008 -2.08517763788431e-05
23.0800000000008 -1.86099968070895e-05
23.0900000000008 -1.63551700641453e-05
23.1000000000008 -1.40902053043271e-05
23.1100000000008 -1.18180109021351e-05
23.1200000000008 -9.54149078251891e-06
23.1300000000008 -7.26354077371265e-06
23.1400000000008 -4.98704498716448e-06
23.1500000000008 -2.71487222903449e-06
23.1600000000008 -4.49872447678713e-07
23.1700000000008 1.80512677855003e-06
23.1800000000008 4.04732370892192e-06
23.1900000000008 6.2739458591806e-06
23.2000000000008 8.48225336751851e-06
23.2100000000008 1.06695423036729e-05
23.2200000000008 1.28331479172205e-05
23.2300000000008 1.4970447821225e-05
23.2400000000008 1.70788651075361e-05
23.2500000000008 1.91558713901065e-05
23.2600000000008 2.11989897728137e-05
23.2700000000008 2.32057977383935e-05
23.2800000000008 2.51739299551842e-05
23.2900000000008 2.71010809985438e-05
23.3000000000008 2.89850079838719e-05
23.3100000000008 3.08235331083329e-05
23.3200000000008 3.26145460984521e-05
23.3300000000008 3.43560065611108e-05
23.3400000000008 3.60459462351641e-05
23.3500000000009 3.76824711414296e-05
23.3600000000009 3.92637636288134e-05
23.3700000000009 4.07880843144634e-05
23.3800000000009 4.22537739159866e-05
23.3900000000009 4.36592549739275e-05
23.4000000000009 4.50030334628518e-05
23.4100000000009 4.62837002895126e-05
23.4200000000009 4.74999326767532e-05
23.4300000000009 4.86504954319341e-05
23.4400000000009 4.97342420988221e-05
23.4500000000009 5.07501159920582e-05
23.4600000000009 5.16971511134444e-05
23.4700000000009 5.25744729494744e-05
23.4800000000009 5.33812991496655e-05
23.4900000000009 5.41169400854305e-05
23.5000000000009 5.47807992893692e-05
23.5100000000009 5.53723737774166e-05
23.5200000000009 5.58912542453877e-05
23.5300000000009 5.63371251508008e-05
23.5400000000009 5.67097646771683e-05
23.5500000000009 5.70090445794811e-05
23.5600000000009 5.72349299129692e-05
23.5700000000009 5.73874786462688e-05
23.5800000000009 5.74668411603751e-05
23.5900000000009 5.74732596350353e-05
23.6000000000009 5.74070673245997e-05
23.6100000000009 5.72686877258294e-05
23.6200000000009 5.70586336408511e-05
23.6300000000009 5.67775061395265e-05
23.6400000000009 5.64259934272772e-05
23.6500000000009 5.60048696275686e-05
23.6600000000009 5.55149934942817e-05
23.6700000000009 5.49573070816739e-05
23.6800000000009 5.43328344279456e-05
23.6900000000009 5.36426803807545e-05
23.7000000000009 5.28880298915497e-05
23.7100000000009 5.20701147049026e-05
23.7200000000009 5.11902632074949e-05
23.7300000000009 5.02498758723431e-05
23.7400000000009 4.92504231449364e-05
23.7500000000009 4.8193443251611e-05
23.7600000000009 4.70805399332916e-05
23.7700000000009 4.59133801077676e-05
23.7800000000009 4.46936914637345e-05
23.7900000000009 4.34232522413e-05
23.8000000000009 4.21039012498725e-05
23.8100000000009 4.0737535485334e-05
23.8200000000009 3.93261019316516e-05
23.8300000000009 3.78715950659236e-05
23.8400000000009 3.63760542285962e-05
23.8500000000009 3.48415609047271e-05
23.8600000000009 3.3270235940022e-05
23.8700000000009 3.16642367056826e-05
23.8800000000009 3.00257542214526e-05
23.8900000000009 2.83570102439055e-05
23.9000000000009 2.66602543257492e-05
23.9100000000009 2.49377608511819e-05
23.9200000000009 2.31918260519505e-05
23.9300000000009 2.14247650084737e-05
23.9400000000009 1.96389086402312e-05
23.9500000000009 1.78366006895217e-05
23.9600000000009 1.60201947026033e-05
23.9700000000009 1.41920510121503e-05
23.9800000000009 1.23545337249426e-05
23.990000000001 1.05100077186229e-05
24.000000000001 8.66083565132887e-06
24.010000000001 6.80937498796397e-06
24.020000000001 4.95797504680475e-06
24.030000000001 3.10897407011076e-06
24.040000000001 1.26469632235122e-06
24.050000000001 -5.72550780409081e-07
24.060000000001 -2.40047950646862e-06
24.070000000001 -4.21682460635522e-06
24.080000000001 -6.01934606924075e-06
24.090000000001 -7.80583183460439e-06
24.100000000001 -9.57410045592641e-06
24.110000000001 -1.13220037132653e-05
24.120000000001 -1.3047429171643e-05
24.130000000001 -1.47483026822741e-05
24.140000000001 -1.64225908237211e-05
24.150000000001 -1.80683032801953e-05
24.160000000001 -1.96834951542652e-05
24.170000000001 -2.12662692113701e-05
24.180000000001 -2.28147780536249e-05
24.190000000001 -2.43272262204852e-05
24.200000000001 -2.58018722139576e-05
24.210000000001 -2.72370304461721e-05
24.220000000001 -2.86310731072555e-05
24.230000000001 -2.99824319513875e-05
24.240000000001 -3.12895999992838e-05
24.250000000001 -3.25511331553145e-05
24.260000000001 -3.37656517376163e-05
24.270000000001 -3.49318419196689e-05
24.280000000001 -3.60484570819389e-05
24.290000000001 -3.71143190723079e-05
24.300000000001 -3.81283193741141e-05
24.310000000001 -3.90894201807834e-05
24.320000000001 -3.99966553761241e-05
24.330000000001 -4.08491314195046e-05
24.340000000001 -4.16460281352481e-05
24.350000000001 -4.23865994057079e-05
24.360000000001 -4.30701737676165e-05
24.370000000001 -4.36961549114247e-05
24.380000000001 -4.4264022083482e-05
24.390000000001 -4.47733303914561e-05
24.400000000001 -4.522371101311e-05
24.410000000001 -4.56148713042384e-05
24.420000000001 -4.59465948151829e-05
24.430000000001 -4.62187412095696e-05
24.440000000001 -4.6431246087799e-05
24.450000000001 -4.65841207161222e-05
24.460000000001 -4.66774516623196e-05
24.470000000001 -4.67114003392022e-05
24.480000000001 -4.66862024574113e-05
24.490000000001 -4.66021673893195e-05
24.500000000001 -4.64596774462918e-05
24.510000000001 -4.62591870722449e-05
24.520000000001 -4.60012219575265e-05
24.530000000001 -4.56863780790006e-05
24.540000000001 -4.53153206756444e-05
24.550000000001 -4.4888783175766e-05
24.560000000001 -4.44075661066579e-05
24.570000000001 -4.38725360528446e-05
24.580000000001 -4.32846248264104e-05
24.590000000001 -4.26448167436372e-05
24.600000000001 -4.19541535838113e-05
24.610000000001 -4.1213745737062e-05
24.620000000001 -4.04247612620166e-05
24.6300000000011 -3.95884241350247e-05
24.6400000000011 -3.8706012437387e-05
24.6500000000011 -3.77788564831793e-05
24.6600000000011 -3.68083368903045e-05
24.6700000000011 -3.5795881663386e-05
24.6800000000011 -3.4742957569723e-05
24.6900000000011 -3.36510859637769e-05
24.7000000000011 -3.25218297225631e-05
24.7100000000011 -3.13567912905303e-05
24.7200000000011 -3.01576105858193e-05
24.7300000000011 -2.892596282071e-05
24.7400000000011 -2.76635562618266e-05
24.7500000000011 -2.63721299443056e-05
24.7600000000011 -2.50534513489194e-05
24.7700000000011 -2.37093140485733e-05
24.7800000000011 -2.23415353292354e-05
24.7900000000011 -2.09519537896207e-05
24.8000000000011 -1.9542426923521e-05
24.8100000000011 -1.81148286884022e-05
24.8200000000011 -1.66710470637414e-05
24.8300000000011 -1.52129816024689e-05
24.8400000000011 -1.37425409787864e-05
24.8500000000011 -1.22616405356056e-05
24.8600000000011 -1.07721998347781e-05
24.8700000000011 -9.27614021326977e-06
24.8800000000011 -7.77538234837729e-06
24.8900000000011 -6.27184383506336e-06
24.9000000000011 -4.76743677843157e-06
24.9100000000011 -3.26406540434496e-06
24.9200000000011 -1.76362369112895e-06
24.9300000000011 -2.67993025271711e-07
24.9400000000011 1.22096011602312e-06
24.9500000000011 2.70138645261679e-06
24.9600000000011 4.17145615678775e-06
24.9700000000011 5.62936107346507e-06
24.9800000000011 7.07331690267754e-06
24.9900000000011 8.50156534165248e-06
25.0000000000011 9.91237618403271e-06
25.0100000000011 1.13040493737646e-05
25.0200000000011 1.26749170112669e-05
25.0300000000011 1.40233453095669e-05
25.0400000000011 1.5347736498173e-05
25.0500000000011 1.66465306725147e-05
25.0600000000011 1.79182075868694e-05
25.0700000000011 1.91612883887774e-05
25.0800000000011 2.03743372930265e-05
25.0900000000011 2.15559631933534e-05
25.1000000000011 2.27048212101968e-05
25.1100000000011 2.38196141727402e-05
25.1200000000011 2.48990940337328e-05
25.1300000000011 2.59420632156219e-05
25.1400000000011 2.69473758866114e-05
25.1500000000011 2.79139391653664e-05
25.1600000000011 2.88407142531739e-05
25.1700000000011 2.97267174924788e-05
25.1800000000011 3.0571021350803e-05
25.1900000000011 3.13727553291594e-05
25.2000000000011 3.21311067941677e-05
25.2100000000011 3.28453217331958e-05
25.2200000000011 3.35147054319285e-05
25.2300000000011 3.413862307389e-05
25.2400000000011 3.47165002615357e-05
25.2500000000011 3.52478234586378e-05
25.2600000000011 3.57321403537937e-05
25.2700000000012 3.61690601449881e-05
25.2800000000012 3.65582537461892e-05
25.2900000000012 3.68994539126823e-05
25.3000000000012 3.71924552896483e-05
25.3100000000012 3.74371143828517e-05
25.3200000000012 3.7633349451258e-05
25.3300000000012 3.7781140322616e-05
25.3400000000012 3.78805281327509e-05
25.3500000000012 3.79316149894702e-05
25.3600000000012 3.79345635621639e-05
25.3700000000012 3.78895965984089e-05
25.3800000000012 3.77969963691907e-05
25.3900000000012 3.76571040447885e-05
25.4000000000012 3.74703190040464e-05
25.4100000000012 3.72370980808538e-05
25.4200000000012 3.69579547536278e-05
25.4300000000012 3.66334582873483e-05
25.4400000000012 3.62642328454619e-05
25.4500000000012 3.58509566066522e-05
25.4600000000012 3.53943609667212e-05
25.4700000000012 3.48952293261143e-05
25.4800000000012 3.43543782925067e-05
25.4900000000012 3.37726867987779e-05
25.5000000000012 3.31510812599964e-05
25.5100000000012 3.24905341767361e-05
25.5200000000012 3.17920626869176e-05
25.5300000000012 3.10567270683057e-05
25.5400000000012 3.02856291938056e-05
25.5500000000012 2.94799109417356e-05
25.5600000000012 2.86407481005868e-05
25.5700000000012 2.77693556180423e-05
25.5800000000012 2.68669866892219e-05
25.5900000000012 2.59349274173213e-05
25.6000000000012 2.49744951563192e-05
25.6100000000012 2.39870367688426e-05
25.6200000000012 2.2973926827526e-05
25.6300000000012 2.19365657746167e-05
25.6400000000012 2.08763780486238e-05
25.6500000000012 1.97948101839697e-05
25.6600000000012 1.86933288881441e-05
25.6700000000012 1.75734191000872e-05
25.6800000000012 1.64365820330895e-05
25.6900000000012 1.52843332052192e-05
25.7000000000012 1.41182004601541e-05
25.7100000000012 1.29397219811653e-05
25.7200000000012 1.1750444300952e-05
25.7300000000012 1.05519203099634e-05
25.7400000000012 9.3457072658043e-06
25.7500000000012 8.13336480629288e-06
25.7600000000012 6.91645296870816e-06
25.7700000000012 5.69653021772226e-06
25.7800000000012 4.47515148450442e-06
25.7900000000012 3.25386621943267e-06
25.8000000000012 2.03421646082378e-06
25.8100000000012 8.17734922062148e-07
25.8200000000012 -3.94056900537315e-07
25.8300000000012 -1.59965059682669e-06
25.8400000000012 -2.79755268662753e-06
25.8500000000012 -3.98628643602262e-06
25.8600000000012 -5.16439364396542e-06
25.8700000000012 -6.33043639708662e-06
25.8800000000012 -7.48299879062892e-06
25.8900000000012 -8.62068861349e-06
25.9000000000012 -9.7421389954147e-06
25.9100000000013 -1.0846010014429e-05
25.9200000000013 -1.19309902626715e-05
25.9300000000013 -1.29957983688374e-05
25.9400000000013 -1.40391844755206e-05
25.9500000000013 -1.50599316697885e-05
25.9600000000013 -1.60568573654127e-05
25.9700000000013 -1.70288146352216e-05
25.9800000000013 -1.79746934921444e-05
25.9900000000013 -1.88934221175779e-05
26.0000000000013 -1.97839680357235e-05
26.0100000000013 -2.06453392327123e-05
26.0200000000013 -2.14765852193489e-05
26.0300000000013 -2.22767980363991e-05
26.0400000000013 -2.30451132014242e-05
26.0500000000013 -2.3780710596241e-05
26.0600000000013 -2.44828152941697e-05
26.0700000000013 -2.51506983263054e-05
26.0800000000013 -2.57836773861452e-05
26.0900000000013 -2.63811174719663e-05
26.1000000000013 -2.69424314664446e-05
26.1100000000013 -2.74670806530842e-05
26.1200000000013 -2.79545751691103e-05
26.1300000000013 -2.84044743945592e-05
26.1400000000013 -2.88163872773918e-05
26.1500000000013 -2.91899725945305e-05
26.1600000000013 -2.9524939148981e-05
26.1700000000013 -2.98210459030818e-05
26.1800000000013 -3.00781020464352e-05
26.1900000000013 -3.02959670021303e-05
26.2000000000013 -3.04745503690303e-05
26.2100000000013 -3.06138118012682e-05
26.2200000000013 -3.07137608254962e-05
26.2300000000013 -3.07744565965562e-05
26.2400000000013 -3.07960075923659e-05
26.2500000000013 -3.07785712489741e-05
26.2600000000013 -3.07223535369494e-05
26.2700000000013 -3.0627608480543e-05
26.2800000000013 -3.04946376214909e-05
26.2900000000013 -3.03237894299854e-05
26.3000000000013 -3.01154586664875e-05
26.3100000000013 -2.98700857001482e-05
26.3200000000013 -2.95881557937745e-05
26.3300000000013 -2.92701983742831e-05
26.3400000000013 -2.89167863292752e-05
26.3500000000013 -2.85285354301354e-05
26.3600000000013 -2.81060959500336e-05
26.3700000000013 -2.76501571099855e-05
26.3800000000013 -2.71614524709291e-05
26.3900000000013 -2.66407535380735e-05
26.4000000000013 -2.60888686055846e-05
26.4100000000013 -2.55066415603027e-05
26.4200000000013 -2.48949506462513e-05
26.4300000000013 -2.42547071917082e-05
26.4400000000013 -2.35868539572344e-05
26.4500000000013 -2.2892359210194e-05
26.4600000000013 -2.21722267079432e-05
26.4700000000013 -2.1427487527148e-05
26.4800000000013 -2.06591987629373e-05
26.4900000000013 -1.98684421417323e-05
26.5000000000013 -1.90563225800553e-05
26.5100000000013 -1.82239667050547e-05
26.5200000000013 -1.73725213455527e-05
26.5300000000013 -1.6503151999264e-05
26.5400000000013 -1.56170412802443e-05
26.5500000000014 -1.47153873498186e-05
26.5600000000014 -1.3799402333777e-05
26.5700000000014 -1.28703107283612e-05
26.5800000000014 -1.19293477974161e-05
26.5900000000014 -1.09777579629712e-05
26.6000000000014 -1.00167931914557e-05
26.6100000000014 -9.04771137770619e-06
26.6200000000014 -8.07177472888561e-06
26.6300000000014 -7.09024815040705e-06
26.6400000000014 -6.1043976359314e-06
26.6500000000014 -5.11548866347856e-06
26.6600000000014 -4.12478459967352e-06
26.6700000000014 -3.13354511411954e-06
26.6800000000014 -2.14302460586949e-06
26.6900000000014 -1.15447064393352e-06
26.7000000000014 -1.69122423735185e-07
26.7100000000014 8.11790758599977e-07
26.7200000000014 1.78705101227601e-06
26.7300000000014 2.75545333636442e-06
26.7400000000014 3.71580708107295e-06
26.7500000000014 4.66693738399733e-06
26.7600000000014 5.60768657967062e-06
26.7700000000014 6.53691558073972e-06
26.7800000000014 7.4535052291656e-06
26.7900000000014 8.35635761587395e-06
26.8000000000014 9.24439736734166e-06
26.8100000000014 1.01165728976466e-05
26.8200000000014 1.09718576245579e-05
26.8300000000014 1.18092511483067e-05
26.8400000000014 1.26277803917168e-05
26.8500000000014 1.34265007004387e-05
26.8600000000014 1.42044969020814e-05
26.8700000000014 1.49608843231233e-05
26.8800000000014 1.56948097624771e-05
26.8900000000014 1.64054524207019e-05
26.9000000000014 1.70920247839079e-05
26.9100000000014 1.77537734614361e-05
26.9200000000014 1.83899799764848e-05
26.9300000000014 1.89999615089023e-05
26.9400000000014 1.95830715894347e-05
26.9500000000014 2.01387007447844e-05
26.9600000000014 2.06662770928988e-05
26.9700000000014 2.11652668879726e-05
26.9800000000014 2.1635175014719e-05
26.9900000000014 2.20755454315286e-05
27.0000000000014 2.2485961562201e-05
27.0100000000014 2.28660466360063e-05
27.0200000000014 2.32154639758965e-05
27.0300000000014 2.35339172347598e-05
27.0400000000014 2.38211505796779e-05
27.0500000000014 2.40769488245659e-05
27.0600000000014 2.43011375100033e-05
27.0700000000014 2.44935829320287e-05
27.0800000000014 2.4654192119569e-05
27.0900000000014 2.47829127605797e-05
27.1000000000014 2.48797330774538e-05
27.1100000000014 2.49446816521895e-05
27.1200000000014 2.49778272019006e-05
27.1300000000014 2.49792783053705e-05
27.1400000000014 2.49491830814898e-05
27.1500000000014 2.48877288206067e-05
27.1600000000014 2.47951415700831e-05
27.1700000000014 2.46716856757577e-05
27.1800000000014 2.4517663281687e-05
27.1900000000015 2.43334137917153e-05
27.2000000000015 2.41193132986889e-05
27.2100000000015 2.38757739917966e-05
27.2200000000015 2.36032435631458e-05
27.2300000000015 2.33022046619052e-05
27.2400000000015 2.29731737965376e-05
27.2500000000015 2.26166907569358e-05
27.2600000000015 2.22333353964624e-05
27.2700000000015 2.18237187781588e-05
27.2800000000015 2.13884822543669e-05
27.2900000000015 2.09282965123581e-05
27.3000000000015 2.04438605874166e-05
27.3100000000015 1.99359008448182e-05
27.3200000000015 1.9405169932162e-05
27.3300000000015 1.88524432656273e-05
27.3400000000015 1.82785214356317e-05
27.3500000000015 1.76842301231502e-05
27.3600000000015 1.70704166344416e-05
27.3700000000015 1.6437948802215e-05
27.3800000000015 1.57877138339688e-05
27.3900000000015 1.5120617124741e-05
27.4000000000015 1.44375810433126e-05
27.4100000000015 1.3739543697319e-05
27.4200000000015 1.30274576810007e-05
27.4300000000015 1.23022888084512e-05
27.4400000000015 1.15650148347536e-05
27.4500000000015 1.08166241671255e-05
27.4600000000015 1.00581145680333e-05
27.4700000000015 9.2904918521467e-06
27.4800000000015 8.51476857893873e-06
27.4900000000015 7.73196274268882e-06
27.5000000000015 6.94309646162742e-06
27.5100000000015 6.14919466791811e-06
27.5200000000015 5.35128380017324e-06
27.5300000000015 4.5503905001576e-06
27.5400000000015 3.74754031533185e-06
27.5500000000015 2.94375640885992e-06
27.5600000000015 2.14005827868549e-06
27.5700000000015 1.33746048726333e-06
27.5800000000015 5.36971403503988e-07
27.5900000000015 -2.60408041528344e-07
27.6000000000015 -1.05368558365441e-06
27.6100000000015 -1.8418788378697e-06
27.6200000000015 -2.62401649243245e-06
27.6300000000015 -3.39913948332487e-06
27.6400000000015 -4.16630214768788e-06
27.6500000000015 -4.92457335488299e-06
27.6600000000015 -5.67303761384174e-06
27.6700000000015 -6.41079615542828e-06
27.6800000000015 -7.13696798855473e-06
27.6900000000015 -7.85069092884427e-06
27.7000000000015 -8.55112259866945e-06
27.7100000000015 -9.23744139743798e-06
27.7200000000015 -9.90884744104182e-06
27.7300000000015 -1.05645634694231e-05
27.7400000000015 -1.1203835721268e-05
27.7500000000015 -1.18259347748734e-05
27.7600000000015 -1.24301563542974e-05
27.7700000000015 -1.30158220999147e-05
27.7800000000015 -1.35822803025925e-05
27.7900000000015 -1.41289066007273e-05
27.8000000000015 -1.46551046394376e-05
27.8100000000015 -1.51603066912579e-05
27.8200000000015 -1.56439742377366e-05
27.8300000000016 -1.61055985113868e-05
27.8400000000016 -1.65447009974975e-05
27.8500000000016 -1.69608338953596e-05
27.8600000000016 -1.73535805385212e-05
27.8700000000016 -1.77225557737381e-05
27.8800000000016 -1.80674062983391e-05
27.8900000000016 -1.83878109557822e-05
27.9000000000016 -1.86834809892318e-05
27.9100000000016 -1.89541602530412e-05
27.9200000000016 -1.91996253820845e-05
27.9300000000016 -1.94196859189921e-05
27.9400000000016 -1.96141843993382e-05
27.9500000000016 -1.97829963942838e-05
27.9600000000016 -1.99260305120959e-05
27.9700000000016 -2.00432283578135e-05
27.9800000000016 -2.01345644516122e-05
27.9900000000016 -2.02000461062238e-05
28.0000000000016 -2.02397132638429e-05
28.0100000000016 -2.02536382930342e-05
28.0200000000016 -2.02419257462512e-05
28.0300000000016 -2.02047120787076e-05
28.0400000000016 -2.01421653295085e-05
28.0500000000016 -2.00544847662049e-05
28.0600000000016 -1.99419004943255e-05
28.0700000000016 -1.98046730341231e-05
28.0800000000016 -1.96430928680035e-05
28.0900000000016 -1.94574799645609e-05
28.1000000000016 -1.92481832904479e-05
28.1100000000016 -1.90155803340621e-05
28.1200000000016 -1.87600767001716e-05
28.1300000000016 -1.84821008716957e-05
28.1400000000016 -1.81821073710685e-05
28.1500000000016 -1.78605793624347e-05
28.1600000000016 -1.75180249219366e-05
28.1700000000016 -1.71549762774039e-05
28.1800000000016 -1.6771989021044e-05
28.1900000000016 -1.63696412963155e-05
28.2000000000016 -1.59485329601714e-05
28.2100000000016 -1.55092846936238e-05
28.2200000000016 -1.50525339464821e-05
28.2300000000016 -1.4578941143323e-05
28.2400000000016 -1.40891846502597e-05
28.2500000000016 -1.35839599108393e-05
28.2600000000016 -1.30639785288717e-05
28.2700000000016 -1.2529967317617e-05
28.2800000000016 -1.19826673248445e-05
28.2900000000016 -1.14228328391535e-05
28.3000000000016 -1.08512303810426e-05
28.3100000000016 -1.02686376812806e-05
28.3200000000016 -9.6758426486368e-06
28.3300000000016 -9.07364232876854e-06
28.3400000000016 -8.46284185588958e-06
28.3500000000016 -7.84425339876293e-06
28.3600000000016 -7.21869510249875e-06
28.3700000000016 -6.58699002759247e-06
28.3800000000016 -5.94996508761853e-06
28.3900000000016 -5.30844998696712e-06
28.4000000000016 -4.6632761599966e-06
28.4100000000016 -4.01527571295724e-06
28.4200000000016 -3.36528037002671e-06
28.4300000000016 -2.71412042478201e-06
28.4400000000016 -2.06262369841446e-06
28.4500000000016 -1.41161450598295e-06
28.4600000000016 -7.61912631975985e-07
28.4700000000017 -1.14332316439544e-07
28.4800000000017 5.30318747096969e-07
28.4900000000017 1.17124040068101e-06
28.5000000000017 1.80764099679549e-06
28.5100000000017 2.43873835784781e-06
28.5200000000017 3.06376071914921e-06
28.5300000000017 3.68194765427067e-06
28.5400000000017 4.29255098168621e-06
28.5500000000017 4.89483565165021e-06
28.5600000000017 5.48808061227749e-06
28.5700000000017 6.07157965383091e-06
28.5800000000017 6.64464223025639e-06
28.5900000000017 7.20659425702923e-06
28.6000000000017 7.75677888441974e-06
28.6100000000017 8.29455724531636e-06
28.6200000000017 8.81930917678539e-06
28.6300000000017 9.33043391457086e-06
28.6400000000017 9.82735075980339e-06
28.6500000000017 1.03094997171883e-05
28.6600000000017 1.07763421040135e-05
28.6700000000017 1.12273611293438e-05
28.6800000000017 1.16620624428131e-05
28.6900000000017 1.20799746524649e-05
28.7000000000017 1.24806498111353e-05
28.7100000000017 1.28636638709149e-05
28.7200000000017 1.32286171052656e-05
28.7300000000017 1.35751344984194e-05
28.7400000000017 1.39028661017163e-05
28.7500000000017 1.42114873565986e-05
28.7600000000017 1.45006993840094e-05
28.7700000000017 1.47702292399927e-05
28.7800000000017 1.50198301373388e-05
28.7900000000017 1.52492816331591e-05
28.8000000000017 1.54583897823225e-05
28.8100000000017 1.56469872567278e-05
28.8200000000017 1.58149334305629e-05
28.8300000000017 1.59621144311496e-05
28.8400000000017 1.60884431560602e-05
28.8500000000017 1.61938592564721e-05
28.8600000000017 1.62783290868726e-05
28.8700000000017 1.63418456214337e-05
28.8800000000017 1.63844283373755e-05
28.8900000000017 1.64061230656941e-05
28.9000000000017 1.64070018097016e-05
28.9100000000017 1.63871625319138e-05
28.9200000000017 1.63467289099316e-05
28.9300000000017 1.62858500621195e-05
28.9400000000017 1.62047002441242e-05
28.9500000000017 1.61034785176646e-05
28.9600000000017 1.59824083937104e-05
28.9700000000017 1.58417374534664e-05
28.9800000000017 1.56817369532687e-05
28.9900000000017 1.55027014256037e-05
29.0000000000017 1.53049483040771e-05
29.0100000000017 1.50888171267271e-05
29.0200000000017 1.48546635870465e-05
29.0300000000017 1.46028690362957e-05
29.0400000000017 1.43338352934981e-05
29.0500000000017 1.40479840406133e-05
29.0600000000017 1.37457561953306e-05
29.0700000000017 1.34276112624553e-05
29.0800000000017 1.30940266648507e-05
29.0900000000017 1.27454970549082e-05
29.1000000000017 1.2382532362759e-05
29.1100000000018 1.20056586420108e-05
29.1200000000018 1.16154183529406e-05
29.1300000000018 1.12123681397437e-05
29.1400000000018 1.07970781033525e-05
29.1500000000018 1.03701310418041e-05
29.1600000000018 9.93212166842893e-06
29.1700000000018 9.48365581330474e-06
29.1800000000018 9.02534961129527e-06
29.1900000000018 8.55782867898745e-06
29.2000000000018 8.08172728232115e-06
29.2100000000018 7.59768749642958e-06
29.2200000000018 7.10635835905307e-06
29.2300000000018 6.60839501879647e-06
29.2400000000018 6.10445787944334e-06
29.2500000000018 5.59521174150034e-06
29.2600000000018 5.08132494212529e-06
29.2700000000018 4.56346849456705e-06
29.2800000000018 4.04231522823935e-06
29.2900000000018 3.51853893052268e-06
29.3000000000018 2.99281349139342e-06
29.3100000000018 2.46581205195344e-06
29.3200000000018 1.93820615792627e-06
29.3300000000018 1.41066491917311e-06
29.3400000000018 8.83854176267363e-07
29.3500000000018 3.58435675147811e-07
29.3600000000018 -1.64933749136986e-07
29.3700000000018 -6.85602978608317e-07
29.3800000000018 -1.20292740748378e-06
29.3900000000018 -1.71626972530066e-06
29.4000000000018 -2.22500068709858e-06
29.4100000000018 -2.72849986975192e-06
29.4200000000018 -3.22615641356312e-06
29.4300000000018 -3.71736974824284e-06
29.4400000000018 -4.20155030244156e-06
29.4500000000018 -4.67812019600769e-06
29.4600000000018 -5.1465139141831e-06
29.4700000000018 -5.60617896296925e-06
29.4800000000018 -6.05657650492363e-06
29.4900000000018 -6.49718197467913e-06
29.5000000000018 -6.92748567350457e-06
29.5100000000018 -7.34699334224999e-06
29.5200000000018 -7.75522671206632e-06
29.5300000000018 -8.15172403229962e-06
29.5400000000018 -8.53604057500797e-06
29.5500000000018 -8.90774911556953e-06
29.5600000000018 -9.26644038889515e-06
29.5700000000018 -9.61172352078247e-06
29.5800000000018 -9.94322643398574e-06
29.5900000000018 -1.02605962286097e-05
29.6000000000018 -1.05634995364701e-05
29.6100000000018 -1.0851622849098e-05
29.6200000000018 -1.11246728191013e-05
29.6300000000018 -1.13823765346284e-05
29.6400000000018 -1.16244817667205e-05
29.6500000000018 -1.18507571893697e-05
29.6600000000018 -1.20609925721371e-05
29.6700000000018 -1.22549989452212e-05
29.6800000000018 -1.24326087369064e-05
29.6900000000018 -1.25936758833509e-05
29.7000000000018 -1.27380759107354e-05
29.7100000000018 -1.28657059898151e-05
29.7200000000018 -1.29764849627115e-05
29.7300000000018 -1.3070353342523e-05
29.7400000000018 -1.31472732855531e-05
29.7500000000019 -1.32072285364366e-05
29.7600000000019 -1.32502243463996e-05
29.7700000000019 -1.32762873649266e-05
29.7800000000019 -1.32854655051674e-05
29.7900000000019 -1.3277827783469e-05
29.8000000000019 -1.32534641334989e-05
29.8100000000019 -1.32124851955228e-05
29.8200000000019 -1.31550220815461e-05
29.8300000000019 -1.3081226117258e-05
29.8400000000019 -1.2991268562094e-05
29.8500000000019 -1.28853403094369e-05
29.8600000000019 -1.27636515703465e-05
29.8700000000019 -1.26264315471856e-05
29.8800000000019 -1.24739281106306e-05
29.8900000000019 -1.23064075131456e-05
29.9000000000019 -1.21241512969488e-05
29.9100000000019 -1.1927458200015e-05
29.9200000000019 -1.17166454447173e-05
29.9300000000019 -1.14920465714754e-05
29.9400000000019 -1.12540109398139e-05
29.9500000000019 -1.10029032116818e-05
29.9600000000019 -1.0739102817824e-05
29.9700000000019 -1.04630034079925e-05
29.9800000000019 -1.01750122857904e-05
29.9900000000019 -9.87554811752895e-06
30.0000000000019 -9.56504404235647e-06
};
\addlegendentry{PI};
\addplot [line width=3pt, color1]
table {%
0 0
0.01 0
0.02 0
0.03 0
0.04 0
0.05 0
0.06 0
0.07 0
0.08 0
0.09 0
0.1 0
0.11 0
0.12 0
0.13 0
0.14 0
0.15 0
0.16 0
0.17 0
0.18 0
0.19 0
0.2 0
0.21 0
0.22 0
0.23 0
0.24 0
0.25 0
0.26 0
0.27 0
0.28 0
0.29 0
0.3 0
0.31 0
0.32 0
0.33 0
0.34 0
0.35 0
0.36 0
0.37 0
0.38 0
0.39 0
0.4 0
0.41 0
0.42 0
0.43 0
0.44 0
0.45 0
0.46 0
0.47 0
0.48 0
0.49 0
0.5 0
0.51 0
0.52 0
0.53 0
0.54 0
0.55 0
0.56 0
0.57 0
0.58 0
0.59 0
0.6 0
0.61 0
0.62 0
0.63 0
0.64 0
0.65 0
0.66 0
0.67 0
0.68 0
0.69 0
0.7 0
0.71 0
0.72 0
0.73 0
0.74 0
0.75 0
0.76 0
0.77 0
0.78 0
0.79 0
0.8 0
0.81 0
0.820000000000001 0
0.830000000000001 0
0.840000000000001 0
0.850000000000001 0
0.860000000000001 0
0.870000000000001 0
0.880000000000001 0
0.890000000000001 0
0.900000000000001 0
0.910000000000001 0
0.920000000000001 0
0.930000000000001 0
0.940000000000001 0
0.950000000000001 0
0.960000000000001 0
0.970000000000001 0
0.980000000000001 0
0.990000000000001 0
1 0
1.01 -4.56702698575438e-08
1.02 -3.14412282210323e-07
1.03 -1.03224858863571e-06
1.04 -2.42692001312371e-06
1.05 -4.72220514793178e-06
1.06 -8.14114550207929e-06
1.07 -1.29054674852236e-05
1.08 -1.92351943429038e-05
1.09 -2.73483098595994e-05
1.1 -3.74604295610264e-05
1.11 -4.97844613317375e-05
1.12 -6.45302582465675e-05
1.13 -8.19042662493712e-05
1.14 -0.000102109168994071
1.15 -0.000125343531876731
1.16 -0.000151801447050597
1.17 -0.00018167218102023
1.18 -0.000215139826246024
1.19 -0.000252382958049325
1.2 -0.000293574297985703
1.21 -0.000338880384746122
1.22 -0.000388461253549982
1.23 -0.000442470124908258
1.24 -0.000501053103557714
1.25 -0.000564348888296994
1.26 -0.000632488493391376
1.27 -0.000705594982154219
1.28 -0.000783783213258725
1.29 -0.000867159600283394
1.3 -0.000955821884947396
1.31 -0.00104985892444807
1.32 -0.00114935049327104
1.33 -0.00125436709980434
1.34 -0.00136496981805016
1.35 -0.00148121013469254
1.36 -0.00160312981174468
1.37 -0.00173076076496695
1.38 -0.00186412495821491
1.39 -0.00200323431384583
1.4 -0.00214809063928263
1.41 -0.00229868556980549
1.42 -0.00245500052761286
1.43 -0.00261700669716682
1.44 -0.00278466501667225
1.45 -0.00295792618619637
1.46 -0.00313673069158975
1.47 -0.00332100884456962
1.48 -0.00351068083878698
1.49 -0.00370565682162715
1.5 -0.00390583698206329
1.51 -0.00411111165360954
1.52 -0.00432136143270295
1.53 -0.0045364573121556
1.54 -0.00475626082943241
1.55 -0.00498062422949103
1.56 -0.0052093906419006
1.57 -0.00544239427193859
1.58 -0.00567946060534774
1.59 -0.00592040662698311
1.6 -0.0061650410541343
1.61 -0.00641316457871866
1.62 -0.00666457012321627
1.63 -0.00691904310879842
1.64 -0.00717636173505131
1.65 -0.00743629727097169
1.66 -0.00769861439029244
1.67 -0.00796307140439868
1.68 -0.00822942060418756
1.69 -0.0084974086039717
1.7 -0.00876677667869147
1.71 -0.00903726110918077
1.72 -0.0093085935349716
1.73 -0.00958050131411414
1.74 -0.00985270788948236
1.75 -0.0101249331610275
1.76 -0.0103968938634371
1.77 -0.01066830394865
1.78 -0.0109388749727432
1.79 -0.0112083164864504
1.8 -0.0114763364288989
1.81 -0.0117426415239977
1.82 -0.0120069376788832
1.83 -0.0122689303838665
1.84 -0.0125283251133168
1.85 -0.0127848277269192
1.86 -0.0130381448707465
1.87 -0.013287984377585
1.88 -0.0135340556659596
1.89 -0.0137760701373067
1.9 -0.0140137415707465
1.91 -0.0142467865149136
1.92 -0.0144749246763098
1.93 -0.0146978793036483
1.94 -0.0149153775676687
1.95 -0.0151271509359078
1.96 -0.0153329355419192
1.97 -0.0155324725484477
1.98 -0.0157255085040693
1.99 -0.0159117956928221
2 -0.0160910924763613
2.01 -0.0162631636281865
2.02 -0.0164277806594973
2.03 -0.0165847221362496
2.04 -0.0167337739869955
2.05 -0.016874729801105
2.06 -0.0170073911170471
2.07 -0.0171315677126815
2.08 -0.0172470778421629
2.09 -0.0173537485094947
2.1 -0.0174514157133097
2.11 -0.0175399246799069
2.12 -0.0176191300842584
2.13 -0.0176888962587173
2.14 -0.017749097389171
2.15 -0.0177996176984012
2.16 -0.0178403516164308
2.17 -0.0178712039376503
2.18 -0.0178920899645405
2.19 -0.0179029356378194
2.2 -0.0179036776528659
2.21 -0.0178942635622843
2.22 -0.0178746518644997
2.23 -0.017844812078285
2.24 -0.0178047248031464
2.25 -0.0177543817655067
2.26 -0.0176937858506483
2.27 -0.017622951120394
2.28 -0.0175419028165252
2.29 -0.01745067734995
2.29999999999999 -0.017349322275659
2.30999999999999 -0.0172378962535171
2.31999999999999 -0.017116468994963
2.32999999999999 -0.0169851211956915
2.33999999999999 -0.0168439444544621
2.34999999999999 -0.0166930411780935
2.35999999999999 -0.0165325244728114
2.36999999999999 -0.0163625180234051
2.37999999999999 -0.0161831559564623
2.38999999999999 -0.0159945826902335
2.39999999999999 -0.0157969527743333
2.40999999999999 -0.0155904307152294
2.41999999999999 -0.0153751907890753
2.42999999999999 -0.0151514168420423
2.43999999999999 -0.014919302078276
2.44999999999999 -0.0146790488355468
2.45999999999999 -0.0144308684617544
2.46999999999999 -0.0141749815215782
2.47999999999999 -0.0139116163226356
2.48999999999999 -0.0136410093442461
2.49999999999999 -0.0133634049583651
2.50999999999999 -0.0130790551423308
2.51999999999999 -0.0127882191833656
2.52999999999999 -0.0124911633749482
2.53999999999999 -0.0121881607052759
2.54999999999999 -0.0118794905381046
2.55999999999999 -0.0115654382863014
2.56999999999999 -0.0112462950784786
2.57999999999999 -0.0109223574190995
2.58999999999999 -0.0105939268424684
2.59999999999999 -0.0102613095610289
2.60999999999999 -0.00992481610840887
2.61999999999999 -0.00958476097765819
2.62999999999999 -0.00924146225513431
2.63999999999999 -0.00889524125049722
2.64999999999999 -0.00854642212328069
2.65999999999999 -0.00819533150651048
2.66999999999999 -0.00784229812784497
2.67999999999999 -0.00748765242871479
2.68999999999999 -0.00713172618194104
2.69999999999999 -0.00677485210831213
2.70999999999999 -0.00641736349259943
2.71999999999999 -0.00605959379949268
2.72999999999999 -0.00570187628993329
2.73999999999999 -0.00534454363832393
2.74999999999999 -0.00498792755108879
2.75999999999999 -0.00463235838705714
2.76999999999998 -0.00427816478013842
2.77999999999998 -0.00392567326475337
2.78999999999998 -0.00357520790448088
2.79999999999998 -0.00322708992437457
2.80999999999998 -0.00288163734739737
2.81999999999998 -0.00253916463541585
2.82999999999998 -0.00219998233518921
2.83999999999998 -0.00186439672977979
2.84999999999998 -0.00153270949580389
2.85999999999998 -0.00120521736693388
2.86999999999998 -0.000882211804052241
2.87999999999998 -0.000563978672449437
2.88999999999998 -0.000250797926447132
2.89999999999998 5.7056698182097e-05
2.90999999999998 0.000359317983625507
2.91999999999998 0.000655725520916634
2.92999999999998 0.00094602599211228
2.93999999999998 0.00122997344314835
2.94999999999998 0.00150732954726987
2.95999999999998 0.00177786385868231
2.96999999999998 0.00204135405613818
2.97999999999998 0.00229758617618606
2.98999999999998 0.00254635483582348
2.99999999999998 0.00278746344430865
3.00999999999998 0.00302072440390081
3.01999999999998 0.00324595929931345
3.02999999999998 0.00346299907567922
3.03999999999998 0.00367168420484046
3.04999999999998 0.00387186483979472
3.05999999999998 0.00406340095713928
3.06999999999998 0.00424616248737467
3.07999999999998 0.0044200294329424
3.08999999999998 0.00458489197388755
3.09999999999998 0.00474065056103012
3.10999999999998 0.0048872159966561
3.11999999999998 0.00502450950257532
3.12999999999998 0.00515246277544806
3.13999999999998 0.00527101802947697
3.14999999999998 0.00538012802639291
3.15999999999998 0.00547975609273206
3.16999999999998 0.00556987612477479
3.17999999999998 0.00565047258172155
3.18999999999998 0.00572154046112989
3.19999999999998 0.00578308526778103
3.20999999999998 0.00583512296784114
3.21999999999998 0.00587767993055738
3.22999999999998 0.00591079285755505
3.23999999999997 0.00593450869979861
3.24999999999997 0.0059488845622667
3.25999999999997 0.00595398759636447
3.26999999999997 0.0059498948800471
3.27999999999997 0.00593669328554226
3.28999999999997 0.0059144793344106
3.29999999999997 0.00588335903942656
3.30999999999997 0.00584345030795974
3.31999999999997 0.00579487613380529
3.32999999999997 0.00573777019728154
3.33999999999997 0.0056722749888938
3.34999999999997 0.00559854160900058
3.35999999999997 0.00551672956104437
3.36999999999997 0.00542700654145486
3.37999999999997 0.00532954823298803
3.38999999999997 0.00522453814266034
3.39999999999997 0.00511216626202445
3.40999999999997 0.00499263031444395
3.41999999999997 0.00486613512761913
3.42999999999997 0.00473289219839315
3.43999999999997 0.00459311941524984
3.44999999999997 0.00444704077435877
3.45999999999997 0.00429488608952313
3.46999999999997 0.00413689069639515
3.47999999999997 0.00397329498435436
3.48999999999997 0.00380434407396633
3.49999999999997 0.00363028814409521
3.50999999999997 0.00345138162365781
3.51999999999997 0.00326788289322278
3.52999999999997 0.00308005397178049
3.53999999999997 0.0028881601958746
3.54999999999997 0.00269246989406197
3.55999999999997 0.00249325405818141
3.56999999999997 0.00229078601232922
3.57999999999997 0.00208534108018961
3.58999999999997 0.00187719258733167
3.59999999999997 0.00166662322865014
3.60999999999997 0.00145391221502489
3.61999999999997 0.00123933943681376
3.62999999999997 0.0010231851268333
3.63999999999997 0.000805729525174629
3.64999999999997 0.000587252546058484
3.65999999999997 0.000368033446989364
3.66999999999997 0.000148350500504407
3.67999999999997 -7.15193311649118e-05
3.68999999999997 -0.000291300718185713
3.69999999999997 -0.000510720281626398
3.70999999999996 -0.000729506908037226
3.71999999999996 -0.00094739205976098
3.72999999999996 -0.00116411008062261
3.73999999999996 -0.0013793984966628
3.74999999999996 -0.00159299831157998
3.75999999999996 -0.00180465429655127
3.76999999999996 -0.00201411527410899
3.77999999999996 -0.00222113439575707
3.78999999999996 -0.00242546941301932
3.79999999999996 -0.00262688294162014
3.80999999999996 -0.00282514271850724
3.81999999999996 -0.00302002185143558
3.82999999999996 -0.00321129906093341
3.83999999999996 -0.00339875891407068
3.84999999999996 -0.00358219205011439
3.85999999999996 -0.00376139539782446
3.86999999999996 -0.00393617238398743
3.87999999999996 -0.0041063331330459
3.88999999999996 -0.00427169465761873
3.89999999999996 -0.0044320810397194
3.90999999999996 -0.00458732360249194
3.91999999999996 -0.0047372610722959
3.92999999999996 -0.00488173973098484
3.93999999999996 -0.00502061355823467
3.94999999999996 -0.00515374436379172
3.95999999999996 -0.00528100190952229
3.96999999999996 -0.00540226402115892
3.97999999999996 -0.00551741668965081
3.98999999999996 -0.00562635416203904
3.99999999999996 -0.00572897902178935
4.00999999999996 -0.00582520225852801
4.01999999999996 -0.00591494332713832
4.02999999999996 -0.00599813019618698
4.03999999999996 -0.00607469938566092
4.04999999999996 -0.00614459599400559
4.05999999999996 -0.00620777371455361
4.06999999999996 -0.00626419484342007
4.07999999999996 -0.00631383026954371
4.08999999999996 -0.00635665946156727
4.09999999999996 -0.00639267050525298
4.10999999999996 -0.00642186025921436
4.11999999999996 -0.00644423377270094
4.12999999999996 -0.0064598045172029
4.13999999999996 -0.00646859432016435
4.14999999999996 -0.00647063328884705
4.15999999999996 -0.00646595972450315
4.16999999999996 -0.00645462002704155
4.17999999999996 -0.00643666859040992
4.18999999999996 -0.00641216768897437
4.19999999999995 -0.00638118735528696
4.20999999999995 -0.00634380524984481
4.21999999999995 -0.00630010652604824
4.22999999999995 -0.00625018368426875
4.23999999999995 -0.00619413643168278
4.24999999999995 -0.00613207155452776
4.25999999999995 -0.00606410190361618
4.26999999999995 -0.00599034744460079
4.27999999999995 -0.00591093479691688
4.28999999999995 -0.00582599683637805
4.29999999999995 -0.00573567250171466
4.30999999999995 -0.00564010659036509
4.31999999999995 -0.00553944954616248
4.32999999999995 -0.00543385724021395
4.33999999999995 -0.00532349074573674
4.34999999999995 -0.00520851610738762
4.35999999999995 -0.00508910410551723
4.36999999999995 -0.00496543001573138
4.37999999999995 -0.00483767336411753
4.38999999999995 -0.00470601767848254
4.39999999999995 -0.00457065023594328
4.40999999999995 -0.00443176180720864
4.41999999999995 -0.00428954639789181
4.42999999999995 -0.00414420098719166
4.43999999999995 -0.00399592526428246
4.44999999999995 -0.00384492136275236
4.45999999999995 -0.00369139359343035
4.46999999999995 -0.0035355453477114
4.47999999999995 -0.0033775876374996
4.48999999999995 -0.00321772965713743
4.49999999999995 -0.00305618170405383
4.50999999999995 -0.00289315490809815
4.51999999999995 -0.00272886096113903
4.52999999999995 -0.00256351184806272
4.53999999999995 -0.00239731957862205
4.54999999999995 -0.00223049592073848
4.55999999999995 -0.00206325213557735
4.56999999999995 -0.00189579871471302
4.57999999999995 -0.00172834511969806
4.58999999999995 -0.00156109952434565
4.59999999999995 -0.00139426856003046
4.60999999999995 -0.00122805706430858
4.61999999999995 -0.00106266783315121
4.62999999999995 -0.00089830137708213
4.63999999999995 -0.000735155681502113
4.64999999999995 -0.000573425971477527
4.65999999999995 -0.000413304481263418
4.66999999999994 -0.000254980228824893
4.67999999999994 -9.86387956123772e-05
4.68999999999994 5.55378881603969e-05
4.69999999999994 0.000207371752486784
4.70999999999994 0.000356688790570934
4.71999999999994 0.000503319255655039
4.72999999999994 0.000647097851396419
4.73999999999994 0.000787863915636973
4.74999999999994 0.000925461597355301
4.75999999999994 0.00105974002481702
4.76999999999994 0.00119055345817226
4.77999999999994 0.00131776147293568
4.78999999999994 0.00144122909161721
4.79999999999994 0.00156082692402237
4.80999999999994 0.00167643129975264
4.81999999999994 0.00178792439277951
4.82999999999994 0.00189519433797494
4.83999999999994 0.00199813533949171
4.84999999999994 0.00209664777096429
4.85999999999994 0.00219063826729141
4.86999999999994 0.00228001980781176
4.87999999999994 0.0023647117913946
4.88999999999994 0.00244463921318899
4.89999999999994 0.00251973564373065
4.90999999999994 0.00258994010222588
4.91999999999994 0.00265519824544995
4.92999999999994 0.0027154623996715
4.93999999999994 0.00277069158401271
4.94999999999994 0.00282085152525523
4.95999999999994 0.00286591466411331
4.96999999999994 0.00290586015519535
4.97999999999994 0.00294067385183946
4.98999999999994 0.00297034828918214
4.99999999999994 0.0029948826568622
5.00999999999994 0.00301428276335714
5.01999999999994 0.00302856099216281
5.02999999999994 0.00303773624991572
5.03999999999994 0.00304183390658215
5.04999999999994 0.00304088572785744
5.05999999999994 0.00303492979994373
5.06999999999994 0.00302401044691617
5.07999999999994 0.00300817814095308
5.08999999999994 0.00298748940680104
5.09999999999994 0.00296200671876972
5.10999999999994 0.00293179839098791
5.11999999999994 0.00289693846825622
5.12999999999994 0.00285750661730387
5.13999999999993 0.00281358803165423
5.14999999999993 0.0027652733840897
5.15999999999993 0.00271265763567614
5.16999999999993 0.0026558409978875
5.17999999999993 0.0025949288549291
5.18999999999993 0.00253003110384417
5.19999999999993 0.00246126205207508
5.20999999999993 0.0023887408913177
5.21999999999993 0.00231259091980459
5.22999999999993 0.00223293940690849
5.23999999999993 0.00214991742817973
5.24999999999993 0.00206365968637641
5.25999999999993 0.00197430432434541
5.26999999999993 0.00188199273232489
5.27999999999993 0.00178686935097737
5.28999999999993 0.00168908147092962
5.29999999999993 0.00158877902935227
5.30999999999993 0.0014861144039938
5.31999999999993 0.00138124220502378
5.32999999999993 0.0012743190650076
5.33999999999993 0.00116550342731696
5.34999999999993 0.00105495533327042
5.35999999999993 0.000942836208291411
5.36999999999993 0.000829308647367516
5.37999999999993 0.00071453620009114
5.38999999999993 0.000598683155559701
5.39999999999993 0.000481914327411189
5.40999999999993 0.000364394839268442
5.41999999999993 0.000246289910863508
5.42999999999993 0.000127764645110726
5.43999999999993 8.98381639447029e-06
5.44999999999993 -0.000109888339665201
5.45999999999993 -0.000228688334709431
5.46999999999993 -0.000347253634143528
5.47999999999993 -0.000465422861393727
5.48999999999993 -0.000583035999662597
5.49999999999993 -0.000699934590957058
5.50999999999993 -0.00081596193213827
5.51999999999993 -0.000930963267752629
5.52999999999993 -0.00104478597942291
5.53999999999993 -0.00115727977157408
5.54999999999993 -0.00126829685327541
5.55999999999993 -0.00137769211598698
5.56999999999993 -0.00148532330700467
5.57999999999993 -0.00159105119840394
5.58999999999993 -0.00169473975128986
5.59999999999993 -0.00179625627516714
5.60999999999992 -0.00189547158225157
5.61999999999992 -0.00199226013655087
5.62999999999992 -0.0020865001975511
5.63999999999992 -0.00217807395835176
5.64999999999992 -0.00226686767810109
5.65999999999992 -0.00235277180859027
5.66999999999992 -0.00243568111487424
5.67999999999992 -0.0025154947897942
5.68999999999992 -0.00259211656228566
5.69999999999992 -0.00266545479936417
5.70999999999992 -0.00273542260168956
5.71999999999992 -0.00280193789261796
5.72999999999992 -0.00286492350065963
5.73999999999992 -0.00292430723526952
5.74999999999992 -0.0029800219559062
5.75999999999992 -0.00303200563430371
5.76999999999992 -0.00308020140990979
5.77999999999992 -0.00312455763845315
5.78999999999992 -0.00316502793361096
5.79999999999992 -0.00320157120175746
5.80999999999992 -0.00323415166978314
5.81999999999992 -0.00326273890769518
5.82999999999992 -0.00328730783989478
5.83999999999992 -0.00330783875224398
5.84999999999992 -0.00332431729318785
5.85999999999992 -0.0033367344671051
5.86999999999992 -0.00334508662086859
5.87999999999992 -0.00334937542367718
5.88999999999992 -0.00334960784023563
5.89999999999992 -0.00334579609737833
5.90999999999992 -0.00333795764425978
5.91999999999992 -0.00332611510627616
5.92999999999992 -0.00331029623295033
5.93999999999992 -0.00329053384013617
5.94999999999992 -0.00326686574713574
5.95999999999992 -0.00323933470981761
5.96999999999992 -0.00320798835193645
5.97999999999992 -0.00317287909960448
5.98999999999992 -0.00313406413139951
5.99999999999992 -0.00309160537036543
6.00999999999992 -0.00304556928628709
6.01999999999992 -0.00299602508243559
6.02999999999992 -0.00294304742333545
6.03999999999992 -0.00288671512814451
6.04999999999992 -0.0028271110495327
6.05999999999992 -0.0027643219480957
6.06999999999992 -0.00269843836225924
6.07999999999991 -0.0026295542340756
6.08999999999991 -0.00255776690913475
6.09999999999991 -0.00248317767256908
6.10999999999991 -0.00240589092875839
6.11999999999991 -0.00232601410087476
6.12999999999991 -0.00224365750158161
6.13999999999991 -0.00215893418978857
6.14999999999991 -0.00207195981944098
6.15999999999991 -0.00198285248307801
6.16999999999991 -0.00189173255157262
6.17999999999991 -0.0017987225108806
6.18999999999991 -0.00170394679634644
6.19999999999991 -0.00160753162190045
6.20999999999991 -0.00150960481450305
6.21999999999991 -0.00141029564562073
6.22999999999991 -0.0013097346559048
6.23999999999991 -0.00120805348233963
6.24999999999991 -0.00110538468471566
6.25999999999991 -0.00100186157166898
6.26999999999991 -0.000897618026522447
6.27999999999991 -0.000792788333159092
6.28999999999991 -0.000687507002154885
6.29999999999991 -0.000581908597394639
6.30999999999991 -0.000476127563392593
6.31999999999991 -0.000370298053535927
6.32999999999991 -0.000264553759467204
6.33999999999991 -0.000159027741818617
6.34999999999991 -5.3852262508059e-05
6.35999999999991 5.08413811959412e-05
6.36999999999991 0.000154923020637339
6.37999999999991 0.000258263777554773
6.38999999999991 0.000360736223514815
6.39999999999991 0.000462214536773481
6.40999999999991 0.000562574656387694
6.41999999999991 0.000661694433395888
6.42999999999991 0.000759458871543323
6.43999999999991 0.000855745217284463
6.44999999999991 0.000950437929328321
6.45999999999991 0.00104342394551219
6.46999999999991 0.00113459281706554
6.47999999999991 0.00122383683890799
6.48999999999991 0.001311051175838
6.49999999999991 0.00139613398447413
6.50999999999991 0.00147898653081626
6.51999999999991 0.00155951330329951
6.52999999999991 0.00163762212121979
6.53999999999991 0.00171322423841503
6.5499999999999 0.00178623444209335
6.5599999999999 0.00185657114670457
6.5699999999999 0.00192414952078153
6.5799999999999 0.00198890259266039
6.5899999999999 0.00205076021142161
6.5999999999999 0.0021096561809031
6.6099999999999 0.00216552832729217
6.6199999999999 0.00221831856131671
6.6299999999999 0.00226797293497879
6.6399999999999 0.00231444169278139
6.6499999999999 0.00235767931740654
6.6599999999999 0.00239764456981063
6.6699999999999 0.00243430052371025
6.6799999999999 0.0024676145944395
6.6899999999999 0.00249755856216717
6.6999999999999 0.0025241085894701
6.7099999999999 0.0025472452348407
6.7199999999999 0.00256695345657212
6.7299999999999 0.00258322261333067
6.7399999999999 0.00259604646021315
6.7499999999999 0.00260542313832703
6.7599999999999 0.00261135515887606
6.7699999999999 0.0026138493818104
6.7799999999999 0.00261291698911464
6.7899999999999 0.0026085734528259
6.7999999999999 0.00260083849790064
6.8099999999999 0.00258973606008942
6.8199999999999 0.00257529423904414
6.8299999999999 0.00255754524700381
6.8399999999999 0.00253652535361347
6.8499999999999 0.00251227482786934
6.8599999999999 0.00248483787910447
6.8699999999999 0.00245426260105603
6.8799999999999 0.00242060092452187
6.8899999999999 0.00238390849986336
6.8999999999999 0.00234424464827162
6.9099999999999 0.00230167141454017
6.9199999999999 0.00225625430184514
6.9299999999999 0.00220806258182843
6.9399999999999 0.00215716874840198
6.9499999999999 0.00210364841166128
6.9599999999999 0.00204758018841835
6.9699999999999 0.00198904511263188
6.9799999999999 0.00192812735112422
6.9899999999999 0.00186491400935433
6.9999999999999 0.00179949472907191
7.00999999999989 0.00173196150846321
7.01999999999989 0.00166240859926983
7.02999999999989 0.00159093239077703
7.03999999999989 0.00151763128595354
7.04999999999989 0.00144260557265333
7.05999999999989 0.00136595729137589
7.06999999999989 0.00128779010044417
7.07999999999989 0.00120820913915048
7.08999999999989 0.00112732088926347
7.09999999999989 0.00104523303520358
7.10999999999989 0.000962054323147272
7.11999999999989 0.000877894419291845
7.12999999999989 0.000792863767496115
7.13999999999989 0.000707073446501297
7.14999999999989 0.000620635026929133
7.15999999999989 0.000533660428249744
7.16999999999989 0.000446261775907354
7.17999999999989 0.000358551258789548
7.18999999999989 0.000270640987222589
7.19999999999989 0.000182642851673209
7.20999999999989 9.46683823346566e-05
7.21999999999989 6.82860976681752e-06
7.22999999999989 -8.07660732364912e-05
7.23999999999989 -0.00016800604834793
7.24999999999989 -0.000254782606853351
7.25999999999989 -0.000340988083063305
7.26999999999989 -0.000426515985819924
7.27999999999989 -0.000511265435743074
7.28999999999989 -0.000595128695006298
7.29999999999989 -0.000678003564573575
7.30999999999989 -0.000759789519002424
7.31999999999989 -0.00084038782648568
7.32999999999989 -0.000919701666064654
7.33999999999989 -0.000997636241882978
7.34999999999989 -0.00107409889435306
7.35999999999989 -0.00114899920811171
7.36999999999989 -0.00122224911664505
7.37999999999989 -0.00129376300322991
7.38999999999989 -0.001363457798627
7.39999999999989 -0.00143125307553262
7.40999999999989 -0.00149707113844429
7.41999999999989 -0.00156083710998829
7.42999999999989 -0.00162247901329206
7.43999999999989 -0.00168192785031396
7.44999999999989 -0.00173911767604824
7.45999999999989 -0.00179397841225827
7.46999999999989 -0.0018464578248494
7.47999999999988 -0.00189649958341368
7.48999999999988 -0.00194405066786694
7.49999999999988 -0.00198906142093598
7.50999999999988 -0.00203148559617937
7.51999999999988 -0.00207128040149992
7.52999999999988 -0.00210840653811307
7.53999999999988 -0.00214282823494169
7.54999999999988 -0.00217451327841414
7.55999999999988 -0.00220343303791404
7.56999999999988 -0.00222956248575201
7.57999999999988 -0.00225288021325329
7.58999999999988 -0.00227336844400992
7.59999999999988 -0.00229101303727037
7.60999999999988 -0.00230580349103359
7.61999999999988 -0.00231773294002165
7.62999999999988 -0.00232679814864851
7.63999999999988 -0.00233299949949029
7.64999999999988 -0.00233634097731353
7.65999999999988 -0.0023368301487349
7.66999999999988 -0.00233447813760962
7.67999999999988 -0.00232929959628168
7.68999999999988 -0.00232131267288649
7.69999999999988 -0.0023105389749946
7.70999999999988 -0.00229700353006224
7.71999999999988 -0.00228073474349196
7.72999999999988 -0.00226176435580089
7.73999999999988 -0.00224012738790183
7.74999999999988 -0.00221586206834461
7.75999999999988 -0.0021890097955682
7.76999999999988 -0.0021596150959424
7.77999999999988 -0.00212772559382635
7.78999999999988 -0.0020933906602953
7.79999999999988 -0.00205666341062924
7.80999999999988 -0.00201759974120621
7.81999999999988 -0.00197625824612597
7.82999999999988 -0.00193270013092413
7.83999999999988 -0.00188698912347598
7.84999999999988 -0.00183919138219689
7.85999999999988 -0.00178937482251545
7.86999999999988 -0.0017376102932181
7.87999999999988 -0.00168397075529473
7.88999999999988 -0.00162853118715017
7.89999999999988 -0.00157136851820769
7.90999999999988 -0.00151256149706831
7.91999999999988 -0.00145219057401148
7.92999999999988 -0.00139033779914695
7.93999999999988 -0.00132708671625768
7.94999999999987 -0.00126252225367742
7.95999999999987 -0.00119673061298357
7.96999999999987 -0.00112979915600753
7.97999999999987 -0.00106181629051712
7.98999999999987 -0.000992871354845572
7.99999999999987 -0.000923054501695055
8.00999999999987 -0.00085245658131511
8.01999999999987 -0.000781169024239489
8.02999999999987 -0.000709283723753719
8.03999999999987 -0.000636892918258511
8.04999999999987 -0.000564089073688896
8.05999999999987 -0.000490964766145133
8.06999999999987 -0.000417612564888299
8.07999999999987 -0.000344124915850528
8.08999999999987 -0.000270594025808506
8.09999999999987 -0.000197111747364978
8.10999999999987 -0.000123769464882233
8.11999999999987 -5.06579815086466e-05
8.12999999999987 2.21325925631992e-05
8.13999999999987 9.45163776568688e-05
8.14999999999987 0.000166401961630369
8.15999999999987 0.000237701975107046
8.16999999999987 0.000308330168557878
8.17999999999987 0.00037820151621999
8.18999999999987 0.00044723231796884
8.19999999999987 0.000515340299030038
8.20999999999987 0.000582444707419217
8.21999999999987 0.000648466409001026
8.22999999999987 0.00071332798006082
8.23999999999987 0.000776953797286301
8.24999999999987 0.000839270125058816
8.25999999999987 0.000900205199957883
8.26999999999987 0.000959689312386331
8.27999999999987 0.00101765488516002
8.28999999999987 0.00107403654876454
8.29999999999987 0.0011287712146525
8.30999999999987 0.00118179814407977
8.31999999999987 0.00123305901407116
8.32999999999987 0.00128249798002975
8.33999999999987 0.00133006173492477
8.34999999999987 0.00137569956499714
8.35999999999987 0.0014193634019255
8.36999999999987 0.00146100787139982
8.37999999999987 0.0015005829145111
8.38999999999987 0.00153805635480369
8.39999999999987 0.00157339122130214
8.40999999999987 0.00160655339730183
8.41999999999986 0.00163751165047405
8.42999999999986 0.00166623765920532
8.43999999999986 0.00169270603515565
8.44999999999986 0.00171689434202522
8.45999999999986 0.00173878311052507
8.46999999999986 0.00175835585036807
8.47999999999986 0.00177559905704414
8.48999999999986 0.00179050221402876
8.49999999999986 0.00180305779381102
8.50999999999986 0.00181326125402478
8.51999999999986 0.00182111102988873
8.52999999999986 0.00182660852300417
8.53999999999986 0.0018297580865748
8.54999999999986 0.0018305670071348
8.55999999999986 0.00182904548290507
8.56999999999986 0.00182520659895015
8.57999999999986 0.00181906629939776
8.58999999999986 0.0018106433571411
8.59999999999986 0.00179995933620257
8.60999999999986 0.00178703854696338
8.61999999999986 0.00177190800993024
8.62999999999986 0.00175459741406812
8.63999999999986 0.00173513907513339
8.64999999999986 0.00171356789760961
8.65999999999986 0.00168992135042917
8.66999999999986 0.00166423886367784
8.67999999999986 0.00163656214111643
8.68999999999986 0.00160693556871751
8.69999999999986 0.00157540578416521
8.70999999999986 0.00154202160724568
8.71999999999986 0.00150683396790039
8.72999999999986 0.00146989583202592
8.73999999999986 0.00143126212511028
8.74999999999986 0.00139098921541599
8.75999999999986 0.00134913577000233
8.76999999999986 0.00130576218457999
8.77999999999986 0.00126093047188826
8.78999999999986 0.00121470418229002
8.79999999999986 0.0011671483201374
8.80999999999986 0.00111832925731685
8.81999999999986 0.00106831465084782
8.82999999999986 0.00101717335476452
8.83999999999986 0.000964975315842185
8.84999999999986 0.00091179148111756
8.85999999999986 0.000857693705452511
8.86999999999986 0.000802754657904446
8.87999999999986 0.000747047727185088
8.88999999999985 0.000690646926430347
8.89999999999985 0.000633626797468552
8.90999999999985 0.000576062314752243
8.91999999999985 0.000518028789105362
8.92999999999985 0.000459601771427792
8.93999999999985 0.000400856956493254
8.94999999999985 0.000341870086971406
8.95999999999985 0.000282716857801801
8.96999999999985 0.000223472821044256
8.97999999999985 0.000164213291327513
8.98999999999985 0.000105013252016191
8.99999999999985 4.59472622131754e-05
9.00999999999985 -1.29106352866928e-05
9.01999999999985 -7.14870049779854e-05
9.02999999999985 -0.000129712272538907
9.03999999999985 -0.000187511753973662
9.04999999999985 -0.000244814371600656
9.05999999999985 -0.000301550001308893
9.06999999999985 -0.000357649556900692
9.07999999999985 -0.00041304507270506
9.08999999999985 -0.000467669784370108
9.09999999999985 -0.00052145820774507
9.10999999999985 -0.000574346215764377
9.11999999999985 -0.00062627111324861
9.12999999999985 -0.000677171709539574
9.13999999999985 -0.000726988388889401
9.14999999999985 -0.000775663178526504
9.15999999999985 -0.00082313981432399
9.16999999999985 -0.000869363803667021
9.17999999999985 -0.000914282486639798
9.18999999999985 -0.000957845093984301
9.19999999999985 -0.00100000280240599
9.20999999999985 -0.00104070878724629
9.21999999999985 -0.0010799182723771
9.22999999999985 -0.00111758857726544
9.23999999999985 -0.0011536791611595
9.24999999999985 -0.00118815166435088
9.25999999999985 -0.00122096284540454
9.26999999999985 -0.00125208612054794
9.27999999999985 -0.00128148995874762
9.28999999999985 -0.00130914514745527
9.29999999999985 -0.00133502481885508
9.30999999999985 -0.00135910447303366
9.31999999999985 -0.00138136199805862
9.32999999999985 -0.0014017776869567
9.33999999999985 -0.00142033425158686
9.34999999999985 -0.00143701683354432
9.35999999999984 -0.00145181301293943
9.36999999999984 -0.00146471281020877
9.37999999999984 -0.00147570868911612
9.38999999999984 -0.00148479555520942
9.39999999999984 -0.00149197075125953
9.40999999999984 -0.00149723404972304
9.41999999999984 -0.00150058764228631
9.42999999999984 -0.00150203612656783
9.43999999999984 -0.00150158649005259
9.44999999999984 -0.00149924808922244
9.45999999999984 -0.00149503262634541
9.46999999999984 -0.00148895412490803
9.47999999999984 -0.00148102890212317
9.48999999999984 -0.00147127553906187
9.49999999999984 -0.00145971484870804
9.50999999999984 -0.00144636984244506
9.51999999999984 -0.00143126569593979
9.52999999999984 -0.00141442971645245
9.53999999999984 -0.00139589131675814
9.54999999999984 -0.00137568188548628
9.55999999999984 -0.00135383413193229
9.56999999999984 -0.00133038322291184
9.57999999999984 -0.00130536619298665
9.58999999999984 -0.00127882188854218
9.59999999999984 -0.00125079090992076
9.60999999999984 -0.00122131555167681
9.61999999999984 -0.00119043974102567
9.62999999999984 -0.00115820871996682
9.63999999999984 -0.0011246694078592
9.64999999999984 -0.00108987030787925
9.65999999999984 -0.00105386127617251
9.66999999999984 -0.0010166934588677
9.67999999999984 -0.00097841922495073
9.68999999999984 -0.000939092096496868
9.69999999999984 -0.00089876667700411
9.70999999999984 -0.000857498578249446
9.71999999999984 -0.000815344345940021
9.72999999999984 -0.000772361384356347
9.73999999999984 -0.000728607880344804
9.74999999999984 -0.000684142730719668
9.75999999999984 -0.000639025460472026
9.76999999999984 -0.000593316142760465
9.77999999999984 -0.000547075320981842
9.78999999999984 -0.000500363930479687
9.79999999999984 -0.000453243220024529
9.80999999999984 -0.000405774673190206
9.81999999999984 -0.000358019929742412
9.82999999999983 -0.000310040707150967
9.83999999999983 -0.000261898722333029
9.84999999999983 -0.000213655613731913
9.85999999999983 -0.000165372863833075
9.86999999999983 -0.000117111722013049
9.87999999999983 -6.89331283653665e-05
9.88999999999983 -2.08976385447133e-05
9.89999999999983 2.69346513641876e-05
9.90999999999983 7.45041786834786e-05
9.91999999999983 0.000121751986976972
9.92999999999983 0.000168623077870492
9.93999999999983 0.000215057073634406
9.94999999999983 0.000260997256692289
9.95999999999983 0.000306387791058368
9.96999999999983 0.000351173789091529
9.97999999999983 0.000395301376719666
9.98999999999983 0.000438717757062323
9.99999999999983 0.000481371272381266
10.0099999999998 0.000523211464290874
10.0199999999998 0.000564189132161459
10.0299999999998 0.000604256389651809
10.0399999999998 0.000643366719308715
10.0499999999998 0.000681475025041648
10.0599999999998 0.000718537682710876
10.0699999999998 0.000754512588878808
10.0799999999998 0.000789359206913711
10.0899999999998 0.000823038611081555
10.0999999999998 0.000855513528385812
10.1099999999998 0.000886748378111065
10.1199999999998 0.00091670255985169
10.1299999999998 0.000945350800653097
10.1399999999998 0.000972662872950677
10.1499999999998 0.000998610377853765
10.1599999999998 0.0010231667727536
10.1699999999998 0.00104630739645606
10.1799999999998 0.0010680094918202
10.1899999999998 0.00108825222588731
10.1999999999998 0.00110701670748855
10.2099999999998 0.00112428600232277
10.2199999999998 0.00114004514550015
10.2299999999998 0.00115428115155105
10.2399999999998 0.00116698302272623
10.2499999999998 0.00117814175261023
10.2599999999998 0.00118775032866808
10.2699999999998 0.00119580373260837
10.2799999999998 0.00120229893784951
10.2899999999998 0.00120723490388575
10.2999999999998 0.00121061256838372
10.3099999999998 0.00121243483711803
10.3199999999998 0.00121270657150578
10.3299999999998 0.00121143457378336
10.3399999999998 0.00120862756988004
10.3499999999998 0.00120429619005826
10.3599999999998 0.00119845294741539
10.3699999999998 0.00119111221438164
10.3799999999998 0.0011822901974225
10.3899999999998 0.00117200491028896
10.3999999999998 0.0011602761464453
10.4099999999998 0.00114712545194677
10.4199999999998 0.0011325761016784
10.4299999999998 0.001116653087591
10.4399999999998 0.00109938250628217
10.4499999999998 0.00108079228458852
10.4599999999998 0.00106091200538854
10.4699999999998 0.00103977275102846
10.4799999999998 0.00101740705677433
10.4899999999998 0.000993848862702358
10.4999999999998 0.000969133464084768
10.5099999999998 0.000943297460331971
10.5199999999998 0.000916378385439326
10.5299999999998 0.000888415351245459
10.5399999999998 0.000859448596510812
10.5499999999998 0.000829519435177971
10.5599999999998 0.000798670203267155
10.5699999999998 0.000766944202976052
10.5799999999998 0.000734385644905735
10.5899999999998 0.000701039588895752
10.5999999999998 0.000666951883756984
10.6099999999998 0.000632169106097437
10.6199999999998 0.000596738498388166
10.6299999999998 0.00056070790639091
10.6399999999998 0.000524125716054054
10.6499999999998 0.000487040789975337
10.6599999999998 0.000449502403232602
10.6699999999998 0.000411560179603056
10.6799999999998 0.000373264027534041
10.6899999999998 0.000334664075440996
10.6999999999998 0.00029581060743574
10.7099999999998 0.000256754000125449
10.7199999999998 0.000217544657272024
10.7299999999998 0.000178232945553915
10.7399999999998 0.000138869131146439
10.7499999999998 9.95033166967193e-05
10.7599999999998 6.01853787757444e-05
10.7699999999998 2.09649058873616e-05
10.7799999999998 -1.81088628871264e-05
10.7899999999998 -5.69870985319079e-05
10.7999999999998 -9.56214419523214e-05
10.8099999999998 -0.000133964062971636
10.8199999999998 -0.000171967718393726
10.8299999999998 -0.000209585809063855
10.8399999999998 -0.00024677607236811
10.8499999999998 -0.000283490120027084
10.8599999999998 -0.000319683601245528
10.8699999999998 -0.000355313022698709
10.8799999999998 -0.000390335799766559
10.8899999999998 -0.000424710306368844
10.8999999999998 -0.000458395923346253
10.9099999999998 -0.000491353085334423
10.9199999999998 -0.000523543326079269
10.9299999999998 -0.000554929322099044
10.9399999999998 -0.000585474934581354
10.9499999999998 -0.000615145250145122
10.9599999999998 -0.000643906619234987
10.9699999999998 -0.000671720654112875
10.9799999999998 -0.000698562306054374
10.9899999999998 -0.000724401983760126
10.9999999999998 -0.000749211497757891
11.0099999999998 -0.000772964089744391
11.0199999999998 -0.000795634459979633
11.0299999999998 -0.000817198792709197
11.0399999999998 -0.000837634779592808
11.0499999999998 -0.000856921641119825
11.0599999999998 -0.000875040145995421
11.0699999999998 -0.000891972628483832
11.0799999999998 -0.000907703003892881
11.0899999999998 -0.00092221678176409
11.0999999999998 -0.000935501076675129
11.1099999999998 -0.00094754461727536
11.1199999999998 -0.000958337753480105
11.1299999999998 -0.000967872461527353
11.1399999999998 -0.00097614234562745
11.1499999999998 -0.000983142639429546
11.1599999999998 -0.000988870204708516
11.1699999999998 -0.000993323528011727
11.1799999999998 -0.000996502715283516
11.1899999999998 -0.000998409484489393
11.1999999999998 -0.000999047156266731
11.2099999999998 -0.000998420642634794
11.2199999999998 -0.000996536433804875
11.2299999999998 -0.00099340258310352
11.2399999999998 -0.000989028689966679
11.2499999999998 -0.00098342588173966
11.2599999999998 -0.000976606793504984
11.2699999999998 -0.000968585546705244
11.2799999999998 -0.000959377726845096
11.2899999999998 -0.000949000361043
11.2999999999998 -0.000937471897100896
11.3099999999998 -0.00092481218842962
11.3199999999998 -0.000911042260379857
11.3299999999998 -0.00089618423633222
11.3399999999998 -0.000880261784919938
11.3499999999998 -0.00086329981588172
11.3599999999998 -0.000845324442589411
11.3699999999998 -0.000826362943278396
11.3799999999998 -0.000806443721026517
11.3899999999998 -0.000785596262530244
11.3999999999998 -0.0007638509474297
11.4099999999998 -0.00074123921663512
11.4199999999998 -0.000717793594404038
11.4299999999998 -0.000693547500549578
11.4399999999998 -0.000668535208645431
11.4499999999998 -0.000642791801300231
11.4599999999998 -0.000616353123592802
11.4699999999998 -0.00058925573520641
11.4799999999998 -0.000561536861552821
11.4899999999998 -0.000533234344092381
11.4999999999998 -0.00050438658997747
11.5099999999998 -0.000475032521130089
11.5199999999998 -0.00044521152284641
11.5299999999998 -0.000414963392011692
11.5399999999998 -0.000384328285003565
11.5499999999998 -0.00035334666535805
11.5599999999998 -0.0003220592512707
11.5699999999998 -0.000290506963003237
11.5799999999998 -0.000258730870265259
11.5899999999998 -0.000226772139639505
11.5999999999998 -0.000194671982118118
11.6099999999998 -0.000162471600817
11.6199999999998 -0.000130212138934117
11.6299999999998 -9.79346280171962e-05
11.6399999999998 -6.56799366052351e-05
11.6499999999998 -3.348871930742e-05
11.6599999999998 -1.40136638221108e-06
11.6699999999998 3.05420461217303e-05
11.6799999999998 6.23018056013361e-05
11.6899999999998 9.38386114545741e-05
11.6999999999998 0.000125113622894027
11.7099999999998 0.00015608850596292
11.7199999999998 0.000186725479697476
11.7299999999998 0.000216987361381003
11.7399999999998 0.000246837610836535
11.7499999999998 0.000276240373706411
11.7599999999998 0.000305160523668528
11.7699999999998 0.000333563703540704
11.7799999999998 0.000361416365226251
11.7899999999998 0.000388685808455418
11.7999999999998 0.000415340218279229
11.8099999999998 0.000441348701273977
11.8199999999998 0.00046668132040069
11.8299999999998 0.000491309128545068
11.8399999999998 0.000515204200607852
11.8499999999998 0.000538339664168755
11.8599999999998 0.000560689728683201
11.8699999999998 0.000582229713178955
11.8799999999998 0.00060293607242446
11.8899999999998 0.000622786421542853
11.8999999999998 0.000641759559047619
11.9099999999998 0.00065983548827808
11.9199999999998 0.000676995437214944
11.9299999999998 0.000693221876658407
11.9399999999998 0.00070849853675339
11.9499999999998 0.000722810421848765
11.9599999999998 0.000736143823679587
11.9699999999998 0.000748486332863608
11.9799999999998 0.000759826848705588
11.9899999999998 0.000770155587305133
11.9999999999998 0.000779464087966187
12.0099999999998 0.000787745218459989
12.0199999999998 0.000794993177015
12.0299999999998 0.000801203493973973
12.0399999999998 0.000806373031573689
12.0499999999998 0.000810499981918136
12.0599999999998 0.000813583863292891
12.0699999999998 0.000815625514837007
12.0799999999998 0.000816627089592465
12.0899999999998 0.000816592045955631
12.0999999999998 0.000815525137560909
12.1099999999998 0.000813432401634672
12.1199999999998 0.00081032114586907
12.1299999999998 0.000806199933883561
12.1399999999998 0.000801078569372492
12.1499999999998 0.000794968079092888
12.1599999999998 0.000787880694870428
12.1699999999998 0.000779829834262052
12.1799999999998 0.00077083008415309
12.1899999999998 0.000760897184964323
12.1999999999998 0.000750048024754471
12.2099999999998 0.000738300126486093
12.2199999999998 0.000725672386080596
12.2299999999998 0.000712184739546754
12.2399999999998 0.000697858120465698
12.2499999999998 0.000682714428759921
12.2599999999998 0.000666776498421705
12.2699999999998 0.000650068064240012
12.2799999999998 0.000632613712491771
12.2899999999998 0.000614438676816775
12.2999999999998 0.000595569229509482
12.3099999999998 0.000576032388523409
12.3199999999998 0.000555855885017776
12.3299999999998 0.000535068127753012
12.3399999999998 0.000513698165645348
12.3499999999998 0.000491775649083596
12.3599999999998 0.000469330790326007
12.3699999999998 0.000446394323168457
12.3799999999998 0.000422997462013633
12.3899999999998 0.000399171860439475
12.3999999999998 0.000374949569348031
12.4099999999998 0.000350362994765995
12.4199999999998 0.000325444855362654
12.4299999999998 0.000300228139747347
12.4399999999998 0.000274746063606199
12.4499999999998 0.000249032026736475
12.4599999999998 0.000223119570035758
12.4699999999998 0.000197042332502164
12.4799999999998 0.000170834008301292
12.4899999999998 0.000144528303954771
12.4999999999998 0.000118158895704687
12.5099999999998 9.17593871075887e-05
12.5199999999998 6.53632669110663e-05
12.5299999999998 3.90038672651812e-05
12.5399999999998 1.27143223203448e-05
12.5499999999998 -1.34724727375367e-05
12.5599999999998 -3.95239021644559e-05
12.5699999999998 -6.54076695997141e-05
12.5799999999998 -9.10918374229726e-05
12.5899999999998 -0.000116544865479221
12.5999999999998 -0.000141735649125977
12.6099999999998 -0.000166633556557545
12.6199999999998 -0.000191208465362316
12.6299999999998 -0.000215430798270437
12.6399999999998 -0.000239271558050149
12.6499999999998 -0.000262702361512677
12.6599999999998 -0.000285695472586933
12.6699999999998 -0.000308223834425647
12.6799999999998 -0.00033026110050673
12.6899999999998 -0.000351781664696648
12.6999999999998 -0.000372760690236389
12.7099999999998 -0.000393174137630943
12.7199999999998 -0.000412998791411805
12.7299999999998 -0.000432212285717449
12.7399999999998 -0.000450793128692548
12.7499999999998 -0.000468720725671753
12.7599999999998 -0.000485975401124327
12.7699999999998 -0.000502538419337567
12.7799999999998 -0.000518392003818632
12.7899999999998 -0.000533519355396173
12.7999999999998 -0.000547904669004873
12.8099999999998 -0.000561533149137797
12.8199999999998 -0.000574391023953152
12.8299999999998 -0.000586465558023957
12.8399999999998 -0.000597745063720844
12.8499999999998 -0.000608218911220027
12.8599999999998 -0.000617877537130377
12.8699999999998 -0.000626712451735285
12.8799999999998 -0.000634716244846936
12.8899999999998 -0.000641882590456069
12.8999999999998 -0.000648206249735277
12.9099999999998 -0.000653683072290751
12.9199999999998 -0.000658309996810792
12.9299999999998 -0.000662085050014318
12.9399999999998 -0.000665007344238559
12.9499999999998 -0.000667077073678104
12.9599999999998 -0.000668295509290251
12.9699999999998 -0.000668664992384906
12.9799999999998 -0.000668188926921394
12.9899999999998 -0.000666871770540051
12.9999999999998 -0.000664719024364269
13.0099999999998 -0.000661737221620589
13.0199999999998 -0.000657933915143796
13.0299999999998 -0.000653317663867968
13.0399999999998 -0.00064789801846748
13.0499999999998 -0.000641685506439323
13.0599999999998 -0.000634691617198835
13.0699999999998 -0.00062692878844871
13.0799999999998 -0.000618410397161727
13.0899999999998 -0.000609150547372807
13.0999999999998 -0.000599164155265655
13.1099999999998 -0.000588467135408715
13.1199999999998 -0.000577076228955665
13.1299999999998 -0.000565008978514589
13.1399999999998 -0.000552283702152224
13.1499999999998 -0.000538919466564531
13.1599999999998 -0.000524936059446805
13.1699999999998 -0.000510353857985212
13.1799999999998 -0.00049519395333336
13.1899999999998 -0.00047947815310164
13.1999999999998 -0.000463228860010652
13.2099999999998 -0.00044646904385846
13.2199999999998 -0.000429222211560478
13.2299999999998 -0.000411512375975141
13.2399999999998 -0.000393364023867637
13.2499999999998 -0.000374802083210319
13.2599999999998 -0.000355851889946886
13.2699999999998 -0.00033653915431163
13.2799999999998 -0.000316889926776154
13.2899999999998 -0.000296930563685383
13.2999999999998 -0.000276687692638663
13.3099999999998 -0.000256188177668025
13.3199999999998 -0.000235459084263311
13.3299999999998 -0.000214527644292453
13.3399999999998 -0.000193421220863999
13.3499999999998 -0.000172167273178294
13.3599999999998 -0.000150793321412969
13.3699999999998 -0.000129326911687925
13.3799999999998 -0.000107795581154347
13.3899999999998 -8.62268232519189e-05
13.3999999999998 -6.46480531777433e-05
13.4099999999998 -4.30865736099457e-05
13.4199999999998 -2.15695407284658e-05
13.4299999999998 -1.23930574753726e-07
13.4399999999998 2.12234942084823e-05
13.4499999999998 4.24462172166787e-05
13.4599999999998 6.35180006633231e-05
13.4699999999998 8.44129172601673e-05
13.4799999999998 0.000105105381560324
13.4899999999998 0.000125570180726871
13.4999999999998 0.000145782504690656
13.5099999999998 0.00016571797566191
13.5199999999998 0.000185352676961279
13.5299999999998 0.000204663181136887
13.5399999999998 0.000223626577335153
13.5499999999998 0.000242220497894078
13.5599999999998 0.000260423144129003
13.5699999999998 0.000278213311281852
13.5799999999998 0.000295570412606146
13.5899999999998 0.000312474502554216
13.5999999999998 0.000328906299070244
13.6099999999998 0.000344847204921847
13.6199999999998 0.000360279328074798
13.6299999999998 0.000375185501084747
13.6399999999998 0.000389549299484965
13.6499999999998 0.000403355059151462
13.6599999999998 0.000416587892628225
13.6699999999998 0.000429233704396786
13.6799999999998 0.000441279205075615
13.6899999999998 0.000452711924536411
13.6999999999998 0.000463520223925712
13.7099999999998 0.000473693306581745
13.7199999999998 0.000483221227837876
13.7299999999998 0.000492094903705548
13.7399999999998 0.000500306118431022
13.7499999999998 0.000507847530921774
13.7599999999998 0.00051471268003992
13.7699999999998 0.000520895988761528
13.7799999999998 0.000526392767518287
13.7899999999998 0.000531199215464227
13.7999999999998 0.000535312421513307
13.8099999999998 0.00053873036401838
13.8199999999997 0.000541451909299712
13.8299999999997 0.000543476809066968
13.8399999999997 0.000544805696745794
13.8499999999997 0.000545440082722629
13.8599999999997 0.000545382348524372
13.8699999999997 0.000544635739953442
13.8799999999997 0.000543204359204115
13.8899999999997 0.000541093155993982
13.8999999999997 0.000538307917756852
13.9099999999997 0.000534855258964528
13.9199999999997 0.000530742609683293
13.9299999999997 0.00052597820354462
13.9399999999997 0.000520571065466293
13.9499999999997 0.000514530999822583
13.9599999999997 0.000507868580715254
13.9699999999997 0.000500595127720137
13.9799999999997 0.00049272239865299
13.9899999999997 0.00048426307583165
13.9999999999997 0.000475230525885534
14.0099999999997 0.000465638779558102
14.0199999999997 0.000455502510790686
14.0299999999997 0.000444837015112752
14.0399999999997 0.00043365818736531
14.0499999999997 0.000421982490879879
14.0599999999997 0.000409826814616371
14.0699999999997 0.000397208740889301
14.0799999999997 0.000384146347554886
14.0899999999997 0.000370658186273882
14.0999999999997 0.000356763258689438
14.1099999999997 0.000342480991384212
14.1199999999997 0.000327831210016088
14.1299999999997 0.000312834112843845
14.1399999999997 0.000297510243770106
14.1499999999997 0.000281880464988202
14.1599999999997 0.000265965929298539
14.1699999999997 0.000249788052148692
14.1799999999997 0.000233368483444971
14.1899999999997 0.000216729079178962
14.1999999999997 0.000199891872911196
14.2099999999997 0.000182879047151379
14.2199999999997 0.000165712904674274
14.2299999999997 0.000148415839809358
14.2399999999997 0.000131010309741863
14.2499999999997 0.000113518805862263
14.2599999999997 9.59638252008713e-05
14.2699999999997 7.83678419837327e-05
14.2799999999997 6.075327934559e-05
14.2899999999997 4.31424812352525e-05
14.2999999999997 2.55576845482626e-05
14.3099999999997 8.02099152118917e-06
14.3199999999997 -9.44565757858089e-06
14.3299999999997 -2.68205114341666e-05
14.3399999999997 -4.40820342991861e-05
14.3499999999997 -6.12089322241658e-05
14.3599999999997 -7.81801788579029e-05
14.3699999999997 -9.4975040793098e-05
14.3799999999997 -0.000111573102426291
14.3899999999997 -0.000127954290302784
14.3999999999997 -0.000144098896918226
14.4099999999997 -0.000159987603949039
14.4199999999997 -0.000175601504885103
14.4299999999997 -0.000190922127038651
14.4399999999997 -0.00020593145290449
14.4499999999997 -0.000220611940847436
14.4599999999997 -0.000234946545093928
14.4699999999997 -0.000248918735003361
14.4799999999997 -0.000262512513604963
14.4899999999997 -0.000275712435376946
14.4999999999997 -0.000288503623240529
14.5099999999997 -0.000300871784761494
14.5199999999997 -0.000312803227538963
14.5299999999997 -0.000324284873765623
14.5399999999997 -0.000335304273944892
14.5499999999997 -0.000345849619751499
14.5599999999997 -0.000355909756023239
14.5699999999997 -0.00036547419187273
14.5799999999997 -0.00037453311090925
14.5899999999997 -0.000383077380561859
14.5999999999997 -0.000391098560496256
14.6099999999997 -0.000398588910118968
14.6199999999997 -0.00040554139516374
14.6299999999997 -0.000411949693356171
14.6399999999997 -0.00041780819915389
14.6499999999997 -0.000423112027560819
14.6599999999997 -0.000427857017125596
14.6699999999997 -0.000432039731807685
14.6799999999997 -0.000435657461857833
14.6899999999997 -0.000438708224082937
14.6999999999997 -0.000441190761054182
14.7099999999997 -0.000443104539408011
14.7199999999997 -0.000444449747248244
14.7299999999997 -0.000445227290659507
14.7399999999997 -0.000445438789344384
14.7499999999997 -0.000445086571399511
14.7599999999997 -0.000444173667249582
14.7699999999997 -0.000442703802763609
14.7799999999997 -0.000440681391586029
14.7899999999997 -0.000438111526728665
14.7999999999997 -0.000434999971493248
14.8099999999997 -0.000431353149838365
14.8199999999997 -0.000427178136394454
14.8299999999997 -0.000422482646529211
14.8399999999997 -0.000417275027356721
14.8499999999997 -0.000411564252068164
14.8599999999997 -0.000405359751186129
14.8699999999997 -0.000398671537729897
14.8799999999997 -0.000391510272044295
14.8899999999997 -0.000383887164272268
14.8999999999997 -0.000375813957542335
14.9099999999997 -0.00036730291057942
14.9199999999997 -0.000358366779760519
14.9299999999997 -0.000349018800637729
14.9399999999997 -0.000339272599088553
14.9499999999997 -0.000329142278597994
14.9599999999997 -0.000318642416260387
14.9699999999997 -0.000307787984310291
14.9799999999997 -0.000296594331337459
14.9899999999997 -0.000285077162245733
14.9999999999997 -0.000273252517416017
15.0099999999997 -0.000261136751302373
15.0199999999997 -0.000248746510591161
15.0299999999997 -0.000236098712006838
15.0399999999997 -0.000223210519824795
15.0499999999997 -0.000210099323139228
15.0599999999997 -0.000196782712927191
15.0699999999997 -0.000183278458945897
15.0799999999997 -0.000169604486498003
15.0899999999997 -0.000155778853098012
15.0999999999997 -0.00014181972507195
15.1099999999997 -0.00012774535412175
15.1199999999997 -0.000113574053885211
15.1299999999997 -9.93241765220583e-05
15.1399999999997 -8.50140893561023e-05
15.1499999999997 -7.06621516032854e-05
15.1599999999997 -5.62866912149803e-05
15.1699999999997 -4.19059818654979e-05
15.1799999999997 -2.75382201125308e-05
15.1899999999997 -1.32015027587111e-05
15.1999999999997 1.08619555785604e-06
15.2099999999997 1.53070445166859e-05
15.2199999999997 2.94433799846837e-05
15.2299999999997 4.347772557639e-05
15.2399999999997 5.73928138784727e-05
15.2499999999997 7.11716073133438e-05
15.2599999999997 8.47973186170318e-05
15.2699999999997 9.82534309072437e-05
15.2799999999997 0.00011152371731801
15.2899999999997 0.000124592260178115
15.2999999999997 0.000137443469711101
15.3099999999997 0.000150062102235414
15.3199999999997 0.000162433277843905
15.3299999999997 0.000174542497542792
15.3399999999997 0.000186375659830808
15.3499999999997 0.000197919076700188
15.3599999999997 0.000209159489038473
15.3699999999997 0.000220084081428708
15.3799999999997 0.000230680496309742
15.3899999999997 0.000240936847497108
15.3999999999997 0.00025084173304633
15.4099999999997 0.00026038424744558
15.4199999999997 0.000269553993125176
15.4299999999997 0.00027834109127258
15.4399999999997 0.000286736191942486
15.4499999999997 0.000294730483452417
15.4599999999997 0.000302315701055311
15.4699999999997 0.000309484134881529
15.4799999999997 0.000316228637143604
15.4899999999997 0.000322542628598155
15.4999999999997 0.000328420104260284
15.5099999999997 0.000333855638366791
15.5199999999997 0.000338844388585564
15.5299999999997 0.000343382099469487
15.5399999999997 0.000347465105154237
15.5499999999997 0.000351090331464411
15.5599999999997 0.000354255296762022
15.5699999999997 0.000356958112552326
15.5799999999997 0.000359197483175658
15.5899999999997 0.000360972704752772
15.5999999999997 0.000362283663393191
15.6099999999997 0.000363130832674143
15.6199999999997 0.000363515270399348
15.6299999999997 0.000363438614648972
15.6399999999997 0.000362903079134718
15.6499999999997 0.000361911447877712
15.6599999999997 0.000360467069232354
15.6699999999997 0.000358573849287954
15.6799999999997 0.000356236244694802
15.6899999999997 0.000353459254988175
15.6999999999997 0.000350248414536189
15.7099999999997 0.000346609784348866
15.7199999999997 0.000342549944246512
15.7299999999997 0.000338075986578952
15.7399999999997 0.000333195478828543
15.7499999999997 0.00032791631303948
15.7599999999997 0.000322246982511102
15.7699999999997 0.000316196435345503
15.7799999999997 0.000309774060945842
15.7899999999997 0.000302989676035517
15.7999999999997 0.000295853510215451
15.8099999999997 0.000288376191077714
15.8199999999997 0.000280568719914616
15.8299999999997 0.000272442389506248
15.8399999999997 0.000264008947049981
15.8499999999997 0.000255280468634137
15.8599999999997 0.000246269344619837
15.8699999999997 0.000236988263689352
15.8799999999997 0.000227450196105225
15.8899999999997 0.000217668376434334
15.8999999999997 0.000207656285872788
15.9099999999997 0.000197427634254188
15.9199999999997 0.000186996341797874
15.9299999999997 0.000176376520640289
15.9399999999997 0.000165582456185285
15.9499999999997 0.000154628588304979
15.9599999999997 0.00014352949242031
15.9699999999997 0.00013229986048885
15.9799999999997 0.000120954481926492
15.9899999999997 0.000109508224488883
15.9999999999997 9.79760151380422e-05
16.0099999999997 8.6372820919174e-05
16.0199999999997 7.47136298723655e-05
16.0299999999997 6.30134320035475e-05
16.0399999999997 5.12872003388595e-05
16.0499999999997 3.95498720862197e-05
16.0599999999997 2.78163299275566e-05
16.0699999999997 1.61013834649946e-05
16.0799999999997 4.41975084383428e-06
16.0899999999997 -7.21395942522759e-06
16.0999999999997 -1.87852664223604e-05
16.1099999999997 -3.02798345317034e-05
16.1199999999997 -4.16834908900279e-05
16.1299999999997 -5.29822425498677e-05
16.1399999999997 -6.41622933369173e-05
16.1499999999997 -7.5210060381574e-05
16.1599999999997 -8.61121903053639e-05
16.1699999999997 -9.68555750432135e-05
16.1799999999997 -0.000107427367283297
16.1899999999997 -0.000117814995506663
16.1999999999997 -0.000128006178609458
16.2099999999997 -0.000137988940091125
16.2199999999997 -0.000147751621792717
16.2299999999997 -0.000157282897169914
16.2399999999997 -0.000166571784084938
16.2499999999997 -0.000175607657107521
16.2599999999997 -0.000184380259307848
16.2699999999997 -0.000192879713527718
16.2799999999997 -0.000201096533121402
16.2899999999997 -0.00020902163215382
16.2999999999997 -0.000216646335045607
16.3099999999998 -0.000223962385655478
16.3199999999998 -0.000230961955790988
16.3299999999998 -0.000237637653139606
16.3399999999998 -0.000243982528612737
16.3499999999998 -0.000249990083096183
16.3599999999998 -0.00025565427360128
16.3699999999998 -0.000260969518811745
16.3799999999998 -0.00026593070402207
16.3899999999998 -0.000270533185464141
16.3999999999998 -0.000274772794019525
16.4099999999998 -0.000278645838315718
16.4199999999998 -0.000282149107205478
16.4299999999998 -0.000285279871688387
16.4399999999998 -0.000288035886087828
16.4499999999998 -0.000290415388632699
16.4599999999998 -0.000292417101535751
16.4699999999998 -0.000294040230405254
16.4799999999998 -0.000295284463052803
16.4899999999998 -0.000296149967702934
16.4999999999998 -0.000296637390611438
16.5099999999998 -0.000296747853100793
16.5199999999998 -0.000296482948023044
16.5299999999998 -0.000295844735663009
16.5399999999998 -0.000294835739098396
16.5499999999998 -0.000293458939039077
16.5599999999998 -0.000291717768177102
16.5699999999998 -0.000289616105095518
16.5799999999998 -0.000287158267815051
16.5899999999998 -0.000284349007121042
16.5999999999998 -0.000281193499954307
16.6099999999998 -0.000277697343502155
16.6199999999998 -0.000273866551691125
16.6299999999998 -0.000269707422278754
16.6399999999998 -0.000265226672964003
16.6499999999998 -0.000260431437717867
16.6599999999998 -0.000255329215640533
16.6699999999998 -0.000249927859731138
16.6799999999998 -0.000244235565273478
16.6899999999998 -0.000238260857852414
16.6999999999998 -0.000232012581016237
16.7099999999998 -0.000225499835146731
16.7199999999998 -0.000218732042922583
16.7299999999998 -0.000211718937306798
16.7399999999998 -0.000204470513267915
16.7499999999998 -0.000196997015188277
16.7599999999998 -0.000189308923465667
16.7699999999998 -0.000181416940599212
16.7799999999998 -0.000173331976905628
16.7899999999998 -0.000165065135949577
16.7999999999998 -0.000156627699742412
16.8099999999998 -0.000148031113748886
16.8199999999998 -0.000139286971733351
16.8299999999998 -0.000130407000472676
16.8399999999998 -0.000121403044360502
16.8499999999998 -0.000112287049925779
16.8599999999998 -0.000103071050287724
16.8699999999998 -9.37671495684927e-05
16.8799999999998 -8.43875072845467e-05
16.8899999999998 -7.49443227371702e-05
16.8999999999998 -6.54498194225108e-05
16.9099999999998 -5.59162294811001e-05
16.9199999999998 -4.63557782066174e-05
16.9299999999998 -3.67806686334613e-05
16.9399999999998 -2.72030662224278e-05
16.9499999999999 -1.76350836635117e-05
16.9599999999999 -8.08876581463694e-06
16.9699999999999 1.42392520517798e-06
16.9799999999999 1.08911247500326e-05
16.9899999999999 2.03010801908268e-05
16.9999999999999 2.96421652508952e-05
17.0099999999999 3.8902894116356e-05
17.0199999999999 4.80719353043131e-05
17.0299999999999 5.71381252725339e-05
17.0399999999999 6.60904817545119e-05
17.0499999999999 7.49182168043149e-05
17.0599999999999 8.361074953607e-05
17.0699999999999 9.21577185433376e-05
17.0799999999999 0.000100548993984163
17.0899999999999 0.00010877468931804
17.0999999999999 0.000116825172681543
17.1099999999999 0.000124691077889941
17.1199999999999 0.000132363315052417
17.1299999999999 0.00013983308078846
17.1399999999999 0.000147091868039235
17.1499999999999 0.000154131475454973
17.1599999999999 0.000160944016354576
17.1699999999999 0.000167521927246503
17.1799999999999 0.00017385797590212
17.1899999999999 0.000179945268973432
17.1999999999999 0.000185777259147661
17.2099999999999 0.000191347751831784
17.2199999999999 0.000196650911360754
17.2299999999999 0.000201681266723796
17.2399999999999 0.000206433716803793
17.2499999999999 0.000210903535125419
17.2599999999999 0.000215086374108345
17.2699999999999 0.000218978268822478
17.2799999999999 0.000222575640242907
17.2899999999999 0.000225875298002796
17.2999999999999 0.000228874442643263
17.3099999999999 0.000231570667363172
17.3199999999999 0.00023396195932822
17.3299999999999 0.000236046700264384
17.3399999999999 0.000237823666777077
17.3499999999999 0.000239292030094811
17.3599999999999 0.000240451355318456
17.3699999999999 0.000241301600180291
17.3799999999999 0.000241843113317976
17.3899999999999 0.000242076632069718
17.3999999999999 0.000242003279798261
17.4099999999999 0.000241624562753162
17.4199999999999 0.000240942366483277
17.4299999999999 0.00023995895181518
17.4399999999999 0.000238676950419124
17.4499999999999 0.000237099359994379
17.4599999999999 0.000235229539124308
17.4699999999999 0.000233071201887964
17.4799999999999 0.000230628412392831
17.4899999999999 0.000227905579576868
17.4999999999999 0.000224907453120876
17.5099999999999 0.00022163908604422
17.5199999999999 0.000218105767557617
17.5299999999999 0.000214313177442644
17.5399999999999 0.00021026729736693
17.5499999999999 0.0002059744018726
17.5599999999999 0.000201441049045591
17.5699999999999 0.000196674070877678
17.5799999999999 0.000191680563333645
17.59 0.000186467868638302
17.6 0.000181043526729359
17.61 0.000175415375467569
17.62 0.000169591470738311
17.63 0.000163580076640777
17.64 0.00015738965482331
17.65 0.000151028853308551
17.66 0.000144506494970464
17.67 0.000137831565750479
17.68 0.000131013202666315
17.69 0.000124060681650378
17.7 0.000116983405246071
17.71 0.00010979089018569
17.72 0.000102492754870739
17.73 9.5098706774054e-05
17.74 8.76185297820071e-05
17.75 8.00620714944532e-05
17.76 7.24392304996463e-05
17.77 6.47599436410384e-05
17.78 5.70341732925301e-05
17.79 4.92718946586547e-05
17.8 4.14830831158681e-05
17.81 3.36777016109797e-05
17.82 2.58656881325781e-05
17.83 1.8056943271003e-05
17.84 1.02613178823891e-05
17.85 2.48860087187034e-06
17.86 -5.25149288903318e-06
17.87 -1.29493344961474e-05
17.88 -2.05953927830916e-05
17.89 -2.81802459142444e-05
17.9 -3.56945927863623e-05
17.91 -4.31292642253166e-05
17.92 -5.04752339646898e-05
17.93 -5.77236293933911e-05
17.94 -6.48657420596879e-05
17.95 -7.18930379195649e-05
17.96 -7.87971673175307e-05
17.97 -8.5569974688594e-05
17.98 -9.22035079703068e-05
17.99 -9.8690027714406e-05
18 -0.000105022015887863
18.01 -0.000111192184353053
18.02 -0.000117193483019977
18.03 -0.000123019107659289
18.04 -0.000128662507368214
18.05 -0.000134117391682432
18.06 -0.000139377737326168
18.07 -0.000144437794593653
18.08 -0.000149292093355541
18.09 -0.000153935448684492
18.1 -0.000158362966094545
18.11 -0.000162570046389495
18.12 -0.000166552390115936
18.13 -0.000170306001617235
18.14 -0.000173827192685178
18.15 -0.000177112585806588
18.16 -0.000180159117002739
18.17 -0.000182964038259944
18.18 -0.000185524919550225
18.19 -0.000187839650441539
18.2 -0.000189906441326759
18.21 -0.000191723824173938
18.22 -0.000193290652894754
18.2300000000001 -0.000194606103341759
18.2400000000001 -0.000195669672878467
18.2500000000001 -0.000196481179548034
18.2600000000001 -0.000197040760844361
18.2700000000001 -0.000197348872090255
18.2800000000001 -0.000197406284428341
18.2900000000001 -0.000197214082431657
18.3000000000001 -0.000196773661342609
18.3100000000001 -0.000196086723951449
18.3200000000001 -0.000195155277129272
18.3300000000001 -0.000193981628036892
18.3400000000001 -0.00019256838004215
18.3500000000001 -0.000190918428399485
18.3600000000001 -0.00018903495578914
18.3700000000001 -0.000186921427911147
18.3800000000001 -0.000184581589575068
18.3900000000001 -0.000182019462467739
18.4000000000001 -0.000179239246128863
18.4100000000001 -0.000176245437079746
18.4200000000001 -0.000173042800039701
18.4300000000001 -0.000169636341909171
18.4400000000001 -0.000166031304281529
18.4500000000001 -0.000162233155699388
18.4600000000001 -0.00015824758366545
18.4700000000001 -0.000154080486418322
18.4800000000001 -0.000149737932418223
18.4900000000001 -0.000145226204681209
18.5000000000001 -0.000140551790402309
18.5100000000001 -0.000135721350306954
18.5200000000001 -0.000130741710229271
18.5300000000001 -0.000125619852170882
18.5400000000001 -0.000120362905025091
18.5500000000001 -0.000114978135060073
18.5600000000001 -0.000109472936215084
18.5700000000001 -0.00010385482024506
18.5800000000001 -9.81314067394107e-05
18.5900000000001 -9.23104130358379e-05
18.6000000000001 -8.63996440470726e-05
18.6100000000001 -8.0406982016876e-05
18.6200000000001 -7.4340376220504e-05
18.6300000000001 -6.82078326242989e-05
18.6400000000001 -6.20174035185766e-05
18.6500000000001 -5.57771771377054e-05
18.6600000000001 -4.94952672810228e-05
18.6700000000001 -4.31798029480541e-05
18.6800000000001 -3.68389180013242e-05
18.6900000000001 -3.04807408698945e-05
18.7000000000001 -2.41133843066018e-05
18.7100000000001 -1.77449352118121e-05
18.7200000000001 -1.13834445363402e-05
18.7300000000001 -5.03691727599343e-06
18.7400000000001 1.28669742995898e-06
18.7500000000001 7.57951608433873e-06
18.7600000000001 1.38337304897279e-05
18.7700000000001 2.00416172666491e-05
18.7800000000001 2.61955472207966e-05
18.7900000000001 3.22879945481819e-05
18.8000000000001 3.83115458672909e-05
18.8100000000001 4.42589090675972e-05
18.8200000000001 5.01229219641109e-05
18.8300000000001 5.58965607478972e-05
18.8400000000001 6.15729482227965e-05
18.8500000000001 6.71453618189511e-05
18.8600000000001 7.26072413740148e-05
18.8700000000002 7.7952196673269e-05
18.8800000000002 8.31740147402523e-05
18.8900000000002 8.82666668697166e-05
18.9000000000002 9.32243153950786e-05
18.9100000000002 9.80413201846984e-05
18.9200000000002 0.000102712244856848
18.9300000000002 0.000107231862709263
18.9400000000002 0.000111595162356403
18.9500000000002 0.000115797353068666
18.9600000000002 0.000119833869808187
18.9700000000002 0.000123700377956294
18.9800000000002 0.000127392777728043
18.9900000000002 0.000130907208269752
19.0000000000002 0.000134240051435817
19.0100000000002 0.000137387935241551
19.0200000000002 0.00014034773698921
19.0300000000002 0.000143116586064803
19.0400000000002 0.000145691866403723
19.0500000000002 0.000148071218623661
19.0600000000002 0.000150252541823748
19.0700000000002 0.000152233995049248
19.0800000000002 0.000154013998424881
19.0900000000002 0.000155591233973516
19.1000000000002 0.000156964646019808
19.1100000000002 0.000158133441351855
19.1200000000002 0.000159097089020378
19.1300000000002 0.000159855319809487
19.1400000000002 0.000160408125381852
19.1500000000002 0.000160755757101743
19.1600000000002 0.000160898724540136
19.1700000000002 0.000160837793667034
19.1800000000002 0.000160573984737301
19.1900000000002 0.000160108569878045
19.2000000000002 0.000159443070388051
19.2100000000002 0.000158579253763777
19.2200000000002 0.000157519130473308
19.2300000000002 0.000156264950512201
19.2400000000002 0.00015481919979993
19.2500000000002 0.000153184596528792
19.2600000000002 0.000151364087703279
19.2700000000002 0.000149360846449397
19.2800000000002 0.000147178240141636
19.2900000000002 0.000144819805392293
19.3000000000002 0.000142289332193134
19.3100000000002 0.000139590810633388
19.3200000000002 0.000136728424896922
19.3300000000002 0.000133706547047066
19.3400000000002 0.000130529730607248
19.3500000000002 0.000127202703945865
19.3600000000002 0.000123730358258664
19.3700000000002 0.000120117717365813
19.3800000000002 0.00011637000055443
19.3900000000002 0.000112492571461101
19.4000000000002 0.000108490931501085
19.4100000000002 0.000104370712750969
19.4200000000002 0.000100137670502039
19.4300000000002 9.57976755874988e-05
19.4400000000002 9.13567065396012e-05
19.4500000000002 8.68208416112591e-05
19.4600000000002 8.21962506861712e-05
19.4700000000002 7.74891870960609e-05
19.4800000000002 7.27059793605423e-05
19.4900000000002 6.78530228634217e-05
19.5000000000002 6.29367714782225e-05
19.5100000000003 5.79637291550611e-05
19.5200000000003 5.29404414805778e-05
19.5300000000003 4.78734872223441e-05
19.5400000000003 4.27694698689516e-05
19.5500000000003 3.763500917684e-05
19.5600000000003 3.24767327347129e-05
19.5700000000003 2.73012675563435e-05
19.5800000000003 2.21152317123977e-05
19.5900000000003 1.69252260117349e-05
19.6000000000003 1.17378257426155e-05
19.6100000000003 6.55957248400141e-06
19.6200000000003 1.39696599701296e-06
19.6300000000003 -3.74354379347316e-06
19.6400000000003 -8.85556471740444e-06
19.6500000000003 -1.39327701687081e-05
19.6600000000003 -1.89689067956227e-05
19.6700000000003 -2.3957802063027e-05
19.6800000000003 -2.88933716778071e-05
19.6900000000003 -3.37696268684445e-05
19.7000000000003 -3.85806815103429e-05
19.7100000000003 -4.33207590885622e-05
19.7200000000003 -4.7984199489896e-05
19.7300000000003 -5.25654656165312e-05
19.7400000000003 -5.70591498137036e-05
19.7500000000003 -6.14599801041074e-05
19.7600000000003 -6.57628262220793e-05
19.7700000000003 -6.99627054408245e-05
19.7800000000003 -7.40547881860271e-05
19.7900000000003 -7.80344034307656e-05
19.8000000000003 -8.18970438646185e-05
19.8100000000003 -8.56383708319395e-05
19.8200000000003 -8.92542190343569e-05
19.8300000000003 -9.27406009925156e-05
19.8400000000003 -9.60937112625283e-05
19.8500000000003 -9.93099304029623e-05
19.8600000000003 -0.000102385828688503
19.8700000000003 -0.000105318169566793
19.8800000000003 -0.000108103912855272
19.8900000000003 -0.000110740217675206
19.9000000000003 -0.000113224445120443
19.9100000000003 -0.000115554160658764
19.9200000000003 -0.000117727136264073
19.9300000000003 -0.000119741352278032
19.9400000000003 -0.000121594999000077
19.9500000000003 -0.000123286478005144
19.9600000000003 -0.000124814403188784
19.9700000000003 -0.000126177601553098
19.9800000000003 -0.000127375113687212
19.9900000000003 -0.000128406193994064
20.0000000000003 -0.000129270310658016
20.0100000000003 -0.000129967145335412
20.0200000000003 -0.000130496592578733
20.0300000000003 -0.000130858758996913
20.0400000000003 -0.000131053962154922
20.0500000000003 -0.000131082729216439
20.0600000000003 -0.000130945795334215
20.0700000000003 -0.000130644101793956
20.0800000000003 -0.000130178793919129
20.0900000000003 -0.000129551218746715
20.1000000000003 -0.000128762922488146
20.1100000000003 -0.000127815647797177
20.1200000000003 -0.000126711330880739
20.1300000000003 -0.000125452098518227
20.1400000000003 -0.000124040265120973
20.1500000000004 -0.00012247833013151
20.1600000000004 -0.000120768976567812
20.1700000000004 -0.000118914999710049
20.1800000000004 -0.000116919402216504
20.1900000000004 -0.000114785359502098
20.2000000000004 -0.000112516207363871
20.2100000000004 -0.000110115436998156
20.2200000000004 -0.000107586689847971
20.2300000000004 -0.00010493375228748
20.2400000000004 -0.000102160550150589
20.2500000000004 -9.92711224671312e-05
20.2600000000004 -9.62696500817623e-05
20.2700000000004 -9.31604482110503e-05
20.2800000000004 -8.99479468489516e-05
20.2900000000004 -8.66366851429855e-05
20.3000000000004 -8.32313054372806e-05
20.3100000000004 -7.97365470993919e-05
20.3200000000004 -7.61572401905647e-05
20.3300000000004 -7.24982990141284e-05
20.3400000000004 -6.87647155648821e-05
20.3500000000004 -6.49615528963233e-05
20.3600000000004 -6.10939384193158e-05
20.3700000000004 -5.71670571440433e-05
20.3800000000004 -5.31861448759447e-05
20.3900000000004 -4.91564813757578e-05
20.4000000000004 -4.50833834933321e-05
20.4100000000004 -4.09721982846204e-05
20.4200000000004 -3.68282961210325e-05
20.4300000000004 -3.26570638002265e-05
20.4400000000004 -2.84638976672418e-05
20.4500000000004 -2.42541967547885e-05
20.4600000000004 -2.0033355951413e-05
20.4700000000004 -1.5806759206121e-05
20.4800000000004 -1.15797727779786e-05
20.4900000000004 -7.35773853903639e-06
20.5000000000004 -3.14596733888618e-06
20.5100000000004 1.05026756105539e-06
20.5200000000004 5.2257369754957e-06
20.5300000000004 9.37526219762155e-06
20.5400000000004 1.34937213080996e-05
20.5500000000004 1.75760553832051e-05
20.5600000000004 2.16172745946684e-05
20.5700000000004 2.56124641940092e-05
20.5800000000004 2.9556790374347e-05
20.5900000000004 3.34455060028219e-05
20.6000000000004 3.72739562169823e-05
20.6100000000004 4.10375838786684e-05
20.6200000000004 4.47319348791976e-05
20.6300000000004 4.8352663289794e-05
20.6400000000004 5.18955363514879e-05
20.6500000000004 5.53564392989092e-05
20.6600000000004 5.87313800126096e-05
20.6700000000004 6.2016493494824e-05
20.6800000000004 6.52080461643742e-05
20.6900000000004 6.83024399649858e-05
20.7000000000004 7.12962162836003e-05
20.7100000000004 7.41860596743443e-05
20.7200000000004 7.69688013843807e-05
20.7300000000004 7.96414226780877e-05
20.7400000000004 8.22010579563222e-05
20.7500000000004 8.46449976677753e-05
20.7600000000004 8.69706910097212e-05
20.7700000000004 8.91757484157308e-05
20.7800000000004 9.12579438282193e-05
20.7900000000005 9.32152167539697e-05
20.8000000000005 9.50456741010626e-05
20.8100000000005 9.6747591795947e-05
20.8200000000005 9.83194161796569e-05
20.8300000000005 9.97597651824875e-05
20.8400000000005 0.000101067429276741
20.8500000000005 0.000102241372209384
20.8600000000005 0.000103280731518638
20.8700000000005 0.000104184818799085
20.8800000000005 0.000104953119781424
20.8900000000005 0.000105585294180307
20.9000000000005 0.000106081175324381
20.9100000000005 0.000106440769570435
20.9200000000005 0.000106664255503957
20.9300000000005 0.000106751982928916
20.9400000000005 0.00010670447165018
20.9500000000005 0.000106522410052777
20.9600000000005 0.000106206653483313
20.9700000000005 0.000105758222440515
20.9800000000005 0.000105178300584513
20.9900000000005 0.000104468232579035
21.0000000000005 0.000103629521789005
21.0100000000005 0.000102663827872589
21.0200000000005 0.00010157296434226
21.0300000000005 0.000100358896254364
21.0400000000005 9.90237384184577e-05
21.0500000000005 9.75697303998404e-05
21.0600000000005 9.59992315655516e-05
21.0700000000005 9.43147653378306e-05
21.0800000000005 9.25189875501657e-05
21.0900000000005 9.06146824571283e-05
21.1000000000005 8.86047586031081e-05
21.1100000000005 8.64922445555236e-05
21.1200000000005 8.42802845082531e-05
21.1300000000005 8.19721304245578e-05
21.1400000000005 7.95711227941133e-05
21.1500000000005 7.70807301069743e-05
21.1600000000005 7.45045162887325e-05
21.1700000000005 7.18461363062874e-05
21.1800000000005 6.91093314256162e-05
21.1900000000005 6.62979242580273e-05
21.2000000000005 6.34158136601908e-05
21.2100000000005 6.04669695236705e-05
21.2200000000005 5.74554274761751e-05
21.2300000000005 5.43852835101166e-05
21.2400000000005 5.12606885505568e-05
21.2500000000005 4.80858429727575e-05
21.2600000000005 4.48649910783849e-05
21.2700000000005 4.16024155388188e-05
21.2800000000005 3.83024318135781e-05
21.2900000000005 3.49693825515926e-05
21.3000000000005 3.16076319828969e-05
21.3100000000005 2.82215603081551e-05
21.3200000000005 2.48155580933184e-05
21.3300000000005 2.13940206766564e-05
21.3400000000005 1.79613425952537e-05
21.3500000000005 1.45219120380454e-05
21.3600000000005 1.10801053323242e-05
21.3700000000005 7.64028147058629e-06
21.3800000000005 4.20677668451274e-06
21.3900000000005 7.83899072711053e-07
21.4000000000005 -2.62407671115906e-06
21.4100000000005 -6.01291470347833e-06
21.4200000000005 -9.37842280259691e-06
21.4300000000006 -1.27164578614149e-05
21.4400000000006 -1.60229306945262e-05
21.4500000000006 -1.92938109939844e-05
21.4600000000006 -2.25251321478999e-05
21.4700000000006 -2.57129959562092e-05
21.4800000000006 -2.8853577238142e-05
21.4900000000006 -3.19431283260604e-05
21.5000000000006 -3.49779834405057e-05
21.5100000000006 -3.79545629414855e-05
21.5200000000006 -4.08693774511815e-05
21.5300000000006 -4.37190318434933e-05
21.5400000000006 -4.65002290959582e-05
21.5500000000006 -4.92097739997259e-05
21.5600000000006 -5.18445767240138e-05
21.5700000000006 -5.44016562305932e-05
21.5800000000006 -5.68781435349655e-05
21.5900000000006 -5.92712848108822e-05
21.6000000000006 -6.1578444334948e-05
21.6100000000006 -6.37971072683553e-05
21.6200000000006 -6.59248822729711e-05
21.6300000000006 -6.79595039592719e-05
21.6400000000006 -6.98988351638062e-05
21.6500000000006 -7.17408690541074e-05
21.6600000000006 -7.34837310592297e-05
21.6700000000006 -7.51256806242705e-05
21.6800000000006 -7.66651127875097e-05
21.6900000000006 -7.81005595790251e-05
21.7000000000006 -7.94306912398612e-05
21.7100000000006 -8.06543172610939e-05
21.7200000000006 -8.17703872423507e-05
21.7300000000006 -8.27779915695995e-05
21.7400000000006 -8.36763619180614e-05
21.7500000000006 -8.44648715597963e-05
21.7600000000006 -8.51430355007051e-05
21.7700000000006 -8.57105104416448e-05
21.7800000000006 -8.61670945584891e-05
21.7900000000006 -8.65127271056826e-05
21.8000000000006 -8.6747487844998e-05
21.8100000000006 -8.68715963015787e-05
21.8200000000006 -8.68854108497832e-05
21.8300000000006 -8.6789427631904e-05
21.8400000000006 -8.65842793135861e-05
21.8500000000006 -8.62707336808482e-05
21.8600000000006 -8.58496920852894e-05
21.8700000000006 -8.53221877468368e-05
21.8800000000006 -8.46893839283287e-05
21.8900000000006 -8.3952572005619e-05
21.9000000000006 -8.31131694763035e-05
21.9100000000006 -8.21727179940996e-05
21.9200000000006 -8.11328816277261e-05
21.9300000000006 -7.99954458799011e-05
21.9400000000006 -7.87622680294679e-05
21.9500000000006 -7.74353480965704e-05
21.9600000000006 -7.60167975243658e-05
21.9700000000006 -7.45088337879745e-05
21.9800000000006 -7.29137770859865e-05
21.9900000000006 -7.12340469193063e-05
22.0000000000006 -6.94721585620026e-05
22.0100000000006 -6.76307194289456e-05
22.0200000000006 -6.57124124588267e-05
22.0300000000006 -6.37200136333051e-05
22.0400000000006 -6.1656387362588e-05
22.0500000000006 -5.95244738624397e-05
22.0600000000006 -5.73272854049951e-05
22.0700000000007 -5.5067902357012e-05
22.0800000000007 -5.27494690790553e-05
22.0900000000007 -5.0375189723396e-05
22.1000000000007 -4.79483239527667e-05
22.1100000000007 -4.5472182594667e-05
22.1200000000007 -4.29501232421561e-05
22.1300000000007 -4.03855458100005e-05
22.1400000000007 -3.77818880539324e-05
22.1500000000007 -3.51426210600699e-05
22.1600000000007 -3.24712447111605e-05
22.1700000000007 -2.97712831360385e-05
22.1800000000007 -2.70462801484971e-05
22.1900000000007 -2.42997946816707e-05
22.2000000000007 -2.15353962239033e-05
22.2100000000007 -1.87566602620202e-05
22.2200000000007 -1.59671637378122e-05
22.2300000000007 -1.31704805235129e-05
22.2400000000007 -1.03701769219492e-05
22.2500000000007 -7.56980719699162e-06
22.2600000000007 -4.77290913984015e-06
22.2700000000007 -1.98299967661828e-06
22.2800000000007 7.96429477346643e-07
22.2900000000007 3.56191611125199e-06
22.3000000000007 6.31003175537634e-06
22.3100000000007 9.03738585534724e-06
22.3200000000007 1.17406298791333e-05
22.3300000000007 1.44164613518656e-05
22.3400000000007 1.70616278137331e-05
22.3500000000007 1.96729306962883e-05
22.3600000000007 2.2247229112639e-05
22.3700000000007 2.47814435571564e-05
22.3800000000007 2.72725595104041e-05
22.3900000000007 2.97176309452002e-05
22.4000000000007 3.21137837298374e-05
22.4100000000007 3.44582189246163e-05
22.4200000000007 3.67482159680504e-05
22.4300000000007 3.89811357491804e-05
22.4400000000007 4.11544235626936e-05
22.4500000000007 4.3265611943807e-05
22.4600000000007 4.53123233794902e-05
22.4700000000007 4.72922728935318e-05
22.4800000000007 4.92032705026496e-05
22.4900000000007 5.10432235411639e-05
22.5000000000007 5.28101388519077e-05
22.5100000000007 5.45021248412256e-05
22.5200000000007 5.61173933961176e-05
22.5300000000007 5.76542616617382e-05
22.5400000000007 5.91111536776775e-05
22.5500000000007 6.04866018716173e-05
22.5600000000007 6.1779248409155e-05
22.5700000000007 6.29878463987786e-05
22.5800000000007 6.41112609511596e-05
22.5900000000007 6.51484700921384e-05
22.6000000000007 6.60985655289485e-05
22.6100000000007 6.69607532694482e-05
22.6200000000007 6.77343540952714e-05
22.6300000000007 6.84188038896604e-05
22.6400000000007 6.90136538077484e-05
22.6500000000007 6.95185703142477e-05
22.6600000000007 6.99333350710681e-05
22.6700000000007 7.02578446807672e-05
22.6800000000007 7.049211028709e-05
22.6900000000007 7.06362570341385e-05
22.7000000000007 7.06905233860301e-05
22.7100000000008 7.06552603092986e-05
22.7200000000008 7.0530930320807e-05
22.7300000000008 7.0318106404659e-05
22.7400000000008 7.00174708026647e-05
22.7500000000008 6.96298136846368e-05
22.7600000000008 6.91560317077353e-05
22.7700000000008 6.85971264794991e-05
22.7800000000008 6.79542029499467e-05
22.7900000000008 6.72284677813362e-05
22.8000000000008 6.64212277997778e-05
22.8100000000008 6.55338887855908e-05
22.8200000000008 6.45679371356373e-05
22.8300000000008 6.35249428236355e-05
22.8400000000008 6.24065819481899e-05
22.8500000000008 6.12146180750125e-05
22.8600000000008 5.99508995906215e-05
22.8700000000008 5.86173569624185e-05
22.8800000000008 5.7215999908966e-05
22.8900000000008 5.57489144843476e-05
22.9000000000008 5.42182581751611e-05
22.9100000000008 5.26262471924626e-05
22.9200000000008 5.09751814340625e-05
22.9300000000008 4.92674237989612e-05
22.9400000000008 4.75053972537704e-05
22.9500000000008 4.56915816787013e-05
22.9600000000008 4.38285105783846e-05
22.9700000000008 4.19187676985659e-05
22.9800000000008 3.9964983571303e-05
22.9900000000008 3.79698320028667e-05
23.0000000000008 3.59360265143669e-05
23.0100000000008 3.38663167429597e-05
23.0200000000008 3.17634848102843e-05
23.0300000000008 2.9630341664078e-05
23.0400000000008 2.74697233985279e-05
23.0500000000008 2.52844875586128e-05
23.0600000000008 2.30775094335669e-05
23.0700000000008 2.08516783444451e-05
23.0800000000008 1.86098939306852e-05
23.0900000000008 1.63550624405108e-05
23.1000000000008 1.40900930299253e-05
23.1100000000008 1.1817894075025e-05
23.1200000000008 9.5413695022661e-06
23.1300000000008 7.26341514130013e-06
23.1400000000008 4.9869151049045e-06
23.1500000000008 2.71473820048072e-06
23.1600000000008 4.49734377539264e-07
23.1700000000008 -1.8052687844987e-06
23.1800000000008 -4.04746954392028e-06
23.1900000000008 -6.27409541557026e-06
23.2000000000008 -8.48240653682553e-06
23.2100000000008 -1.06696989766906e-05
23.2200000000008 -1.28333079840908e-05
23.2300000000008 -1.49706111715194e-05
23.2400000000008 -1.70790316303345e-05
23.2500000000008 -1.91560409740758e-05
23.2600000000008 -2.11991623062854e-05
23.2700000000008 -2.32059731094396e-05
23.2800000000008 -2.5174108051692e-05
23.2900000000008 -2.71012617082902e-05
23.3000000000008 -2.89851911945956e-05
23.3100000000008 -3.08237187078059e-05
23.3200000000008 -3.26147339745499e-05
23.3300000000008 -3.43561966018811e-05
23.3400000000008 -3.60461383288947e-05
23.3500000000009 -3.76826651767141e-05
23.3600000000009 -3.92639594946162e-05
23.3700000000009 -4.07882819001838e-05
23.3800000000009 -4.22539731115206e-05
23.3900000000009 -4.36594556697285e-05
23.4000000000009 -4.50032355499906e-05
23.4100000000009 -4.62839036597347e-05
23.4200000000009 -4.75001372225361e-05
23.4300000000009 -4.86507010465422e-05
23.4400000000009 -4.97344486763605e-05
23.4500000000009 -5.07503234275252e-05
23.4600000000009 -5.16973593027822e-05
23.4700000000009 -5.25746817896185e-05
23.4800000000009 -5.33815085385932e-05
23.4900000000009 -5.41171499222069e-05
23.5000000000009 -5.47810094741921e-05
23.5100000000009 -5.53725842116607e-05
23.5200000000009 -5.58914648316465e-05
23.5300000000009 -5.63373357929271e-05
23.5400000000009 -5.67099752803134e-05
23.5500000000009 -5.70092550501327e-05
23.5600000000009 -5.72351401589878e-05
23.5700000000009 -5.73876885769223e-05
23.5800000000009 -5.74670506863723e-05
23.5900000000009 -5.74734686685578e-05
23.6000000000009 -5.74072757793323e-05
23.6100000000009 -5.72688955169893e-05
23.6200000000009 -5.70588406852153e-05
23.6300000000009 -5.67777123554583e-05
23.6400000000009 -5.64261987347511e-05
23.6500000000009 -5.60050739481935e-05
23.6600000000009 -5.55151967513235e-05
23.6700000000009 -5.49575092000764e-05
23.6800000000009 -5.433303533435e-05
23.6900000000009 -5.36428800035188e-05
23.7000000000009 -5.2888228160767e-05
23.7100000000009 -5.20703115523797e-05
23.7200000000009 -5.11904585667974e-05
23.7300000000009 -5.02500696788076e-05
23.7400000000009 -4.92506153356822e-05
23.7500000000009 -4.81936337655498e-05
23.7600000000009 -4.70807287111361e-05
23.7700000000009 -4.59135670920392e-05
23.7800000000009 -4.46938765987689e-05
23.7900000000009 -4.34234354732452e-05
23.8000000000009 -4.21040825266981e-05
23.8100000000009 -4.0737714756838e-05
23.8200000000009 -3.93262791494611e-05
23.8300000000009 -3.78717701834946e-05
23.8400000000009 -3.63762272012127e-05
23.8500000000009 -3.48417316894982e-05
23.8600000000009 -3.32704044958794e-05
23.8700000000009 -3.16644029933761e-05
23.8800000000009 -3.00259182035449e-05
23.8900000000009 -2.83571718847662e-05
23.9000000000009 -2.66604135915467e-05
23.9100000000009 -2.49379177098762e-05
23.9200000000009 -2.31919804732835e-05
23.9300000000009 -2.1424916963959e-05
23.9400000000009 -1.96390581031433e-05
23.9500000000009 -1.78367476348847e-05
23.9600000000009 -1.6020339107177e-05
23.9700000000009 -1.41921928544177e-05
23.9800000000009 -1.23546729850948e-05
23.990000000001 -1.0510144378544e-05
24.000000000001 -8.66096969458021e-06
24.010000000001 -6.80950639976735e-06
24.020000000001 -4.95810381402488e-06
24.030000000001 -3.10910018123704e-06
24.040000000001 -1.26481976747905e-06
24.050000000001 5.72430009598164e-07
24.060000000001 2.40036141672674e-06
24.070000000001 4.21670920288908e-06
24.080000000001 6.01923335573298e-06
24.090000000001 7.80572181323548e-06
24.100000000001 9.57399312739722e-06
24.110000000001 1.132189907682e-05
24.120000000001 1.30473272250925e-05
24.130000000001 1.47482034220203e-05
24.140000000001 1.64224942447814e-05
24.150000000001 1.8068209376227e-05
24.160000000001 1.96834039175909e-05
24.170000000001 2.12661806330039e-05
24.180000000001 2.28146921232977e-05
24.190000000001 2.4327142926672e-05
24.200000000001 2.58017915439031e-05
24.210000000001 2.72369523859189e-05
24.220000000001 2.86309976416704e-05
24.230000000001 2.99823590641898e-05
24.240000000001 3.12895296730726e-05
24.250000000001 3.25510653715969e-05
24.260000000001 3.37655864768355e-05
24.270000000001 3.49317791612324e-05
24.280000000001 3.60483968042477e-05
24.290000000001 3.71142612527847e-05
24.300000000001 3.81282639892321e-05
24.310000000001 3.90893672060951e-05
24.320000000001 3.99966047862906e-05
24.330000000001 4.08490831883243e-05
24.340000000001 4.16459822356859e-05
24.350000000001 4.23865558099246e-05
24.360000000001 4.30701324469974e-05
24.370000000001 4.36961158366087e-05
24.380000000001 4.42639852243909e-05
24.390000000001 4.47732957173233e-05
24.400000000001 4.52236784925099e-05
24.410000000001 4.56148409051148e-05
24.420000000001 4.5946566504878e-05
24.430000000001 4.62187149548527e-05
24.440000000001 4.64312218548949e-05
24.450000000001 4.65840984707396e-05
24.460000000001 4.66774313696793e-05
24.470000000001 4.67113819640651e-05
24.480000000001 4.66861859641064e-05
24.490000000001 4.66021527417716e-05
24.500000000001 4.64596646080491e-05
24.510000000001 4.62591760065057e-05
24.520000000001 4.60012126271668e-05
24.530000000001 4.56863704466005e-05
24.540000000001 4.5315314703515e-05
24.550000000001 4.48887788259755e-05
24.560000000001 4.44075633410578e-05
24.570000000001 4.38725348330962e-05
24.580000000001 4.32846251140101e-05
24.590000000001 4.2644818499937e-05
24.600000000001 4.19541567700415e-05
24.610000000001 4.12137503143591e-05
24.620000000001 4.04247671914477e-05
24.6300000000011 3.95884313776112e-05
24.6400000000011 3.8706020954128e-05
24.6500000000011 3.77788662350745e-05
24.6600000000011 3.68083478383765e-05
24.6700000000011 3.57958937687027e-05
24.6800000000011 3.47429707934154e-05
24.6900000000011 3.36511002670653e-05
24.7000000000011 3.25218450667775e-05
24.7100000000011 3.13568076371308e-05
24.7200000000011 3.01576278964166e-05
24.7300000000011 2.89259810570846e-05
24.7400000000011 2.76635753859483e-05
24.7500000000011 2.63721499183524e-05
24.7600000000011 2.50534721352954e-05
24.7700000000011 2.37093356099272e-05
24.7800000000011 2.23415576284775e-05
24.7900000000011 2.09519767899402e-05
24.8000000000011 1.95424505884029e-05
24.8100000000011 1.81148529816429e-05
24.8200000000011 1.66710719494647e-05
24.8300000000011 1.52130070451409e-05
24.8400000000011 1.37425669432307e-05
24.8500000000011 1.22616669870175e-05
24.8600000000011 1.07722267387385e-05
24.8700000000011 9.27616753575853e-06
24.8800000000011 7.77541005578613e-06
24.8900000000011 6.27187189420824e-06
24.9000000000011 4.76746515656484e-06
24.9100000000011 3.26409406916704e-06
24.9200000000011 1.76365261079925e-06
24.9300000000011 2.68022168419353e-07
24.9400000000011 -1.22093078028911e-06
24.9500000000011 -2.7013569546977e-06
24.9600000000011 -4.17142652658601e-06
24.9700000000011 -5.62933134037525e-06
24.9800000000011 -7.07328709557811e-06
24.9900000000011 -8.50153548889785e-06
25.0000000000011 -9.91234631344581e-06
25.0100000000011 -1.13040195126295e-05
25.0200000000011 -1.26748871863224e-05
25.0300000000011 -1.40233155470005e-05
25.0400000000011 -1.53477068236146e-05
25.0500000000011 -1.6646501111032e-05
25.0600000000011 -1.79181781629623e-05
25.0700000000011 -1.91612591263739e-05
25.0800000000011 -2.03743082154788e-05
25.0900000000011 -2.15559343234336e-05
25.1000000000011 -2.27047925700943e-05
25.1100000000011 -2.38195857840585e-05
25.1200000000011 -2.48990659174869e-05
25.1300000000011 -2.59420353922366e-05
25.1400000000011 -2.69473483759194e-05
25.1500000000011 -2.79139119866069e-05
25.1600000000011 -2.88406874249916e-05
25.1700000000011 -2.97266910329239e-05
25.1800000000011 -3.05709952773303e-05
25.1900000000011 -3.13727296586286e-05
25.2000000000011 -3.21310815428442e-05
25.2100000000011 -3.28452969167514e-05
25.2200000000011 -3.35146810654425e-05
25.2300000000011 -3.41385991718507e-05
25.2400000000011 -3.47164768378421e-05
25.2500000000011 -3.52478005266023e-05
25.2600000000011 -3.57321179261439e-05
25.2700000000012 -3.61690382338701e-05
25.2800000000012 -3.65582323631704e-05
25.2900000000012 -3.68994330687546e-05
25.3000000000012 -3.71924349952321e-05
25.3100000000012 -3.74370946477996e-05
25.3200000000012 -3.76333302848589e-05
25.3300000000012 -3.77811217336001e-05
25.3400000000012 -3.78805101292937e-05
25.3500000000012 -3.79315975791977e-05
25.3600000000012 -3.79345467521581e-05
25.3700000000012 -3.7889580395213e-05
25.3800000000012 -3.77969807788145e-05
25.3900000000012 -3.7657089072715e-05
25.4000000000012 -3.74703046552375e-05
25.4100000000012 -3.72370843597563e-05
25.4200000000012 -3.69579416641802e-05
25.4300000000012 -3.66334458329875e-05
25.4400000000012 -3.62642210291301e-05
25.4500000000012 -3.5850945430804e-05
25.4600000000012 -3.53943504333305e-05
25.4700000000012 -3.48952194366821e-05
25.4800000000012 -3.43543690480715e-05
25.4900000000012 -3.37726781999201e-05
25.5000000000012 -3.31510733068463e-05
25.5100000000012 -3.24905268689816e-05
25.5200000000012 -3.17920560238124e-05
25.5300000000012 -3.10567210486775e-05
25.5400000000012 -3.02856238160648e-05
25.5500000000012 -2.94799062038833e-05
25.5600000000012 -2.86407440002241e-05
25.5700000000012 -2.77693521523786e-05
25.5800000000012 -2.68669838550833e-05
25.5900000000012 -2.59349252111593e-05
25.6000000000012 -2.49744935742196e-05
25.6100000000012 -2.39870358065347e-05
25.6200000000012 -2.29739264803914e-05
25.6300000000012 -2.19365660376989e-05
25.6400000000012 -2.08763789166368e-05
25.6500000000012 -1.97948116513072e-05
25.6600000000012 -1.86933309488888e-05
25.6700000000012 -1.75734217480201e-05
25.6800000000012 -1.6436585261699e-05
25.6900000000012 -1.52843370077106e-05
25.7000000000012 -1.41182048294587e-05
25.7100000000012 -1.29397269099495e-05
25.7200000000012 -1.1750449781627e-05
25.7300000000012 -1.05519263346942e-05
25.7400000000012 -9.34571382651866e-06
25.7500000000012 -8.133371894691e-06
25.7600000000012 -6.91646057627148e-06
25.7700000000012 -5.69653833572308e-06
25.7800000000012 -4.47516010401479e-06
25.7900000000012 -3.25387533133347e-06
25.8000000000012 -2.03422605581394e-06
25.8100000000012 -8.17744990667471e-07
25.8200000000012 3.94046367955083e-07
25.8300000000012 1.59963961006094e-06
25.8400000000012 2.79754125561781e-06
25.8500000000012 3.98627457084615e-06
25.8600000000012 5.16438135482832e-06
25.8700000000012 6.33042369431486e-06
25.8800000000012 7.48298568466035e-06
25.8900000000012 8.6206751148653e-06
25.9000000000012 9.74212511476938e-06
25.9100000000013 1.08459957624848e-05
25.9200000000013 1.19309756502285e-05
25.9300000000013 1.29957834067655e-05
25.9400000000013 1.40391691747517e-05
25.9500000000013 1.50599160413088e-05
25.9600000000013 1.6056841420254e-05
25.9700000000013 1.70287983844547e-05
25.9800000000013 1.79746769468707e-05
25.9900000000013 1.88934052889223e-05
26.0000000000013 1.9783950934827e-05
26.0100000000013 2.06453218707247e-05
26.0200000000013 2.14765676074218e-05
26.0300000000013 2.2276780185678e-05
26.0400000000013 2.30450951230427e-05
26.0500000000013 2.37806923013135e-05
26.0600000000013 2.44827967937846e-05
26.0700000000013 2.51506796315193e-05
26.0800000000013 2.57836585079758e-05
26.0900000000013 2.63810984213863e-05
26.1000000000013 2.69424122543758e-05
26.1100000000013 2.74670612903914e-05
26.1200000000013 2.79545556665953e-05
26.1300000000013 2.84044547629557e-05
26.1400000000013 2.8816367527359e-05
26.1500000000013 2.91899527366482e-05
26.1600000000013 2.95249191937441e-05
26.1700000000013 2.98210258608954e-05
26.1800000000013 3.00780819276096e-05
26.1900000000013 3.02959468168758e-05
26.2000000000013 3.0474530127453e-05
26.2100000000013 3.06137915133648e-05
26.2200000000013 3.07137405011499e-05
26.2300000000013 3.07744362455327e-05
26.2400000000013 3.0795987224309e-05
26.2500000000013 3.07785508734018e-05
26.2600000000013 3.07223331632501e-05
26.2700000000013 3.06275881179716e-05
26.2800000000013 3.04946172791652e-05
26.2900000000013 3.03237691168832e-05
26.3000000000013 3.01154383914424e-05
26.3100000000013 2.98700654718476e-05
26.3200000000013 2.95881356207556e-05
26.3300000000013 2.92701782649307e-05
26.3400000000013 2.89167662918185e-05
26.3500000000013 2.85285154726456e-05
26.3600000000013 2.81060760804231e-05
26.3700000000013 2.76501373360054e-05
26.3800000000013 2.71614328001662e-05
26.3900000000013 2.66407339779479e-05
26.4000000000013 2.6088849163348e-05
26.4100000000013 2.5506622243037e-05
26.4200000000013 2.48949314608663e-05
26.4300000000013 2.42546881449407e-05
26.4400000000013 2.35868350556467e-05
26.4500000000013 2.28923404601736e-05
26.4600000000013 2.21722081157006e-05
26.4700000000013 2.14274690987157e-05
26.4800000000013 2.0659180504169e-05
26.4900000000013 1.98684240583023e-05
26.5000000000013 1.90563046774583e-05
26.5100000000013 1.82239489886046e-05
26.5200000000013 1.73725038203833e-05
26.5300000000013 1.65031346703278e-05
26.5400000000013 1.56170241523128e-05
26.5500000000014 1.47153704274824e-05
26.5600000000014 1.37993856214458e-05
26.5700000000014 1.28702942302641e-05
26.5800000000014 1.19293315176018e-05
26.5900000000014 1.09777419053084e-05
26.6000000000014 1.00167773596339e-05
26.6100000000014 9.04769577523624e-06
26.6200000000014 8.07175935910043e-06
26.6300000000014 7.09023301646246e-06
26.6400000000014 6.10438274080692e-06
26.6500000000014 5.11547400997859e-06
26.6600000000014 4.12477019042827e-06
26.6700000000014 3.13353095158643e-06
26.6800000000014 2.1430106923342e-06
26.6900000000014 1.15445698151154e-06
26.7000000000014 1.6910901437312e-07
26.7100000000014 -8.11803913122744e-07
26.7200000000014 -1.78706391034576e-06
26.7300000000014 -2.75546597653134e-06
26.7400000000014 -3.71581946204956e-06
26.7500000000014 -4.66694950465651e-06
26.7600000000014 -5.60769843904379e-06
26.7700000000014 -6.53692717801497e-06
26.7800000000014 -7.45351656368561e-06
26.7900000000014 -8.35636868713422e-06
26.8000000000014 -9.24440817498779e-06
26.8100000000014 -1.01165834414725e-05
26.8200000000014 -1.09718679045036e-05
26.8300000000014 -1.18092611644557e-05
26.8400000000014 -1.2627790144294e-05
26.8500000000014 -1.34265101898081e-05
26.8600000000014 -1.42045061287436e-05
26.8700000000014 -1.49608932877132e-05
26.8800000000014 -1.56948184657608e-05
26.8900000000014 -1.64054608635747e-05
26.9000000000014 -1.70920329673915e-05
26.9100000000014 -1.7753781386676e-05
26.9200000000014 -1.83899876447478e-05
26.9300000000014 -1.89999689215735e-05
26.9400000000014 -1.95830787480154e-05
26.9500000000014 -2.01387076508887e-05
26.9600000000014 -2.06662837482512e-05
26.9700000000014 -2.11652732944051e-05
26.9800000000014 -2.16351811741683e-05
26.9900000000014 -2.20755513460333e-05
27.0000000000014 -2.2485967233899e-05
27.0100000000014 -2.28660520671317e-05
27.0200000000014 -2.32154691687766e-05
27.0300000000014 -2.35339221918127e-05
27.0400000000014 -2.38211553034093e-05
27.0500000000014 -2.40769533175664e-05
27.0600000000014 -2.43011417749454e-05
27.0700000000014 -2.44935869716638e-05
27.0800000000014 -2.4654195936725e-05
27.0900000000014 -2.47829163581577e-05
27.1000000000014 -2.48797364584253e-05
27.1100000000014 -2.49446848195936e-05
27.1200000000014 -2.49778301588411e-05
27.1300000000014 -2.49792810550132e-05
27.1400000000014 -2.49491856270594e-05
27.1500000000014 -2.48877311653841e-05
27.1600000000014 -2.47951437174025e-05
27.1700000000014 -2.46716876290041e-05
27.1800000000014 -2.45176650442928e-05
27.1900000000015 -2.43334153671579e-05
27.2000000000015 -2.41193146904883e-05
27.2100000000015 -2.38757752035121e-05
27.2200000000015 -2.36032445983735e-05
27.2300000000015 -2.33022055242752e-05
27.2400000000015 -2.29731744897112e-05
27.2500000000015 -2.26166912846042e-05
27.2600000000015 -2.2233335762343e-05
27.2700000000015 -2.18237189859927e-05
27.2800000000015 -2.13884823079161e-05
27.2900000000015 -2.09282964154035e-05
27.3000000000015 -2.04438603437553e-05
27.3100000000015 -1.99359004582609e-05
27.3200000000015 -1.94051694065306e-05
27.3300000000015 -1.88524426047528e-05
27.3400000000015 -1.82785206433517e-05
27.3500000000015 -1.76842292033062e-05
27.3600000000015 -1.7070415590877e-05
27.3700000000015 -1.64379476387726e-05
27.3800000000015 -1.57877125544889e-05
27.3900000000015 -1.5120615733059e-05
27.4000000000015 -1.44375795432568e-05
27.4100000000015 -1.37395420927087e-05
27.4200000000015 -1.30274559756443e-05
27.4300000000015 -1.23022870061436e-05
27.4400000000015 -1.15650129392747e-05
27.4500000000015 -1.08166221822381e-05
27.4600000000015 -1.00581124974811e-05
27.4700000000015 -9.29048969965275e-06
27.4800000000015 -8.51476634820335e-06
27.4900000000015 -7.73196043738795e-06
27.5000000000015 -6.94309408541082e-06
27.5100000000015 -6.14919222440782e-06
27.5200000000015 -5.35128129296185e-06
27.5300000000015 -4.55038793280669e-06
27.5400000000015 -3.74753769137059e-06
27.5500000000015 -2.94375373178341e-06
27.5600000000015 -2.14005555195343e-06
27.5700000000015 -1.33745771429852e-06
27.5800000000015 -5.36968587691055e-07
27.5900000000015 2.60410896844616e-07
27.6000000000015 1.05368847516987e-06
27.6100000000015 1.84188176232228e-06
27.6200000000015 2.6240194466035e-06
27.6300000000015 3.39914246404008e-06
27.6400000000015 4.16630515181866e-06
27.6500000000015 4.92457637934701e-06
27.6600000000015 5.67304065560436e-06
27.6700000000015 6.41079921150312e-06
27.6800000000015 7.13697105600485e-06
27.6900000000015 7.85069400478301e-06
27.7000000000015 8.55112568026135e-06
27.7100000000015 9.23744448189946e-06
27.7200000000015 9.90885052564168e-06
27.7300000000015 1.05645665514835e-05
27.7400000000015 1.12038387981653e-05
27.7500000000015 1.18259378440382e-05
27.7600000000015 1.24301594132156e-05
27.7700000000015 1.30158251461277e-05
27.7800000000015 1.35822833336976e-05
27.7900000000015 1.41289096143786e-05
27.8000000000015 1.46551076333459e-05
27.8100000000015 1.51603096631914e-05
27.8200000000015 1.56439771855211e-05
27.8300000000016 1.61056014329061e-05
27.8400000000016 1.65447038906933e-05
27.8500000000016 1.69608367582321e-05
27.8600000000016 1.73535833691289e-05
27.8700000000016 1.77225585701981e-05
27.8800000000016 1.80674090588274e-05
27.8900000000016 1.83878136785332e-05
27.9000000000016 1.86834836725388e-05
27.9100000000016 1.89541628952564e-05
27.9200000000016 1.91996279816184e-05
27.9300000000016 1.94196884743139e-05
27.9400000000016 1.96141869089758e-05
27.9500000000016 1.97829988568233e-05
27.9600000000016 1.99260329261813e-05
27.9700000000016 2.0043230722147e-05
27.9800000000016 2.01345667649535e-05
27.9900000000016 2.020004836739e-05
28.0000000000016 2.02397154717083e-05
28.0100000000016 2.02536404465296e-05
28.0200000000016 2.02419278443639e-05
28.0300000000016 2.02047141204808e-05
28.0400000000016 2.01421673140413e-05
28.0500000000016 2.00544866926511e-05
28.0600000000016 1.99419023618937e-05
28.0700000000016 1.9804674842076e-05
28.0800000000016 1.96430946156574e-05
28.0900000000016 1.94574816512851e-05
28.1000000000016 1.92481849156642e-05
28.1100000000016 1.90155818972445e-05
28.1200000000016 1.87600782008454e-05
28.1300000000016 1.84821023094367e-05
28.1400000000016 1.81821087455026e-05
28.1500000000016 1.78605806732369e-05
28.1600000000016 1.75180261688308e-05
28.1700000000016 1.71549774601623e-05
28.1800000000016 1.67719901394857e-05
28.1900000000016 1.63696423503065e-05
28.2000000000016 1.59485339496234e-05
28.2100000000016 1.55092856184934e-05
28.2200000000016 1.50525348067703e-05
28.2300000000016 1.45789419390742e-05
28.2400000000016 1.40891853815613e-05
28.2500000000016 1.35839605778203e-05
28.2600000000016 1.30639791317022e-05
28.2700000000016 1.25299678565074e-05
28.2800000000016 1.19826678000445e-05
28.2900000000016 1.14228332509515e-05
28.3000000000016 1.08512307297647e-05
28.3100000000016 1.02686379672893e-05
28.3200000000016 9.67584287233083e-06
28.3300000000016 9.07364249058167e-06
28.3400000000016 8.4628419562895e-06
28.3500000000016 7.84425343825058e-06
28.3600000000016 7.21869508160752e-06
28.3700000000016 6.58698994688723e-06
28.3800000000016 5.9499649476946e-06
28.3900000000016 5.3084497884495e-06
28.4000000000016 4.66327590353887e-06
28.4100000000016 4.01527539924085e-06
28.4200000000016 3.36527999975988e-06
28.4300000000016 2.71411999869907e-06
28.4400000000016 2.0626232172747e-06
28.4500000000016 1.41161397056988e-06
28.4600000000016 7.61912043096371e-07
28.4700000000017 1.1433167492236e-07
28.4800000000017 -5.30319440401301e-07
28.4900000000017 -1.17124114490161e-06
28.5000000000017 -1.8076417910419e-06
28.5100000000017 -2.4387392012108e-06
28.5200000000017 -3.06376161070173e-06
28.5300000000017 -3.68194859306888e-06
28.5400000000017 -4.29255196677034e-06
28.5500000000017 -4.89483668204507e-06
28.5600000000017 -5.48808168699378e-06
28.5700000000017 -6.07158077186596e-06
28.5800000000017 -6.64464339059489e-06
28.5900000000017 -7.20659545864417e-06
28.6000000000017 -7.75678012627321e-06
28.6100000000017 -8.29455852636065e-06
28.6200000000017 -8.81931049596345e-06
28.6300000000017 -9.33043527081726e-06
28.6400000000017 -9.82735215204508e-06
28.6500000000017 -1.03095011443457e-05
28.6600000000017 -1.07763435650009e-05
28.6700000000017 -1.12273626230705e-05
28.6800000000017 -1.16620639681839e-05
28.6900000000017 -1.20799762083806e-05
28.7000000000017 -1.24806513964941e-05
28.7100000000017 -1.28636654846128e-05
28.7200000000017 -1.32286187461973e-05
28.7300000000017 -1.35751361654787e-05
28.7400000000017 -1.39028677937971e-05
28.7500000000017 -1.42114890725955e-05
28.7600000000017 -1.45007011228182e-05
28.7700000000017 -1.47702310005112e-05
28.7800000000017 -1.50198319184676e-05
28.7900000000017 -1.52492834338021e-05
28.8000000000017 -1.54583916013876e-05
28.8100000000017 -1.56469890931277e-05
28.8200000000017 -1.5814935283215e-05
28.8300000000017 -1.59621162989771e-05
28.8400000000017 -1.60884450379929e-05
28.8500000000017 -1.61938611514466e-05
28.8600000000017 -1.62783309938328e-05
28.8700000000017 -1.63418475393314e-05
28.8800000000017 -1.63844302651712e-05
28.8900000000017 -1.6406125002357e-05
28.9000000000017 -1.64070037542105e-05
28.9100000000017 -1.63871644832575e-05
28.9200000000017 -1.63467308671096e-05
28.9300000000017 -1.62858520241418e-05
28.9400000000017 -1.62047022100121e-05
28.9500000000017 -1.61034804864512e-05
28.9600000000017 -1.5982410364441e-05
28.9700000000017 -1.58417394251987e-05
28.9800000000017 -1.56817389250735e-05
28.9900000000017 -1.55027033965651e-05
29.0000000000017 -1.53049502732927e-05
29.0100000000017 -1.50888190933085e-05
29.0200000000017 -1.48546655501193e-05
29.0300000000017 -1.46028709949999e-05
29.0400000000017 -1.43338372469888e-05
29.0500000000017 -1.40479859880605e-05
29.0600000000017 -1.37457581359199e-05
29.0700000000017 -1.34276131953881e-05
29.0800000000017 -1.30940285893439e-05
29.0900000000017 -1.2745498970195e-05
29.1000000000017 -1.23825342680888e-05
29.1100000000018 -1.20056605366497e-05
29.1200000000018 -1.16154202361712e-05
29.1300000000018 -1.12123700108654e-05
29.1400000000018 -1.07970799616819e-05
29.1500000000018 -1.03701328866749e-05
29.1600000000018 -9.93212349919213e-06
29.1700000000018 -9.4836576293286e-06
29.1800000000018 -9.0253514119656e-06
29.1900000000018 -8.55783046370774e-06
29.2000000000018 -8.08172905051269e-06
29.2100000000018 -7.59768924753125e-06
29.2200000000018 -7.10636009252159e-06
29.2300000000018 -6.60839673410639e-06
29.2400000000018 -6.10445957608712e-06
29.2500000000018 -5.59521341898826e-06
29.2600000000018 -5.0813265999855e-06
29.2700000000018 -4.56347013234567e-06
29.2800000000018 -4.04231684550037e-06
29.2900000000018 -3.51854052684794e-06
29.3000000000018 -2.99281506638256e-06
29.3100000000018 -2.46581360522379e-06
29.3200000000018 -1.93820768911304e-06
29.3300000000018 -1.41066642792923e-06
29.3400000000018 -8.83855662263178e-07
29.3500000000018 -3.5843713807131e-07
29.3600000000018 1.64932309580393e-07
29.3700000000018 6.85601562696071e-07
29.3800000000018 1.20292601547604e-06
29.3900000000018 1.71626835744047e-06
29.4000000000018 2.22499934361211e-06
29.4100000000018 2.72849855084858e-06
29.4200000000018 3.22615511943552e-06
29.4300000000018 3.71736847906714e-06
29.4400000000018 4.20154905837761e-06
29.4500000000018 4.67811897719907e-06
29.4600000000018 5.1465127207575e-06
29.4700000000018 5.60617779503839e-06
29.4800000000018 6.05657536258353e-06
29.4900000000018 6.49718085801049e-06
29.5000000000018 6.92748458257256e-06
29.5100000000018 7.34699227710489e-06
29.5200000000018 7.75522567274336e-06
29.5300000000018 8.15172301881958e-06
29.5400000000018 8.53603958737714e-06
29.5500000000018 8.90774815377989e-06
29.5600000000018 9.26643945292497e-06
29.5700000000018 9.61172261059637e-06
29.5800000000018 9.94322554953485e-06
29.5900000000018 1.02605953698318e-05
29.6000000000018 1.05634987032899e-05
29.6100000000018 1.08516220414277e-05
29.6200000000018 1.11246720368404e-05
29.6300000000018 1.13823757776643e-05
29.6400000000018 1.16244810349288e-05
29.6500000000018 1.18507564826142e-05
29.6600000000018 1.20609918902699e-05
29.6700000000018 1.22549982880835e-05
29.6800000000018 1.24326081043286e-05
29.6900000000018 1.25936752751528e-05
29.7000000000018 1.27380753267262e-05
29.7100000000018 1.28657054297943e-05
29.7200000000018 1.29764844264685e-05
29.7300000000018 1.30703528298378e-05
29.7400000000018 1.31472727961965e-05
29.7500000000019 1.32072280701706e-05
29.7600000000019 1.32502239029773e-05
29.7700000000019 1.3276286944093e-05
29.7800000000019 1.32854651066593e-05
29.7900000000019 1.32778274070154e-05
29.8000000000019 1.32534637788214e-05
29.8100000000019 1.32124848623355e-05
29.8200000000019 1.31550217695564e-05
29.8300000000019 1.30812258261666e-05
29.8400000000019 1.29912682915952e-05
29.8500000000019 1.28853400592189e-05
29.8600000000019 1.27636513400917e-05
29.8700000000019 1.26264313365707e-05
29.8800000000019 1.24739279193274e-05
29.8900000000019 1.23064073408208e-05
29.9000000000019 1.2124151143264e-05
29.9100000000019 1.1927458064628e-05
29.9200000000019 1.17166453272812e-05
29.9300000000019 1.14920464716396e-05
29.9400000000019 1.1254010857224e-05
29.9500000000019 1.10029031459803e-05
29.9600000000019 1.07391027686502e-05
29.9700000000019 1.04630033749831e-05
29.9800000000019 1.01750122685791e-05
29.9900000000019 9.87554811574742e-06
30.0000000000019 9.56504405563409e-06
};
\addlegendentry{DDPG};
\end{axis}

\end{tikzpicture}

		\end{figure}
		\begin{figure}\scriptsize
			\hspace{3cm}
			% This file was created by tikzplotlib v0.9.1.
\begin{tikzpicture}[trim axis right,trim axis left]

\definecolor{color0}{rgb}{0.12156862745098,0.466666666666667,0.705882352941177}
\definecolor{color1}{rgb}{1,0.498039215686275,0.0549019607843137}

\begin{axis}[
compat=newest,
tick align=outside,
tick pos=left,
x grid style={white!69.0196078431373!black},
xmin=-1.50000000000009, xmax=31.500000000002,
xtick style={color=black},
y grid style={white!69.0196078431373!black},
ymin=-0.00331591221720112, ymax=0.0276834385040403,
ytick style={color=black},
%yticklabel style={
%        /pgf/number format/.cd,
%        	fixed,
%        	fixed zerofill,
%         	precision=3,
%        /tikz/.cd
%},
scaled y ticks=true,
scaled y ticks=base 10:2,
width=7cm,
height=7cm,
xlabel=Time (sec),
ylabel=Control Signal,
y label style={at={(-0.2,0.5)}}
]
\addplot [ultra thick, blue!20!gray, dotted]
table {%
0 0
0.01 0
0.02 0
0.03 0
0.04 0
0.05 0
0.06 0
0.07 0
0.08 0
0.09 0
0.1 0
0.11 0
0.12 0
0.13 0
0.14 0
0.15 0
0.16 0
0.17 0
0.18 0
0.19 0
0.2 0
0.21 0
0.22 0
0.23 0
0.24 0
0.25 0
0.26 0
0.27 0
0.28 0
0.29 0
0.3 0
0.31 0
0.32 0
0.33 0
0.34 0
0.35 0
0.36 0
0.37 0
0.38 0
0.39 0
0.4 0
0.41 0
0.42 0
0.43 0
0.44 0
0.45 0
0.46 0
0.47 0
0.48 0
0.49 0
0.5 0
0.51 0
0.52 0
0.53 0
0.54 0
0.55 0
0.56 0
0.57 0
0.58 0
0.59 0
0.6 0
0.61 0
0.62 0
0.63 0
0.64 0
0.65 0
0.66 0
0.67 0
0.68 0
0.69 0
0.7 0
0.71 0
0.72 0
0.73 0
0.74 0
0.75 0
0.76 0
0.77 0
0.78 0
0.79 0
0.8 0
0.81 0
0.820000000000001 0
0.830000000000001 0
0.840000000000001 0
0.850000000000001 0
0.860000000000001 0
0.870000000000001 0
0.880000000000001 0
0.890000000000001 0
0.900000000000001 0
0.910000000000001 0
0.920000000000001 0
0.930000000000001 0
0.940000000000001 0
0.950000000000001 0
0.960000000000001 0
0.970000000000001 0
0.980000000000001 0
0.990000000000001 0
1 -4.3927576726343e-19
1.01 6.51165931843092e-09
1.02 6.88168037885038e-08
1.03 2.57142679871956e-07
1.04 6.43147531775966e-07
1.05 1.29756621960344e-06
1.06 2.29157793207718e-06
1.07 3.69658887283981e-06
1.08 5.584093999423e-06
1.09 8.02556017579946e-06
1.1 1.10923121262385e-05
1.11 1.48554134452624e-05
1.12 1.93855436454516e-05
1.13 2.47528721793769e-05
1.14 3.10269302598353e-05
1.15 3.82764812014668e-05
1.16 4.65693899243068e-05
1.17 5.59724921926305e-05
1.18 6.65514641066273e-05
1.19 7.83706923172054e-05
1.2 9.14931453935508e-05
1.21 0.000105980246737521
1.22 0.000121891749407602
1.23 0.000139285613187178
1.24 0.000158217884206755
1.25 0.000178742577407031
1.26 0.000200911562108968
1.27 0.000224774450938019
1.28 0.000250378492332079
1.29 0.000277768466846435
1.3 0.000306986587453742
1.31 0.00033807240402273
1.32 0.000371062712145771
1.33 0.000405991466472624
1.34 0.000442889698695366
1.35 0.000481785440317723
1.36 0.000522703650330712
1.37 0.000565666147905504
1.38 0.000610691550203821
1.39 0.000657795215395797
1.4 0.000706989190965142
1.41 0.000758282167371637
1.42 0.000811679437131271
1.43 0.000867182859364844
1.44 0.000924790829798501
1.45 0.000984498256476486
1.46 0.0010462965408854
1.47 0.00111017356469161
1.48 0.00117611368207008
1.49 0.00124409771757355
1.5 0.0013141029697316
1.51 0.00138610322003686
1.52 0.00146006874751303
1.53 0.00153596634877289
1.54 0.00161375936352312
1.55 0.00169340770546521
1.56 0.00177486789853425
1.57 0.00185809311841062
1.58 0.00194303323923262
1.59 0.00202963488566609
1.6 0.00211784149071758
1.61 0.00220759335694106
1.62 0.00229882772312578
1.63 0.00239147883588126
1.64 0.00248547802593192
1.65 0.0025807537890481
1.66 0.00267723188544788
1.67 0.0027748353965996
1.68 0.00287348482752241
1.69 0.00297309821011458
1.7 0.00307359120393008
1.71 0.00317487720060498
1.72 0.0032768674317769
1.73 0.0033794710803356
1.74 0.00348259539483972
1.75 0.00358614580693015
1.76 0.00369002605156765
1.77 0.00379413828991845
1.78 0.00389838323473531
1.79 0.00400266027798059
1.8 0.00410686762056379
1.81 0.00421090240400543
1.82 0.00431466084382664
1.83 0.00441803836447643
1.84 0.00452092973560502
1.85 0.0046232292094902
1.86 0.00472483065942296
1.87 0.00482562771885778
1.88 0.00492551392113265
1.89 0.00502438283956368
1.9 0.00512212822771936
1.91 0.00521864415967983
1.92 0.00531382517008712
1.93 0.0054075663937933
1.94 0.0054997637049146
1.95 0.00559031385510083
1.96 0.00567911461083126
1.97 0.00576606488954984
1.98 0.00585106489445481
1.99 0.00593401624776022
2 0.00601482212224916
2.01 0.00609338737094173
2.02 0.00616961865470337
2.03 0.00624342456762271
2.04 0.00631471575999138
2.05 0.00638340505872182
2.06 0.00644940758507138
2.07 0.00651264087473041
2.08 0.00657302497693673
2.09 0.00663048257039227
2.1 0.00668493906862615
2.11 0.00673632272189139
2.12 0.00678456471546638
2.13 0.00682959926423663
2.14 0.0068713637034368
2.15 0.00690979857543807
2.16 0.00694484771247119
2.17 0.00697645831518065
2.18 0.00700458102691147
2.19 0.0070291700036354
2.2 0.00705018297942955
2.21 0.00706758132742611
2.22 0.00708133011615843
2.23 0.00709139816123435
2.24 0.00709775807227424
2.25 0.00710038629505759
2.26 0.00709926314882795
2.27 0.00709437285871289
2.28 0.00708570358322166
2.29 0.00707324743678968
2.29999999999999 0.00705700050734544
2.30999999999999 0.00703696286888161
2.31999999999999 0.00701313858901834
2.32999999999999 0.00698553573154843
2.33999999999999 0.00695416635398159
2.34999999999999 0.00691904650006853
2.35999999999999 0.00688019618732991
2.36999999999999 0.00683763939015508
2.37999999999999 0.00679140401686327
2.38999999999999 0.00674152188174172
2.39999999999999 0.0066880286733582
2.40999999999999 0.00663096391740871
2.41999999999999 0.00657037093470004
2.42999999999999 0.00650629679428298
2.43999999999999 0.00643879226173841
2.44999999999999 0.00636791174259503
2.45999999999999 0.00629371326792681
2.46999999999999 0.00621625861761676
2.47999999999999 0.00613561274399405
2.48999999999999 0.00605184398791189
2.49999999999999 0.00596502399992463
2.50999999999999 0.00587522765802701
2.51999999999999 0.00578253298187999
2.52999999999999 0.00568702104352003
2.53999999999999 0.00558877587459342
2.54999999999999 0.00548788437018674
2.55999999999999 0.00538443618934432
2.56999999999999 0.00527852365237848
2.57999999999999 0.0051702416350889
2.58999999999999 0.00505968746001577
2.59999999999999 0.00494696078485864
2.60999999999999 0.00483216348819829
2.61999999999999 0.00471539955266385
2.62999999999999 0.00459677494569157
2.63999999999999 0.00447639749802542
2.64999999999999 0.00435437678011289
2.65999999999999 0.00423082397655175
2.66999999999999 0.00410585175874703
2.67999999999999 0.00397957415593846
2.68999999999999 0.00385210642476167
2.69999999999999 0.00372356491750706
2.70999999999999 0.00359406694924249
2.71999999999999 0.00346373066396653
2.72999999999999 0.00333267489995996
2.73999999999999 0.00320101905450428
2.74999999999999 0.00306888294813569
2.75999999999999 0.00293638668860385
2.76999999999998 0.00280365053470432
2.77999999999998 0.00267079476015323
2.78999999999998 0.00253793951767278
2.79999999999998 0.00240520470345469
2.80999999999998 0.0022727098221684
2.81999999999998 0.0021405738526793
2.82999999999998 0.0020089151146414
2.83999999999998 0.00187785113612658
2.84999999999998 0.00174749852245142
2.85999999999998 0.0016179728263604
2.86999999999998 0.00148938841972212
2.87999999999998 0.00136185836689281
2.88999999999998 0.00123549429989886
2.89999999999998 0.00111040629558756
2.90999999999998 0.000986702754897498
2.91999999999998 0.000864490284394157
2.92999999999998 0.000743873580147423
2.93999999999998 0.00062495531420673
2.94999999999998 0.000507836023720285
2.95999999999998 0.000392614002849654
2.96999999999998 0.000279385197604974
2.97999999999998 0.000168243103721868
2.98999999999998 5.92786676971333e-05
2.99999999999998 -4.74198089041051e-05
3.00999999999998 -0.000151766761761477
3.01999999999998 -0.00025367945262902
3.02999999999998 -0.000353078054582202
3.03999999999998 -0.000449885733587982
3.04999999999998 -0.000544028726308518
3.05999999999998 -0.000635436414053836
3.06999999999998 -0.000724041392803974
3.07999999999998 -0.000809779539226145
3.08999999999998 -0.000892590072617571
3.09999999999998 -0.000972415612700121
3.10999999999998 -0.00104920223324566
3.11999999999998 -0.00112289951144076
3.12999999999998 -0.00119346057292027
3.13999999999998 -0.00126084213248062
3.14999999999998 -0.00132500453041183
3.15999999999998 -0.00138591176441535
3.16999999999998 -0.00144353151722942
3.17999999999998 -0.00149783518017011
3.18999999999998 -0.00154879787005109
3.19999999999998 -0.00159639844411324
3.20999999999998 -0.00164061950953295
3.21999999999998 -0.00168144742840524
3.22999999999998 -0.00171887231819065
3.23999999999997 -0.00175288804761336
3.24999999999997 -0.00178349222799154
3.25999999999997 -0.0018106861999693
3.26999999999997 -0.00183447501559835
3.27999999999997 -0.00185486741568077
3.28999999999997 -0.0018718758022218
3.29999999999997 -0.00188551620573351
3.30999999999997 -0.00189580931544134
3.31999999999997 -0.00190277768604433
3.32999999999997 -0.00190644810017757
3.33999999999997 -0.00190685082078106
3.34999999999997 -0.00190401954193931
3.35999999999997 -0.00189799133713649
3.36999999999997 -0.00188880660618272
3.37999999999997 -0.00187650902358327
3.38999999999997 -0.00186114550550286
3.39999999999997 -0.00184276568571318
3.40999999999997 -0.00182142246932561
3.41999999999997 -0.00179717180542219
3.42999999999997 -0.00177007253959398
3.43999999999997 -0.00174018633193757
3.44999999999997 -0.00170757757207629
3.45999999999997 -0.00167231329131423
3.46999999999997 -0.00163446307203615
3.47999999999997 -0.00159409888467813
3.48999999999997 -0.00155129498481572
3.49999999999997 -0.00150612808134272
3.50999999999997 -0.00145867703047713
3.51999999999997 -0.00140902274166476
3.52999999999997 -0.00135724807672237
3.53999999999997 -0.00130343774517851
3.54999999999997 -0.00124767819701319
3.55999999999997 -0.00119005751337967
3.56999999999997 -0.00113066529565068
3.57999999999997 -0.00106959255302818
3.58999999999997 -0.00100693005397189
3.59999999999997 -0.000942773105068836
3.60999999999997 -0.000877216209631927
3.61999999999997 -0.000810354825441027
3.62999999999997 -0.00074228524637907
3.63999999999997 -0.000673104483918077
3.64999999999997 -0.000602910148517982
3.65999999999997 -0.000531800331025578
3.66999999999997 -0.000459873484176939
3.67999999999997 -0.000387228304317299
3.68999999999997 -0.000313963613459522
3.69999999999997 -0.000240178241808913
3.70999999999996 -0.000165970910878039
3.71999999999996 -9.14401173227679e-05
3.72999999999996 -1.66840176338577e-05
3.73999999999996 5.81996861869997e-05
3.74999999999996 0.00013311385983626
3.75999999999996 0.000207962048673402
3.76999999999996 0.000282648588313445
3.77999999999996 0.000357078713720786
3.78999999999996 0.000431158666679813
3.79999999999996 0.000504795801519935
3.80999999999996 0.000577898688975066
3.81999999999996 0.000650377218060189
3.82999999999996 0.000722142695889003
3.83999999999996 0.000793107945187091
3.84999999999996 0.000863187399531708
3.85999999999996 0.00093229719621012
3.86999999999996 0.00100035526652126
3.87999999999996 0.00106728142345293
3.88999999999996 0.00113299744663936
3.89999999999996 0.00119742716450776
3.90999999999996 0.00126049653352611
3.91999999999996 0.00132213371446853
3.92999999999996 0.0013822691456183
3.93999999999996 0.00144083561283274
3.94999999999996 0.00149776831639835
3.95999999999996 0.00155300493460856
3.96999999999996 0.00160648568400082
3.97999999999996 0.00165815337619379
3.98999999999996 0.00170795347126981
3.99999999999996 0.00175583412765186
4.00999999999996 0.0018017462484285
4.01999999999996 0.00184564352408424
4.02999999999996 0.00188748247159704
4.03999999999996 0.00192722246986788
4.04999999999996 0.00196482579145146
4.05999999999996 0.00200025763059678
4.06999999999996 0.00203348612843875
4.07999999999996 0.00206448239141095
4.08999999999996 0.00209322050898761
4.09999999999996 0.00211967758471578
4.10999999999996 0.00214383384217253
4.11999999999996 0.00216567239900753
4.12999999999996 0.00218517938890401
4.13999999999996 0.00220234395989691
4.14999999999996 0.00221715826903158
4.15999999999996 0.00222961747339494
4.16999999999996 0.00223971971756126
4.17999999999996 0.00224746611751002
4.18999999999996 0.00225286074109805
4.19999999999995 0.00225591058521286
4.20999999999995 0.00225662554982292
4.21999999999995 0.00225501841023307
4.22999999999995 0.00225110478494982
4.23999999999995 0.00224490310607931
4.24999999999995 0.00223643459499029
4.25999999999995 0.00222572286926878
4.26999999999995 0.00221279440842352
4.27999999999995 0.00219767839059988
4.28999999999995 0.00218040655597345
4.29999999999995 0.00216101315499718
4.30999999999995 0.00213953489203037
4.31999999999995 0.00211601086542408
4.32999999999995 0.00209048250457282
4.33999999999995 0.00206299350421918
4.34999999999995 0.00203358975620237
4.35999999999995 0.00200231927879811
4.36999999999995 0.00196923214377681
4.37999999999995 0.0019343804012975
4.38999999999995 0.00189781800275036
4.39999999999995 0.00185960072165951
4.40999999999995 0.00181978607275689
4.41999999999995 0.00177843322933907
4.42999999999995 0.0017356029390193
4.43999999999995 0.00169135743798831
4.44999999999995 0.00164576036389831
4.45999999999995 0.00159887666748558
4.46999999999995 0.00155077133842975
4.47999999999995 0.00150151299806922
4.48999999999995 0.00145116998134318
4.49999999999995 0.0013998115740688
4.50999999999995 0.00134750791979499
4.51999999999995 0.00129432992603328
4.52999999999995 0.00124034917032461
4.53999999999995 0.00118563780589199
4.54999999999995 0.00113026846711211
4.55999999999995 0.00107431417492269
4.56999999999995 0.0010178482422817
4.57999999999995 0.000960944179794845
4.58999999999995 0.000903675601626453
4.59999999999995 0.000846116131808848
4.60999999999995 0.000788339311063952
4.61999999999995 0.00073041850424989
4.62999999999995 0.000672426808544472
4.63999999999995 0.000614436962475647
4.64999999999995 0.000556521255907823
4.65999999999995 0.000498751441091128
4.66999999999994 0.000441198644879147
4.67999999999994 0.000383933282218462
4.68999999999994 0.000327024971011616
4.69999999999994 0.000270542448456266
4.70999999999994 0.000214553488950477
4.71999999999994 0.000159124823652942
4.72999999999994 0.00010432206181466
4.73999999999994 5.02096139513599e-05
4.74999999999994 -3.14938305113241e-06
4.75999999999994 -5.56931380565489e-05
4.76999999999994 -0.000107361272155847
4.77999999999994 -0.000158094899040353
4.78999999999994 -0.000207836684572148
4.79999999999994 -0.000256530910834074
4.80999999999994 -0.000304123537785906
4.81999999999994 -0.000350562262461688
4.82999999999994 -0.000395796575646166
4.83999999999994 -0.000439777815971676
4.84999999999994 -0.000482459221408591
4.85999999999994 -0.00052379597803319
4.86999999999994 -0.000563745265975748
4.87999999999994 -0.000602266302751015
4.88999999999994 -0.000639320004224856
4.89999999999994 -0.000674870257107564
4.90999999999994 -0.000708882621370187
4.91999999999994 -0.000741324840472462
4.92999999999994 -0.000772166870166031
4.93999999999994 -0.000801380904354541
4.94999999999994 -0.000828941397991575
4.95999999999994 -0.000854825087001539
4.96999999999994 -0.000879011006129877
4.97999999999994 -0.000901480501009033
4.98999999999994 -0.000922217239996031
4.99999999999994 -0.000941207222168692
5.00999999999994 -0.000958438782285843
5.01999999999994 -0.000973902592772246
5.02999999999994 -0.000987591662742295
5.03999999999994 -0.000999501334086339
5.04999999999994 -0.0010096292746512
5.05999999999994 -0.00101797546855662
5.06999999999994 -0.00102454220370637
5.07999999999994 -0.00102933405657978
5.08999999999994 -0.00103235787484939
5.09999999999994 -0.00103362275708039
5.10999999999994 -0.00103314002978099
5.11999999999994 -0.00103092322484262
5.12999999999994 -0.00102698805725571
5.13999999999993 -0.00102135240857475
5.14999999999993 -0.00101403633027024
5.15999999999993 -0.00100506157143968
5.16999999999993 -0.000994452002197018
5.17999999999993 -0.000982233604141436
5.18999999999993 -0.000968434218470915
5.19999999999993 -0.000953083526118324
5.20999999999993 -0.000936213268075238
5.21999999999993 -0.000917856944126288
5.22999999999993 -0.000898049778984971
5.23999999999993 -0.000876828675880681
5.24999999999993 -0.000854232164064432
5.25999999999993 -0.000830300342652613
5.26999999999993 -0.000805074821856125
5.27999999999993 -0.000778598662116011
5.28999999999993 -0.0007509163114444
5.29999999999993 -0.000722073541168671
5.30999999999993 -0.000692117380227792
5.31999999999993 -0.00066109604814543
5.32999999999993 -0.000629058886791364
5.33999999999993 -0.000596056291035962
5.34999999999993 -0.000562139638398915
5.35999999999993 -0.000527361217791289
5.36999999999993 -0.000491774157449087
5.37999999999993 -0.000455432352155878
5.38999999999993 -0.000418390389851839
5.39999999999993 -0.000380703477726443
5.40999999999993 -0.000342427367891833
5.41999999999993 -0.000303618282733829
5.42999999999993 -0.000264332840037299
5.43999999999993 -0.000224627977982316
5.44999999999993 -0.000184560880107265
5.45999999999993 -0.000144188900334443
5.46999999999993 -0.000103569488153246
5.47999999999993 -6.27601140552497e-05
5.48999999999993 -2.18181953174986e-05
5.49999999999993 1.91989777805952e-05
5.50999999999993 6.02343152104784e-05
5.51999999999993 0.000101230999824659
5.52999999999993 0.000142132559277971
5.53999999999993 0.000182882937151174
5.54999999999993 0.000223426563190206
5.55999999999993 0.000263708422576222
5.56999999999993 0.00030367412414288
5.57999999999993 0.000343269967459302
5.58999999999993 0.000382443008698875
5.59999999999993 0.000421141125215975
5.60999999999992 0.000459313078754736
5.61999999999992 0.00049690857721606
5.62999999999992 0.000533878334911257
5.63999999999992 0.000570174131232952
5.64999999999992 0.000605748867676303
5.65999999999992 0.000640556623145826
5.66999999999992 0.00067455270748582
5.67999999999992 0.000707693713174787
5.68999999999992 0.00073993756512688
5.69999999999992 0.000771243568546151
5.70999999999992 0.000801572454782016
5.71999999999992 0.000830886425137261
5.72999999999992 0.000859149192582599
5.73999999999992 0.000886326021334829
5.74999999999992 0.000912383764258455
5.75999999999992 0.000937290898053686
5.76999999999992 0.000961017556196665
5.77999999999992 0.000983535559600959
5.78999999999992 0.00100481844497224
5.79999999999992 0.00102484149083145
5.80999999999992 0.00104358174118469
5.81999999999992 0.00106101802753849
5.82999999999992 0.0010771309866893
5.83999999999992 0.00109190307723672
5.84999999999992 0.00110531859408056
5.85999999999992 0.00111736368011865
5.86999999999992 0.00112802633553383
5.87999999999992 0.00113729642467469
5.88999999999992 0.00114516568054071
5.89999999999992 0.00115162770688973
5.90999999999992 0.00115667797799711
5.91999999999992 0.0011603138361121
5.92999999999992 0.00116253448668561
5.93999999999992 0.00116334099149402
5.94999999999992 0.00116273625988244
5.95999999999992 0.001160725038556
5.96999999999992 0.0011573139008095
5.97999999999992 0.00115251123722857
5.98999999999992 0.00114632725302703
5.99999999999992 0.00113877398292089
6.00999999999992 0.00112986522742565
6.01999999999992 0.00111961581796481
6.02999999999992 0.00110804276953544
6.03999999999992 0.00109516475487291
6.04999999999992 0.00108100207257312
6.05999999999992 0.00106557661333116
6.06999999999992 0.00104891182425416
6.07999999999991 0.00103103257134074
6.08999999999991 0.0010119651576257
6.09999999999991 0.0009917375649034
6.10999999999991 0.000970379130114412
6.11999999999991 0.000947920521468604
6.12999999999991 0.000924393702352676
6.13999999999991 0.000899831889218375
6.14999999999991 0.000874269505925382
6.15999999999991 0.000847742135658953
6.16999999999991 0.000820286470991603
6.17999999999991 0.000791940262413891
6.18999999999991 0.000762742265543077
6.19999999999991 0.000732732185874542
6.20999999999991 0.000701950625383575
6.21999999999991 0.000670439027535378
6.22999999999991 0.000638239619079793
6.23999999999991 0.000605395352480285
6.24999999999991 0.000571949847650462
6.25999999999991 0.000537947333082864
6.26999999999991 0.000503432586452664
6.27999999999991 0.000468450874777746
6.28999999999991 0.000433047894215704
6.29999999999991 0.000397269709577696
6.30999999999991 0.000361162693638737
6.31999999999991 0.000324773466323368
6.32999999999991 0.00028814883384543
6.33999999999991 0.000251335727880021
6.34999999999991 0.000214381144845327
6.35999999999991 0.000177332085371417
6.36999999999991 0.000140235494032275
6.37999999999991 0.000103138199420407
6.38999999999991 6.60868546273609e-05
6.39999999999991 2.91278782166639e-05
6.40999999999991 -7.69260424183745e-06
6.41999999999991 -4.43288180054996e-05
6.42999999999991 -8.0737518974195e-05
6.43999999999991 -0.000116871776915786
6.44999999999991 -0.000152687209045219
6.45999999999991 -0.000188140005100654
6.46999999999991 -0.000223186981099874
6.47999999999991 -0.000257785632045113
6.48999999999991 -0.000291894183515762
6.49999999999991 -0.000325471642089776
6.50999999999991 -0.000358477844536155
6.51999999999991 -0.000390873505722421
6.52999999999991 -0.000422620265182755
6.53999999999991 -0.00045368073229404
6.5499999999999 -0.000484018530009089
6.5599999999999 -0.000513598337098017
6.5699999999999 -0.000542383014896397
6.5799999999999 -0.000570342430823082
6.5899999999999 -0.000597444684929147
6.5999999999999 -0.000623659087769146
6.6099999999999 -0.000648956194463405
6.6199999999999 -0.000673307837033337
6.6299999999999 -0.000696687154975508
6.6399999999999 -0.000719068624042821
6.6499999999999 -0.000740428083203709
6.6599999999999 -0.000760742759752735
6.6699999999999 -0.000779991292548694
6.6799999999999 -0.000798153753358845
6.6899999999999 -0.00081521166629057
6.6999999999999 -0.000831148025294513
6.7099999999999 -0.000845947310385325
6.7199999999999 -0.000859595500241081
6.7299999999999 -0.000872080083798817
6.7399999999999 -0.000883390070331681
6.7499999999999 -0.0008935159971717
6.7599999999999 -0.000902449935470658
6.7699999999999 -0.000910185494006413
6.7799999999999 -0.000916717821047698
6.7899999999999 -0.000922043604297847
6.7999999999999 -0.000926161068948769
6.8099999999999 -0.000929069973893019
6.8199999999999 -0.000930771606168833
6.8299999999999 -0.000931268773763252
6.8399999999999 -0.000930565796985264
6.8499999999999 -0.000928668498802934
6.8599999999999 -0.000925584194922016
6.8699999999999 -0.00092132168526862
6.8799999999999 -0.000915891249154558
6.8899999999999 -0.000909304611441432
6.8999999999999 -0.00090157493735924
6.9099999999999 -0.000892716453742649
6.9199999999999 -0.000882744769740168
6.9299999999999 -0.000871677020622556
6.9399999999999 -0.000859531655350611
6.9499999999999 -0.000846328407488606
6.9599999999999 -0.000832088264657493
6.9699999999999 -0.000816833237685479
6.9799999999999 -0.000800586673674915
6.9899999999999 -0.000783373190023493
6.9999999999999 -0.000765218521541429
7.00999999999989 -0.000746149460065306
7.01999999999989 -0.000726193825970503
7.02999999999989 -0.000705380434029113
7.03999999999989 -0.00068373905580009
7.04999999999989 -0.000661300379749047
7.05999999999989 -0.000638095969705484
7.06999999999989 -0.000614158221999586
7.07999999999989 -0.0005895203214924
7.08999999999989 -0.000564216196647835
7.09999999999989 -0.000538280473759659
7.10999999999989 -0.000511748430427448
7.11999999999989 -0.00048465594836397
7.12999999999989 -0.000457039465610106
7.13999999999989 -0.000428935928229273
7.14999999999989 -0.000400382741550813
7.15999999999989 -0.0003714177210304
7.16999999999989 -0.000342079042794242
7.17999999999989 -0.000312405193933345
7.18999999999989 -0.000282434922613374
7.19999999999989 -0.000252207188065278
7.20999999999989 -0.000221761110521385
7.21999999999989 -0.000191135921159048
7.22999999999989 -0.000160370912115443
7.23999999999989 -0.00012950538665105
7.24999999999989 -9.85786095060889e-05
7.25999999999989 -6.7629757518765e-05
7.26999999999989 -3.66978705700653e-05
7.27999999999989 -5.82001171819669e-06
7.28999999999989 2.49635407144578e-05
7.29999999999989 5.56144495473391e-05
7.30999999999989 8.60947044008799e-05
7.31999999999989 0.000116366668359133
7.32999999999989 0.000146393123943096
7.33999999999989 0.00017613731834166
7.34999999999989 0.000205563007847536
7.35999999999989 0.000234634501446879
7.36999999999989 0.000263316703512158
7.37999999999989 0.000291575155450231
7.38999999999989 0.00031937607648496
7.39999999999989 0.000346686403575021
7.40999999999989 0.000373473829904068
7.41999999999989 0.000399706842376132
7.42999999999989 0.000425354757938294
7.43999999999989 0.000450387758689532
7.44999999999989 0.000474776925736196
7.45999999999989 0.000498491235157021
7.46999999999989 0.000521506744269154
7.47999999999988 0.000543797442180646
7.48999999999988 0.000565338340641269
7.49999999999988 0.000586105501095078
7.50999999999988 0.000606076060294338
7.51999999999988 0.000625228254448438
7.52999999999988 0.000643541441883425
7.53999999999988 0.000660996124189963
7.54999999999988 0.000677573965839678
7.55999999999988 0.00069325781236641
7.56999999999988 0.000708031706620531
7.57999999999988 0.000721880903755601
7.58999999999988 0.00073479188537712
7.59999999999988 0.000746752369900651
7.60999999999988 0.000757751323101262
7.61999999999988 0.000767778966829675
7.62999999999988 0.000776826785929624
7.63999999999988 0.000784887533553268
7.64999999999988 0.000791955234884144
7.65999999999988 0.000798025189283944
7.66999999999988 0.00080309397088899
7.67999999999988 0.000807159427697048
7.68999999999988 0.000810220679208747
7.69999999999988 0.000812278112728514
7.70999999999988 0.00081333337850332
7.71999999999988 0.000813389384017868
7.72999999999988 0.000812450288053412
7.73999999999988 0.000810521489949887
7.74999999999988 0.000807609611508065
7.75999999999988 0.000803722493507435
7.76999999999988 0.000798869190549301
7.77999999999988 0.000793059970973194
7.78999999999988 0.00078630576743738
7.79999999999988 0.000778619020172148
7.80999999999988 0.000770013288159776
7.81999999999988 0.000760503226861238
7.82999999999988 0.000750104564708008
7.83999999999988 0.000738834078385235
7.84999999999988 0.00072670956693569
7.85999999999988 0.000713749583113468
7.86999999999988 0.000699973935897631
7.87999999999988 0.000685403360498981
7.88999999999988 0.000670059490887449
7.89999999999988 0.000653964844043959
7.90999999999988 0.000637142776776137
7.91999999999988 0.000619617448332432
7.92999999999988 0.000601413789366157
7.93999999999988 0.000582557468920349
7.94999999999987 0.000563074859980503
7.95999999999987 0.000542993003907879
7.96999999999987 0.000522339573949935
7.97999999999987 0.000501142837963488
7.98999999999987 0.000479431620452833
7.99999999999987 0.000457235264006297
8.00999999999987 0.000434583590203277
8.01999999999987 0.000411506860057427
8.02999999999987 0.00038803573405719
8.03999999999987 0.000364201231862421
8.04999999999987 0.000340034691714031
8.05999999999987 0.000315567729612398
8.06999999999987 0.000290832198319488
8.07999999999987 0.000265860146238804
8.08999999999987 0.000240683776227219
8.09999999999987 0.000215335404391568
8.10999999999987 0.000189847418923089
8.11999999999987 0.000164252239021999
8.12999999999987 0.000138582273964022
8.13999999999987 0.000112868462392435
8.14999999999987 8.7144327899443e-05
8.15999999999987 6.14420028545988e-05
8.16999999999987 3.57934503952166e-05
8.17999999999987 1.0230424840906e-05
8.18999999999987 -1.52155674362094e-05
8.19999999999987 -4.05133066403858e-05
8.20999999999987 -6.56318976355855e-05
8.21999999999987 -9.05408074391579e-05
8.22999999999987 -0.000115209902082366
8.23999999999987 -0.000139609482795208
8.24999999999987 -0.000163710321473577
8.25999999999987 -0.000187483695387855
8.26999999999987 -0.000210901421093137
8.27999999999987 -0.000233935887474376
8.28999999999987 -0.000256560087800094
8.29999999999987 -0.000278747651354379
8.30999999999987 -0.000300472873601856
8.31999999999987 -0.000321710745544478
8.32999999999987 -0.000342436982063528
8.33999999999987 -0.000362628049215174
8.34999999999987 -0.000382261190448976
8.35999999999987 -0.000401314451720089
8.36999999999987 -0.000419766705466999
8.37999999999987 -0.000437594565322085
8.38999999999987 -0.000454781822508391
8.39999999999987 -0.000471309988040943
8.40999999999987 -0.000487161484862129
8.41999999999986 -0.000502319665851657
8.42999999999986 -0.000516768830585999
8.43999999999986 -0.000530494240832151
8.44999999999986 -0.000543482134762364
8.45999999999986 -0.000555719739878455
8.46999999999986 -0.000567195284977926
8.47999999999986 -0.000577898010217318
8.48999999999986 -0.00058781817553063
8.49999999999986 -0.000596947068808741
8.50999999999986 -0.000605277012277963
8.51999999999986 -0.000612801367569032
8.52999999999986 -0.000619514539485376
8.53999999999986 -0.000625411978485807
8.54999999999986 -0.000630490181905725
8.55999999999986 -0.000634746693954546
8.56999999999986 -0.00063818010454902
8.57999999999986 -0.000640790047078914
8.58999999999986 -0.00064257719526727
8.59999999999986 -0.000643543257127718
8.60999999999986 -0.000643690966088981
8.61999999999986 -0.000643024075748282
8.62999999999986 -0.000641547352454532
8.63999999999986 -0.000639266567973389
8.64999999999986 -0.000636188493722351
8.65999999999986 -0.000632320900809666
8.66999999999986 -0.00062767231945417
8.67999999999986 -0.000622252179122643
8.68999999999986 -0.000616070988449339
8.69999999999986 -0.000609140166106649
8.70999999999986 -0.000601472022033637
8.71999999999986 -0.000593079737665097
8.72999999999986 -0.000583977345183755
8.73999999999986 -0.000574179705820694
8.74999999999986 -0.000563702304312367
8.75999999999986 -0.000552561615708362
8.76999999999986 -0.000540774877753776
8.77999999999986 -0.000528360054196516
8.78999999999986 -0.00051533581140106
8.79999999999986 -0.000501721493073425
8.80999999999986 -0.000487537093674446
8.81999999999986 -0.000472803233384314
8.82999999999986 -0.000457541130533917
8.83999999999986 -0.000441772567119649
8.84999999999986 -0.000425519859052948
8.85999999999986 -0.000408805826240352
8.86999999999986 -0.000391653761969027
8.87999999999986 -0.000374087401705414
8.88999999999985 -0.000356130891390453
8.89999999999985 -0.0003378087553003
8.90999999999985 -0.00031914586353264
8.91999999999985 -0.000300167399173401
8.92999999999985 -0.000280898825194959
8.93999999999985 -0.000261365851134712
8.94999999999985 -0.000241594399601246
8.95999999999985 -0.000221610572654195
8.96999999999985 -0.000201440618103103
8.97999999999985 -0.000181110895769825
8.98999999999985 -0.000160647843758595
8.99999999999985 -0.000140077944777109
9.00999999999985 -0.000119427692551848
9.01999999999985 -9.87235583803074e-05
9.02999999999985 -7.79906480720846e-05
9.03999999999985 -5.7256425282872e-05
9.04999999999985 -3.65470930131651e-05
9.05999999999985 -1.58887021304518e-05
9.06999999999985 4.69288088570665e-06
9.07999999999985 2.51720058045781e-05
9.08999999999985 4.55232700746239e-05
9.09999999999985 6.57215497873146e-05
9.10999999999985 8.57420301387419e-05
9.11999999999985 0.00010556023535397
9.12999999999985 0.000125152058039655
9.13999999999985 0.000144493787931155
9.14999999999985 0.000163562140001165
9.15999999999985 0.000182334281897641
9.16999999999985 0.000200787860541249
9.17999999999985 0.000218901028347084
9.18999999999985 0.000236652468423627
9.19999999999985 0.0002540214189856
9.20999999999985 0.000270987696985738
9.21999999999985 0.000287531720901533
9.22999999999985 0.000303634532651297
9.23999999999985 0.000319277818614868
9.24999999999985 0.000334443929735309
9.25999999999985 0.000349112928559157
9.26999999999985 0.000363271593135913
9.27999999999985 0.000376904406623755
9.28999999999985 0.000389996588753379
9.29999999999985 0.000402534111105128
9.30999999999985 0.000414503711368149
9.31999999999985 0.000425892906568844
9.32999999999985 0.000436690005257449
9.33999999999985 0.000446884118643201
9.34999999999985 0.000456465170727281
9.35999999999984 0.000465423907779174
9.36999999999984 0.000473751905121898
9.37999999999984 0.000481441574625354
9.38999999999984 0.000488486170506872
9.39999999999984 0.000494879794065599
9.40999999999984 0.000500617397359133
9.41999999999984 0.000505694785836645
9.42999999999984 0.000510108619951028
9.43999999999984 0.000513856415771249
9.44999999999984 0.000516936543756351
9.45999999999984 0.000519348226681352
9.46999999999984 0.000521091537110691
9.47999999999984 0.000522167393766589
9.48999999999984 0.000522577557010436
9.49999999999984 0.000522324623551474
9.50999999999984 0.000521412020584409
9.51999999999984 0.000519843999747797
9.52999999999984 0.000517625631737965
9.53999999999984 0.000514762803729688
9.54999999999984 0.000511262173817611
9.55999999999984 0.000507130906933957
9.56999999999984 0.00050237715601023
9.57999999999984 0.000497009824623378
9.58999999999984 0.00049103855207422
9.59999999999984 0.000484473697639623
9.60999999999984 0.000477326324016009
9.61999999999984 0.000469608179973957
9.62999999999984 0.000461331576032831
9.63999999999984 0.000452509543717814
9.64999999999984 0.000443155804801596
9.65999999999984 0.000433284683194432
9.66999999999984 0.000422911086683687
9.67999999999984 0.000412050486848568
9.68999999999984 0.00040071889776643
9.69999999999984 0.000388932853811973
9.70999999999984 0.000376709386716464
9.71999999999984 0.000364066001991814
9.72999999999984 0.00035102065479318
9.73999999999984 0.000337591725360972
9.74999999999984 0.000323797995733857
9.75999999999984 0.000309658622458662
9.76999999999984 0.000295193109864874
9.77999999999984 0.000280421284024876
9.78999999999984 0.000265363266374922
9.79999999999984 0.000250039447045846
9.80999999999984 0.000234470457948399
9.81999999999984 0.000218677145655244
9.82999999999983 0.000202680544119826
9.83999999999983 0.000186501847270987
9.84999999999983 0.000170162381521314
9.85999999999983 0.00015368357822631
9.86999999999983 0.000137086946045393
9.87999999999983 0.000120394043469399
9.88999999999983 0.000103626451528469
9.89999999999983 8.68057462599968e-05
9.90999999999983 6.99534714630358e-05
9.91999999999983 5.30911116141907e-05
9.92999999999983 3.62387060330179e-05
9.93999999999983 1.94187253799691e-05
9.94999999999983 2.65231664988914e-06
9.95999999999983 -1.40395368649848e-05
9.96999999999983 -3.06360414284512e-05
9.97999999999983 -4.71166183454376e-05
9.98999999999983 -6.34609290495802e-05
9.99999999999983 -7.964889976446e-05
10.0099999999998 -9.56607457102827e-05
10.0199999999998 -0.000111476994827946
10.0299999999998 -0.000127078510993448
10.0399999999998 -0.000142446516695849
10.0499999999998 -0.00015756261509766
10.0599999999998 -0.00017240881157533
10.0699999999998 -0.000186967534758854
10.0799999999998 -0.000201221656730533
10.0899999999998 -0.000215154512645181
10.0999999999998 -0.000228749919668896
10.1099999999998 -0.000241992195214833
10.1199999999998 -0.000254863351862101
10.1299999999998 -0.000267351597149971
10.1399999999998 -0.000279442876254378
10.1499999999998 -0.000291123701108601
10.1599999999998 -0.000302381164893086
10.1699999999998 -0.000313202955723211
10.1799999999998 -0.000323577369521979
10.1899999999998 -0.000333493322065881
10.1999999999998 -0.000342940360193277
10.2099999999998 -0.000351908672166025
10.2199999999998 -0.000360389097176292
10.2299999999998 -0.000368373133992056
10.2399999999998 -0.00037585294908076
10.2499999999998 -0.000382821382958865
10.2599999999998 -0.000389271956271507
10.2699999999998 -0.000395198875130496
10.2799999999998 -0.000400597035413581
10.2899999999998 -0.000405462025923608
10.2999999999998 -0.000409790130741297
10.3099999999998 -0.000413578330814968
10.3199999999998 -0.00041682430467916
10.3299999999998 -0.000419526428312231
10.3399999999998 -0.000421683774147435
10.3499999999998 -0.000423296109258454
10.3599999999998 -0.000424363892750419
10.3699999999998 -0.000424888272404212
10.3799999999998 -0.000424871080652386
10.3899999999998 -0.000424314830021245
10.3999999999998 -0.00042322270829292
10.4099999999998 -0.000421598573909095
10.4199999999998 -0.000419446952820644
10.4299999999998 -0.000416773040375553
10.4399999999998 -0.000413582453551226
10.4499999999998 -0.000409881539522336
10.4599999999998 -0.000405677309764068
10.4699999999998 -0.000400977381661156
10.4799999999998 -0.000395789965951871
10.4899999999998 -0.000390123853503803
10.4999999999998 -0.000383988401436961
10.5099999999998 -0.000377393518611267
10.5199999999998 -0.000370349518175526
10.5299999999998 -0.000362867392387164
10.5399999999998 -0.00035495863136209
10.5499999999998 -0.000346635208074377
10.5599999999998 -0.000337909562708243
10.5699999999998 -0.000328794585760155
10.5799999999998 -0.000319303600268504
10.5899999999998 -0.000309450343365207
10.5999999999998 -0.000299248947262535
10.6099999999998 -0.000288713919749524
10.6199999999998 -0.000277860124252442
10.6299999999998 -0.000266702759503227
10.6399999999998 -0.000255257338853743
10.6499999999998 -0.000243539669270443
10.6599999999998 -0.000231565829919811
10.6699999999998 -0.000219352150765011
10.6799999999998 -0.000206915190902497
10.6899999999998 -0.00019427171645491
10.6999999999998 -0.000181438678475361
10.7099999999998 -0.00016843319112743
10.7199999999998 -0.000155272508792692
10.7299999999998 -0.00014197400345371
10.7399999999998 -0.000128555142230988
10.7499999999998 -0.000115033464893068
10.7599999999998 -0.000101426561370291
10.7699999999998 -8.77520493017954e-05
10.7799999999998 -7.40275516452067e-05
10.7899999999998 -6.02706743779184e-05
10.7999999999998 -4.6498984318326e-05
10.8099999999998 -3.27299870943226e-05
10.8199999999998 -1.89811052881049e-05
10.8299999999998 -5.26965678355284e-06
10.8399999999998 8.38867648610009e-06
10.8499999999998 2.19754998038374e-05
10.8599999999998 3.54739320882484e-05
10.8699999999998 4.88672796249112e-05
10.8799999999998 6.21390563718082e-05
10.8899999999998 7.52730039028754e-05
10.8999999999998 8.82531109667092e-05
10.9099999999998 0.000101063632638033
10.9199999999998 0.00011368910903986
10.9299999999998 0.000126114383596082
10.9399999999998 0.000138324620766731
10.9499999999998 0.00015030532352701
10.9599999999998 0.000162042350075052
10.9699999999998 0.000173519407206441
10.9799999999998 0.000184725595541076
10.9899999999998 0.000195647955099068
10.9999999999998 0.00020627394369705
11.0099999999998 0.000216591450939308
11.0199999999998 0.000226588811597533
11.0299999999998 0.00023625481836555
11.0399999999998 0.000245578733976364
11.0499999999998 0.000254550302669571
11.0599999999998 0.000263159760998297
11.0699999999998 0.000271397847965665
11.0799999999998 0.000279255814565155
11.0899999999998 0.000286725432537443
11.0999999999998 0.000293799002296601
11.1099999999998 0.000300469360281673
11.1199999999998 0.000306729885697886
11.1299999999998 0.000312574506518572
11.1399999999998 0.000317997704210236
11.1499999999998 0.000322994518522197
11.1599999999998 0.000327560551251689
11.1699999999998 0.00033169196928639
11.1799999999998 0.000335385506925691
11.1899999999998 0.00033863846748352
11.1999999999998 0.000341448724177477
11.2099999999998 0.000343814720311429
11.2199999999998 0.000345735468761929
11.2299999999998 0.000347210550767423
11.2399999999998 0.00034824011399644
11.2499999999998 0.000348824870190953
11.2599999999998 0.000348966092057585
11.2699999999998 0.000348665609718026
11.2799999999998 0.000347925806830119
11.2899999999998 0.000346749616693737
11.2999999999998 0.000345140519029529
11.3099999999998 0.000343102539231469
11.3199999999998 0.000340640156811537
11.3299999999998 0.000337758280220572
11.3399999999998 0.000334462438358695
11.3499999999998 0.000330758659443635
11.3599999999998 0.000326653460988978
11.3699999999998 0.000322153839230542
11.3799999999998 0.000317267258013242
11.3899999999998 0.000312001637151941
11.3999999999998 0.000306365278409416
11.4099999999998 0.000300366941069578
11.4199999999998 0.000294015856629991
11.4299999999998 0.000287321655940409
11.4399999999998 0.000280294357110507
11.4499999999998 0.000272944352129318
11.4599999999998 0.000265282392645924
11.4699999999998 0.000257319575130028
11.4799999999998 0.000249067325527898
11.4899999999998 0.000240537383493884
11.4999999999998 0.000231741786244986
11.5099999999998 0.000222692852078985
11.5199999999998 0.000213403163589453
11.5299999999998 0.000203885550607045
11.5399999999998 0.000194153072894427
11.5499999999998 0.000184219002620832
11.5599999999998 0.000174096806641575
11.5699999999998 0.000163800128607182
11.5799999999998 0.000153342770926682
11.5899999999998 0.000142738676609263
11.5999999999998 0.000132001911008352
11.6099999999998 0.000121146643492146
11.6199999999998 0.000110187129064356
11.6299999999998 9.91376899589591e-05
11.6399999999998 8.80126972324999e-05
11.6499999999998 7.68265523774129e-05
11.6599999999998 6.55936689796519e-05
11.6699999999998 5.43284544437054e-05
11.6799999999998 4.30452918081996e-05
11.6899999999998 3.17585216737061e-05
11.6999999999998 2.0482424266415e-05
11.7099999999998 9.23120165916638e-06
11.7199999999998 -1.98103982827363e-06
11.7299999999998 -1.31403070282009e-05
11.7399999999998 -2.42327371021105e-05
11.7499999999998 -3.52446144752656e-05
11.7599999999998 -4.61623875083366e-05
11.7699999999998 -5.69726848862118e-05
11.7799999999998 -6.76623317045492e-05
11.7899999999998 -7.82183652350661e-05
11.7999999999998 -8.86280503510962e-05
11.8099999999998 -9.88788945954662e-05
11.8199999999998 -0.000108958662866602
11.8299999999998 -0.000118855391732709
11.8399999999998 -0.000128557403318403
11.8499999999998 -0.000138053318771994
11.8599999999998 -0.000147332071294823
11.8699999999998 -0.000156382918717021
11.8799999999998 -0.00016519545560588
11.8899999999998 -0.000173759624893681
11.8999999999998 -0.000182065729012476
11.9099999999998 -0.000190104440524053
11.9199999999998 -0.00019786681223394
11.9299999999998 -0.000205344286779107
11.9399999999998 -0.000212528705679665
11.9499999999998 -0.000219412317845665
11.9599999999998 -0.000225987787530811
11.9699999999998 -0.000232248201725677
11.9799999999998 -0.000238187076983789
11.9899999999998 -0.00024379836567471
11.9999999999998 -0.000249076461659089
12.0099999999998 -0.00025401620561244
12.0199999999998 -0.00025861288910376
12.0299999999998 -0.000262862258652076
12.0399999999998 -0.000266760519112318
12.0499999999998 -0.000270304336414848
12.0599999999998 -0.000273490839715184
12.0699999999998 -0.00027631762295561
12.0799999999998 -0.000278782745841821
12.0899999999998 -0.000280884734239486
12.0999999999998 -0.000282622579997933
12.1099999999998 -0.00028399574021138
12.1199999999998 -0.000285004135932844
12.1299999999998 -0.000285648150363454
12.1399999999998 -0.000285928626552509
12.1499999999998 -0.000285846864666869
12.1599999999998 -0.000285404618898398
12.1699999999998 -0.000284604093855026
12.1799999999998 -0.000283447942205422
12.1899999999998 -0.000281939262610169
12.1999999999998 -0.000280081601801891
12.2099999999998 -0.00027787874595726
12.2199999999998 -0.000275335032228845
12.2299999999998 -0.000272455214605746
12.2399999999998 -0.000269244450771584
12.2499999999998 -0.000265708293657336
12.2599999999998 -0.000261852682548231
12.2699999999998 -0.000257683933755463
12.2799999999998 -0.000253208724575756
12.2899999999998 -0.000248434012599461
12.2999999999998 -0.000243367203288803
12.3099999999998 -0.000238016032319367
12.3199999999998 -0.000232388556430822
12.3299999999998 -0.000226493142911462
12.3399999999998 -0.00022033845825841
12.3499999999998 -0.000213933456260308
12.3599999999998 -0.000207287365630328
12.3699999999998 -0.000200409677264535
12.3799999999998 -0.000193310131174972
12.3899999999998 -0.000185998703133858
12.3999999999998 -0.000178485591058207
12.4099999999998 -0.000170781201160184
12.4199999999998 -0.000162896133886325
12.4299999999998 -0.000154841169667338
12.4399999999998 -0.000146627254499394
12.4499999999998 -0.000138265485377336
12.4599999999998 -0.000129767095599942
12.4699999999998 -0.00012114343996706
12.4799999999998 -0.000112405979888448
12.4899999999998 -0.000103566268423889
12.4999999999998 -9.46359352741395e-05
12.5099999999998 -8.56266717421443e-05
12.5199999999998 -7.65502156838581e-05
12.5299999999998 -6.74183364678735e-05
12.5399999999998 -5.82428199629647e-05
12.5499999999998 -4.9035453572481e-05
12.5599999999998 -3.98080113344824e-05
12.5699999999998 -3.05722391059661e-05
12.5799999999998 -2.13398398496719e-05
12.5899999999998 -1.21224590418597e-05
12.5999999999998 -2.93167021878997e-06
12.6099999999998 6.22103932037881e-06
12.6199999999998 1.5324282636869e-05
12.6299999999998 2.43667870846868e-05
12.6399999999998 3.33374080353344e-05
12.6499999999998 4.22251423784704e-05
12.6599999999998 5.10191417825535e-05
12.6699999999998 5.97087256994599e-05
12.6799999999998 6.82833940977946e-05
12.6899999999998 7.6732839910669e-05
12.6999999999998 8.50469611809216e-05
12.7099999999998 9.32158728951233e-05
12.7199999999998 0.000101229918492765
12.7299999999998 0.000109079681026485
12.7399999999998 0.000116755993972417
12.7499999999998 0.000124249951674948
12.7599999999998 0.000131552919414375
12.7699999999998 0.000138656543086468
12.7799999999998 0.000145552758483475
12.7899999999998 0.000152233800166695
12.7999999999998 0.000158692209921278
12.8099999999998 0.000164920844784536
12.8199999999998 0.000170912884639575
12.8299999999998 0.000176661839366712
12.8399999999998 0.000182161555545725
12.8499999999998 0.000187406222702593
12.8599999999998 0.000192390379095032
12.8699999999998 0.000197108917031756
12.8799999999998 0.000201557087721042
12.8899999999998 0.000205730505721756
12.8999999999998 0.000209625152808554
12.9099999999998 0.000213237381202813
12.9199999999998 0.000216563916645543
12.9299999999998 0.000219601860850678
12.9399999999998 0.000222348693476021
12.9499999999998 0.000224802273612776
12.9599999999998 0.000226960840795664
12.9699999999998 0.000228823015536942
12.9799999999998 0.000230387799389271
12.9899999999998 0.000231654574544603
12.9999999999998 0.000232623102979498
13.0099999999998 0.000233293525162139
13.0199999999998 0.000233666358344396
13.0299999999998 0.000233742494476399
13.0399999999998 0.000233523197807321
13.0499999999998 0.000233010102289174
13.0599999999998 0.000232205209017361
13.0699999999998 0.000231110884228316
13.0799999999998 0.000229729859241977
13.0899999999998 0.000228065145967398
13.0999999999998 0.000226120075997402
13.1099999999998 0.000223898381724716
13.1199999999998 0.000221404128497022
13.1299999999998 0.000218641707880387
13.1399999999998 0.000215615830554209
13.1499999999998 0.000212331518846189
13.1599999999998 0.000208794098916668
13.1699999999998 0.000205009149580064
13.1799999999998 0.0002009825578044
13.1899999999998 0.000196720523415201
13.1999999999998 0.000192229512126881
13.2099999999998 0.000187516247412674
13.2199999999998 0.00018258770152621
13.2299999999998 0.000177451085968344
13.2399999999998 0.000172113841542281
13.2499999999998 0.00016658362807594
13.2599999999998 0.000160868313860737
13.2699999999998 0.000154975964841026
13.2799999999998 0.000148914833580688
13.2899999999998 0.000142693348029016
13.2999999999998 0.000136320100105571
13.3099999999998 0.000129803834122338
13.3199999999998 0.000123153435060502
13.3299999999998 0.000116377916718801
13.3399999999998 0.000109486409749963
13.3499999999998 0.000102488149601587
13.3599999999998 9.53924643776669e-05
13.3699999999998 8.82087626368406e-05
13.3799999999998 8.09465211433587e-05
13.3899999999998 7.36152725867219e-05
13.3999999999998 6.62245932858046e-05
13.4099999999998 5.87840908931987e-05
13.4199999999998 5.13033921154747e-05
13.4299999999998 4.37921304648517e-05
13.4399999999998 3.62599340576881e-05
13.4499999999998 2.87164134752157e-05
13.4599999999998 2.1171149701178e-05
13.4699999999998 1.36336821518415e-05
13.4799999999998 6.11349681284037e-06
13.4899999999998 -1.3799855025505e-06
13.4999999999998 -8.83742075940876e-06
13.5099999999998 -1.62495531573552e-05
13.5199999999998 -2.36072264231202e-05
13.5299999999998 -3.09013949259442e-05
13.5399999999998 -3.81231346025717e-05
13.5499999999998 -4.52636536788601e-05
13.5599999999998 -5.23143031754042e-05
13.5699999999998 -5.92665871848564e-05
13.5799999999998 -6.61121729089912e-05
13.5899999999998 -7.28429004409423e-05
13.5999999999998 -7.94507922935344e-05
13.6099999999998 -8.59280626447302e-05
13.6199999999998 -9.22671263011419e-05
13.6299999999998 -9.84606073675941e-05
13.6399999999998 -0.000104501347612716
13.6499999999998 -0.000110382414521393
13.6599999999998 -0.000116097109025328
13.6699999999998 -0.000121638972903466
13.6799999999998 -0.000127001795844406
13.6899999999998 -0.000132179622163502
13.6999999999998 -0.000137166757167727
13.7099999999998 -0.000141957773161949
13.7199999999998 -0.000146547515090687
13.7299999999998 -0.000150931105809972
13.7399999999998 -0.00015510395098442
13.7499999999998 -0.000159061743605132
13.7599999999998 -0.000162800468124599
13.7699999999998 -0.000166316404205277
13.7799999999998 -0.000169606130211467
13.7899999999998 -0.000172666525915588
13.7999999999998 -0.000175494775186623
13.8099999999998 -0.000178088368186482
13.8199999999997 -0.000180445103157679
13.8299999999997 -0.000182563087817244
13.8399999999997 -0.000184440740358125
13.8499999999997 -0.000186076790060241
13.8599999999997 -0.000187470277514595
13.8699999999997 -0.000188620554465394
13.8799999999997 -0.000189527283277327
13.8899999999997 -0.000190190436038414
13.8999999999997 -0.000190610293314014
13.9099999999997 -0.000190787442576343
13.9199999999997 -0.000190722776349818
13.9299999999997 -0.000190417490143296
13.9399999999997 -0.000189873080305484
13.9499999999997 -0.000189091342090915
13.9599999999997 -0.000188074368621008
13.9699999999997 -0.000186824543833238
13.9799999999997 -0.000185344417761556
13.9899999999997 -0.000183636911840236
13.9999999999997 -0.000181705221878513
14.0099999999997 -0.000179552812697703
14.0199999999997 -0.000177183412463943
14.0299999999997 -0.000174601006723312
14.0399999999997 -0.000171809832146763
14.0499999999997 -0.000168814366694857
14.0599999999997 -0.000165619272996262
14.0699999999997 -0.0001622295129339
14.0799999999997 -0.000158650268172632
14.0899999999997 -0.000154886934018262
14.0999999999997 -0.00015094511237202
14.1099999999997 -0.000146830604137838
14.1199999999997 -0.000142549401245824
14.1299999999997 -0.000138107678376895
14.1399999999997 -0.000133511784438523
14.1499999999997 -0.000128768233824589
14.1599999999997 -0.000123883697483621
14.1699999999997 -0.000118864993815037
14.1799999999997 -0.000113719079410358
14.1899999999997 -0.000108453039654671
14.1999999999997 -0.000103074079203162
14.2099999999997 -9.75895123464896e-05
14.2199999999997 -9.20067532786939e-05
14.2299999999997 -8.63333062810775e-05
14.2399999999997 -8.05767558353422e-05
14.2499999999997 -7.47447566791664e-05
14.2599999999997 -6.88450238173333e-05
14.2699999999997 -6.28853225014429e-05
14.2799999999997 -5.687345819118e-05
14.2899999999997 -5.08172665100287e-05
14.2999999999997 -4.47246032082745e-05
14.3099999999997 -3.86033341460028e-05
14.3199999999997 -3.2461325308744e-05
14.3299999999997 -2.63064328683151e-05
14.3399999999997 -2.01464933011743e-05
14.3499999999997 -1.39893135765962e-05
14.3599999999997 -7.84266142681837e-06
14.3699999999997 -1.71425571107987e-06
14.3799999999997 4.38824311469184e-06
14.3899999999997 1.04572424109299e-05
14.3999999999997 1.64852266388915e-05
14.4099999999997 2.2464766505302e-05
14.4199999999997 2.83885279563177e-05
14.4299999999997 3.42492810100725e-05
14.4399999999997 4.00399084173981e-05
14.4499999999997 4.5753414140539e-05
14.4599999999997 5.13829316399862e-05
14.4699999999997 5.69217319588313e-05
14.4799999999997 6.23632315982424e-05
14.4899999999997 6.77010001737403e-05
14.4999999999997 7.29287678400665e-05
14.5099999999997 7.80404324807288e-05
14.5199999999997 8.3030066652771e-05
14.5299999999997 8.78919242790978e-05
14.5399999999997 9.26204470810883e-05
14.5499999999997 9.72102707445426e-05
14.5599999999997 0.000101656230812425
14.5699999999997 0.000105953368298213
14.5799999999997 0.000110096935014075
14.5899999999997 0.000114082398608467
14.5999999999997 0.000117905447308167
14.6099999999997 0.000121561994360145
14.6199999999997 0.000125048182169085
14.6299999999997 0.000128360386126829
14.6399999999997 0.00013149521813037
14.6499999999997 0.000134449529785535
14.6599999999997 0.000137220415340079
14.6699999999997 0.000139805214211478
14.6799999999997 0.000142201513167818
14.6899999999997 0.000144407148314047
14.6999999999997 0.000146420206696826
14.7099999999997 0.000148239027587616
14.7199999999997 0.000149862203444675
14.7299999999997 0.000151288580555393
14.7399999999997 0.000152517259361242
14.7499999999997 0.000153547594468791
14.7599999999997 0.000154379194351715
14.7699999999997 0.000155011920750962
14.7799999999997 0.000155445887783607
14.7899999999997 0.000155681460776547
14.7999999999997 0.000155719254851002
14.8099999999997 0.000155560133302185
14.8199999999997 0.00015520520585593
14.8299999999997 0.000154655826966869
14.8399999999997 0.000153913594527385
14.8499999999997 0.000152980349975484
14.8599999999997 0.000151858110619467
14.8699999999997 0.000150549124171091
14.8799999999997 0.000149055898137647
14.8899999999997 0.000147381161679164
14.8999999999997 0.000145527861094146
14.9099999999997 0.000143499155059366
14.9199999999997 0.000141298409629632
14.9299999999997 0.00013892919300392
14.9399999999997 0.000136395240913442
14.9499999999997 0.000133700495414749
14.9599999999997 0.000130849105667824
14.9699999999997 0.000127845397634393
14.9799999999997 0.000124693868616314
14.9899999999997 0.000121399181240169
14.9999999999997 0.000117966157077419
15.0099999999997 0.000114399769993055
15.0199999999997 0.000110705139274323
15.0299999999997 0.000106887522571831
15.0399999999997 0.000102952308675692
15.0499999999997 9.89050101442495e-05
15.0599999999997 9.4751255800084e-05
15.0699999999997 9.04967831064337e-05
15.0799999999997 8.61474304362106e-05
15.0899999999997 8.17091292451937e-05
15.0999999999997 7.7187896160684e-05
15.1099999999997 7.25898249966538e-05
15.1199999999997 6.79210787062764e-05
15.1299999999997 6.31878812826705e-05
15.1399999999997 5.83965096185604e-05
15.1499999999997 5.35532853355349e-05
15.1599999999997 4.86645665935356e-05
15.1699999999997 4.37367398911003e-05
15.1799999999997 3.87762118668934e-05
15.1899999999997 3.37894011129309e-05
15.1999999999997 2.87827300098579e-05
15.2099999999997 2.37626165945561e-05
15.2199999999997 1.87354664702585e-05
15.2299999999997 1.37076647691488e-05
15.2399999999997 8.68556817752515e-06
15.2499999999997 3.67549703325187e-06
15.2599999999997 -1.31627249477988e-06
15.2699999999997 -6.28351613568863e-06
15.2799999999997 -1.12200691447459e-05
15.2899999999997 -1.61198338214535e-05
15.2999999999997 -2.09767869085339e-05
15.3099999999997 -2.57849868630332e-05
15.3199999999997 -3.05385809909006e-05
15.3299999999997 -3.52318124366802e-05
15.3399999999997 -3.98590270201068e-05
15.3499999999997 -4.44146799116884e-05
15.3599999999997 -4.88933421381129e-05
15.3699999999997 -5.32897069160596e-05
15.3799999999997 -5.75985957978224e-05
15.3899999999997 -6.18149646282974e-05
15.3999999999997 -6.593390930501e-05
15.4099999999997 -6.99506713348625e-05
15.4199999999997 -7.38606431814534e-05
15.4299999999997 -7.76593733971917e-05
15.4399999999997 -8.13425715347381e-05
15.4499999999997 -8.49061128325716e-05
15.4599999999997 -8.8346042669818e-05
15.4699999999997 -9.16585807858065e-05
15.4799999999997 -9.48401252601055e-05
15.4899999999997 -9.78872562491662e-05
15.4999999999997 -0.000100796739475997
15.5099999999997 -0.000103565529469654
15.5199999999997 -0.000106190772551671
15.5299999999997 -0.000108669809566905
15.5399999999997 -0.000111000178356637
15.5499999999997 -0.000113179616040821
15.5599999999997 -0.000115206060828939
15.5699999999997 -0.000117077653781276
15.5799999999997 -0.000118792740238449
15.5899999999997 -0.000120349870986798
15.5999999999997 -0.000121747803161331
15.6099999999997 -0.000122985500887126
15.6199999999997 -0.000124062135660694
15.6299999999997 -0.000124977086473675
15.6399999999997 -0.000125729939682278
15.6499999999997 -0.000126320488627394
15.6599999999997 -0.000126748733012575
15.6699999999997 -0.000127014878050664
15.6799999999997 -0.000127119333396012
15.6899999999997 -0.000127062711890409
15.6999999999997 -0.000126845828172685
15.7099999999997 -0.000126469697248398
15.7199999999997 -0.000125935533224695
15.7299999999997 -0.000125244748704456
15.7399999999997 -0.000124398941262267
15.7499999999997 -0.0001233998328462
15.7599999999997 -0.000122249386899321
15.7699999999997 -0.000120949749373478
15.7799999999997 -0.000119503245138165
15.7899999999997 -0.000117912374186206
15.7999999999997 -0.000116179807641024
15.8099999999997 -0.000114308383570581
15.8199999999997 -0.000112301098867078
15.8299999999997 -0.000110161076928776
15.8399999999997 -0.000107891637568243
15.8499999999997 -0.000105496246673583
15.8599999999997 -0.000102978512058966
15.8699999999997 -0.000100342178735561
15.8799999999997 -9.759112382775e-05
15.8899999999997 -9.47293512385054e-05
15.8999999999997 -9.17609861184898e-05
15.9099999999997 -8.86902691711926e-05
15.9199999999997 -8.55215508156275e-05
15.9299999999997 -8.22592852225391e-05
15.9399999999997 -7.89080242370539e-05
15.9499999999997 -7.54724111990073e-05
15.9599999999997 -7.19571746712091e-05
15.9699999999997 -6.83671220852888e-05
15.9799999999997 -6.47071333144551e-05
15.9899999999997 -6.09821541822282e-05
15.9999999999997 -5.71971899161192e-05
16.0099999999997 -5.33572985550932e-05
16.0199999999997 -4.94675843196037e-05
16.0299999999997 -4.55331909529259e-05
16.0399999999997 -4.1559295042489e-05
16.0499999999997 -3.75510993298404e-05
16.0599999999997 -3.3513826017812e-05
16.0699999999997 -2.94527100834604e-05
16.0799999999997 -2.53729926052533e-05
16.0899999999997 -2.12799141128752e-05
16.0999999999997 -1.71787079680554e-05
16.1099999999997 -1.3074593784617e-05
16.1199999999997 -8.97277089589621e-06
16.1299999999997 -4.87841187766967e-06
16.1399999999997 -7.9665613444985e-07
16.1499999999997 3.26739644294804e-06
16.1599999999997 7.30869174092752e-06
16.1699999999997 1.13222276001105e-05
16.1799999999997 1.53030598984202e-05
16.1899999999997 1.92463085329739e-05
16.1999999999997 2.31471632936911e-05
16.2099999999997 2.70008896216609e-05
16.2199999999997 3.08028342455559e-05
16.2299999999997 3.45484306894854e-05
16.2399999999997 3.82332046454144e-05
16.2499999999997 4.1852779205719e-05
16.2599999999997 4.5402879948323e-05
16.2699999999997 4.88793398681585e-05
16.2799999999997 5.22781041508229e-05
16.2899999999997 5.55952347826034e-05
16.2999999999997 5.88269149917794e-05
16.3099999999998 6.19694535163967e-05
16.3199999999998 6.50192886939032e-05
16.3299999999998 6.79729923683263e-05
16.3399999999998 7.08272736108842e-05
16.3499999999998 7.35789822502173e-05
16.3599999999998 7.62251122086779e-05
16.3699999999998 7.8762804641367e-05
16.3799999999998 8.11893508748923e-05
16.3899999999998 8.35021951431075e-05
16.3999999999998 8.56989371173426e-05
16.4099999999998 8.77773342289501e-05
16.4199999999998 8.97353037822674e-05
16.4299999999998 9.15709248811844e-05
16.4399999999998 9.32824400895874e-05
16.4499999999998 9.486825688623e-05
16.4599999999998 9.63269489508253e-05
16.4699999999998 9.76572572114251e-05
16.4799999999998 9.88580906774644e-05
16.4899999999998 9.99285270589702e-05
16.4999999999998 0.000100867813172928
16.5099999999998 0.000101675365138403
16.5199999999998 0.000102350768362754
16.5299999999998 0.000102893777322367
16.5399999999998 0.000103304315142783
16.5499999999998 0.000103582472985505
16.5599999999998 0.000103728509252606
16.5699999999998 0.000103742848627111
16.5799999999998 0.000103626080980073
16.5899999999998 0.000103378960201635
16.5999999999998 0.00010300240307225
16.6099999999998 0.000102497488437247
16.6199999999998 0.000101865457392045
16.6299999999998 0.000101107659668936
16.6399999999998 0.000100225611903388
16.6499999999998 9.92209977137847e-05
16.6599999999998 9.80956480527143e-05
16.6699999999998 9.68515381865414e-05
16.6799999999998 9.54907845109981e-05
16.6899999999998 9.40156412069109e-05
16.6999999999998 9.24284967404234e-05
16.7099999999998 9.07318499931673e-05
16.7199999999998 8.89283391570808e-05
16.7299999999998 8.70207383545777e-05
16.7399999999998 8.50119391149289e-05
16.7499999999998 8.29049467013226e-05
16.7599999999998 8.07028760814444e-05
16.7699999999998 7.84089476611711e-05
16.7799999999998 7.60264828405656e-05
16.7899999999998 7.35588994253732e-05
16.7999999999998 7.10097069149538e-05
16.8099999999998 6.83825016814632e-05
16.8199999999998 6.56809620517885e-05
16.8299999999998 6.29088433019649e-05
16.8399999999998 6.00699725728017e-05
16.8499999999998 5.71682437147323e-05
16.8599999999998 5.42076120696627e-05
16.8699999999998 5.11920891972693e-05
16.8799999999998 4.81257375531353e-05
16.8899999999998 4.50126651259202e-05
16.8999999999998 4.18570200408176e-05
16.9099999999998 3.86629851364158e-05
16.9199999999998 3.54347725220571e-05
16.9299999999998 3.21766181227801e-05
16.9399999999998 2.8892776218868e-05
16.9499999999999 2.55875139869832e-05
16.9599999999999 2.22651060498306e-05
16.9699999999999 1.89298290412449e-05
16.9799999999999 1.55859561935233e-05
16.9899999999999 1.22377519537713e-05
16.9999999999999 8.8894666359488e-06
17.0099999999999 5.54533111522596e-06
17.0199999999999 2.20955157120612e-06
17.0299999999999 -1.11369571359734e-06
17.0399999999999 -4.42026949366369e-06
17.0499999999999 -7.70606863254118e-06
17.0599999999999 -1.0967037100995e-05
17.0699999999999 -1.41991688952278e-05
17.0799999999999 -1.73985128693339e-05
17.0899999999999 -2.05611774762574e-05
17.0999999999999 -2.36833354116829e-05
17.1099999999999 -2.67612281554449e-05
17.1199999999999 -2.97911704051084e-05
17.1299999999999 -3.27695543962796e-05
17.1399999999999 -3.56928541067562e-05
17.1499999999999 -3.85576293362148e-05
17.1599999999999 -4.13605296594003e-05
17.1699999999999 -4.40982982477482e-05
17.1799999999999 -4.67677755551857e-05
17.1899999999999 -4.9365902864106e-05
17.1999999999999 -5.18897256876779e-05
17.2099999999999 -5.43363970248538e-05
17.2199999999999 -5.67031804646591e-05
17.2299999999999 -5.89874531365354e-05
17.2399999999999 -6.11867085037467e-05
17.2499999999999 -6.32985589970484e-05
17.2599999999999 -6.53207384860548e-05
17.2699999999999 -6.72511045859559e-05
17.2799999999999 -6.90876407974769e-05
17.2899999999999 -7.08284584781686e-05
17.2999999999999 -7.24717986433967e-05
17.3099999999999 -7.40160335969873e-05
17.3199999999999 -7.54596684151009e-05
17.3299999999999 -7.68013421669589e-05
17.3399999999999 -7.80398290554049e-05
17.3499999999999 -7.91740393501969e-05
17.3599999999999 -8.02030201463774e-05
17.3699999999999 -8.11259559479616e-05
17.3799999999999 -8.19421690775636e-05
17.3899999999999 -8.26511199130202e-05
17.3999999999999 -8.32524069526149e-05
17.4099999999999 -8.37457667112434e-05
17.4199999999999 -8.41310734508744e-05
17.4299999999999 -8.44083387502119e-05
17.4399999999999 -8.45777109209228e-05
17.4499999999999 -8.46394742820213e-05
17.4599999999999 -8.45940483117434e-05
17.4699999999999 -8.44419867114085e-05
17.4799999999999 -8.41839764482108e-05
17.4899999999999 -8.38208369204222e-05
17.4999999999999 -8.33535195939214e-05
17.5099999999999 -8.27830937975393e-05
17.5199999999999 -8.21107201376686e-05
17.5299999999999 -8.1337716065123e-05
17.5399999999999 -8.04655202779973e-05
17.5499999999999 -7.949569032088e-05
17.5599999999999 -7.84299000486005e-05
17.5699999999999 -7.72699369578152e-05
17.5799999999999 -7.60176993899605e-05
17.59 -7.46751904808589e-05
17.6 -7.32444992224323e-05
17.61 -7.17278435614173e-05
17.62 -7.01275384139824e-05
17.63 -6.84459928685763e-05
17.64 -6.66857070169543e-05
17.65 -6.48492685553015e-05
17.66 -6.29393492216211e-05
17.67 -6.09587011043522e-05
17.68 -5.89101528431566e-05
17.69 -5.67966057358724e-05
17.7 -5.46210297620936e-05
17.71 -5.2386459531918e-05
17.72 -5.00959901672583e-05
17.73 -4.77527731225412e-05
17.74 -4.53600119511922e-05
17.75 -4.29209580240953e-05
17.76 -4.04389062060713e-05
17.77 -3.79171904963347e-05
17.78 -3.53591796387918e-05
17.79 -3.27682727080519e-05
17.8 -3.01478946769403e-05
17.81 -2.75014919713009e-05
17.82 -2.48325280178407e-05
17.83 -2.21444787907004e-05
17.84 -1.94408283624734e-05
17.85 -1.67250644652702e-05
17.86 -1.40006740674396e-05
17.87 -1.12711389714924e-05
17.88 -8.53993143869339e-06
17.89 -5.81050984575684e-06
17.9 -3.08631437900471e-06
17.91 -3.70762771256034e-07
17.92 2.33275391333799e-06
17.93 5.02087544141941e-06
17.94 7.69027651445991e-06
17.95 1.0337670809813e-05
17.96 1.2959814953925e-05
17.97 1.55535124230364e-05
17.98 1.8115617366724e-05
17.99 2.06430383498498e-05
18 2.31327420085349e-05
18.01 2.55817566156837e-05
18.02 2.79871755528813e-05
18.03 3.03461606836832e-05
18.04 3.26559456246459e-05
18.05 3.49138389108236e-05
18.06 3.71172270520457e-05
18.07 3.92635774766255e-05
18.08 4.13504413592858e-05
18.09 4.3375456330282e-05
18.1 4.53363490628463e-05
18.11 4.72309377362596e-05
18.12 4.9057134372014e-05
18.13 5.08129470407209e-05
18.14 5.24964819375878e-05
18.15 5.41059453244692e-05
18.16 5.56396453366755e-05
18.17 5.70959936529266e-05
18.18 5.84735070269977e-05
18.19 5.97708086798288e-05
18.2 6.09866295632645e-05
18.21 6.21198094535849e-05
18.22 6.31692979141225e-05
18.2300000000001 6.41341551303187e-05
18.2400000000001 6.50135525926762e-05
18.2500000000001 6.58067736371325e-05
18.2600000000001 6.65132138432133e-05
18.2700000000001 6.71323812906527e-05
18.2800000000001 6.76638966755634e-05
18.2900000000001 6.81074932877547e-05
18.3000000000001 6.84630168515023e-05
18.3100000000001 6.87304252330867e-05
18.3200000000001 6.8909788020016e-05
18.3300000000001 6.90012859794698e-05
18.3400000000001 6.90052104081918e-05
18.3500000000001 6.89219623948923e-05
18.3600000000001 6.87520520343942e-05
18.3700000000001 6.84960976735227e-05
18.3800000000001 6.81548253712241e-05
18.3900000000001 6.77290690645121e-05
18.4000000000001 6.72197304841627e-05
18.4100000000001 6.66278298115544e-05
18.4200000000001 6.59544947677113e-05
18.4300000000001 6.52009508774263e-05
18.4400000000001 6.43685194522517e-05
18.4500000000001 6.34586154642882e-05
18.4600000000001 6.24727453135967e-05
18.4700000000001 6.14125044922552e-05
18.4800000000001 6.02795617667758e-05
18.4900000000001 5.9075678727806e-05
18.5000000000001 5.78027065462735e-05
18.5100000000001 5.64625742611206e-05
18.5200000000001 5.50572863156043e-05
18.5300000000001 5.35889198635127e-05
18.5400000000001 5.20596219212424e-05
18.5500000000001 5.04716064036103e-05
18.5600000000001 4.88271510647883e-05
18.5700000000001 4.71285943579635e-05
18.5800000000001 4.53783322233702e-05
18.5900000000001 4.35788148122873e-05
18.6000000000001 4.17325431533883e-05
18.6100000000001 3.98420657672237e-05
18.6200000000001 3.79099752341671e-05
18.6300000000001 3.5938904720954e-05
18.6400000000001 3.39315244707889e-05
18.6500000000001 3.18905382618996e-05
18.6600000000001 2.98186798393651e-05
18.6700000000001 2.771870932499e-05
18.6800000000001 2.55934096099837e-05
18.6900000000001 2.34455827351543e-05
18.7000000000001 2.12780462633247e-05
18.7100000000001 1.909362964863e-05
18.7200000000001 1.68951706073513e-05
18.7300000000001 1.46855114948811e-05
18.7400000000001 1.2467495693413e-05
18.7500000000001 1.02439640148754e-05
18.7600000000001 8.01775112361579e-06
18.7700000000001 5.79168198326415e-06
18.7800000000001 3.56856833218255e-06
18.7900000000001 1.3512051918235e-06
18.8000000000001 -8.57632587735204e-07
18.8100000000001 -3.05519374086891e-06
18.8200000000001 -5.23875395398172e-06
18.8300000000001 -7.40561918352255e-06
18.8400000000001 -9.55312892046503e-06
18.8500000000001 -1.1678659397385e-05
18.8600000000001 -1.37796267343354e-05
18.8700000000002 -1.58534900198185e-05
18.8800000000002 -1.78977543232742e-05
18.8900000000002 -1.99099736355369e-05
18.9000000000002 -2.18877537338232e-05
18.9100000000002 -2.38287549686762e-05
18.9200000000002 -2.57306949683715e-05
18.9300000000002 -2.75913512587712e-05
18.9400000000002 -2.94085637954151e-05
18.9500000000002 -3.11802374050613e-05
18.9600000000002 -3.29043441340133e-05
18.9700000000002 -3.45789255007044e-05
18.9800000000002 -3.62020946501314e-05
18.9900000000002 -3.77720384078858e-05
19.0000000000002 -3.92870192316507e-05
19.0100000000002 -4.07453770581839e-05
19.0200000000002 -4.21455310439507e-05
19.0300000000002 -4.34859811977124e-05
19.0400000000002 -4.47653099035287e-05
19.0500000000002 -4.59821833327758e-05
19.0600000000002 -4.7135352743951e-05
19.0700000000002 -4.82236556691635e-05
19.0800000000002 -4.92460169877565e-05
19.0900000000002 -5.02014498931963e-05
19.1000000000002 -5.10890567103186e-05
19.1100000000002 -5.19080296342946e-05
19.1200000000002 -5.26576513401019e-05
19.1300000000002 -5.33372954757178e-05
19.1400000000002 -5.39464270392109e-05
19.1500000000002 -5.44846026401676e-05
19.1600000000002 -5.49514706461615e-05
19.1700000000002 -5.53467712153621e-05
19.1800000000002 -5.56703362168507e-05
19.1900000000002 -5.5922089040904e-05
19.2000000000002 -5.61020443025368e-05
19.2100000000002 -5.6210307443258e-05
19.2200000000002 -5.62470742388446e-05
19.2300000000002 -5.62126302261749e-05
19.2400000000002 -5.61073500724856e-05
19.2500000000002 -5.59316969325815e-05
19.2600000000002 -5.56862218921248e-05
19.2700000000002 -5.53715637375284e-05
19.2800000000002 -5.49884365973233e-05
19.2900000000002 -5.45376204297126e-05
19.3000000000002 -5.40199969203405e-05
19.3100000000002 -5.34365281817678e-05
19.3200000000002 -5.27882551519672e-05
19.3300000000002 -5.20762959025943e-05
19.3400000000002 -5.13018438593518e-05
19.3500000000002 -5.04661659368536e-05
19.3600000000002 -4.95705984143579e-05
19.3700000000002 -4.86165352081678e-05
19.3800000000002 -4.76054549491088e-05
19.3900000000002 -4.65389005502434e-05
19.4000000000002 -4.54184773257699e-05
19.4100000000002 -4.42458508714824e-05
19.4200000000002 -4.30227447964537e-05
19.4300000000002 -4.17509383480003e-05
19.4400000000002 -4.0432263952368e-05
19.4500000000002 -3.90686046846143e-05
19.4600000000002 -3.76618916767979e-05
19.4700000000002 -3.62141014713245e-05
19.4800000000002 -3.47272533250402e-05
19.4900000000002 -3.3203406468965e-05
19.5000000000002 -3.16446573281778e-05
19.5100000000003 -3.00531367060861e-05
19.5200000000003 -2.84310069371907e-05
19.5300000000003 -2.67804590123508e-05
19.5400000000003 -2.51037096805015e-05
19.5500000000003 -2.34029985307368e-05
19.5600000000003 -2.16805850586262e-05
19.5700000000003 -1.99387457206332e-05
19.5800000000003 -1.8179770980477e-05
19.5900000000003 -1.64059623512277e-05
19.6000000000003 -1.46196294369599e-05
19.6100000000003 -1.28230869777149e-05
19.6200000000003 -1.10186519015054e-05
19.6300000000003 -9.2086403871047e-06
19.6400000000003 -7.39536494126465e-06
19.6500000000003 -5.58113149401514e-06
19.6600000000003 -3.76823651564641e-06
19.6700000000003 -1.95896415892397e-06
19.6800000000003 -1.55583430025713e-07
19.6900000000003 1.63965460832677e-06
19.7000000000003 3.42451959817508e-06
19.7100000000003 5.19680460872847e-06
19.7200000000003 6.95432881659578e-06
19.7300000000003 8.69494014066855e-06
19.7400000000003 1.04165178285131e-05
19.7500000000003 1.21169749912337e-05
19.7600000000003 1.3794261083847e-05
19.7700000000003 1.54463643282685e-05
19.7800000000003 1.70713140760108e-05
19.7900000000003 1.86671831083173e-05
19.8000000000003 2.02320898705659e-05
19.8100000000003 2.17642006386181e-05
19.8200000000003 2.32617316147858e-05
19.8300000000003 2.47229509510378e-05
19.8400000000003 2.61461806972238e-05
19.8500000000003 2.75297986722082e-05
19.8600000000003 2.88722402559055e-05
19.8700000000003 3.01720001003286e-05
19.8800000000003 3.14276337578624e-05
19.8900000000003 3.26377592251025e-05
19.9000000000003 3.38010584007046e-05
19.9100000000003 3.49162784558132e-05
19.9200000000003 3.59822331157564e-05
19.9300000000003 3.69978038518184e-05
19.9400000000003 3.79619409820188e-05
19.9500000000003 3.88736646799637e-05
19.9600000000003 3.97320658909483e-05
19.9700000000003 4.05363071602421e-05
19.9800000000003 4.12856233534702e-05
19.9900000000003 4.19793222899562e-05
20.0000000000003 4.26167852859903e-05
20.0100000000003 4.31974675997763e-05
20.0200000000003 4.37208987816916e-05
20.0300000000003 4.41866829301047e-05
20.0400000000003 4.45944988532179e-05
20.0500000000003 4.49441001376654e-05
20.0600000000003 4.52353151249402e-05
20.0700000000003 4.54680467971841e-05
20.0800000000003 4.5642272574563e-05
20.0900000000003 4.57580440274999e-05
20.1000000000003 4.58154865088071e-05
20.1100000000003 4.58147987138816e-05
20.1200000000003 4.57562521830907e-05
20.1300000000003 4.56401907727274e-05
20.1400000000003 4.5467030148582e-05
20.1500000000004 4.52372574261633e-05
20.1600000000004 4.49514312924657e-05
20.1700000000004 4.46101530716179e-05
20.1800000000004 4.42141070287581e-05
20.1900000000004 4.37640466695804e-05
20.2000000000004 4.32607903175348e-05
20.2100000000004 4.27052197698897e-05
20.2200000000004 4.20982788811647e-05
20.2300000000004 4.14409720758766e-05
20.2400000000004 4.07343627926706e-05
20.2500000000004 3.9979563246517e-05
20.2600000000004 3.9177747068551e-05
20.2700000000004 3.8330146921086e-05
20.2800000000004 3.74380470348899e-05
20.2900000000004 3.65027815591887e-05
20.3000000000004 3.5525732763773e-05
20.3100000000004 3.45083291411878e-05
20.3200000000004 3.3452043433113e-05
20.3300000000004 3.23583905946345e-05
20.3400000000004 3.12289257051826e-05
20.3500000000004 3.0065241832427e-05
20.3600000000004 2.8868967854065e-05
20.3700000000004 2.76417662417413e-05
20.3800000000004 2.63853308108652e-05
20.3900000000004 2.51013844398844e-05
20.4000000000004 2.37916767623924e-05
20.4100000000004 2.24579818353749e-05
20.4200000000004 2.11020957868207e-05
20.4300000000004 1.97258344459053e-05
20.4400000000004 1.8331030958912e-05
20.4500000000004 1.69195333940394e-05
20.4600000000004 1.54932023382348e-05
20.4700000000004 1.40539084891551e-05
20.4800000000004 1.26035302453724e-05
20.4900000000004 1.11439512978788e-05
20.5000000000004 9.67705822597325e-06
20.5100000000004 8.20473810054469e-06
20.5200000000004 6.72887609777288e-06
20.5300000000004 5.25135312621112e-06
20.5400000000004 3.77404347021403e-06
20.5500000000004 2.29881245260363e-06
20.5600000000004 8.27514119453214e-07
20.5700000000004 -6.38011050179084e-07
20.5800000000004 -2.09593840679594e-06
20.5900000000004 -3.54446136254093e-06
20.6000000000004 -4.98179359013643e-06
20.6100000000004 -6.40617118604181e-06
20.6200000000004 -7.81585479527766e-06
20.6300000000004 -9.20913169539351e-06
20.6400000000004 -1.05843178371396e-05
20.6500000000004 -1.19397598394608e-05
20.6600000000004 -1.32738369364845e-05
20.6700000000004 -1.45849628742638e-05
20.6800000000004 -1.5871587755339e-05
20.6900000000004 -1.71321998285505e-05
20.7000000000004 -1.83653272224743e-05
20.7100000000004 -1.95695396204405e-05
20.7200000000004 -2.07434498753006e-05
20.7300000000004 -2.18857155621759e-05
20.7400000000004 -2.29950404675203e-05
20.7500000000004 -2.40701760129137e-05
20.7600000000004 -2.51099226120923e-05
20.7700000000004 -2.6113130959818e-05
20.7800000000004 -2.70787032512812e-05
20.7900000000005 -2.80055943308271e-05
20.8000000000005 -2.88928127688902e-05
20.8100000000005 -2.9739421866127e-05
20.8200000000005 -3.0544540583826e-05
20.8300000000005 -3.13073443997877e-05
20.8400000000005 -3.20270660889617e-05
20.8500000000005 -3.2702996429054e-05
20.8600000000005 -3.33344848322043e-05
20.8700000000005 -3.39209398873185e-05
20.8800000000005 -3.44618298400584e-05
20.8900000000005 -3.49566829904365e-05
20.9000000000005 -3.5405088013246e-05
20.9100000000005 -3.58066942014598e-05
20.9200000000005 -3.61612116328841e-05
20.9300000000005 -3.6468411260554e-05
20.9400000000005 -3.67281249275901e-05
20.9500000000005 -3.69402453075702e-05
20.9600000000005 -3.71047257719074e-05
20.9700000000005 -3.72215801864196e-05
20.9800000000005 -3.72908826403681e-05
20.9900000000005 -3.73127671131383e-05
21.0000000000005 -3.72874270872086e-05
21.0100000000005 -3.72151151229445e-05
21.0200000000005 -3.70961424255912e-05
21.0300000000005 -3.69308784702348e-05
21.0400000000005 -3.67197508471947e-05
21.0500000000005 -3.64632354592572e-05
21.0600000000005 -3.61618550515658e-05
21.0700000000005 -3.58161982149735e-05
21.0800000000005 -3.54269067982309e-05
21.0900000000005 -3.49946748421101e-05
21.1000000000005 -3.45202474535331e-05
21.1100000000005 -3.40044196212948e-05
21.1200000000005 -3.34480349750534e-05
21.1300000000005 -3.28519830969889e-05
21.1400000000005 -3.22171920722607e-05
21.1500000000005 -3.1544645528621e-05
21.1600000000005 -3.08353696419946e-05
21.1700000000005 -3.00904318732276e-05
21.1800000000005 -2.93109395527107e-05
21.1900000000005 -2.84980383691577e-05
21.2000000000005 -2.76529107891312e-05
21.2100000000005 -2.67767744215837e-05
21.2200000000005 -2.58708803360644e-05
21.2300000000005 -2.49365113404754e-05
21.2400000000005 -2.39749802228215e-05
21.2500000000005 -2.29876279606288e-05
21.2600000000005 -2.19758219012364e-05
21.2700000000005 -2.09409539159356e-05
21.2800000000005 -1.98844385307596e-05
21.2900000000005 -1.88077110366328e-05
21.3000000000005 -1.771222558154e-05
21.3100000000005 -1.65994532473293e-05
21.3200000000005 -1.54708801137315e-05
21.3300000000005 -1.43280053121826e-05
21.3400000000005 -1.31723390719876e-05
21.3500000000005 -1.20054007613839e-05
21.3600000000005 -1.08287169260229e-05
21.3700000000005 -9.64381932737941e-06
21.3800000000005 -8.4522429836044e-06
21.3900000000005 -7.25552421527377e-06
21.4000000000005 -6.05519869851948e-06
21.4100000000005 -4.85279952795138e-06
21.4200000000005 -3.64985529180305e-06
21.4300000000006 -2.44788816166923e-06
21.4400000000006 -1.24841199919089e-06
21.4500000000006 -5.29304820083333e-08
21.4600000000006 1.13706474873915e-06
21.4700000000006 2.32009591711867e-06
21.4800000000006 3.49470091511394e-06
21.4900000000006 4.65943507718103e-06
21.5000000000006 5.81287292453469e-06
21.5100000000006 6.95360987710167e-06
21.5200000000006 8.08026393112155e-06
21.5300000000006 9.19147730044553e-06
21.5400000000006 1.02859180196112e-05
21.5500000000006 1.13622815068073e-05
21.5600000000006 1.24192920851293e-05
21.5700000000006 1.34557044601387e-05
21.5800000000006 1.44703051521757e-05
21.5900000000006 1.5461913881852e-05
21.6000000000006 1.64293849071651e-05
21.6100000000006 1.73716083107774e-05
21.6200000000006 1.82875112360579e-05
21.6300000000006 1.91760590705767e-05
21.6400000000006 2.0036256575795e-05
21.6500000000006 2.08671489617797e-05
21.6600000000006 2.16678229058502e-05
21.6700000000006 2.24374075141291e-05
21.6800000000006 2.31750752250598e-05
21.6900000000006 2.38800426540297e-05
21.7000000000006 2.45515713783119e-05
21.7100000000006 2.51889686616324e-05
21.7200000000006 2.57915881177414e-05
21.7300000000006 2.63588303124599e-05
21.7400000000006 2.68901433061851e-05
21.7500000000006 2.73850231278212e-05
21.7600000000006 2.78430141899657e-05
21.7700000000006 2.82637096426398e-05
21.7800000000006 2.86467516628811e-05
21.7900000000006 2.89918316815621e-05
21.8000000000006 2.92986905476e-05
21.8100000000006 2.9567118629873e-05
21.8200000000006 2.97969558573272e-05
21.8300000000006 2.99880916979815e-05
21.8400000000006 3.01404650778481e-05
21.8500000000006 3.02540642412218e-05
21.8600000000006 3.03289265544939e-05
21.8700000000006 3.03651382567942e-05
21.8800000000006 3.03628341628226e-05
21.8900000000006 3.03221973271481e-05
21.9000000000006 3.02434586873537e-05
21.9100000000006 3.01268967217274e-05
21.9200000000006 2.9972837203852e-05
21.9300000000006 2.97816532769165e-05
21.9400000000006 2.95537452770649e-05
21.9500000000006 2.92895707472979e-05
21.9600000000006 2.89896318720262e-05
21.9700000000006 2.86544737166687e-05
21.9800000000006 2.8284683334276e-05
21.9900000000006 2.78808888239165e-05
22.0000000000006 2.74437583421729e-05
22.0100000000006 2.6973999069153e-05
22.0200000000006 2.64723507521473e-05
22.0300000000006 2.59395934791586e-05
22.0400000000006 2.53765462304449e-05
22.0500000000006 2.47840620834403e-05
22.0600000000006 2.416302710952e-05
22.0700000000007 2.35143591761544e-05
22.0800000000007 2.28390066845752e-05
22.0900000000007 2.21379472581984e-05
22.1000000000007 2.14121863905247e-05
22.1100000000007 2.0662756058147e-05
22.1200000000007 1.989071330293e-05
22.1300000000007 1.90971387865832e-05
22.1400000000007 1.8283135320391e-05
22.1500000000007 1.74498263725833e-05
22.1600000000007 1.65983545556848e-05
22.1700000000007 1.57298800960796e-05
22.1800000000007 1.48455792879645e-05
22.1900000000007 1.39466429338381e-05
22.2000000000007 1.30342747736342e-05
22.2100000000007 1.21096899046071e-05
22.2200000000007 1.1174113194043e-05
22.2300000000007 1.02287776868795e-05
22.2400000000007 9.27492301029063e-06
22.2500000000007 8.31379377729208e-06
22.2600000000007 7.3466379913988e-06
22.2700000000007 6.37470545436428e-06
22.2800000000007 5.39924617900491e-06
22.2900000000007 4.42150880910603e-06
22.3000000000007 3.44273904837299e-06
22.3100000000007 2.46417810038066e-06
22.3200000000007 1.48706112144402e-06
22.3300000000007 5.12615688312187e-07
22.3400000000007 -4.57939717450034e-07
22.3500000000007 -1.42339720654636e-06
22.3600000000007 -2.38256095869237e-06
22.3700000000007 -3.33424867725242e-06
22.3800000000007 -4.27729302001946e-06
22.3900000000007 -5.21054300445021e-06
22.4000000000007 -6.13286538569813e-06
22.4100000000007 -7.04314600581837e-06
22.4200000000007 -7.94029111258666e-06
22.4300000000007 -8.82322864638532e-06
22.4400000000007 -9.6909094936994e-06
22.4500000000007 -1.05423087058561e-05
22.4600000000007 -1.13764266814611e-05
22.4700000000007 -1.21922903113543e-05
22.4800000000007 -1.29889540847627e-05
22.4900000000007 -1.37655011554426e-05
22.5000000000007 -1.45210443666498e-05
22.5100000000007 -1.52547272338334e-05
22.5200000000007 -1.59657248840162e-05
22.5300000000007 -1.66532449508742e-05
22.5400000000007 -1.73165284245969e-05
22.5500000000007 -1.79548504556699e-05
22.5600000000007 -1.85675211117828e-05
22.5700000000007 -1.91538860871329e-05
22.5800000000007 -1.97133273634573e-05
22.5900000000007 -2.02452638221967e-05
22.6000000000007 -2.07491518072525e-05
22.6100000000007 -2.12244856378788e-05
22.6200000000007 -2.16707980717179e-05
22.6300000000007 -2.20876607179106e-05
22.6400000000007 -2.24746843947665e-05
22.6500000000007 -2.2831519442006e-05
22.6600000000007 -2.31578559798672e-05
22.6700000000007 -2.34534241171039e-05
22.6800000000007 -2.37179941079614e-05
22.6900000000007 -2.39513764583246e-05
22.7000000000007 -2.41534219813562e-05
22.7100000000008 -2.43240218031048e-05
22.7200000000008 -2.44631073187647e-05
22.7300000000008 -2.45706501005691e-05
22.7400000000008 -2.46466617587334e-05
22.7500000000008 -2.46911937575851e-05
22.7600000000008 -2.47043371902355e-05
22.7700000000008 -2.46862225174123e-05
22.7800000000008 -2.46370192805494e-05
22.7900000000008 -2.45569358089198e-05
22.8000000000008 -2.44462189637929e-05
22.8100000000008 -2.43051540262996e-05
22.8200000000008 -2.41340574623059e-05
22.8300000000008 -2.39332785452161e-05
22.8400000000008 -2.37032091293082e-05
22.8500000000008 -2.34442762709754e-05
22.8600000000008 -2.31569415210735e-05
22.8700000000008 -2.28417001774316e-05
22.8800000000008 -2.24990804986317e-05
22.8900000000008 -2.21296428801966e-05
22.9000000000008 -2.17339781991931e-05
22.9100000000008 -2.13127028953503e-05
22.9200000000008 -2.08664697638298e-05
22.9300000000008 -2.03959597160242e-05
22.9400000000008 -1.99018809326702e-05
22.9500000000008 -1.93849679203862e-05
22.9600000000008 -1.88459805067607e-05
22.9700000000008 -1.82857027906888e-05
22.9800000000008 -1.77049420569702e-05
22.9900000000008 -1.71045276606805e-05
23.0000000000008 -1.64853098850923e-05
23.0100000000008 -1.58481587760235e-05
23.0200000000008 -1.51939629550029e-05
23.0300000000008 -1.45236284133537e-05
23.0400000000008 -1.38380772891532e-05
23.0500000000008 -1.31382466288995e-05
23.0600000000008 -1.24250871356929e-05
23.0700000000008 -1.1699561905672e-05
23.0800000000008 -1.09626451544348e-05
23.0900000000008 -1.02153209351632e-05
23.1000000000008 -9.45858185014093e-06
23.1100000000008 -8.69342775736517e-06
23.1200000000008 -7.9208644739227e-06
23.1300000000008 -7.14190247781209e-06
23.1400000000008 -6.35755560986807e-06
23.1500000000008 -5.56883977743923e-06
23.1600000000008 -4.77677166146139e-06
23.1700000000008 -3.98236742854961e-06
23.1800000000008 -3.18664144971727e-06
23.1900000000008 -2.39060502731843e-06
23.2000000000008 -1.59526513178796e-06
23.2100000000008 -8.01623149729969e-07
23.2200000000008 -1.0673644886134e-08
23.2300000000008 7.76596866500486e-07
23.2400000000008 1.55921112445355e-06
23.2500000000008 2.33620231670394e-06
23.2600000000008 3.10661525137637e-06
23.2700000000008 3.86950750952878e-06
23.2800000000008 4.62395057617637e-06
23.2900000000008 5.3690309484798e-06
23.3000000000008 6.10385121979374e-06
23.3100000000008 6.82753113832245e-06
23.3200000000008 7.53920863914581e-06
23.3300000000008 8.23804084851046e-06
23.3400000000008 8.92320505914025e-06
23.3500000000009 9.59389967552162e-06
23.3600000000009 1.02493451281142e-05
23.3700000000009 1.08887847554727e-05
23.3800000000009 1.1511485653311e-05
23.3900000000009 1.21167394895956e-05
23.4000000000009 1.27038632847991e-05
23.4100000000009 1.32722001564873e-05
23.4200000000009 1.38211200274759e-05
23.4300000000009 1.43500202968303e-05
23.4400000000009 1.48583264730373e-05
23.4500000000009 1.53454927687357e-05
23.4600000000009 1.5811002656434e-05
23.4700000000009 1.62543693847076e-05
23.4800000000009 1.6675136454413e-05
23.4900000000009 1.70728780545193e-05
23.5000000000009 1.74471994572082e-05
23.5100000000009 1.77977373729518e-05
23.5200000000009 1.81241602617104e-05
23.5300000000009 1.84261686044644e-05
23.5400000000009 1.87034951335709e-05
23.5500000000009 1.89559050210707e-05
23.5600000000009 1.91831960254551e-05
23.5700000000009 1.93851985970054e-05
23.5800000000009 1.95617759419117e-05
23.5900000000009 1.97128240454888e-05
23.6000000000009 1.98382716549476e-05
23.6100000000009 1.99380802223831e-05
23.6200000000009 2.00122438089191e-05
23.6300000000009 2.0060788951397e-05
23.6400000000009 2.0083774493735e-05
23.6500000000009 2.00812913863971e-05
23.6600000000009 2.00534624599309e-05
23.6700000000009 2.00004421837311e-05
23.6800000000009 1.99224164330088e-05
23.6900000000009 1.98196023171195e-05
23.7000000000009 1.96922482052738e-05
23.7100000000009 1.95406201844203e-05
23.7200000000009 1.93650231256389e-05
23.7300000000009 1.91657907739357e-05
23.7400000000009 1.89432851887567e-05
23.7500000000009 1.86978961516116e-05
23.7600000000009 1.84300405417193e-05
23.7700000000009 1.81401616806002e-05
23.7800000000009 1.78287286465719e-05
23.7900000000009 1.74962323258792e-05
23.8000000000009 1.71431898999252e-05
23.8100000000009 1.67701441658267e-05
23.8200000000009 1.6377660427304e-05
23.8300000000009 1.59663257583204e-05
23.8400000000009 1.55367482065328e-05
23.8500000000009 1.50895559553352e-05
23.8600000000009 1.46253964540495e-05
23.8700000000009 1.41449355217757e-05
23.8800000000009 1.36488564284863e-05
23.8900000000009 1.31378589559769e-05
23.9000000000009 1.2612658440762e-05
23.9100000000009 1.20739848007071e-05
23.9200000000009 1.15225815470363e-05
23.9300000000009 1.09592047832407e-05
23.9400000000009 1.03846221923605e-05
23.9500000000009 9.79961201407836e-06
23.9600000000009 9.20496201303628e-06
23.9700000000009 8.60146843976749e-06
23.9800000000009 7.98993498563454e-06
23.990000000001 7.37117173314731e-06
24.000000000001 6.74599410303059e-06
24.010000000001 6.11522179940628e-06
24.020000000001 5.47967775444085e-06
24.030000000001 4.84018707380382e-06
24.040000000001 4.1975759842798e-06
24.050000000001 3.5526707848541e-06
24.060000000001 2.90629680258699e-06
24.070000000001 2.25927735458185e-06
24.080000000001 1.61243271732969e-06
24.090000000001 9.6657910470323e-07
24.100000000001 3.22527655851777e-07
24.110000000001 -3.18916565767817e-07
24.120000000001 -9.56955561006877e-07
24.130000000001 -1.59079937010056e-06
24.140000000001 -2.21966703302531e-06
24.150000000001 -2.84278753400281e-06
24.160000000001 -3.45940072902043e-06
24.170000000001 -4.06875825527976e-06
24.180000000001 -4.67012442151098e-06
24.190000000001 -5.26277707811122e-06
24.200000000001 -5.84600846609944e-06
24.210000000001 -6.41912604392572e-06
24.220000000001 -6.98145329120685e-06
24.230000000001 -7.53233048842821e-06
24.240000000001 -8.07111547178604e-06
24.250000000001 -8.59718436232168e-06
24.260000000001 -9.10993226854595e-06
24.270000000001 -9.60877396179001e-06
24.280000000001 -1.00931445235603e-05
24.290000000001 -1.05624999642123e-05
24.300000000001 -1.10163178122938e-05
24.310000000001 -1.1454097673959e-05
24.320000000001 -1.18753617618827e-05
24.330000000001 -1.22796553931581e-05
24.340000000001 -1.26665474556957e-05
24.350000000001 -1.30356308426874e-05
24.360000000001 -1.33865228547442e-05
24.370000000001 -1.37188655693569e-05
24.380000000001 -1.40323261773792e-05
24.390000000001 -1.43265972864499e-05
24.400000000001 -1.46013971911489e-05
24.410000000001 -1.48564701078656e-05
24.420000000001 -1.5091586378043e-05
24.430000000001 -1.53065426368666e-05
24.440000000001 -1.55011619481699e-05
24.450000000001 -1.56752939056104e-05
24.460000000001 -1.58288147002468e-05
24.470000000001 -1.59616271547216e-05
24.480000000001 -1.60736607243608e-05
24.490000000001 -1.61648714656292e-05
24.500000000001 -1.62352419725722e-05
24.510000000001 -1.62847812821483e-05
24.520000000001 -1.63135247498132e-05
24.530000000001 -1.63215338974864e-05
24.540000000001 -1.63088962374619e-05
24.550000000001 -1.62757250786587e-05
24.560000000001 -1.62221593277511e-05
24.570000000001 -1.61483633124645e-05
24.580000000001 -1.60545266949447e-05
24.590000000001 -1.5940859437537e-05
24.600000000001 -1.58075941246948e-05
24.610000000001 -1.56549908604875e-05
24.620000000001 -1.5483332970423e-05
24.6300000000011 -1.52929265328503e-05
24.6400000000011 -1.50840998839649e-05
24.6500000000011 -1.48572030971775e-05
24.6600000000011 -1.46126074376219e-05
24.6700000000011 -1.43507044027686e-05
24.6800000000011 -1.40719023774363e-05
24.6900000000011 -1.37766334860291e-05
24.7000000000011 -1.3465348399598e-05
24.7100000000011 -1.31385157689928e-05
24.7200000000011 -1.27966215971293e-05
24.7300000000011 -1.24401685721032e-05
24.7400000000011 -1.20696753715341e-05
24.7500000000011 -1.16856759437824e-05
24.7600000000011 -1.12887187695201e-05
24.7700000000011 -1.08793661060607e-05
24.7800000000011 -1.04581932162999e-05
24.7900000000011 -1.0025787583813e-05
24.8000000000011 -9.58274811548336e-06
24.8100000000011 -9.12968433293215e-06
24.8200000000011 -8.6672155539662e-06
24.8300000000011 -8.19597006521965e-06
24.8400000000011 -7.71658428713991e-06
24.8500000000011 -7.22970193246165e-06
24.8600000000011 -6.73597315929236e-06
24.8700000000011 -6.2360537199365e-06
24.8800000000011 -5.73060410656694e-06
24.8900000000011 -5.22028869485828e-06
24.9000000000011 -4.70577488667995e-06
24.9100000000011 -4.18773225295056e-06
24.9200000000011 -3.66683167773825e-06
24.9300000000011 -3.14374450469267e-06
24.9400000000011 -2.61914168687634e-06
24.9500000000011 -2.09369294106068e-06
24.9600000000011 -1.56806590753699e-06
24.9700000000011 -1.04292531647628e-06
24.9800000000011 -5.18932161870918e-07
24.9900000000011 3.2571159399567e-09
25.0000000000011 5.22991438162737e-07
25.0100000000011 1.03962588488453e-06
25.0200000000011 1.55252248089112e-06
25.0300000000011 2.06105096902669e-06
25.0400000000011 2.56458957017938e-06
25.0500000000011 3.06252572899225e-06
25.0600000000011 3.55425684442342e-06
25.0700000000011 4.03919098430525e-06
25.0800000000011 4.51674758307338e-06
25.0900000000011 4.98635812185401e-06
25.1000000000011 5.44746679016668e-06
25.1100000000011 5.89953112844746e-06
25.1200000000011 6.34202265069298e-06
25.1300000000011 6.7744274465345e-06
25.1400000000011 7.19624676207407e-06
25.1500000000011 7.60699755885088e-06
25.1600000000011 8.00621305033266e-06
25.1700000000011 8.39344321536385e-06
25.1800000000011 8.76825528802878e-06
25.1900000000011 9.13023422342548e-06
25.2000000000011 9.47898313887322e-06
25.2100000000011 9.81412373011921e-06
25.2200000000011 1.0135296662134e-05
25.2300000000011 1.04421619341283e-05
25.2400000000011 1.07343992184545e-05
25.2500000000011 1.1011708173093e-05
25.2600000000011 1.12738087274599e-05
25.2700000000012 1.15204413413097e-05
25.2800000000012 1.1751367236937e-05
25.2900000000012 1.1966368603088e-05
25.3000000000012 1.21652487722443e-05
25.3100000000012 1.23478323705849e-05
25.3200000000012 1.25139654403192e-05
25.3300000000012 1.26635155345921e-05
25.3400000000012 1.27963717850308e-05
25.3500000000012 1.29124449420673e-05
25.3600000000012 1.30116673882417e-05
25.3700000000012 1.30939931247789e-05
25.3800000000012 1.31593977318607e-05
25.3900000000012 1.32078783031885e-05
25.4000000000012 1.32394533557117e-05
25.4100000000012 1.32541627158586e-05
25.4200000000012 1.32520673844233e-05
25.4300000000012 1.32332493838309e-05
25.4400000000012 1.31978115947463e-05
25.4500000000012 1.31458775963737e-05
25.4600000000012 1.30775915436767e-05
25.4700000000012 1.29931178737146e-05
25.4800000000012 1.28926336924489e-05
25.4900000000012 1.27763411059771e-05
25.5000000000012 1.26444612480066e-05
25.5100000000012 1.24972339098927e-05
25.5200000000012 1.23349171488953e-05
25.5300000000012 1.21577868752792e-05
25.5400000000012 1.19661364188922e-05
25.5500000000012 1.17602760758686e-05
25.5600000000012 1.15405307732502e-05
25.5700000000012 1.13072424623792e-05
25.5800000000012 1.10607699456904e-05
25.5900000000012 1.08014868543017e-05
25.6000000000012 1.05297811573685e-05
25.6100000000012 1.02460546334386e-05
25.6200000000012 9.95072231538693e-06
25.6300000000012 9.64421191485831e-06
25.6400000000012 9.32696322965832e-06
25.6500000000012 8.99942753636178e-06
25.6600000000012 8.66206696979785e-06
25.6700000000012 8.31535389075801e-06
25.6800000000012 7.95977024309186e-06
25.6900000000012 7.59580690125042e-06
25.7000000000012 7.22396300928295e-06
25.7100000000012 6.84474531224737e-06
25.7200000000012 6.45866748098242e-06
25.7300000000012 6.06624943116756e-06
25.7400000000012 5.66801663758965e-06
25.7500000000012 5.26449944452889e-06
25.7600000000012 4.85623237317364e-06
25.7700000000012 4.44375342696134e-06
25.7800000000012 4.02760339574807e-06
25.7900000000012 3.60832515969759e-06
25.8000000000012 3.18646299377485e-06
25.8100000000012 2.76256187372791e-06
25.8200000000012 2.33716678442977e-06
25.8300000000012 1.91082203144797e-06
25.8400000000012 1.4840705566983e-06
25.8500000000012 1.05745325902916e-06
25.8600000000012 6.31508320575359e-07
25.8700000000012 2.06770539706496e-07
25.8800000000012 -2.16229328618649e-07
25.8900000000012 -6.3696522428958e-07
25.9000000000012 -1.05491642449059e-06
25.9100000000013 -1.46956817638418e-06
25.9200000000013 -1.88041231927858e-06
25.9300000000013 -2.28694789554096e-06
25.9400000000013 -2.68868174953995e-06
25.9500000000013 -3.08512911391243e-06
25.9600000000013 -3.47581418247755e-06
25.9700000000013 -3.8602706691314e-06
25.9800000000013 -4.23804235208851e-06
25.9900000000013 -4.60868360285635e-06
26.0000000000013 -4.97175989932533e-06
26.0100000000013 -5.32684832242051e-06
26.0200000000013 -5.67353803575879e-06
26.0300000000013 -6.01143074778636e-06
26.0400000000013 -6.34014115589668e-06
26.0500000000013 -6.65929737205144e-06
26.0600000000013 -6.96854132945664e-06
26.0700000000013 -7.26752916986786e-06
26.0800000000013 -7.55593161113153e-06
26.0900000000013 -7.83343429458934e-06
26.1000000000013 -8.09973811200563e-06
26.1100000000013 -8.35455951170302e-06
26.1200000000013 -8.59763078362105e-06
26.1300000000013 -8.82870032304035e-06
26.1400000000013 -9.04753287274565e-06
26.1500000000013 -9.25390974342875e-06
26.1600000000013 -9.44762901223359e-06
26.1700000000013 -9.628505699293e-06
26.1800000000013 -9.79637192147629e-06
26.1900000000013 -9.951077024677e-06
26.2000000000013 -1.00924876935282e-05
26.2100000000013 -1.02204880388325e-05
26.2200000000013 -1.03349796627441e-05
26.2300000000013 -1.04358817017842e-05
26.2400000000013 -1.05231308478226e-05
26.2500000000013 -1.05966813472202e-05
26.2600000000013 -1.0656504978414e-05
26.2700000000013 -1.0702591008337e-05
26.2800000000013 -1.07349461282428e-05
26.2900000000013 -1.07535943697775e-05
26.3000000000013 -1.07585770026198e-05
26.3100000000013 -1.07499524158833e-05
26.3200000000013 -1.07277959872134e-05
26.3300000000013 -1.06921999472756e-05
26.3400000000013 -1.06432732563864e-05
26.3500000000013 -1.05811415349897e-05
26.3600000000013 -1.05059437464363e-05
26.3700000000013 -1.04178342177551e-05
26.3800000000013 -1.03169850542934e-05
26.3900000000013 -1.02035836469657e-05
26.4000000000013 -1.00778323628071e-05
26.4100000000013 -9.93994821807381e-06
26.4200000000013 -9.79016253441279e-06
26.4300000000013 -9.62872057862841e-06
26.4400000000013 -9.45588104325054e-06
26.4500000000013 -9.27191374119772e-06
26.4600000000013 -9.07710393583606e-06
26.4700000000013 -8.87174910158421e-06
26.4800000000013 -8.65615854507033e-06
26.4900000000013 -8.43065298825839e-06
26.5000000000013 -8.1955641268228e-06
26.5100000000013 -7.95123417014428e-06
26.5200000000013 -7.69801536641734e-06
26.5300000000013 -7.43626951504192e-06
26.5400000000013 -7.16636746781865e-06
26.5500000000014 -6.88868862012749e-06
26.5600000000014 -6.60362039308622e-06
26.5700000000014 -6.31155770757806e-06
26.5800000000014 -6.0129024509788e-06
26.5900000000014 -5.70806293737585e-06
26.6000000000014 -5.39745336205122e-06
26.6100000000014 -5.08149325098414e-06
26.6200000000014 -4.76060690612321e-06
26.6300000000014 -4.4352228471681e-06
26.6400000000014 -4.10577325060147e-06
26.6500000000014 -3.77269338670084e-06
26.6600000000014 -3.43642105526349e-06
26.6700000000014 -3.09739602076789e-06
26.6800000000014 -2.75605944769551e-06
26.6900000000014 -2.41285333672841e-06
26.7000000000014 -2.0682199625341e-06
26.7100000000014 -1.72260131384458e-06
26.7200000000014 -1.37643853652795e-06
26.7300000000014 -1.03017138034317e-06
26.7400000000014 -6.84237650060807e-07
26.7500000000014 -3.3907266162727e-07
26.7600000000014 4.8912959690091e-09
26.7700000000014 3.4722549246992e-07
26.7800000000014 6.87505281119383e-07
26.7900000000014 1.02531061580213e-06
26.8000000000014 1.36022655993254e-06
26.8100000000014 1.69184378649e-06
26.8200000000014 2.01975906860725e-06
26.8300000000014 2.34357576014056e-06
26.8400000000014 2.66290426565824e-06
26.8500000000014 2.97736249930455e-06
26.8600000000014 3.28657633200956e-06
26.8700000000014 3.5901800265464e-06
26.8800000000014 3.88781665992627e-06
26.8900000000014 4.17913853266595e-06
26.9000000000014 4.46380756447601e-06
26.9100000000014 4.74149567592818e-06
26.9200000000014 5.011885155691e-06
26.9300000000014 5.27466901293557e-06
26.9400000000014 5.52955131453808e-06
26.9500000000014 5.77624750672484e-06
26.9600000000014 6.01448472082913e-06
26.9700000000014 6.24400206284789e-06
26.9800000000014 6.46455088651248e-06
26.9900000000014 6.67589504960677e-06
27.0000000000014 6.87781115329067e-06
27.0100000000014 7.07008876420959e-06
27.0200000000014 7.25253061919385e-06
27.0300000000014 7.42495281237642e-06
27.0400000000014 7.58718496458128e-06
27.0500000000014 7.73907037500556e-06
27.0600000000014 7.88046615455747e-06
27.0700000000014 8.01124334144684e-06
27.0800000000014 8.13128699874324e-06
27.0900000000014 8.24049629378251e-06
27.1000000000014 8.33878455950036e-06
27.1100000000014 8.42607933773951e-06
27.1200000000014 8.50232240461405e-06
27.1300000000014 8.5674697780597e-06
27.1400000000014 8.62149170775579e-06
27.1500000000014 8.66437264768032e-06
27.1600000000014 8.69611121166845e-06
27.1700000000014 8.71672011251287e-06
27.1800000000014 8.72622608542412e-06
27.1900000000015 8.7246697971598e-06
27.2000000000015 8.71210574307665e-06
27.2100000000015 8.68860213630968e-06
27.2200000000015 8.65424079771902e-06
27.2300000000015 8.60911706661486e-06
27.2400000000015 8.55333948274414e-06
27.2500000000015 8.48702552115744e-06
27.2600000000015 8.41030871765987e-06
27.2700000000015 8.32333511932708e-06
27.2800000000015 8.22626304055577e-06
27.2900000000015 8.11926280472379e-06
27.3000000000015 8.00251647188998e-06
27.3100000000015 7.87621755296254e-06
27.3200000000015 7.74057071077281e-06
27.3300000000015 7.59579043075042e-06
27.3400000000015 7.44210215882293e-06
27.3500000000015 7.27974240255133e-06
27.3600000000015 7.10895742034776e-06
27.3700000000015 6.93000289511171e-06
27.3800000000015 6.74314358410942e-06
27.3900000000015 6.54865295212805e-06
27.4000000000015 6.34681279152476e-06
27.4100000000015 6.13791283129692e-06
27.4200000000015 5.92225033658427e-06
27.4300000000015 5.70012969965451e-06
27.4400000000015 5.4718620232307e-06
27.4500000000015 5.2377646969103e-06
27.4600000000015 4.99816096736307e-06
27.4700000000015 4.75337950296258e-06
27.4800000000015 4.50375395348175e-06
27.4900000000015 4.24962250546764e-06
27.5000000000015 3.9913274339082e-06
27.5100000000015 3.72921465078948e-06
27.5200000000015 3.46363325114795e-06
27.5300000000015 3.19493505720849e-06
27.5400000000015 2.92347416120498e-06
27.5500000000015 2.64960646747113e-06
27.5600000000015 2.37368923438927e-06
27.5700000000015 2.09608061678048e-06
27.5800000000015 1.81713920931334e-06
27.5900000000015 1.53722359150759e-06
27.6000000000015 1.25669187490139e-06
27.6100000000015 9.75901252944999e-07
27.6200000000015 6.95207554178548e-07
27.6300000000015 4.14964799242656e-07
27.6400000000015 1.35524762267551e-07
27.6500000000015 -1.4276346283002e-07
27.6600000000015 -4.19553890611847e-07
27.6700000000015 -6.94504066433467e-07
27.6800000000015 -9.6727548230104e-07
27.6900000000015 -1.2375339857758e-06
27.7000000000015 -1.504950181442e-06
27.7100000000015 -1.76919982446601e-06
27.7200000000015 -2.02996420578734e-06
27.7300000000015 -2.28693052849239e-06
27.7400000000015 -2.53979227493961e-06
27.7500000000015 -2.78824956421348e-06
27.7600000000015 -3.03200949950787e-06
27.7700000000015 -3.27078650503627e-06
27.7800000000015 -3.50430265210215e-06
27.7900000000015 -3.73228797396643e-06
27.8000000000015 -3.95448076916748e-06
27.8100000000015 -4.17062789296504e-06
27.8200000000015 -4.38048503659697e-06
27.8300000000016 -4.58381699405453e-06
27.8400000000016 -4.78039791609888e-06
27.8500000000016 -4.97001155125833e-06
27.8600000000016 -5.15245147356542e-06
27.8700000000016 -5.32752129681003e-06
27.8800000000016 -5.49503487510294e-06
27.8900000000016 -5.65481648956332e-06
27.9000000000016 -5.80670102096292e-06
27.9100000000016 -5.9505341081771e-06
27.9200000000016 -6.08617229231497e-06
27.9300000000016 -6.21348314644373e-06
27.9400000000016 -6.33234539081467e-06
27.9500000000016 -6.44264899326869e-06
27.9600000000016 -6.54429525529661e-06
27.9700000000016 -6.63719688332832e-06
27.9800000000016 -6.72127804535735e-06
27.9900000000016 -6.79647441292282e-06
28.0000000000016 -6.86273318850039e-06
28.0100000000016 -6.92001311838442e-06
28.0200000000016 -6.96828449118443e-06
28.0300000000016 -7.0075291221084e-06
28.0400000000016 -7.03774032327569e-06
28.0500000000016 -7.05892286040656e-06
28.0600000000016 -7.07109289639884e-06
28.0700000000016 -7.07427792258354e-06
28.0800000000016 -7.0685166789688e-06
28.0900000000016 -7.053859065804e-06
28.1000000000016 -7.03036605101373e-06
28.1100000000016 -6.99810958337703e-06
28.1200000000016 -6.95717253600583e-06
28.1300000000016 -6.90764663831336e-06
28.1400000000016 -6.84963390400361e-06
28.1500000000016 -6.78324782654382e-06
28.1600000000016 -6.7086119382655e-06
28.1700000000016 -6.62585960659955e-06
28.1800000000016 -6.53513381880535e-06
28.1900000000016 -6.43658695554773e-06
28.2000000000016 -6.33038055367484e-06
28.2100000000016 -6.21668504676943e-06
28.2200000000016 -6.09567818298869e-06
28.2300000000016 -5.96754772282124e-06
28.2400000000016 -5.8324894503415e-06
28.2500000000016 -5.69070692036984e-06
28.2600000000016 -5.54241118216736e-06
28.2700000000016 -5.38782048763507e-06
28.2800000000016 -5.22715998786157e-06
28.2900000000016 -5.06066142014456e-06
28.3000000000016 -4.88856278682086e-06
28.3100000000016 -4.71110802685423e-06
28.3200000000016 -4.52854668092596e-06
28.3300000000016 -4.34113355066639e-06
28.3400000000016 -4.14912835259828e-06
28.3500000000016 -3.95279536733172e-06
28.3600000000016 -3.75240308452723e-06
28.3700000000016 -3.54822384412865e-06
28.3800000000016 -3.34053347436324e-06
28.3900000000016 -3.12961092699725e-06
28.4000000000016 -2.91573791033445e-06
28.4100000000016 -2.69919852044142e-06
28.4200000000016 -2.48027887108024e-06
28.4300000000016 -2.25926672282869e-06
28.4400000000016 -2.03645111186251e-06
28.4500000000016 -1.81212197887601e-06
28.4600000000016 -1.58656979860976e-06
28.4700000000017 -1.36008521045321e-06
28.4800000000017 -1.13295865058591e-06
28.4900000000017 -9.05479986114798e-07
28.5000000000017 -6.779381516629e-07
28.5100000000017 -4.50620788856752e-07
28.5200000000017 -2.2381388915527e-07
28.5300000000017 2.19855954419423e-09
28.5400000000017 2.27134922092562e-07
28.5500000000017 4.50716260648321e-07
28.5600000000017 6.72666674195223e-07
28.5700000000017 8.92713632611882e-07
28.5800000000017 1.11058830489943e-06
28.5900000000017 1.32602588117749e-06
28.6000000000017 1.5387658880725e-06
28.6100000000017 1.74855249713084e-06
28.6200000000017 1.95513482590108e-06
28.6300000000017 2.15826723133653e-06
28.6400000000017 2.35770959519071e-06
28.6500000000017 2.55322760107535e-06
28.6600000000017 2.74459300287478e-06
28.6700000000017 2.9315838842185e-06
28.6800000000017 3.11398490872617e-06
28.6900000000017 3.29158756075315e-06
28.7000000000017 3.46419037637835e-06
28.7100000000017 3.6315991643883e-06
28.7200000000017 3.79362721702685e-06
28.7300000000017 3.95009551029383e-06
28.7400000000017 4.10083289358883e-06
28.7500000000017 4.24567626851302e-06
28.7600000000017 4.38447075665569e-06
28.7700000000017 4.51706985620604e-06
28.7800000000017 4.6433355872485e-06
28.7900000000017 4.76313862561272e-06
28.8000000000017 4.87635842516686e-06
28.8100000000017 4.98288332845691e-06
28.8200000000017 5.08261066566586e-06
28.8300000000017 5.1754468416337e-06
28.8400000000017 5.26130741112904e-06
28.8500000000017 5.34011714226174e-06
28.8600000000017 5.41181006798436e-06
28.8700000000017 5.47632952571428e-06
28.8800000000017 5.53362818510506e-06
28.8900000000017 5.58366806401889e-06
28.9000000000017 5.62642053278011e-06
28.9100000000017 5.66186630682352e-06
28.9200000000017 5.68999542789761e-06
28.9300000000017 5.71080723404621e-06
28.9400000000017 5.72431031869177e-06
28.9500000000017 5.73052247930421e-06
28.9600000000017 5.72947065642601e-06
28.9700000000017 5.72119086436571e-06
28.9800000000017 5.70572811599771e-06
28.9900000000017 5.68313634665423e-06
29.0000000000017 5.65347834862207e-06
29.0100000000017 5.61682553070256e-06
29.0200000000017 5.57325553192303e-06
29.0300000000017 5.52285626937698e-06
29.0400000000017 5.46572386738051e-06
29.0500000000017 5.40196249708127e-06
29.0600000000017 5.33168420659076e-06
29.0700000000017 5.25500874193017e-06
29.0800000000017 5.17206335907876e-06
29.0900000000017 5.08298262741719e-06
29.1000000000017 4.98790770515942e-06
29.1100000000018 4.88698678054438e-06
29.1200000000018 4.7803752790951e-06
29.1300000000018 4.6682350247946e-06
29.1400000000018 4.55073402336977e-06
29.1500000000018 4.42804623090726e-06
29.1600000000018 4.30035131198126e-06
29.1700000000018 4.16783438946313e-06
29.1800000000018 4.03068578730018e-06
29.1900000000018 3.88910076713284e-06
29.2000000000018 3.74327925940484e-06
29.2100000000018 3.59342558950975e-06
29.2200000000018 3.43974819945348e-06
29.2300000000018 3.28245936547722e-06
29.2400000000018 3.12177491206385e-06
29.2500000000018 2.95791392273717e-06
29.2600000000018 2.79109844805807e-06
29.2700000000018 2.6215532112146e-06
29.2800000000018 2.44950531160245e-06
29.2900000000018 2.27518392678416e-06
29.3000000000018 2.09882001322231e-06
29.3100000000018 1.92064600617217e-06
29.3200000000018 1.74089551912079e-06
29.3300000000018 1.55980304315837e-06
29.3400000000018 1.377603646663e-06
29.3500000000018 1.19453267567861e-06
29.3600000000018 1.01082545536404e-06
29.3700000000018 8.26716992883342e-07
29.3800000000018 6.42441682110466e-07
29.3900000000018 4.58233010510082e-07
29.4000000000018 2.7432326855778e-07
29.4100000000018 9.09432620537888e-08
29.4200000000018 -9.16779723207873e-08
29.4300000000018 -2.73313447857792e-07
29.4400000000018 -4.53738504751377e-07
29.4500000000018 -6.3273108276896e-07
29.4600000000018 -8.10071989341057e-07
29.4700000000018 -9.85545162751846e-07
29.4800000000018 -1.15893793012168e-06
29.4900000000018 -1.33004125988059e-06
29.5000000000018 -1.49865000843987e-06
29.5100000000018 -1.66456316077619e-06
29.5200000000018 -1.82758406465674e-06
29.5300000000018 -1.9875206582367e-06
29.5400000000018 -2.14418569077457e-06
29.5500000000018 -2.29739693621793e-06
29.5600000000018 -2.44697739942629e-06
29.5700000000018 -2.59275551480378e-06
29.5800000000018 -2.734565337128e-06
29.5900000000018 -2.87224672437111e-06
29.6000000000018 -3.0056455123208e-06
29.6100000000018 -3.13461368081993e-06
29.6200000000018 -3.25900951145518e-06
29.6300000000018 -3.37869773653701e-06
29.6400000000018 -3.49354967922469e-06
29.6500000000018 -3.6034433846629e-06
29.6600000000018 -3.70826374200764e-06
29.6700000000018 -3.80790259723215e-06
29.6800000000018 -3.90225885661651e-06
29.6900000000018 -3.99123858083598e-06
29.7000000000018 -4.07475506958539e-06
29.7100000000018 -4.15272893668433e-06
29.7200000000018 -4.22508817551842e-06
29.7300000000018 -4.29176821498165e-06
29.7400000000018 -4.35271196575539e-06
29.7500000000019 -4.40786985696059e-06
29.7600000000019 -4.45719986319717e-06
29.7700000000019 -4.50066752200145e-06
29.7800000000019 -4.53824594177282e-06
29.7900000000019 -4.56991580024393e-06
29.8000000000019 -4.59566533359965e-06
29.8100000000019 -4.61549031639018e-06
29.8200000000019 -4.6293940324441e-06
29.8300000000019 -4.63738723708078e-06
29.8400000000019 -4.63948811108125e-06
29.8500000000019 -4.63572220716706e-06
29.8600000000019 -4.62612239031225e-06
29.8700000000019 -4.61072877445617e-06
29.8800000000019 -4.58958866116127e-06
29.8900000000019 -4.562756493945e-06
29.9000000000019 -4.5302926445837e-06
29.9100000000019 -4.49226428013966e-06
29.9200000000019 -4.44874597326946e-06
29.9300000000019 -4.39981887377086e-06
29.9400000000019 -4.34557057480532e-06
29.9500000000019 -4.28609497155883e-06
29.9600000000019 -4.22149211257816e-06
29.9700000000019 -4.15186804401969e-06
29.9800000000019 -4.07733464705056e-06
29.9900000000019 -3.9980087539758e-06
30.0000000000019 -3.91401351719794e-06
};
\addlegendentry{PI};
\addplot [ultra thick, blue!20!gray]
table {%
0 0
0.01 0.0262743771076202
0.02 0.0250437378883362
0.03 0.0217091783881187
0.04 0.0185855552554131
0.05 0.0160840809345245
0.06 0.0141661375761032
0.07 0.0127087831497192
0.08 0.0115955509245396
0.09 0.010734298825264
0.1 0.010056383907795
0.11 0.00951158478856087
0.12 0.00906379371881485
0.13 0.0086871400475502
0.14 0.00836313962936401
0.15 0.00807849168777466
0.16 0.00782388597726822
0.17 0.00759248808026314
0.18 0.00737959817051888
0.19 0.00718163177371025
0.2 0.00699625909328461
0.21 0.00682154670357704
0.22 0.00665621235966682
0.23 0.00649922862648964
0.24 0.00634992048144341
0.25 0.00620771422982216
0.26 0.00607215128839016
0.27 0.00594289153814316
0.28 0.00581970773637295
0.29 0.00570222660899162
0.3 0.00559027306735516
0.31 0.00548365227878094
0.32 0.0053821999579668
0.33 0.00528572946786881
0.34 0.00519412197172642
0.35 0.0051071785390377
0.36 0.00502474568784237
0.37 0.00494676269590855
0.38 0.00487300269305706
0.39 0.00480335280299187
0.4 0.00473767966032028
0.41 0.00467587113380432
0.42 0.00461780056357384
0.43 0.00456339344382286
0.44 0.00451250523328781
0.45 0.00446494445204735
0.46 0.00442070104181767
0.47 0.00437962673604488
0.48 0.00434165336191654
0.49 0.00430662520229816
0.5 0.00427450090646744
0.51 0.00424517951905727
0.52 0.00421852469444275
0.53 0.00419447347521782
0.54 0.00417296849191189
0.55 0.00415394604206085
0.56 0.00413726009428501
0.57 0.00412291400134563
0.58 0.00411073192954063
0.59 0.0041006863117218
0.6 0.00409279130399227
0.61 0.00408691540360451
0.62 0.00408298075199127
0.63 0.00408092476427555
0.64 0.00408075787127018
0.65 0.00408237800002098
0.66 0.00408569984138012
0.67 0.00409077405929565
0.68 0.00409743264317512
0.69 0.00410571433603764
0.7 0.00411550588905811
0.71 0.00412681438028812
0.72 0.0041395790874958
0.73 0.00415378250181675
0.74 0.00416934788227081
0.75 0.00418624319136143
0.76 0.00420449003577232
0.77 0.00422396250069141
0.78 0.0042446069419384
0.79 0.00426655672490597
0.8 0.00428963638842106
0.81 0.00431381948292255
0.820000000000001 0.00433910116553307
0.830000000000001 0.00436549447476864
0.840000000000001 0.00439285263419151
0.850000000000001 0.00442126616835594
0.860000000000001 0.00445068255066872
0.870000000000001 0.00448099076747894
0.880000000000001 0.00451226569712162
0.890000000000001 0.00454439371824265
0.900000000000001 0.00457742027938366
0.910000000000001 0.0046112384647131
0.920000000000001 0.00464588850736618
0.930000000000001 0.00468133836984634
0.940000000000001 0.00471754260361195
0.950000000000001 0.00475445911288261
0.960000000000001 0.00479209721088409
0.970000000000001 0.00483040101826191
0.980000000000001 0.00486934147775173
0.990000000000001 0.00490894764661789
1 0.00494916625320911
1.01 0.00498985610902309
1.02 0.00394409038126469
1.03 0.00400295816361904
1.04 0.00430034995079041
1.05 0.00462174080312252
1.06 0.00492244362831116
1.07 0.00519902296364307
1.08 0.00545642003417015
1.09 0.00569975785911083
1.1 0.00593307949602604
1.11 0.00615925379097462
1.12 0.00638032481074333
1.13 0.00659743919968605
1.14 0.00681138187646866
1.15 0.00702266916632652
1.16 0.00723148435354233
1.17 0.00743797942996025
1.18 0.00764211267232895
1.19 0.00784372836351395
1.2 0.00804281309247017
1.21 0.00823922008275986
1.22 0.00843271687626839
1.23 0.00862330123782158
1.24 0.00881073549389839
1.25 0.00899497643113136
1.26 0.00917582586407661
1.27 0.00935320481657982
1.28 0.00952708646655083
1.29 0.00969740971922874
1.3 0.00986398980021477
1.31 0.0100269868969917
1.32 0.010186143964529
1.33 0.0103415846824646
1.34 0.0104932680726051
1.35 0.0106412082910538
1.36 0.0107854664325714
1.37 0.0109259761869907
1.38 0.0110628373920918
1.39 0.0111960239708424
1.4 0.0113256476819515
1.41 0.0114517517387867
1.42 0.0115744091570377
1.43 0.0116936191916466
1.44 0.0118095234036446
1.45 0.0119221456348896
1.46 0.0120316565036774
1.47 0.0121380344033241
1.48 0.0122413516044617
1.49 0.0123418278992176
1.5 0.0124394036829472
1.51 0.0125342652201653
1.52 0.0126264840364456
1.53 0.0127161130309105
1.54 0.0128032729029655
1.55 0.0128880321979523
1.56 0.0129704877734184
1.57 0.013050776720047
1.58 0.0131289422512054
1.59 0.0132051527500153
1.6 0.0132792368531227
1.61 0.0133516371250153
1.62 0.0134221807122231
1.63 0.0134910762310028
1.64 0.0135583281517029
1.65 0.0136240676045418
1.66 0.0136883810162544
1.67 0.0137513443827629
1.68 0.0138129055500031
1.69 0.0138732418417931
1.7 0.0139324679970741
1.71 0.0139905720949173
1.72 0.0140476033091545
1.73 0.0141037300229073
1.74 0.0144411534070969
1.75 0.0148102432489395
1.76 0.0151597544550896
1.77 0.0154921680688858
1.78 0.0158088713884354
1.79 0.0161108613014221
1.8 0.0163986325263977
1.81 0.0166727036237717
1.82 0.0169335246086121
1.83 0.0171813443303108
1.84 0.0174166187644005
1.85 0.0176394864916801
1.86 0.0178502932190895
1.87 0.0180492326617241
1.88 0.0181794449687004
1.89 0.0180848881602287
1.9 0.0180144384503365
1.91 0.0179502457380295
1.92 0.0179037347435951
1.93 0.0178703218698502
1.94 0.0178470030426979
1.95 0.0178315058350563
1.96 0.0178222641348839
1.97 0.0178178980946541
1.98 0.0178175806999207
1.99 0.017820517718792
2 0.0178261250257492
2.01 0.01783397346735
2.02 0.0178437456488609
2.03 0.0178552031517029
2.04 0.0178681775927544
2.05 0.0178824901580811
2.06 0.0178980857133865
2.07 0.0179148659110069
2.08 0.0179328590631485
2.09 0.0179519131779671
2.1 0.0179720982909203
2.11 0.0179933294653893
2.12 0.0180155694484711
2.13 0.0180388912558556
2.14 0.018063285946846
2.15 0.0180886298418045
2.16 0.0181150034070015
2.17 0.0181423515081406
2.18 0.0181706413626671
2.19 0.0181999132037163
2.2 0.0182301551103592
2.21 0.0182613000273705
2.22 0.0182933583855629
2.23 0.0183262228965759
2.24 0.0183600276708603
2.25 0.0183945268392563
2.26 0.0184298440814018
2.27 0.0184659495949745
2.28 0.0185027480125427
2.29 0.0185401827096939
2.29999999999999 0.0185782983899117
2.30999999999999 0.018617045879364
2.31999999999999 0.0186563327908516
2.32999999999999 0.0186961233615875
2.33999999999999 0.018736420571804
2.34999999999999 0.0187772020697594
2.35999999999999 0.0188183397054672
2.36999999999999 0.0188598975539207
2.37999999999999 0.0189018085598946
2.38999999999999 0.0189440384507179
2.39999999999999 0.0189864531159401
2.40999999999999 0.0190291613340378
2.41999999999999 0.0190720587968826
2.42999999999999 0.0191151320934296
2.43999999999999 0.0191582798957825
2.44999999999999 0.0192015260457993
2.45999999999999 0.0192449018359184
2.46999999999999 0.0192882373929024
2.47999999999999 0.0193316534161568
2.48999999999999 0.019374917447567
2.49999999999999 0.0194182381033897
2.50999999999999 0.0194614574313164
2.51999999999999 0.0195045202970505
2.52999999999999 0.019547463953495
2.53999999999999 0.0195902794599533
2.54999999999999 0.0196329087018967
2.55999999999999 0.0196753412485123
2.56999999999999 0.0197174906730652
2.57999999999999 0.0197594910860062
2.58999999999999 0.0198011681437492
2.59999999999999 0.0198427200317383
2.60999999999999 0.0198838487267494
2.61999999999999 0.0199247613549232
2.62999999999999 0.0199653133749962
2.63999999999999 0.0200055927038193
2.64999999999999 0.020045568048954
2.65999999999999 0.0200851172208786
2.66999999999999 0.0201243460178375
2.67999999999999 0.0201632976531982
2.68999999999999 0.0202018544077873
2.69999999999999 0.0202400475740433
2.70999999999999 0.0202778324484825
2.71999999999999 0.0203152701258659
2.72999999999999 0.0203523740172386
2.73999999999999 0.0203890442848206
2.74999999999999 0.0204253748059273
2.75999999999999 0.0204612761735916
2.76999999999998 0.0204969093203545
2.77999999999998 0.0205321326851845
2.78999999999998 0.0205669343471527
2.79999999999998 0.0206014141440392
2.80999999999998 0.0206355258822441
2.81999999999998 0.0206691697239876
2.82999999999998 0.0207025423645973
2.83999999999998 0.0207355529069901
2.84999999999998 0.0207681834697723
2.85999999999998 0.0208004862070084
2.86999999999998 0.0208323985338211
2.87999999999998 0.020864000916481
2.88999999999998 0.0208952680230141
2.89999999999998 0.0209261611104012
2.90999999999998 0.020956788957119
2.91999999999998 0.0209870785474777
2.92999999999998 0.0210169792175293
2.93999999999998 0.0210465863347054
2.94999999999998 0.0210759460926056
2.95999999999998 0.0211049437522888
2.96999999999998 0.0211337238550186
2.97999999999998 0.0211621701717377
2.98999999999998 0.0211903482675552
2.99999999999998 0.021218179166317
3.00999999999998 0.0212458267807961
3.01999999999998 0.0212731301784515
3.02999999999998 0.0213002547621727
3.03999999999998 0.0213270485401154
3.04999999999998 0.0213536083698273
3.05999999999998 0.0213798820972443
3.06999999999998 0.0214059934020042
3.07999999999998 0.0214318051934242
3.08999999999998 0.0214573889970779
3.09999999999998 0.0214828118681908
3.10999999999998 0.0215079799294472
3.11999999999998 0.0215328708291054
3.12999999999998 0.0215576127171516
3.13999999999998 0.0215821161866188
3.14999999999998 0.02160634547472
3.15999999999998 0.0216305285692215
3.16999999999998 0.02165437489748
3.17999999999998 0.0216780841350555
3.18999999999998 0.0217015996575356
3.19999999999998 0.0217248663306236
3.20999999999998 0.0217480212450027
3.21999999999998 0.0217709794640541
3.22999999999998 0.0217936739325523
3.23999999999997 0.0218162804841995
3.24999999999997 0.0218386709690094
3.25999999999997 0.0218608975410461
3.26999999999997 0.0218829602003098
3.27999999999997 0.0219048351049423
3.28999999999997 0.0219265580177307
3.29999999999997 0.0219480529427528
3.30999999999997 0.0219694465398788
3.31999999999997 0.0219906404614449
3.32999999999997 0.0220117315649986
3.33999999999997 0.0220325917005539
3.34999999999997 0.0220533400774002
3.35999999999997 0.022073882818222
3.36999999999997 0.0220942810177803
3.37999999999997 0.022114485502243
3.38999999999997 0.0221346393227577
3.39999999999997 0.0221545144915581
3.40999999999997 0.0221743509173393
3.41999999999997 0.022193931043148
3.42999999999997 0.0222134068608284
3.43999999999997 0.0222327351570129
3.44999999999997 0.0222519010305405
3.45999999999997 0.0222709149122238
3.46999999999997 0.0222897365689278
3.47999999999997 0.0223085209727287
3.48999999999997 0.0223269760608673
3.49999999999997 0.0223454385995865
3.50999999999997 0.0223637253046036
3.51999999999997 0.0223818063735962
3.52999999999997 0.0223997488617897
3.53999999999997 0.0224175468087196
3.54999999999997 0.0224352315068245
3.55999999999997 0.0224527701735497
3.56999999999997 0.0224700644612312
3.57999999999997 0.0224873214960098
3.58999999999997 0.0225043907761574
3.59999999999997 0.0225212454795837
3.60999999999997 0.022538036108017
3.61999999999997 0.0225546523928642
3.62999999999997 0.0225711703300476
3.63999999999997 0.0225874826312065
3.64999999999997 0.0226036503911018
3.65999999999997 0.0226196959614754
3.66999999999997 0.022635580599308
3.67999999999997 0.0226513564586639
3.68999999999997 0.0226669639348984
3.69999999999997 0.0226824522018433
3.70999999999996 0.022697776556015
3.71999999999996 0.0227129399776459
3.72999999999996 0.0227279514074326
3.73999999999996 0.0227428779006004
3.74999999999996 0.0227576285600662
3.75999999999996 0.0227722451090813
3.76999999999996 0.0227867037057877
3.77999999999996 0.0228010550141335
3.78999999999996 0.0228153005242348
3.79999999999996 0.022829370200634
3.80999999999996 0.0228432834148407
3.81999999999996 0.0228571012616158
3.82999999999996 0.0228706821799278
3.83999999999996 0.022884313762188
3.84999999999996 0.0228976726531982
3.85999999999996 0.02291090041399
3.86999999999996 0.022924117743969
3.87999999999996 0.0229371026158333
3.88999999999996 0.0229499518871307
3.89999999999996 0.0229627192020416
3.90999999999996 0.0229753851890564
3.91999999999996 0.0229878425598145
3.92999999999996 0.0230002865195274
3.93999999999996 0.023012475669384
3.94999999999996 0.0230245545506477
3.95999999999996 0.0230365693569183
3.96999999999996 0.0230485126376152
3.97999999999996 0.0230602636933327
3.98999999999996 0.0230719342827797
3.99999999999996 0.0230834722518921
4.00999999999996 0.0230948641896248
4.01999999999996 0.0231061264872551
4.02999999999996 0.0231173500418663
4.03999999999996 0.023128479719162
4.04999999999996 0.0231394112110138
4.05999999999996 0.0231502637267113
4.06999999999996 0.0231609746813774
4.07999999999996 0.0231716275215149
4.08999999999996 0.0231822058558464
4.09999999999996 0.0231926470994949
4.10999999999996 0.0232009798288345
4.11999999999996 0.0232082083821297
4.12999999999996 0.023215463757515
4.13999999999996 0.0232229515910149
4.14999999999996 0.0232305258512497
4.15999999999996 0.0232382357120514
4.16999999999996 0.0232460588216782
4.17999999999996 0.0232539057731628
4.18999999999996 0.0232618898153305
4.19999999999995 0.0232698768377304
4.20999999999995 0.023277884721756
4.21999999999995 0.023285773396492
4.22999999999995 0.0232937410473824
4.23999999999995 0.023301699757576
4.24999999999995 0.0233095645904541
4.25999999999995 0.0233174309134483
4.26999999999995 0.0233252838253975
4.27999999999995 0.0233330428600311
4.28999999999995 0.0233408883213997
4.29999999999995 0.0233486011624336
4.30999999999995 0.023356319963932
4.31999999999995 0.0233638927340508
4.32999999999995 0.0233714774250984
4.33999999999995 0.0233790993690491
4.34999999999995 0.0233865529298782
4.35999999999995 0.0233939871191978
4.36999999999995 0.023401452600956
4.37999999999995 0.023408804833889
4.38999999999995 0.0234161257743835
4.39999999999995 0.0234234258532524
4.40999999999995 0.0234306529164314
4.41999999999995 0.0234378069639206
4.42999999999995 0.0234450176358223
4.43999999999995 0.0234520331025124
4.44999999999995 0.0234591022133827
4.45999999999995 0.0234660893678665
4.46999999999995 0.0234730079770088
4.47999999999995 0.0234799236059189
4.48999999999995 0.0234867349267006
4.49999999999995 0.023493555188179
4.50999999999995 0.0235002934932709
4.51999999999995 0.0235069677233696
4.52999999999995 0.0235136061906815
4.53999999999995 0.0235202014446259
4.54999999999995 0.0235267505049706
4.55999999999995 0.0235332533717155
4.56999999999995 0.0235396862030029
4.57999999999995 0.0235460668802261
4.58999999999995 0.0235524281859398
4.59999999999995 0.0235586941242218
4.60999999999995 0.0235648959875107
4.61999999999995 0.0235710635781288
4.62999999999995 0.0235771745443344
4.63999999999995 0.0235832616686821
4.64999999999995 0.0235892951488495
4.65999999999995 0.0235952749848366
4.66999999999994 0.0236011922359467
4.67999999999994 0.0236070141196251
4.68999999999994 0.0236128553748131
4.69999999999994 0.0236186027526855
4.70999999999994 0.0236243367195129
4.71999999999994 0.0236300185322762
4.72999999999994 0.0236356303095818
4.73999999999994 0.0236411795020103
4.74999999999994 0.0236466586589813
4.75999999999994 0.0236521139740944
4.76999999999994 0.0236575797200203
4.77999999999994 0.0236628860235214
4.78999999999994 0.0236681550741196
4.79999999999994 0.0236734226346016
4.80999999999994 0.0236786246299744
4.81999999999994 0.023683774471283
4.82999999999994 0.0236889004707336
4.83999999999994 0.0236939266324043
4.84999999999994 0.0236989244818687
4.85999999999994 0.0237039238214493
4.86999999999994 0.0237088188529015
4.87999999999994 0.0237137034535408
4.88999999999994 0.02371846139431
4.89999999999994 0.0237232491374016
4.90999999999994 0.0237279579043388
4.91999999999994 0.0237326517701149
4.92999999999994 0.0237372398376465
4.93999999999994 0.0237418130040169
4.94999999999994 0.0237463563680649
4.95999999999994 0.0237508505582809
4.96999999999994 0.0237552732229233
4.97999999999994 0.0237596362829208
4.98999999999994 0.0237639546394348
4.99999999999994 0.02376828789711
5.00999999999994 0.0237725675106049
5.01999999999994 0.0237767890095711
5.02999999999994 0.0237809121608734
5.03999999999994 0.0237850770354271
5.04999999999994 0.0237891376018524
5.05999999999994 0.0237931743264198
5.06999999999994 0.0237972036004066
5.07999999999994 0.023801127076149
5.08999999999994 0.0238050580024719
5.09999999999994 0.0238089382648468
5.10999999999994 0.0238127291202545
5.11999999999994 0.0238165602087975
5.12999999999994 0.0238203138113022
5.13999999999993 0.0238239929080009
5.14999999999993 0.0238276764750481
5.15999999999993 0.0238312736153603
5.16999999999993 0.0238348975777626
5.17999999999993 0.0238384246826172
5.18999999999993 0.0238419681787491
5.19999999999993 0.0238454550504684
5.20999999999993 0.0238489136099815
5.21999999999993 0.0238523155450821
5.22999999999993 0.0238556653261185
5.23999999999993 0.0238589406013489
5.24999999999993 0.0238622531294823
5.25999999999993 0.023865532875061
5.26999999999993 0.0238687872886658
5.27999999999993 0.0238719269633293
5.28999999999993 0.0238751009106636
5.29999999999993 0.0238781958818436
5.30999999999993 0.0238812774419785
5.31999999999993 0.0238843128085136
5.32999999999993 0.0238873213529587
5.33999999999993 0.0238902777433395
5.34999999999993 0.0238932579755783
5.35999999999993 0.0238961711525917
5.36999999999993 0.0238990366458893
5.37999999999993 0.0239018946886063
5.38999999999993 0.0239046633243561
5.39999999999993 0.0239074870944023
5.40999999999993 0.023910228908062
5.41999999999993 0.02391297519207
5.42999999999993 0.0239156171679497
5.43999999999993 0.0239182949066162
5.44999999999993 0.0239209026098251
5.45999999999993 0.0239235281944275
5.46999999999993 0.0239261016249657
5.47999999999993 0.0239286214113235
5.48999999999993 0.0239311784505844
5.49999999999993 0.0239336714148521
5.50999999999993 0.0239361345767975
5.51999999999993 0.0239385366439819
5.52999999999993 0.0239409297704697
5.53999999999993 0.0239432573318481
5.54999999999993 0.0239456534385681
5.55999999999993 0.0239479705691338
5.56999999999993 0.0239501789212227
5.57999999999993 0.0239525079727173
5.58999999999993 0.0239547535777092
5.59999999999993 0.0239570051431656
5.60999999999992 0.0239591374993324
5.61999999999992 0.0239613547921181
5.62999999999992 0.0239634692668915
5.63999999999992 0.0239655867218971
5.64999999999992 0.0239676401019096
5.65999999999992 0.0239697098731995
5.66999999999992 0.0239717990159988
5.67999999999992 0.0239738285541534
5.68999999999992 0.0239757776260376
5.69999999999992 0.0239777117967606
5.70999999999992 0.0239797160029411
5.71999999999992 0.023981648683548
5.72999999999992 0.0239835098385811
5.73999999999992 0.0239853844046593
5.74999999999992 0.0239872649312019
5.75999999999992 0.0239891529083252
5.76999999999992 0.0239909291267395
5.77999999999992 0.0239927351474762
5.78999999999992 0.023994491994381
5.79999999999992 0.0239962786436081
5.80999999999992 0.0239979982376099
5.81999999999992 0.0239997372031212
5.82999999999992 0.0240014016628265
5.83999999999992 0.0240030884742737
5.84999999999992 0.0240047439932823
5.85999999999992 0.0240063533186913
5.86999999999992 0.024008010327816
5.87999999999992 0.0240096881985664
5.88999999999992 0.0240111947059631
5.89999999999992 0.0240127608180046
5.90999999999992 0.0240143150091171
5.91999999999992 0.0240158259868622
5.92999999999992 0.024017408490181
5.93999999999992 0.0240188255906105
5.94999999999992 0.0240203648805618
5.95999999999992 0.0240218162536621
5.96999999999992 0.0240232437849045
5.97999999999992 0.0240246579051018
5.98999999999992 0.0240261346101761
5.99999999999992 0.0240275114774704
6.00999999999992 0.0240288406610489
6.01999999999992 0.0240302518010139
6.02999999999992 0.0240315645933151
6.03999999999992 0.0240329653024673
6.04999999999992 0.0240343034267426
6.05999999999992 0.024035581946373
6.06999999999992 0.0240368485450745
6.07999999999991 0.0240381047129631
6.08999999999991 0.024039414525032
6.09999999999991 0.0240406438708305
6.10999999999991 0.0240418761968613
6.11999999999991 0.0240431129932404
6.12999999999991 0.0240443184971809
6.13999999999991 0.0240455687046051
6.14999999999991 0.0240467488765717
6.15999999999991 0.0240479245781899
6.16999999999991 0.0240490436553955
6.17999999999991 0.0240502089262009
6.18999999999991 0.0240513116121292
6.19999999999991 0.024052457511425
6.20999999999991 0.0240535527467728
6.21999999999991 0.024054628610611
6.22999999999991 0.0240556925535202
6.23999999999991 0.024056801199913
6.24999999999991 0.0240578874945641
6.25999999999991 0.024058897793293
6.26999999999991 0.0240599885582924
6.27999999999991 0.0240610152482986
6.28999999999991 0.0240620151162148
6.29999999999991 0.0240630686283112
6.30999999999991 0.0240640386939049
6.31999999999991 0.0240649923682213
6.32999999999991 0.0240659847855568
6.33999999999991 0.0240669623017311
6.34999999999991 0.0240679308772087
6.35999999999991 0.024068908393383
6.36999999999991 0.0240697354078293
6.37999999999991 0.0240706890821457
6.38999999999991 0.0240716814994812
6.39999999999991 0.0240725472569466
6.40999999999991 0.0240734159946442
6.41999999999991 0.0240743175148964
6.42999999999991 0.0240752294659615
6.43999999999991 0.0240760892629623
6.44999999999991 0.0240769758820534
6.45999999999991 0.0240777850151062
6.46999999999991 0.0240786656737328
6.47999999999991 0.0240794718265533
6.48999999999991 0.0240803316235542
6.49999999999991 0.0240811511874199
6.50999999999991 0.0240819439291954
6.51999999999991 0.0240827858448029
6.52999999999991 0.0240835517644882
6.53999999999991 0.0240843087434769
6.5499999999999 0.0240850999951363
6.5599999999999 0.0240859001874924
6.5699999999999 0.0240866392850876
6.5799999999999 0.0240874007344246
6.5899999999999 0.0240882068872452
6.5999999999999 0.0240889444947243
6.6099999999999 0.0240896090865135
6.6199999999999 0.0240903601050377
6.6299999999999 0.0240910947322845
6.6399999999999 0.0240917891263962
6.6499999999999 0.0240924909710884
6.6599999999999 0.024093197286129
6.6699999999999 0.024093896150589
6.6799999999999 0.0240945518016815
6.6899999999999 0.0240952551364899
6.6999999999999 0.0240959048271179
6.7099999999999 0.0240966111421585
6.7199999999999 0.0240972325205803
6.7299999999999 0.0240979045629501
6.7399999999999 0.0240985199809074
6.7499999999999 0.0240991860628128
6.7599999999999 0.0240998595952988
6.7699999999999 0.0241004183888435
6.7799999999999 0.0241010442376137
6.7899999999999 0.0241016402840614
6.7999999999999 0.0241022601723671
6.8099999999999 0.024102845788002
6.8199999999999 0.0241034433245659
6.8299999999999 0.0241040408611298
6.8399999999999 0.0241046249866486
6.8499999999999 0.0241052627563477
6.8599999999999 0.0241057857871056
6.8699999999999 0.0241063579916954
6.8799999999999 0.0241069048643112
6.8899999999999 0.0241074323654175
6.8999999999999 0.0241080164909363
6.9099999999999 0.0241085603833199
6.9199999999999 0.0241091340780258
6.9299999999999 0.0241096675395966
6.9399999999999 0.0241101324558258
6.9499999999999 0.0241107076406479
6.9599999999999 0.0241112276911736
6.9699999999999 0.02411168217659
6.9799999999999 0.0241122290492058
6.9899999999999 0.0241127774119377
6.9999999999999 0.0241132065653801
7.00999999999989 0.0241137281060219
7.01999999999989 0.0241142243146896
7.02999999999989 0.0241146832704544
7.03999999999989 0.0241151735186577
7.04999999999989 0.0241156503558159
7.05999999999989 0.0241161286830902
7.06999999999989 0.0241165474057198
7.07999999999989 0.0241170212626457
7.08999999999989 0.024117474257946
7.09999999999989 0.0241179212927818
7.10999999999989 0.0241183802485466
7.11999999999989 0.0241187706589699
7.12999999999989 0.0241192951798439
7.13999999999989 0.0241197049617767
7.14999999999989 0.0241201058030128
7.15999999999989 0.0241205543279648
7.16999999999989 0.0241209402680397
7.17999999999989 0.0241214022040367
7.18999999999989 0.0241218253970146
7.19999999999989 0.0241222128272057
7.20999999999989 0.0241225808858871
7.21999999999989 0.0241230025887489
7.22999999999989 0.0241233259439468
7.23999999999989 0.0241237670183182
7.24999999999989 0.0241241574287415
7.25999999999989 0.0241245731711388
7.26999999999989 0.0241249397397041
7.27999999999989 0.024125312268734
7.28999999999989 0.0241256594657898
7.29999999999989 0.0241260260343552
7.30999999999989 0.0241264075040817
7.31999999999989 0.0241267636418343
7.32999999999989 0.0241271138191223
7.33999999999989 0.0241274610161781
7.34999999999989 0.0241278082132339
7.35999999999989 0.0241281524300575
7.36999999999989 0.0241285070776939
7.37999999999989 0.0241288602352142
7.38999999999989 0.0241291627287865
7.39999999999989 0.024129530787468
7.40999999999989 0.0241298973560333
7.41999999999989 0.0241301342844963
7.42999999999989 0.0241305217146873
7.43999999999989 0.0241307690739632
7.44999999999989 0.0241310968995094
7.45999999999989 0.0241313919425011
7.46999999999989 0.024131715297699
7.47999999999988 0.0241320267319679
7.48999999999988 0.0241323783993721
7.49999999999988 0.0241326302289963
7.50999999999988 0.0241329446434975
7.51999999999988 0.0241332784295082
7.52999999999988 0.0241335242986679
7.53999999999988 0.0241337731480598
7.54999999999988 0.0241340607404709
7.55999999999988 0.0241343215107918
7.56999999999988 0.0241346031427383
7.57999999999988 0.0241349220275879
7.58999999999988 0.0241351798176765
7.59999999999988 0.0241354361176491
7.60999999999988 0.0241357117891312
7.61999999999988 0.0241359576582909
7.62999999999988 0.0241362705826759
7.63999999999988 0.0241365075111389
7.64999999999988 0.0241367325186729
7.65999999999988 0.0241370633244514
7.66999999999988 0.0241372346878052
7.67999999999988 0.0241375163197517
7.68999999999988 0.0241377219557762
7.69999999999988 0.0241379991173744
7.70999999999988 0.0241382628679276
7.71999999999988 0.0241384267807007
7.72999999999988 0.0241386890411377
7.73999999999988 0.0241389498114586
7.74999999999988 0.0241391316056252
7.75999999999988 0.0241394028067589
7.76999999999988 0.0241396054625511
7.77999999999988 0.024139815568924
7.78999999999988 0.0241400703787804
7.79999999999988 0.0241402581334114
7.80999999999988 0.0241404622793198
7.81999999999988 0.024140690267086
7.82999999999988 0.0241409286856651
7.83999999999988 0.0241410851478577
7.84999999999988 0.0241413697600365
7.85999999999988 0.0241415455937386
7.86999999999988 0.0241417467594147
7.87999999999988 0.0241419091820717
7.88999999999988 0.0241421207785606
7.89999999999988 0.0241423219442368
7.90999999999988 0.0241424560546875
7.91999999999988 0.024142698943615
7.92999999999988 0.0241429209709167
7.93999999999988 0.0241431072354317
7.94999999999987 0.0241432830691338
7.95999999999987 0.0241434186697006
7.96999999999987 0.0241436421871185
7.97999999999987 0.0241438001394272
7.98999999999987 0.0241439774632454
7.99999999999987 0.0241441249847412
8.00999999999987 0.0241443529725075
8.01999999999987 0.0241444796323776
8.02999999999987 0.0241446301341057
8.03999999999987 0.0241448506712914
8.04999999999987 0.0241450324654579
8.05999999999987 0.0241451561450958
8.06999999999987 0.0241453632712364
8.07999999999987 0.0241455137729645
8.08999999999987 0.0241456776857376
8.09999999999987 0.0241457477211952
8.10999999999987 0.0241459473967552
8.11999999999987 0.0241460874676704
8.12999999999987 0.0241463035345078
8.13999999999987 0.024146443605423
8.14999999999987 0.0241465717554092
8.15999999999987 0.0241466999053955
8.16999999999987 0.0241468846797943
8.17999999999987 0.0241470158100128
8.18999999999987 0.0241471603512764
8.19999999999987 0.0241472527384758
8.20999999999987 0.0241474270820618
8.21999999999987 0.0241475641727448
8.22999999999987 0.0241476997733116
8.23999999999987 0.0241478323936462
8.24999999999987 0.0241480112075806
8.25999999999987 0.0241481468081474
8.26999999999987 0.0241482600569725
8.27999999999987 0.0241483688354492
8.28999999999987 0.0241485208272934
8.29999999999987 0.0241486579179764
8.30999999999987 0.0241487219929695
8.31999999999987 0.0241489425301552
8.32999999999987 0.0241489678621292
8.33999999999987 0.0241491258144379
8.34999999999987 0.0241492420434952
8.35999999999987 0.0241493940353394
8.36999999999987 0.0241494834423065
8.37999999999987 0.024149614572525
8.38999999999987 0.0241496816277504
8.39999999999987 0.0241497844457626
8.40999999999987 0.0241499304771423
8.41999999999986 0.0241500154137611
8.42999999999986 0.0241501286625862
8.43999999999986 0.0241502314805985
8.44999999999986 0.0241503730416298
8.45999999999986 0.0241504237055779
8.46999999999986 0.024150612950325
8.47999999999986 0.0241507172584534
8.48999999999986 0.0241508036851883
8.49999999999986 0.0241509407758713
8.50999999999986 0.0241509959101677
8.51999999999986 0.0241510912775993
8.52999999999986 0.0241512179374695
8.53999999999986 0.0241513192653656
8.54999999999986 0.0241514429450035
8.55999999999986 0.0241514846682549
8.56999999999986 0.0241515293717384
8.57999999999986 0.0241516798734665
8.58999999999986 0.0241518214344978
8.59999999999986 0.0241518750786781
8.60999999999986 0.0241519600152969
8.61999999999986 0.0241520747542381
8.62999999999986 0.0241521045565605
8.63999999999986 0.0241522133350372
8.64999999999986 0.0241523161530495
8.65999999999986 0.0241523608565331
8.66999999999986 0.0241524860262871
8.67999999999986 0.0241525590419769
8.68999999999986 0.0241526484489441
8.69999999999986 0.024152734875679
8.70999999999986 0.0241528436541557
8.71999999999986 0.0241528898477554
8.72999999999986 0.0241529911756516
8.73999999999986 0.0241530656814575
8.74999999999986 0.0241530999541283
8.75999999999986 0.0241532236337662
8.76999999999986 0.0241533026099205
8.77999999999986 0.0241533949971199
8.78999999999986 0.0241534605622292
8.79999999999986 0.0241535797715187
8.80999999999986 0.0241536498069763
8.81999999999986 0.0241537094116211
8.82999999999986 0.0241537511348724
8.83999999999986 0.0241538017988205
8.84999999999986 0.0241538897156715
8.85999999999986 0.0241539865732193
8.86999999999986 0.0241540297865868
8.87999999999986 0.0241541042923927
8.88999999999985 0.0241542056202888
8.89999999999985 0.0241542175412178
8.90999999999985 0.0241543158888817
8.91999999999985 0.024154332280159
8.92999999999985 0.0241544336080551
8.93999999999985 0.0241545006632805
8.94999999999985 0.0241545468568802
8.95999999999985 0.0241546347737312
8.96999999999985 0.0241547122597694
8.97999999999985 0.0241547539830208
8.98999999999985 0.0241548597812653
8.99999999999985 0.0241548374295235
9.00999999999985 0.0241549044847488
9.01999999999985 0.0241549864411354
9.02999999999985 0.02415501922369
9.03999999999985 0.0241550996899605
9.04999999999985 0.0241551414132118
9.05999999999985 0.0241552487015724
9.06999999999985 0.0241553202271462
9.07999999999985 0.0241553321480751
9.08999999999985 0.0241554111242294
9.09999999999985 0.0241554170846939
9.10999999999985 0.0241554453969002
9.11999999999985 0.0241555362939835
9.12999999999985 0.0241555944085121
9.13999999999985 0.0241556063294411
9.14999999999985 0.0241556704044342
9.15999999999985 0.0241557538509369
9.16999999999985 0.0241558089852333
9.17999999999985 0.0241558223962784
9.18999999999985 0.0241559073328972
9.19999999999985 0.024155955016613
9.20999999999985 0.0241560027003288
9.21999999999985 0.0241560697555542
9.22999999999985 0.0241561368107796
9.23999999999985 0.02415621727705
9.24999999999985 0.0241561934351921
9.25999999999985 0.0241562277078629
9.26999999999985 0.0241562992334366
9.27999999999985 0.0241562932729721
9.28999999999985 0.0241563558578491
9.29999999999985 0.0241564035415649
9.30999999999985 0.024156442284584
9.31999999999985 0.0241565093398094
9.32999999999985 0.0241566002368927
9.33999999999985 0.0241566002368927
9.34999999999985 0.024156628549099
9.35999999999984 0.0241567119956017
9.36999999999984 0.0241567328572273
9.37999999999984 0.0241567760705948
9.38999999999984 0.024156765639782
9.39999999999984 0.0241568267345428
9.40999999999984 0.0241568699479103
9.41999999999984 0.0241569310426712
9.42999999999984 0.0241569474339485
9.43999999999984 0.0241569876670837
9.44999999999984 0.0241570264101028
9.45999999999984 0.0241570860147476
9.46999999999984 0.0241570979356766
9.47999999999984 0.0241571739315987
9.48999999999984 0.0241571739315987
9.49999999999984 0.0241572365164757
9.50999999999984 0.0241572722792625
9.51999999999984 0.0241573005914688
9.52999999999984 0.0241573512554169
9.53999999999984 0.0241573616862297
9.54999999999984 0.0241573974490166
9.55999999999984 0.0241574466228485
9.56999999999984 0.0241574421525002
9.57999999999984 0.0241574704647064
9.58999999999984 0.0241575017571449
9.59999999999984 0.0241575360298157
9.60999999999984 0.0241575911641121
9.61999999999984 0.0241576328873634
9.62999999999984 0.0241576626896858
9.63999999999984 0.0241576343774796
9.64999999999984 0.0241577461361885
9.65999999999984 0.0241576939821243
9.66999999999984 0.024157777428627
9.67999999999984 0.0241577699780464
9.68999999999984 0.0241578489542007
9.69999999999984 0.0241578876972199
9.70999999999984 0.024157926440239
9.71999999999984 0.0241579353809357
9.72999999999984 0.024157939851284
9.73999999999984 0.024157977104187
9.74999999999984 0.0241579920053482
9.75999999999984 0.0241580173373222
9.76999999999984 0.0241580352187157
9.77999999999984 0.0241581067442894
9.78999999999984 0.0241580590605736
9.79999999999984 0.0241581037640572
9.80999999999984 0.024158176779747
9.81999999999984 0.0241581931710243
9.82999999999983 0.0241582497954369
9.83999999999983 0.0241582185029984
9.84999999999983 0.0241582334041595
9.85999999999983 0.0241582870483398
9.86999999999983 0.0241583004593849
9.87999999999983 0.0241583734750748
9.88999999999983 0.0241583898663521
9.89999999999983 0.0241583958268166
9.90999999999983 0.0241584077477455
9.91999999999983 0.0241584002971649
9.92999999999983 0.0241584524512291
9.93999999999983 0.0241584986448288
9.94999999999983 0.0241585269570351
9.95999999999983 0.0241585165262222
9.96999999999983 0.0241585463285446
9.97999999999983 0.0241585984826088
9.98999999999983 0.024158576130867
9.99999999999983 0.0241586253046989
10.0099999999998 0.0241586565971375
10.0199999999998 0.0241586565971375
10.0299999999998 0.0241586714982986
10.0399999999998 0.0241586968302727
10.0499999999998 0.0241587445139885
10.0599999999998 0.0241587668657303
10.0699999999998 0.0241587415337563
10.0799999999998 0.0241587311029434
10.0899999999998 0.0241587683558464
10.0999999999998 0.0241587981581688
10.1099999999998 0.0241588592529297
10.1199999999998 0.024158863723278
10.1299999999998 0.0241589099168777
10.1399999999998 0.0241589546203613
10.1499999999998 0.0241589024662971
10.1599999999998 0.024158938229084
10.1699999999998 0.0241589680314064
10.1799999999998 0.0241589725017548
10.1899999999998 0.0241589769721031
10.1999999999998 0.0241589963436127
10.2099999999998 0.024159049987793
10.2199999999998 0.024159000813961
10.2299999999998 0.0241590708494186
10.2399999999998 0.0241590902209282
10.2499999999998 0.0241591110825539
10.2599999999998 0.0241591140627861
10.2699999999998 0.0241591274738312
10.2799999999998 0.0241591453552246
10.2899999999998 0.0241591453552246
10.2999999999998 0.0241591736674309
10.3099999999998 0.0241592198610306
10.3199999999998 0.0241592094302177
10.3299999999998 0.0241592168807983
10.3399999999998 0.0241592168807983
10.3499999999998 0.0241592571139336
10.3599999999998 0.0241592988371849
10.3699999999998 0.0241592884063721
10.3799999999998 0.0241592973470688
10.3899999999998 0.0241592943668365
10.3999999999998 0.02415931224823
10.4099999999998 0.0241593345999718
10.4199999999998 0.0241593271493912
10.4299999999998 0.0241593867540359
10.4399999999998 0.0241593420505524
10.4499999999998 0.0241593942046165
10.4599999999998 0.0241593882441521
10.4699999999998 0.0241594314575195
10.4799999999998 0.0241593837738037
10.4899999999998 0.0241594061255455
10.4999999999998 0.024159437417984
10.5099999999998 0.0241594463586807
10.5199999999998 0.0241594418883324
10.5299999999998 0.0241594970226288
10.5399999999998 0.0241594702005386
10.5499999999998 0.0241595223546028
10.5599999999998 0.0241595074534416
10.5699999999998 0.0241595044732094
10.5799999999998 0.0241595447063446
10.5899999999998 0.0241595447063446
10.5999999999998 0.0241595163941383
10.6099999999998 0.0241595551371574
10.6199999999998 0.0241595849394798
10.6299999999998 0.0241596266627312
10.6399999999998 0.0241596207022667
10.6499999999998 0.0241596013307571
10.6599999999998 0.0241596832871437
10.6699999999998 0.0241596475243568
10.6799999999998 0.0241596579551697
10.6899999999998 0.0241596221923828
10.6999999999998 0.0241596400737762
10.7099999999998 0.0241596683859825
10.7199999999998 0.0241596862673759
10.7299999999998 0.0241597175598145
10.7399999999998 0.0241597160696983
10.7499999999998 0.0241597309708595
10.7599999999998 0.0241597071290016
10.7699999999998 0.0241597235202789
10.7799999999998 0.024159774184227
10.7899999999998 0.0241597473621368
10.7999999999998 0.0241597771644592
10.8099999999998 0.0241598024964333
10.8199999999998 0.0241597965359688
10.8299999999998 0.0241598144173622
10.8399999999998 0.0241598069667816
10.8499999999998 0.0241598159074783
10.8599999999998 0.0241598099470139
10.8699999999998 0.02415991127491
10.8799999999998 0.0241598412394524
10.8899999999998 0.0241598382592201
10.8999999999998 0.0241598889231682
10.9099999999998 0.0241598412394524
10.9199999999998 0.0241598501801491
10.9299999999998 0.0241598695516586
10.9399999999998 0.0241599068045616
10.9499999999998 0.0241598978638649
10.9599999999998 0.0241599291563034
10.9699999999998 0.0241599276661873
10.9799999999998 0.024159924685955
10.9899999999998 0.0241599336266518
10.9999999999998 0.0241599440574646
11.0099999999998 0.0241599097847939
11.0199999999998 0.0241599664092064
11.0299999999998 0.0241599872708321
11.0399999999998 0.0241599947214127
11.0499999999998 0.0241599842905998
11.0599999999998 0.0241599589586258
11.0699999999998 0.0241600155830383
11.0799999999998 0.0241599887609482
11.0899999999998 0.0241599723696709
11.0999999999998 0.0241600096225739
11.1099999999998 0.0241599887609482
11.1199999999998 0.0241600155830383
11.1299999999998 0.0241600334644318
11.1399999999998 0.0241600185632706
11.1499999999998 0.0241600126028061
11.1599999999998 0.0241600319743156
11.1699999999998 0.0241600349545479
11.1799999999998 0.0241600319743156
11.1899999999998 0.0241600707173347
11.1999999999998 0.0241600751876831
11.2099999999998 0.0241600677371025
11.2199999999998 0.0241600498557091
11.2299999999998 0.0241600841283798
11.2399999999998 0.0241600915789604
11.2499999999998 0.0241601213812828
11.2599999999998 0.0241600796580315
11.2699999999998 0.0241601005196571
11.2799999999998 0.0241600796580315
11.2899999999998 0.0241600960493088
11.2999999999998 0.0241601333022118
11.3099999999998 0.0241601154208183
11.3199999999998 0.0241601020097733
11.3299999999998 0.0241601139307022
11.3399999999998 0.0241601377725601
11.3499999999998 0.0241601869463921
11.3599999999998 0.0241601556539536
11.3699999999998 0.0241601184010506
11.3799999999998 0.0241601601243019
11.3899999999998 0.0241601884365082
11.3999999999998 0.0241601571440697
11.4099999999998 0.0241602137684822
11.4199999999998 0.0241601869463921
11.4299999999998 0.0241602152585983
11.4399999999998 0.0241601660847664
11.4499999999998 0.0241601690649986
11.4599999999998 0.0241602078080177
11.4699999999998 0.0241601809859276
11.4799999999998 0.0241602182388306
11.4899999999998 0.0241602301597595
11.4999999999998 0.0241602197289467
11.5099999999998 0.0241602331399918
11.5199999999998 0.0241602391004562
11.5299999999998 0.0241602346301079
11.5399999999998 0.0241602584719658
11.5499999999998 0.0241602867841721
11.5599999999998 0.024160236120224
11.5699999999998 0.0241602227091789
11.5799999999998 0.0241602316498756
11.5899999999998 0.024160261452198
11.5999999999998 0.0241602778434753
11.6099999999998 0.0241602823138237
11.6199999999998 0.0241602540016174
11.6299999999998 0.0241602540016174
11.6399999999998 0.0241603031754494
11.6499999999998 0.0241602823138237
11.6599999999998 0.0241602584719658
11.6699999999998 0.0241602674126625
11.6799999999998 0.0241602882742882
11.6899999999998 0.0241603091359139
11.6999999999998 0.0241603121161461
11.7099999999998 0.0241603091359139
11.7199999999998 0.0241603374481201
11.7299999999998 0.0241602882742882
11.7399999999998 0.0241603061556816
11.7499999999998 0.0241603031754494
11.7599999999998 0.0241603046655655
11.7699999999998 0.0241603314876556
11.7799999999998 0.0241603538393974
11.7899999999998 0.0241603314876556
11.7999999999998 0.024160324037075
11.8099999999998 0.0241603091359139
11.8199999999998 0.0241603493690491
11.8299999999998 0.0241603806614876
11.8399999999998 0.0241603344678879
11.8499999999998 0.0241603657603264
11.8599999999998 0.0241603627800941
11.8699999999998 0.0241603851318359
11.8799999999998 0.024160361289978
11.8899999999998 0.0241603910923004
11.8999999999998 0.024160361289978
11.9099999999998 0.0241603657603264
11.9199999999998 0.0241603791713715
11.9299999999998 0.0241603940725327
11.9399999999998 0.0241603940725327
11.9499999999998 0.0241603523492813
11.9599999999998 0.0241603672504425
11.9699999999998 0.0241603866219521
11.9799999999998 0.0241604089736938
11.9899999999998 0.0241604134440422
11.9999999999998 0.0241603791713715
12.0099999999998 0.0241603583097458
12.0199999999998 0.0241604149341583
12.0299999999998 0.0241603970527649
12.0399999999998 0.0241604432463646
12.0499999999998 0.0241604074835777
12.0599999999998 0.0241603747010231
12.0699999999998 0.024160398542881
12.0799999999998 0.024160461127758
12.0899999999998 0.0241604417562485
12.0999999999998 0.0241604328155518
12.1099999999998 0.0241603851318359
12.1199999999998 0.0241603821516037
12.1299999999998 0.02416041046381
12.1399999999998 0.0241604521870613
12.1499999999998 0.0241604223847389
12.1599999999998 0.0241604119539261
12.1699999999998 0.0241603910923004
12.1799999999998 0.0241604536771774
12.1899999999998 0.0241604417562485
12.1999999999998 0.024160461127758
12.2099999999998 0.0241604521870613
12.2199999999998 0.0241604506969452
12.2299999999998 0.0241604417562485
12.2399999999998 0.0241604462265968
12.2499999999998 0.024160435795784
12.2599999999998 0.024160461127758
12.2699999999998 0.024160461127758
12.2799999999998 0.0241604402661324
12.2899999999998 0.0241604194045067
12.2999999999998 0.0241604253649712
12.3099999999998 0.0241604462265968
12.3199999999998 0.0241604819893837
12.3299999999998 0.0241604804992676
12.3399999999998 0.0241604626178741
12.3499999999998 0.0241604864597321
12.3599999999998 0.0241604506969452
12.3699999999998 0.0241604760289192
12.3799999999998 0.0241604924201965
12.3899999999998 0.0241605058312416
12.3999999999998 0.02416051030159
12.4099999999998 0.0241604775190353
12.4199999999998 0.0241604715585709
12.4299999999998 0.0241604760289192
12.4399999999998 0.0241604760289192
12.4499999999998 0.0241605192422867
12.4599999999998 0.0241604998707771
12.4699999999998 0.0241605028510094
12.4799999999998 0.0241604715585709
12.4899999999998 0.0241604700684547
12.4999999999998 0.02416051030159
12.5099999999998 0.0241604864597321
12.5199999999998 0.0241604998707771
12.5299999999998 0.0241605222225189
12.5399999999998 0.0241605252027512
12.5499999999998 0.0241605162620544
12.5599999999998 0.0241605028510094
12.5699999999998 0.0241605192422867
12.5799999999998 0.0241604968905449
12.5899999999998 0.0241605535149574
12.5999999999998 0.0241604998707771
12.6099999999998 0.0241604939103127
12.6199999999998 0.0241605252027512
12.6299999999998 0.0241604998707771
12.6399999999998 0.0241605073213577
12.6499999999998 0.0241605281829834
12.6599999999998 0.0241604715585709
12.6699999999998 0.0241605326533318
12.6799999999998 0.0241605162620544
12.6899999999998 0.0241605058312416
12.6999999999998 0.0241605073213577
12.7099999999998 0.0241604998707771
12.7199999999998 0.0241605073213577
12.7299999999998 0.0241605147719383
12.7399999999998 0.0241605535149574
12.7499999999998 0.0241605207324028
12.7599999999998 0.0241605147719383
12.7699999999998 0.0241605117917061
12.7799999999998 0.0241605117917061
12.7899999999998 0.0241605296730995
12.7999999999998 0.0241605505347252
12.8099999999998 0.0241605505347252
12.8199999999998 0.0241605877876282
12.8299999999998 0.0241605535149574
12.8399999999998 0.0241605594754219
12.8499999999998 0.0241605624556541
12.8599999999998 0.0241604998707771
12.8699999999998 0.0241605311632156
12.8799999999998 0.0241605445742607
12.8899999999998 0.0241605296730995
12.8999999999998 0.0241605326533318
12.9099999999998 0.024160572886467
12.9199999999998 0.0241605311632156
12.9299999999998 0.0241605713963509
12.9399999999998 0.0241605713963509
12.9499999999998 0.0241605684161186
12.9599999999998 0.0241605654358864
12.9699999999998 0.0241605311632156
12.9799999999998 0.0241605386137962
12.9899999999998 0.0241605445742607
12.9999999999998 0.0241605311632156
13.0099999999998 0.024160560965538
13.0199999999998 0.0241605430841446
13.0299999999998 0.0241605713963509
13.0399999999998 0.0241605550050735
13.0499999999998 0.0241605684161186
13.0599999999998 0.0241605579853058
13.0699999999998 0.0241605907678604
13.0799999999998 0.0241606011986732
13.0899999999998 0.0241605445742607
13.0999999999998 0.0241605445742607
13.1099999999998 0.024160547554493
13.1199999999998 0.0241605296730995
13.1299999999998 0.0241605937480927
13.1399999999998 0.0241605803370476
13.1499999999998 0.024160572886467
13.1599999999998 0.0241605654358864
13.1699999999998 0.0241605713963509
13.1799999999998 0.0241605564951897
13.1899999999998 0.0241605818271637
13.1999999999998 0.02416061013937
13.2099999999998 0.0241606056690216
13.2199999999998 0.0241605773568153
13.2299999999998 0.0241605713963509
13.2399999999998 0.024160523712635
13.2499999999998 0.0241605371236801
13.2599999999998 0.0241605594754219
13.2699999999998 0.0241605669260025
13.2799999999998 0.0241605654358864
13.2899999999998 0.0241605713963509
13.2999999999998 0.0241605594754219
13.3099999999998 0.0241605669260025
13.3199999999998 0.0241605713963509
13.3299999999998 0.0241605654358864
13.3399999999998 0.0241605654358864
13.3499999999998 0.0241605669260025
13.3599999999998 0.0241605773568153
13.3699999999998 0.0241605550050735
13.3799999999998 0.0241605967283249
13.3899999999998 0.0241605669260025
13.3999999999998 0.0241605848073959
13.4099999999998 0.0241606011986732
13.4199999999998 0.0241605862975121
13.4299999999998 0.0241605639457703
13.4399999999998 0.0241605967283249
13.4499999999998 0.0241605907678604
13.4599999999998 0.024160572886467
13.4699999999998 0.0241605862975121
13.4799999999998 0.0241605460643768
13.4899999999998 0.024160572886467
13.4999999999998 0.0241605862975121
13.5099999999998 0.0241605877876282
13.5199999999998 0.0241605818271637
13.5299999999998 0.0241605713963509
13.5399999999998 0.0241605937480927
13.5499999999998 0.0241605892777443
13.5599999999998 0.0241605803370476
13.5699999999998 0.0241605669260025
13.5799999999998 0.0241605967283249
13.5899999999998 0.0241606131196022
13.5999999999998 0.0241606160998344
13.6099999999998 0.0241605967283249
13.6199999999998 0.0241605967283249
13.6299999999998 0.0241606280207634
13.6399999999998 0.0241605967283249
13.6499999999998 0.0241605907678604
13.6599999999998 0.0241605907678604
13.6699999999998 0.0241605907678604
13.6799999999998 0.0241605907678604
13.6899999999998 0.0241605907678604
13.6999999999998 0.0241606160998344
13.7099999999998 0.02416061013937
13.7199999999998 0.0241606041789055
13.7299999999998 0.0241606265306473
13.7399999999998 0.0241606265306473
13.7499999999998 0.0241606280207634
13.7599999999998 0.0241606295108795
13.7699999999998 0.0241606637835503
13.7799999999998 0.0241606295108795
13.7899999999998 0.0241606518626213
13.7999999999998 0.0241606041789055
13.8099999999998 0.0241606071591377
13.8199999999997 0.0241606131196022
13.8299999999997 0.0241606131196022
13.8399999999997 0.0241605967283249
13.8499999999997 0.0241605937480927
13.8599999999997 0.0241606026887894
13.8699999999997 0.0241606146097183
13.8799999999997 0.0241606086492538
13.8899999999997 0.0241606011986732
13.8999999999997 0.0241606071591377
13.9099999999997 0.0241605848073959
13.9199999999997 0.0241606190800667
13.9299999999997 0.0241605967283249
13.9399999999997 0.0241606295108795
13.9499999999997 0.0241606146097183
13.9599999999997 0.0241605937480927
13.9699999999997 0.0241605803370476
13.9799999999997 0.0241605997085571
13.9899999999997 0.0241606026887894
13.9999999999997 0.0241606086492538
14.0099999999997 0.0241605803370476
14.0199999999997 0.0241605997085571
14.0299999999997 0.0241606131196022
14.0399999999997 0.0241606131196022
14.0499999999997 0.0241606160998344
14.0599999999997 0.0241606146097183
14.0699999999997 0.0241606146097183
14.0799999999997 0.0241606116294861
14.0899999999997 0.0241606056690216
14.0999999999997 0.0241605952382088
14.1099999999997 0.024160598218441
14.1199999999997 0.0241606444120407
14.1299999999997 0.0241606041789055
14.1399999999997 0.024160598218441
14.1499999999997 0.0241606265306473
14.1599999999997 0.0241606011986732
14.1699999999997 0.0241606295108795
14.1799999999997 0.024160623550415
14.1899999999997 0.0241606324911118
14.1999999999997 0.0241605848073959
14.2099999999997 0.0241606116294861
14.2199999999997 0.0241606116294861
14.2299999999997 0.02416061013937
14.2399999999997 0.0241606265306473
14.2499999999997 0.0241606295108795
14.2599999999997 0.0241606146097183
14.2699999999997 0.0241606086492538
14.2799999999997 0.0241606250405312
14.2899999999997 0.0241606280207634
14.2999999999997 0.0241606310009956
14.3099999999997 0.0241606384515762
14.3199999999997 0.0241606190800667
14.3299999999997 0.0241606190800667
14.3399999999997 0.0241606414318085
14.3499999999997 0.0241606205701828
14.3599999999997 0.0241606205701828
14.3699999999997 0.0241606086492538
14.3799999999997 0.0241606086492538
14.3899999999997 0.0241606220602989
14.3999999999997 0.0241606339812279
14.4099999999997 0.0241606131196022
14.4199999999997 0.0241606414318085
14.4299999999997 0.0241606041789055
14.4399999999997 0.0241606399416924
14.4499999999997 0.02416061013937
14.4599999999997 0.0241606265306473
14.4699999999997 0.0241606414318085
14.4799999999997 0.0241606131196022
14.4899999999997 0.0241605967283249
14.4999999999997 0.024160623550415
14.5099999999997 0.0241606220602989
14.5199999999997 0.0241606071591377
14.5299999999997 0.0241606071591377
14.5399999999997 0.024160598218441
14.5499999999997 0.0241606086492538
14.5599999999997 0.0241606369614601
14.5699999999997 0.0241606071591377
14.5799999999997 0.0241606071591377
14.5899999999997 0.0241606369614601
14.5999999999997 0.0241605967283249
14.6099999999997 0.0241606399416924
14.6199999999997 0.0241606295108795
14.6299999999997 0.024160635471344
14.6399999999997 0.0241606131196022
14.6499999999997 0.024160647392273
14.6599999999997 0.0241606116294861
14.6699999999997 0.0241605997085571
14.6799999999997 0.0241606220602989
14.6899999999997 0.0241606220602989
14.6999999999997 0.0241606220602989
14.7099999999997 0.0241606324911118
14.7199999999997 0.0241606205701828
14.7299999999997 0.0241606280207634
14.7399999999997 0.0241606712341309
14.7499999999997 0.0241606488823891
14.7599999999997 0.0241606518626213
14.7699999999997 0.0241606414318085
14.7799999999997 0.024160647392273
14.7899999999997 0.02416061013937
14.7999999999997 0.024160647392273
14.8099999999997 0.0241606071591377
14.8199999999997 0.0241606533527374
14.8299999999997 0.0241606533527374
14.8399999999997 0.0241606086492538
14.8499999999997 0.0241606503725052
14.8599999999997 0.0241606339812279
14.8699999999997 0.0241606250405312
14.8799999999997 0.0241606682538986
14.8899999999997 0.0241606190800667
14.8999999999997 0.0241606295108795
14.9099999999997 0.0241606295108795
14.9199999999997 0.0241606310009956
14.9299999999997 0.0241606399416924
14.9399999999997 0.0241606310009956
14.9499999999997 0.0241606310009956
14.9599999999997 0.0241606310009956
14.9699999999997 0.0241606310009956
14.9799999999997 0.02416061013937
14.9899999999997 0.0241606295108795
14.9999999999997 0.024160623550415
15.0099999999997 0.0241606310009956
15.0199999999997 0.0241606459021568
15.0299999999997 0.0241606310009956
15.0399999999997 0.024160635471344
15.0499999999997 0.0241606369614601
15.0599999999997 0.024160623550415
15.0699999999997 0.024160623550415
15.0799999999997 0.0241606310009956
15.0899999999997 0.0241606131196022
15.0999999999997 0.0241606414318085
15.1099999999997 0.0241606131196022
15.1199999999997 0.0241606295108795
15.1299999999997 0.0241606414318085
15.1399999999997 0.0241606414318085
15.1499999999997 0.0241606131196022
15.1599999999997 0.0241606324911118
15.1699999999997 0.0241606324911118
15.1799999999997 0.0241606414318085
15.1899999999997 0.024160635471344
15.1999999999997 0.0241606369614601
15.2099999999997 0.0241606369614601
15.2199999999997 0.0241606280207634
15.2299999999997 0.0241606310009956
15.2399999999997 0.0241606265306473
15.2499999999997 0.0241606250405312
15.2599999999997 0.0241606310009956
15.2699999999997 0.0241606310009956
15.2799999999997 0.024160623550415
15.2899999999997 0.0241606310009956
15.2999999999997 0.0241606310009956
15.3099999999997 0.0241606295108795
15.3199999999997 0.024160623550415
15.3299999999997 0.0241606384515762
15.3399999999997 0.0241606310009956
15.3499999999997 0.0241606011986732
15.3599999999997 0.0241606310009956
15.3699999999997 0.024160635471344
15.3799999999997 0.0241606593132019
15.3899999999997 0.024160598218441
15.3999999999997 0.0241606310009956
15.4099999999997 0.0241606533527374
15.4199999999997 0.0241606310009956
15.4299999999997 0.0241606310009956
15.4399999999997 0.0241606310009956
15.4499999999997 0.0241606310009956
15.4599999999997 0.0241606310009956
15.4699999999997 0.0241606310009956
15.4799999999997 0.0241606310009956
15.4899999999997 0.0241606369614601
15.4999999999997 0.0241606280207634
15.5099999999997 0.0241606280207634
15.5199999999997 0.0241606339812279
15.5299999999997 0.0241606250405312
15.5399999999997 0.0241606250405312
15.5499999999997 0.0241606637835503
15.5599999999997 0.0241606041789055
15.5699999999997 0.0241606250405312
15.5799999999997 0.0241606339812279
15.5899999999997 0.0241606339812279
15.5999999999997 0.0241606339812279
15.6099999999997 0.024160635471344
15.6199999999997 0.024160623550415
15.6299999999997 0.024160623550415
15.6399999999997 0.0241606369614601
15.6499999999997 0.024160623550415
15.6599999999997 0.0241606250405312
15.6699999999997 0.024160623550415
15.6799999999997 0.024160623550415
15.6899999999997 0.0241606205701828
15.6999999999997 0.0241606444120407
15.7099999999997 0.0241606414318085
15.7199999999997 0.0241606414318085
15.7299999999997 0.0241606414318085
15.7399999999997 0.0241606190800667
15.7499999999997 0.0241606190800667
15.7599999999997 0.024160623550415
15.7699999999997 0.0241606190800667
15.7799999999997 0.0241606503725052
15.7899999999997 0.0241606190800667
15.7999999999997 0.0241606414318085
15.8099999999997 0.0241606414318085
15.8199999999997 0.024160623550415
15.8299999999997 0.024160623550415
15.8399999999997 0.0241606190800667
15.8499999999997 0.024160623550415
15.8599999999997 0.0241606444120407
15.8699999999997 0.0241606488823891
15.8799999999997 0.0241606459021568
15.8899999999997 0.0241606205701828
15.8999999999997 0.0241606414318085
15.9099999999997 0.0241606459021568
15.9199999999997 0.0241606414318085
15.9299999999997 0.0241606414318085
15.9399999999997 0.0241606429219246
15.9499999999997 0.0241606190800667
15.9599999999997 0.0241606414318085
15.9699999999997 0.0241606414318085
15.9799999999997 0.0241606414318085
15.9899999999997 0.0241606414318085
15.9999999999997 0.0241606414318085
16.0099999999997 0.0241606190800667
16.0199999999997 0.0241606488823891
16.0299999999997 0.0241606190800667
16.0399999999997 0.0241606578230858
16.0499999999997 0.0241605967283249
16.0599999999997 0.0241606578230858
16.0699999999997 0.0241606250405312
16.0799999999997 0.0241606190800667
16.0899999999997 0.0241606578230858
16.0999999999997 0.0241606265306473
16.1099999999997 0.0241606593132019
16.1199999999997 0.0241606488823891
16.1299999999997 0.0241606682538986
16.1399999999997 0.0241606488823891
16.1499999999997 0.0241606488823891
16.1599999999997 0.0241606622934341
16.1699999999997 0.0241606533527374
16.1799999999997 0.0241606667637825
16.1899999999997 0.0241606339812279
16.1999999999997 0.0241606324911118
16.2099999999997 0.0241606339812279
16.2199999999997 0.0241606175899506
16.2299999999997 0.0241606384515762
16.2399999999997 0.0241606488823891
16.2499999999997 0.0241606399416924
16.2599999999997 0.0241606444120407
16.2699999999997 0.0241606444120407
16.2799999999997 0.0241606444120407
16.2899999999997 0.0241606444120407
16.2999999999997 0.0241606518626213
16.3099999999998 0.0241606518626213
16.3199999999998 0.0241606459021568
16.3299999999998 0.0241606459021568
16.3399999999998 0.0241606310009956
16.3499999999998 0.0241606518626213
16.3599999999998 0.024160623550415
16.3699999999998 0.0241606459021568
16.3799999999998 0.0241606518626213
16.3899999999998 0.0241606459021568
16.3999999999998 0.0241606518626213
16.4099999999998 0.0241606548428535
16.4199999999998 0.024160660803318
16.4299999999998 0.024160623550415
16.4399999999998 0.0241606578230858
16.4499999999998 0.0241606250405312
16.4599999999998 0.024160635471344
16.4699999999998 0.0241606265306473
16.4799999999998 0.0241606086492538
16.4899999999998 0.0241606459021568
16.4999999999998 0.0241606459021568
16.5099999999998 0.0241606593132019
16.5199999999998 0.0241606205701828
16.5299999999998 0.0241606459021568
16.5399999999998 0.0241606190800667
16.5499999999998 0.024160647392273
16.5599999999998 0.0241606593132019
16.5699999999998 0.0241606190800667
16.5799999999998 0.0241606459021568
16.5899999999998 0.0241606459021568
16.5999999999998 0.0241606190800667
16.6099999999998 0.0241606459021568
16.6199999999998 0.0241606459021568
16.6299999999998 0.0241606190800667
16.6399999999998 0.024160623550415
16.6499999999998 0.0241606503725052
16.6599999999998 0.0241606533527374
16.6699999999998 0.0241606533527374
16.6799999999998 0.0241606250405312
16.6899999999998 0.0241606533527374
16.6999999999998 0.0241606250405312
16.7099999999998 0.0241606637835503
16.7199999999998 0.0241606250405312
16.7299999999998 0.0241606250405312
16.7399999999998 0.0241606533527374
16.7499999999998 0.0241606250405312
16.7599999999998 0.0241606533527374
16.7699999999998 0.0241606250405312
16.7799999999998 0.0241606533527374
16.7899999999998 0.0241606399416924
16.7999999999998 0.0241606250405312
16.8099999999998 0.0241606533527374
16.8199999999998 0.0241606310009956
16.8299999999998 0.0241606533527374
16.8399999999998 0.0241606310009956
16.8499999999998 0.0241606310009956
16.8599999999998 0.0241606637835503
16.8699999999998 0.0241606310009956
16.8799999999998 0.0241606310009956
16.8899999999998 0.0241606533527374
16.8999999999998 0.0241606533527374
16.9099999999998 0.0241606533527374
16.9199999999998 0.0241606339812279
16.9299999999998 0.0241606533527374
16.9399999999998 0.0241606339812279
16.9499999999999 0.0241606533527374
16.9599999999999 0.0241606429219246
16.9699999999999 0.0241606339812279
16.9799999999999 0.0241606533527374
16.9899999999999 0.0241606339812279
16.9999999999999 0.0241606533527374
17.0099999999999 0.0241606339812279
17.0199999999999 0.0241605862975121
17.0299999999999 0.0241606146097183
17.0399999999999 0.0241606131196022
17.0499999999999 0.0241605877876282
17.0599999999999 0.0241606205701828
17.0699999999999 0.0241606205701828
17.0799999999999 0.0241606190800667
17.0899999999999 0.0241606190800667
17.0999999999999 0.0241606190800667
17.1099999999999 0.024160623550415
17.1199999999999 0.0241606250405312
17.1299999999999 0.0241606250405312
17.1399999999999 0.0241606622934341
17.1499999999999 0.0241606414318085
17.1599999999999 0.0241606637835503
17.1699999999999 0.024160623550415
17.1799999999999 0.0241606578230858
17.1899999999999 0.0241606190800667
17.1999999999999 0.0241606160998344
17.2099999999999 0.0241606160998344
17.2199999999999 0.0241606220602989
17.2299999999999 0.0241606220602989
17.2399999999999 0.0241606220602989
17.2499999999999 0.024160623550415
17.2599999999999 0.0241606220602989
17.2699999999999 0.0241606205701828
17.2799999999999 0.0241606444120407
17.2899999999999 0.0241606205701828
17.2999999999999 0.0241606190800667
17.3099999999999 0.0241606429219246
17.3199999999999 0.0241606429219246
17.3299999999999 0.0241606444120407
17.3399999999999 0.0241606205701828
17.3499999999999 0.0241606160998344
17.3599999999999 0.0241606444120407
17.3699999999999 0.0241606160998344
17.3799999999999 0.0241606831550598
17.3899999999999 0.0241606920957565
17.3999999999999 0.0241606593132019
17.4099999999999 0.0241606384515762
17.4199999999999 0.0241606488823891
17.4299999999999 0.024160660803318
17.4399999999999 0.0241606324911118
17.4499999999999 0.0241606637835503
17.4599999999999 0.0241606712341309
17.4699999999999 0.0241606503725052
17.4799999999999 0.0241606444120407
17.4899999999999 0.0241606399416924
17.4999999999999 0.024160672724247
17.5099999999999 0.0241606399416924
17.5199999999999 0.0241606399416924
17.5299999999999 0.0241606399416924
17.5399999999999 0.0241606399416924
17.5499999999999 0.0241606503725052
17.5599999999999 0.0241606622934341
17.5699999999999 0.0241606250405312
17.5799999999999 0.024160660803318
17.59 0.0241606369614601
17.6 0.0241606414318085
17.61 0.0241606339812279
17.62 0.0241606295108795
17.63 0.0241606637835503
17.64 0.0241606444120407
17.65 0.0241606414318085
17.66 0.0241606324911118
17.67 0.0241606324911118
17.68 0.0241606384515762
17.69 0.0241606384515762
17.7 0.0241606175899506
17.71 0.0241606190800667
17.72 0.0241606593132019
17.73 0.024160623550415
17.74 0.0241606160998344
17.75 0.0241606384515762
17.76 0.0241606324911118
17.77 0.0241606459021568
17.78 0.0241606548428535
17.79 0.0241606265306473
17.8 0.02416061013937
17.81 0.0241606459021568
17.82 0.0241606578230858
17.83 0.0241606548428535
17.84 0.0241606265306473
17.85 0.0241606444120407
17.86 0.0241606503725052
17.87 0.0241606205701828
17.88 0.0241606444120407
17.89 0.0241606444120407
17.9 0.0241606280207634
17.91 0.0241606444120407
17.92 0.0241606444120407
17.93 0.0241606503725052
17.94 0.0241606205701828
17.95 0.0241606444120407
17.96 0.0241606444120407
17.97 0.0241606563329697
17.98 0.024160623550415
17.99 0.0241606205701828
18 0.0241606220602989
18.01 0.024160647392273
18.02 0.0241606667637825
18.03 0.0241606563329697
18.04 0.0241606250405312
18.05 0.0241606563329697
18.06 0.0241606250405312
18.07 0.0241606563329697
18.08 0.0241606250405312
18.09 0.0241606563329697
18.1 0.0241606459021568
18.11 0.0241606384515762
18.12 0.0241606459021568
18.13 0.0241606459021568
18.14 0.0241606384515762
18.15 0.024160647392273
18.16 0.0241606459021568
18.17 0.0241606712341309
18.18 0.0241606324911118
18.19 0.0241606384515762
18.2 0.0241606488823891
18.21 0.0241606503725052
18.22 0.0241606488823891
18.2300000000001 0.0241606414318085
18.2400000000001 0.0241606488823891
18.2500000000001 0.0241606190800667
18.2600000000001 0.0241606280207634
18.2700000000001 0.0241605937480927
18.2800000000001 0.0241606220602989
18.2900000000001 0.024160647392273
18.3000000000001 0.0241606011986732
18.3100000000001 0.02416061013937
18.3200000000001 0.0241606205701828
18.3300000000001 0.0241606339812279
18.3400000000001 0.0241606205701828
18.3500000000001 0.0241606429219246
18.3600000000001 0.0241606116294861
18.3700000000001 0.0241606205701828
18.3800000000001 0.0241606429219246
18.3900000000001 0.0241606116294861
18.4000000000001 0.0241606518626213
18.4100000000001 0.0241606190800667
18.4200000000001 0.0241606190800667
18.4300000000001 0.0241606399416924
18.4400000000001 0.0241606190800667
18.4500000000001 0.0241606384515762
18.4600000000001 0.0241606190800667
18.4700000000001 0.0241606384515762
18.4800000000001 0.0241606414318085
18.4900000000001 0.0241606488823891
18.5000000000001 0.0241606637835503
18.5100000000001 0.0241606190800667
18.5200000000001 0.0241606190800667
18.5300000000001 0.0241606190800667
18.5400000000001 0.0241606160998344
18.5500000000001 0.0241606578230858
18.5600000000001 0.0241606548428535
18.5700000000001 0.024160623550415
18.5800000000001 0.0241606414318085
18.5900000000001 0.0241606190800667
18.6000000000001 0.0241606414318085
18.6100000000001 0.0241606190800667
18.6200000000001 0.0241606414318085
18.6300000000001 0.0241606384515762
18.6400000000001 0.0241606920957565
18.6500000000001 0.0241606682538986
18.6600000000001 0.0241606667637825
18.6700000000001 0.0241606622934341
18.6800000000001 0.0241606324911118
18.6900000000001 0.0241606518626213
18.7000000000001 0.0241606444120407
18.7100000000001 0.0241606503725052
18.7200000000001 0.0241606414318085
18.7300000000001 0.024160623550415
18.7400000000001 0.0241606459021568
18.7500000000001 0.0241606682538986
18.7600000000001 0.024160623550415
18.7700000000001 0.0241606444120407
18.7800000000001 0.0241606444120407
18.7900000000001 0.024160623550415
18.8000000000001 0.0241606682538986
18.8100000000001 0.024160623550415
18.8200000000001 0.0241606667637825
18.8300000000001 0.0241606399416924
18.8400000000001 0.0241606459021568
18.8500000000001 0.0241606399416924
18.8600000000001 0.0241606399416924
18.8700000000002 0.0241606399416924
18.8800000000002 0.0241606399416924
18.8900000000002 0.0241606131196022
18.9000000000002 0.024160623550415
18.9100000000002 0.0241606190800667
18.9200000000002 0.0241606444120407
18.9300000000002 0.0241606444120407
18.9400000000002 0.0241606444120407
18.9500000000002 0.0241606459021568
18.9600000000002 0.0241606518626213
18.9700000000002 0.0241606488823891
18.9800000000002 0.0241606459021568
18.9900000000002 0.0241606459021568
19.0000000000002 0.0241606414318085
19.0100000000002 0.0241606414318085
19.0200000000002 0.0241606414318085
19.0300000000002 0.0241606459021568
19.0400000000002 0.0241606414318085
19.0500000000002 0.0241606414318085
19.0600000000002 0.0241606414318085
19.0700000000002 0.0241606518626213
19.0800000000002 0.0241606190800667
19.0900000000002 0.0241606459021568
19.1000000000002 0.0241606414318085
19.1100000000002 0.0241606503725052
19.1200000000002 0.0241606518626213
19.1300000000002 0.0241606444120407
19.1400000000002 0.0241606444120407
19.1500000000002 0.0241606205701828
19.1600000000002 0.0241606578230858
19.1700000000002 0.0241606459021568
19.1800000000002 0.0241606488823891
19.1900000000002 0.0241606488823891
19.2000000000002 0.0241606488823891
19.2100000000002 0.0241606265306473
19.2200000000002 0.0241606265306473
19.2300000000002 0.024160623550415
19.2400000000002 0.0241606250405312
19.2500000000002 0.0241606444120407
19.2600000000002 0.024160647392273
19.2700000000002 0.024160647392273
19.2800000000002 0.0241606459021568
19.2900000000002 0.0241606459021568
19.3000000000002 0.0241606459021568
19.3100000000002 0.0241606384515762
19.3200000000002 0.024160647392273
19.3300000000002 0.0241606459021568
19.3400000000002 0.024160647392273
19.3500000000002 0.0241606339812279
19.3600000000002 0.024160647392273
19.3700000000002 0.024160647392273
19.3800000000002 0.0241606324911118
19.3900000000002 0.0241606503725052
19.4000000000002 0.0241606488823891
19.4100000000002 0.0241606324911118
19.4200000000002 0.0241606488823891
19.4300000000002 0.0241606265306473
19.4400000000002 0.024160647392273
19.4500000000002 0.0241606265306473
19.4600000000002 0.024160647392273
19.4700000000002 0.024160647392273
19.4800000000002 0.0241606429219246
19.4900000000002 0.0241606265306473
19.5000000000002 0.024160623550415
19.5100000000003 0.0241606518626213
19.5200000000003 0.024160623550415
19.5300000000003 0.0241606533527374
19.5400000000003 0.0241606533527374
19.5500000000003 0.024160647392273
19.5600000000003 0.024160647392273
19.5700000000003 0.024160647392273
19.5800000000003 0.0241606280207634
19.5900000000003 0.0241606503725052
19.6000000000003 0.0241606503725052
19.6100000000003 0.0241606280207634
19.6200000000003 0.0241606503725052
19.6300000000003 0.0241606503725052
19.6400000000003 0.0241606280207634
19.6500000000003 0.0241606503725052
19.6600000000003 0.0241606503725052
19.6700000000003 0.0241606280207634
19.6800000000003 0.0241606503725052
19.6900000000003 0.0241606503725052
19.7000000000003 0.0241606280207634
19.7100000000003 0.0241606503725052
19.7200000000003 0.0241606339812279
19.7300000000003 0.0241606339812279
19.7400000000003 0.0241606310009956
19.7500000000003 0.0241605922579765
19.7600000000003 0.0241606146097183
19.7700000000003 0.0241606116294861
19.7800000000003 0.02416061013937
19.7900000000003 0.0241606488823891
19.8000000000003 0.0241606175899506
19.8100000000003 0.024160623550415
19.8200000000003 0.024160623550415
19.8300000000003 0.0241606205701828
19.8400000000003 0.0241606205701828
19.8500000000003 0.0241606190800667
19.8600000000003 0.024160623550415
19.8700000000003 0.0241606190800667
19.8800000000003 0.0241606190800667
19.8900000000003 0.0241606265306473
19.9000000000003 0.0241606190800667
19.9100000000003 0.0241606190800667
19.9200000000003 0.0241606205701828
19.9300000000003 0.0241606205701828
19.9400000000003 0.0241606205701828
19.9500000000003 0.0241606205701828
19.9600000000003 0.0241606548428535
19.9700000000003 0.0241606205701828
19.9800000000003 0.0241606205701828
19.9900000000003 0.0241606444120407
20.0000000000003 0.0241606205701828
20.0100000000003 0.0241606444120407
20.0200000000003 0.0241606444120407
20.0300000000003 0.0241606444120407
20.0400000000003 0.0241606891155243
20.0500000000003 0.0241606652736664
20.0600000000003 0.0241606265306473
20.0700000000003 0.0241606637835503
20.0800000000003 0.0241606324911118
20.0900000000003 0.0241606205701828
20.1000000000003 0.0241606295108795
20.1100000000003 0.0241606518626213
20.1200000000003 0.0241606533527374
20.1300000000003 0.0241606593132019
20.1400000000003 0.0241606488823891
20.1500000000004 0.0241606488823891
20.1600000000004 0.0241606295108795
20.1700000000004 0.0241606682538986
20.1800000000004 0.0241606667637825
20.1900000000004 0.0241606384515762
20.2000000000004 0.024160635471344
20.2100000000004 0.024160660803318
20.2200000000004 0.0241606622934341
20.2300000000004 0.0241606414318085
20.2400000000004 0.0241606712341309
20.2500000000004 0.0241606399416924
20.2600000000004 0.0241606414318085
20.2700000000004 0.0241606414318085
20.2800000000004 0.0241606339812279
20.2900000000004 0.0241606280207634
20.3000000000004 0.0241606339812279
20.3100000000004 0.024160660803318
20.3200000000004 0.0241606429219246
20.3300000000004 0.0241606384515762
20.3400000000004 0.0241606399416924
20.3500000000004 0.0241606399416924
20.3600000000004 0.0241606399416924
20.3700000000004 0.0241606637835503
20.3800000000004 0.0241606444120407
20.3900000000004 0.0241606295108795
20.4000000000004 0.0241606384515762
20.4100000000004 0.0241606175899506
20.4200000000004 0.0241606190800667
20.4300000000004 0.0241606444120407
20.4400000000004 0.0241606324911118
20.4500000000004 0.0241606622934341
20.4600000000004 0.024160660803318
20.4700000000004 0.0241606175899506
20.4800000000004 0.024160660803318
20.4900000000004 0.0241606175899506
20.5000000000004 0.0241606175899506
20.5100000000004 0.02416061013937
20.5200000000004 0.0241606459021568
20.5300000000004 0.0241606518626213
20.5400000000004 0.0241606548428535
20.5500000000004 0.0241606265306473
20.5600000000004 0.0241606563329697
20.5700000000004 0.024160623550415
20.5800000000004 0.024160623550415
20.5900000000004 0.0241606444120407
20.6000000000004 0.0241606503725052
20.6100000000004 0.0241606265306473
20.6200000000004 0.0241606637835503
20.6300000000004 0.0241606280207634
20.6400000000004 0.0241606190800667
20.6500000000004 0.0241606444120407
20.6600000000004 0.0241606503725052
20.6700000000004 0.024160623550415
20.6800000000004 0.0241606384515762
20.6900000000004 0.0241606533527374
20.7000000000004 0.0241606503725052
20.7100000000004 0.0241606205701828
20.7200000000004 0.0241606503725052
20.7300000000004 0.0241606205701828
20.7400000000004 0.0241606444120407
20.7500000000004 0.0241606622934341
20.7600000000004 0.0241606205701828
20.7700000000004 0.0241606220602989
20.7800000000004 0.0241606503725052
20.7900000000005 0.0241606682538986
20.8000000000005 0.0241606339812279
20.8100000000005 0.0241606444120407
20.8200000000005 0.0241606339812279
20.8300000000005 0.0241606444120407
20.8400000000005 0.024160647392273
20.8500000000005 0.0241606593132019
20.8600000000005 0.0241606697440147
20.8700000000005 0.0241606324911118
20.8800000000005 0.0241606265306473
20.8900000000005 0.0241606459021568
20.9000000000005 0.0241606712341309
20.9100000000005 0.0241606324911118
20.9200000000005 0.0241606488823891
20.9300000000005 0.0241606488823891
20.9400000000005 0.0241606488823891
20.9500000000005 0.0241606384515762
20.9600000000005 0.0241606265306473
20.9700000000005 0.0241606265306473
20.9800000000005 0.024160623550415
20.9900000000005 0.0241606459021568
21.0000000000005 0.0241606459021568
21.0100000000005 0.024160635471344
21.0200000000005 0.024160647392273
21.0300000000005 0.024160647392273
21.0400000000005 0.024160647392273
21.0500000000005 0.024160647392273
21.0600000000005 0.024160647392273
21.0700000000005 0.0241606503725052
21.0800000000005 0.0241606265306473
21.0900000000005 0.0241606280207634
21.1000000000005 0.0241606250405312
21.1100000000005 0.0241606265306473
21.1200000000005 0.0241606578230858
21.1300000000005 0.0241605967283249
21.1400000000005 0.0241606459021568
21.1500000000005 0.0241606518626213
21.1600000000005 0.0241606488823891
21.1700000000005 0.024160635471344
21.1800000000005 0.024160647392273
21.1900000000005 0.0241606459021568
21.2000000000005 0.0241606459021568
21.2100000000005 0.0241606459021568
21.2200000000005 0.0241606459021568
21.2300000000005 0.0241606459021568
21.2400000000005 0.0241606459021568
21.2500000000005 0.0241606459021568
21.2600000000005 0.0241606488823891
21.2700000000005 0.0241606488823891
21.2800000000005 0.0241606488823891
21.2900000000005 0.0241606265306473
21.3000000000005 0.0241606459021568
21.3100000000005 0.0241606265306473
21.3200000000005 0.0241606265306473
21.3300000000005 0.024160623550415
21.3400000000005 0.024160647392273
21.3500000000005 0.024160647392273
21.3600000000005 0.0241606533527374
21.3700000000005 0.024160647392273
21.3800000000005 0.024160647392273
21.3900000000005 0.024160647392273
21.4000000000005 0.0241606503725052
21.4100000000005 0.0241606503725052
21.4200000000005 0.0241606488823891
21.4300000000006 0.0241606295108795
21.4400000000006 0.0241606488823891
21.4500000000006 0.0241606488823891
21.4600000000006 0.0241606488823891
21.4700000000006 0.0241606295108795
21.4800000000006 0.0241606265306473
21.4900000000006 0.0241606488823891
21.5000000000006 0.0241606488823891
21.5100000000006 0.0241606324911118
21.5200000000006 0.0241606324911118
21.5300000000006 0.0241606310009956
21.5400000000006 0.0241606310009956
21.5500000000006 0.0241606369614601
21.5600000000006 0.0241606369614601
21.5700000000006 0.0241606429219246
21.5800000000006 0.0241606369614601
21.5900000000006 0.0241606369614601
21.6000000000006 0.0241606369614601
21.6100000000006 0.0241605937480927
21.6200000000006 0.0241606071591377
21.6300000000006 0.0241606160998344
21.6400000000006 0.0241606220602989
21.6500000000006 0.0241606160998344
21.6600000000006 0.0241606205701828
21.6700000000006 0.0241606190800667
21.6800000000006 0.0241606190800667
21.6900000000006 0.0241606459021568
21.7000000000006 0.0241606190800667
21.7100000000006 0.0241606459021568
21.7200000000006 0.0241606175899506
21.7300000000006 0.0241606175899506
21.7400000000006 0.024160623550415
21.7500000000006 0.0241606116294861
21.7600000000006 0.0241606488823891
21.7700000000006 0.024160623550415
21.7800000000006 0.0241606324911118
21.7900000000006 0.0241606652736664
21.8000000000006 0.024160660803318
21.8100000000006 0.0241606637835503
21.8200000000006 0.0241606310009956
21.8300000000006 0.0241606295108795
21.8400000000006 0.0241606518626213
21.8500000000006 0.0241606295108795
21.8600000000006 0.0241606190800667
21.8700000000006 0.024160647392273
21.8800000000006 0.0241606637835503
21.8900000000006 0.0241606593132019
21.9000000000006 0.0241606205701828
21.9100000000006 0.0241606160998344
21.9200000000006 0.0241606622934341
21.9300000000006 0.0241606593132019
21.9400000000006 0.0241606488823891
21.9500000000006 0.0241606488823891
21.9600000000006 0.0241606667637825
21.9700000000006 0.0241606488823891
21.9800000000006 0.0241606444120407
21.9900000000006 0.0241606444120407
22.0000000000006 0.0241606488823891
22.0100000000006 0.0241606488823891
22.0200000000006 0.0241606488823891
22.0300000000006 0.0241606548428535
22.0400000000006 0.0241606548428535
22.0500000000006 0.0241606459021568
22.0600000000006 0.0241606414318085
22.0700000000007 0.0241606384515762
22.0800000000007 0.0241606414318085
22.0900000000007 0.0241606414318085
22.1000000000007 0.0241606414318085
22.1100000000007 0.0241606384515762
22.1200000000007 0.0241606399416924
22.1300000000007 0.0241606399416924
22.1400000000007 0.024160672724247
22.1500000000007 0.0241606414318085
22.1600000000007 0.0241606414318085
22.1700000000007 0.0241606414318085
22.1800000000007 0.0241606310009956
22.1900000000007 0.0241606414318085
22.2000000000007 0.0241606414318085
22.2100000000007 0.0241606414318085
22.2200000000007 0.0241606414318085
22.2300000000007 0.0241606414318085
22.2400000000007 0.0241606414318085
22.2500000000007 0.0241606414318085
22.2600000000007 0.0241606414318085
22.2700000000007 0.0241606414318085
22.2800000000007 0.0241606414318085
22.2900000000007 0.024160635471344
22.3000000000007 0.0241606682538986
22.3100000000007 0.0241606205701828
22.3200000000007 0.0241606190800667
22.3300000000007 0.0241606190800667
22.3400000000007 0.024160623550415
22.3500000000007 0.0241606563329697
22.3600000000007 0.0241606310009956
22.3700000000007 0.0241606414318085
22.3800000000007 0.0241606414318085
22.3900000000007 0.024160635471344
22.4000000000007 0.024160635471344
22.4100000000007 0.024160635471344
22.4200000000007 0.0241606444120407
22.4300000000007 0.024160623550415
22.4400000000007 0.0241606220602989
22.4500000000007 0.0241606444120407
22.4600000000007 0.0241606280207634
22.4700000000007 0.0241606444120407
22.4800000000007 0.0241606280207634
22.4900000000007 0.0241606444120407
22.5000000000007 0.0241606220602989
22.5100000000007 0.0241606160998344
22.5200000000007 0.0241606160998344
22.5300000000007 0.0241606190800667
22.5400000000007 0.0241606205701828
22.5500000000007 0.024160623550415
22.5600000000007 0.0241606488823891
22.5700000000007 0.0241606190800667
22.5800000000007 0.0241606414318085
22.5900000000007 0.0241606414318085
22.6000000000007 0.0241606429219246
22.6100000000007 0.0241606414318085
22.6200000000007 0.0241606190800667
22.6300000000007 0.0241606190800667
22.6400000000007 0.0241606414318085
22.6500000000007 0.0241606414318085
22.6600000000007 0.0241606414318085
22.6700000000007 0.0241606190800667
22.6800000000007 0.0241606414318085
22.6900000000007 0.0241606414318085
22.7000000000007 0.0241606190800667
22.7100000000008 0.0241606429219246
22.7200000000008 0.0241606414318085
22.7300000000008 0.0241606637835503
22.7400000000008 0.0241606205701828
22.7500000000008 0.0241606637835503
22.7600000000008 0.0241606652736664
22.7700000000008 0.0241606622934341
22.7800000000008 0.0241606563329697
22.7900000000008 0.0241606295108795
22.8000000000008 0.0241606459021568
22.8100000000008 0.024160623550415
22.8200000000008 0.0241606667637825
22.8300000000008 0.0241606265306473
22.8400000000008 0.0241606637835503
22.8500000000008 0.0241606295108795
22.8600000000008 0.0241606384515762
22.8700000000008 0.024160623550415
22.8800000000008 0.0241606205701828
22.8900000000008 0.0241606205701828
22.9000000000008 0.0241606533527374
22.9100000000008 0.0241606503725052
22.9200000000008 0.0241606205701828
22.9300000000008 0.0241606444120407
22.9400000000008 0.0241606563329697
22.9500000000008 0.0241606205701828
22.9600000000008 0.0241606444120407
22.9700000000008 0.0241606503725052
22.9800000000008 0.0241606265306473
22.9900000000008 0.0241606444120407
23.0000000000008 0.0241606444120407
23.0100000000008 0.0241606488823891
23.0200000000008 0.0241606444120407
23.0300000000008 0.0241606265306473
23.0400000000008 0.0241606205701828
23.0500000000008 0.0241606503725052
23.0600000000008 0.0241606503725052
23.0700000000008 0.0241606444120407
23.0800000000008 0.0241606444120407
23.0900000000008 0.0241606488823891
23.1000000000008 0.024160623550415
23.1100000000008 0.0241606190800667
23.1200000000008 0.0241606503725052
23.1300000000008 0.0241606503725052
23.1400000000008 0.0241606205701828
23.1500000000008 0.0241606444120407
23.1600000000008 0.0241606503725052
23.1700000000008 0.0241606444120407
23.1800000000008 0.0241606488823891
23.1900000000008 0.024160623550415
23.2000000000008 0.0241606190800667
23.2100000000008 0.0241606503725052
23.2200000000008 0.0241606444120407
23.2300000000008 0.0241606503725052
23.2400000000008 0.0241606444120407
23.2500000000008 0.0241606488823891
23.2600000000008 0.024160623550415
23.2700000000008 0.0241606190800667
23.2800000000008 0.0241606503725052
23.2900000000008 0.0241606444120407
23.3000000000008 0.0241606444120407
23.3100000000008 0.0241606444120407
23.3200000000008 0.0241606444120407
23.3300000000008 0.0241606563329697
23.3400000000008 0.0241606459021568
23.3500000000009 0.0241606488823891
23.3600000000009 0.0241606712341309
23.3700000000009 0.0241606265306473
23.3800000000009 0.0241606488823891
23.3900000000009 0.0241606488823891
23.4000000000009 0.0241606488823891
23.4100000000009 0.0241606384515762
23.4200000000009 0.0241606265306473
23.4300000000009 0.0241606459021568
23.4400000000009 0.0241606369614601
23.4500000000009 0.0241606459021568
23.4600000000009 0.0241606459021568
23.4700000000009 0.0241606369614601
23.4800000000009 0.0241606459021568
23.4900000000009 0.0241606459021568
23.5000000000009 0.0241606488823891
23.5100000000009 0.0241606160998344
23.5200000000009 0.0241606250405312
23.5300000000009 0.0241606250405312
23.5400000000009 0.0241606459021568
23.5500000000009 0.0241606459021568
23.5600000000009 0.024160635471344
23.5700000000009 0.0241606488823891
23.5800000000009 0.0241606488823891
23.5900000000009 0.0241606488823891
23.6000000000009 0.024160647392273
23.6100000000009 0.024160647392273
23.6200000000009 0.024160647392273
23.6300000000009 0.0241606503725052
23.6400000000009 0.0241606503725052
23.6500000000009 0.0241606324911118
23.6600000000009 0.0241606324911118
23.6700000000009 0.0241606280207634
23.6800000000009 0.0241606265306473
23.6900000000009 0.0241606324911118
23.7000000000009 0.0241606488823891
23.7100000000009 0.0241606563329697
23.7200000000009 0.0241606459021568
23.7300000000009 0.0241606459021568
23.7400000000009 0.0241606459021568
23.7500000000009 0.0241606503725052
23.7600000000009 0.0241606265306473
23.7700000000009 0.0241606503725052
23.7800000000009 0.0241606503725052
23.7900000000009 0.0241606265306473
23.8000000000009 0.0241606503725052
23.8100000000009 0.0241606503725052
23.8200000000009 0.0241606265306473
23.8300000000009 0.0241606503725052
23.8400000000009 0.0241606265306473
23.8500000000009 0.024160647392273
23.8600000000009 0.0241606503725052
23.8700000000009 0.0241606503725052
23.8800000000009 0.0241606488823891
23.8900000000009 0.0241606488823891
23.9000000000009 0.0241606265306473
23.9100000000009 0.0241606488823891
23.9200000000009 0.0241606265306473
23.9300000000009 0.0241606488823891
23.9400000000009 0.0241606324911118
23.9500000000009 0.0241606310009956
23.9600000000009 0.0241606310009956
23.9700000000009 0.0241606369614601
23.9800000000009 0.0241606369614601
23.990000000001 0.0241606429219246
24.000000000001 0.0241606369614601
24.010000000001 0.0241606369614601
24.020000000001 0.0241606369614601
24.030000000001 0.0241605937480927
24.040000000001 0.0241605907678604
24.050000000001 0.0241606160998344
24.060000000001 0.0241606205701828
24.070000000001 0.0241606205701828
24.080000000001 0.0241606190800667
24.090000000001 0.0241606190800667
24.100000000001 0.0241606518626213
24.110000000001 0.0241606190800667
24.120000000001 0.0241606175899506
24.130000000001 0.0241606175899506
24.140000000001 0.0241606205701828
24.150000000001 0.0241606190800667
24.160000000001 0.0241606205701828
24.170000000001 0.0241606160998344
24.180000000001 0.0241606414318085
24.190000000001 0.0241606205701828
24.200000000001 0.0241606160998344
24.210000000001 0.0241606518626213
24.220000000001 0.0241606414318085
24.230000000001 0.0241606414318085
24.240000000001 0.0241606518626213
24.250000000001 0.0241606637835503
24.260000000001 0.0241606190800667
24.270000000001 0.0241606190800667
24.280000000001 0.0241606429219246
24.290000000001 0.0241606429219246
24.300000000001 0.0241606459021568
24.310000000001 0.0241606637835503
24.320000000001 0.0241606593132019
24.330000000001 0.0241606265306473
24.340000000001 0.0241606295108795
24.350000000001 0.0241606533527374
24.360000000001 0.0241606593132019
24.370000000001 0.0241606265306473
24.380000000001 0.0241606518626213
24.390000000001 0.0241606265306473
24.400000000001 0.0241606518626213
24.410000000001 0.0241606265306473
24.420000000001 0.0241606518626213
24.430000000001 0.0241606518626213
24.440000000001 0.0241606622934341
24.450000000001 0.0241606771945953
24.460000000001 0.0241606503725052
24.470000000001 0.0241606459021568
24.480000000001 0.0241606444120407
24.490000000001 0.0241606459021568
24.500000000001 0.0241606414318085
24.510000000001 0.0241606280207634
24.520000000001 0.0241606324911118
24.530000000001 0.024160660803318
24.540000000001 0.0241606384515762
24.550000000001 0.0241606548428535
24.560000000001 0.0241606384515762
24.570000000001 0.0241606384515762
24.580000000001 0.0241606384515762
24.590000000001 0.0241606384515762
24.600000000001 0.0241606175899506
24.610000000001 0.0241606160998344
24.620000000001 0.024160660803318
24.6300000000011 0.0241606160998344
24.6400000000011 0.024160660803318
24.6500000000011 0.0241606384515762
24.6600000000011 0.0241606190800667
24.6700000000011 0.0241606160998344
24.6800000000011 0.024160660803318
24.6900000000011 0.024160623550415
24.7000000000011 0.0241606160998344
24.7100000000011 0.0241606637835503
24.7200000000011 0.0241606533527374
24.7300000000011 0.0241606324911118
24.7400000000011 0.024160660803318
24.7500000000011 0.0241606205701828
24.7600000000011 0.0241606205701828
24.7700000000011 0.024160647392273
24.7800000000011 0.0241606488823891
24.7900000000011 0.0241606503725052
24.8000000000011 0.0241606265306473
24.8100000000011 0.0241606667637825
24.8200000000011 0.0241606265306473
24.8300000000011 0.0241606667637825
24.8400000000011 0.024160623550415
24.8500000000011 0.0241606682538986
24.8600000000011 0.0241606086492538
24.8700000000011 0.024160623550415
24.8800000000011 0.0241606190800667
24.8900000000011 0.0241606444120407
24.9000000000011 0.0241606503725052
24.9100000000011 0.0241606280207634
24.9200000000011 0.0241606563329697
24.9300000000011 0.0241606280207634
24.9400000000011 0.0241606444120407
24.9500000000011 0.0241606503725052
24.9600000000011 0.0241606280207634
24.9700000000011 0.0241606444120407
24.9800000000011 0.0241606444120407
24.9900000000011 0.0241606444120407
25.0000000000011 0.0241606444120407
25.0100000000011 0.0241606339812279
25.0200000000011 0.0241606444120407
25.0300000000011 0.024160647392273
25.0400000000011 0.024160647392273
25.0500000000011 0.0241606459021568
25.0600000000011 0.0241606459021568
25.0700000000011 0.0241606384515762
25.0800000000011 0.0241606459021568
25.0900000000011 0.0241606459021568
25.1000000000011 0.0241606488823891
25.1100000000011 0.0241606369614601
25.1200000000011 0.0241606459021568
25.1300000000011 0.0241606488823891
25.1400000000011 0.0241606369614601
25.1500000000011 0.0241606488823891
25.1600000000011 0.0241606488823891
25.1700000000011 0.0241606369614601
25.1800000000011 0.0241606384515762
25.1900000000011 0.024160647392273
25.2000000000011 0.0241606488823891
25.2100000000011 0.0241606384515762
25.2200000000011 0.0241606488823891
25.2300000000011 0.0241606265306473
25.2400000000011 0.0241606280207634
25.2500000000011 0.0241606250405312
25.2600000000011 0.0241606250405312
25.2700000000012 0.0241606459021568
25.2800000000012 0.024160647392273
25.2900000000012 0.0241606369614601
25.3000000000012 0.0241606488823891
25.3100000000012 0.0241606488823891
25.3200000000012 0.0241606488823891
25.3300000000012 0.024160647392273
25.3400000000012 0.024160647392273
25.3500000000012 0.0241606503725052
25.3600000000012 0.0241606503725052
25.3700000000012 0.0241606503725052
25.3800000000012 0.0241606324911118
25.3900000000012 0.0241606324911118
25.4000000000012 0.0241606295108795
25.4100000000012 0.0241606265306473
25.4200000000012 0.0241606444120407
25.4300000000012 0.0241606250405312
25.4400000000012 0.0241606250405312
25.4500000000012 0.0241606459021568
25.4600000000012 0.024160647392273
25.4700000000012 0.024160647392273
25.4800000000012 0.0241606563329697
25.4900000000012 0.0241606459021568
25.5000000000012 0.0241606459021568
25.5100000000012 0.024160647392273
25.5200000000012 0.0241606503725052
25.5300000000012 0.0241606503725052
25.5400000000012 0.0241606503725052
25.5500000000012 0.0241606533527374
25.5600000000012 0.0241606295108795
25.5700000000012 0.0241606265306473
25.5800000000012 0.0241606488823891
25.5900000000012 0.0241606488823891
25.6000000000012 0.0241606265306473
25.6100000000012 0.0241606488823891
25.6200000000012 0.0241606324911118
25.6300000000012 0.0241606324911118
25.6400000000012 0.0241606310009956
25.6500000000012 0.0241606369614601
25.6600000000012 0.0241606369614601
25.6700000000012 0.0241606369614601
25.6800000000012 0.0241606429219246
25.6900000000012 0.0241606369614601
25.7000000000012 0.0241606369614601
25.7100000000012 0.0241605937480927
25.7200000000012 0.0241606071591377
25.7300000000012 0.0241606160998344
25.7400000000012 0.0241606220602989
25.7500000000012 0.0241606160998344
25.7600000000012 0.0241606190800667
25.7700000000012 0.0241606175899506
25.7800000000012 0.0241606220602989
25.7900000000012 0.0241606116294861
25.8000000000012 0.0241606459021568
25.8100000000012 0.0241606175899506
25.8200000000012 0.024160623550415
25.8300000000012 0.0241606116294861
25.8400000000012 0.0241606444120407
25.8500000000012 0.024160623550415
25.8600000000012 0.0241606160998344
25.8700000000012 0.0241606265306473
25.8800000000012 0.0241606205701828
25.8900000000012 0.0241606384515762
25.9000000000012 0.0241606548428535
25.9100000000013 0.0241606414318085
25.9200000000013 0.0241606399416924
25.9300000000013 0.0241606399416924
25.9400000000013 0.0241606503725052
25.9500000000013 0.0241606637835503
25.9600000000013 0.024160660803318
25.9700000000013 0.024160623550415
25.9800000000013 0.024160647392273
25.9900000000013 0.0241606146097183
26.0000000000013 0.0241606563329697
26.0100000000013 0.0241606533527374
26.0200000000013 0.0241606593132019
26.0300000000013 0.0241606265306473
26.0400000000013 0.0241606518626213
26.0500000000013 0.0241606265306473
26.0600000000013 0.0241606518626213
26.0700000000013 0.0241606265306473
26.0800000000013 0.0241606667637825
26.0900000000013 0.0241606488823891
26.1000000000013 0.0241606667637825
26.1100000000013 0.0241606667637825
26.1200000000013 0.0241606548428535
26.1300000000013 0.0241606384515762
26.1400000000013 0.0241606459021568
26.1500000000013 0.0241606459021568
26.1600000000013 0.0241606459021568
26.1700000000013 0.0241606459021568
26.1800000000013 0.0241606459021568
26.1900000000013 0.0241606459021568
26.2000000000013 0.0241606414318085
26.2100000000013 0.0241606414318085
26.2200000000013 0.0241606295108795
26.2300000000013 0.024160635471344
26.2400000000013 0.0241606190800667
26.2500000000013 0.0241606459021568
26.2600000000013 0.0241606459021568
26.2700000000013 0.0241606190800667
26.2800000000013 0.0241606175899506
26.2900000000013 0.024160660803318
26.3000000000013 0.0241606399416924
26.3100000000013 0.0241606622934341
26.3200000000013 0.0241606190800667
26.3300000000013 0.0241606190800667
26.3400000000013 0.024160660803318
26.3500000000013 0.0241606190800667
26.3600000000013 0.0241606160998344
26.3700000000013 0.024160660803318
26.3800000000013 0.024160623550415
26.3900000000013 0.0241606444120407
26.4000000000013 0.0241606563329697
26.4100000000013 0.024160623550415
26.4200000000013 0.0241606682538986
26.4300000000013 0.024160623550415
26.4400000000013 0.0241606682538986
26.4500000000013 0.0241606459021568
26.4600000000013 0.0241606459021568
26.4700000000013 0.0241606444120407
26.4800000000013 0.0241606429219246
26.4900000000013 0.0241606429219246
26.5000000000013 0.0241606429219246
26.5100000000013 0.0241606086492538
26.5200000000013 0.0241606459021568
26.5300000000013 0.0241606414318085
26.5400000000013 0.0241606459021568
26.5500000000014 0.0241606414318085
26.5600000000014 0.0241606459021568
26.5700000000014 0.024160635471344
26.5800000000014 0.0241606459021568
26.5900000000014 0.0241606414318085
26.6000000000014 0.0241606459021568
26.6100000000014 0.0241606459021568
26.6200000000014 0.0241606459021568
26.6300000000014 0.0241606459021568
26.6400000000014 0.0241606488823891
26.6500000000014 0.0241606488823891
26.6600000000014 0.0241606488823891
26.6700000000014 0.0241606488823891
26.6800000000014 0.0241606205701828
26.6900000000014 0.0241606459021568
26.7000000000014 0.0241606459021568
26.7100000000014 0.0241606205701828
26.7200000000014 0.024160623550415
26.7300000000014 0.0241606459021568
26.7400000000014 0.0241606459021568
26.7500000000014 0.0241606459021568
26.7600000000014 0.0241606459021568
26.7700000000014 0.0241606459021568
26.7800000000014 0.0241606459021568
26.7900000000014 0.0241606384515762
26.8000000000014 0.0241606488823891
26.8100000000014 0.024160647392273
26.8200000000014 0.0241606265306473
26.8300000000014 0.0241606280207634
26.8400000000014 0.0241606265306473
26.8500000000014 0.0241606548428535
26.8600000000014 0.0241605967283249
26.8700000000014 0.0241606369614601
26.8800000000014 0.0241606131196022
26.8900000000014 0.0241606220602989
26.9000000000014 0.02416061013937
26.9100000000014 0.0241606205701828
26.9200000000014 0.0241606190800667
26.9300000000014 0.0241606205701828
26.9400000000014 0.0241606190800667
26.9500000000014 0.0241606190800667
26.9600000000014 0.0241606190800667
26.9700000000014 0.0241606160998344
26.9800000000014 0.0241606160998344
26.9900000000014 0.0241606444120407
27.0000000000014 0.0241606622934341
27.0100000000014 0.0241606205701828
27.0200000000014 0.0241606175899506
27.0300000000014 0.0241606175899506
27.0400000000014 0.0241606384515762
27.0500000000014 0.0241606459021568
27.0600000000014 0.0241606414318085
27.0700000000014 0.0241606414318085
27.0800000000014 0.0241606429219246
27.0900000000014 0.0241606429219246
27.1000000000014 0.0241606190800667
27.1100000000014 0.0241606414318085
27.1200000000014 0.0241606414318085
27.1300000000014 0.0241606190800667
27.1400000000014 0.0241606414318085
27.1500000000014 0.0241606190800667
27.1600000000014 0.0241606444120407
27.1700000000014 0.0241606414318085
27.1800000000014 0.024160623550415
27.1900000000015 0.0241606414318085
27.2000000000015 0.024160623550415
27.2100000000015 0.0241606831550598
27.2200000000015 0.0241606920957565
27.2300000000015 0.0241606548428535
27.2400000000015 0.0241606265306473
27.2500000000015 0.0241606205701828
27.2600000000015 0.0241606459021568
27.2700000000015 0.0241606459021568
27.2800000000015 0.0241606518626213
27.2900000000015 0.0241606384515762
27.3000000000015 0.0241606593132019
27.3100000000015 0.0241606667637825
27.3200000000015 0.0241606220602989
27.3300000000015 0.0241606667637825
27.3400000000015 0.0241606220602989
27.3500000000015 0.024160660803318
27.3600000000015 0.0241606399416924
27.3700000000015 0.0241606399416924
27.3800000000015 0.0241606667637825
27.3900000000015 0.024160623550415
27.4000000000015 0.0241606622934341
27.4100000000015 0.0241606399416924
27.4200000000015 0.0241606220602989
27.4300000000015 0.0241606175899506
27.4400000000015 0.024160660803318
27.4500000000015 0.0241606563329697
27.4600000000015 0.024160623550415
27.4700000000015 0.0241606548428535
27.4800000000015 0.0241606175899506
27.4900000000015 0.0241606622934341
27.5000000000015 0.0241606175899506
27.5100000000015 0.0241606682538986
27.5200000000015 0.0241606175899506
27.5300000000015 0.024160623550415
27.5400000000015 0.0241606682538986
27.5500000000015 0.0241606190800667
27.5600000000015 0.0241606265306473
27.5700000000015 0.0241606503725052
27.5800000000015 0.0241606459021568
27.5900000000015 0.0241606190800667
27.6000000000015 0.0241606444120407
27.6100000000015 0.0241606459021568
27.6200000000015 0.0241606190800667
27.6300000000015 0.0241606444120407
27.6400000000015 0.0241606459021568
27.6500000000015 0.0241606414318085
27.6600000000015 0.0241606459021568
27.6700000000015 0.0241606637835503
27.6800000000015 0.0241606459021568
27.6900000000015 0.0241606459021568
27.7000000000015 0.0241606667637825
27.7100000000015 0.0241606444120407
27.7200000000015 0.0241606324911118
27.7300000000015 0.0241606488823891
27.7400000000015 0.0241606667637825
27.7500000000015 0.0241606444120407
27.7600000000015 0.024160598218441
27.7700000000015 0.0241606459021568
27.7800000000015 0.0241606459021568
27.7900000000015 0.0241606384515762
27.8000000000015 0.024160647392273
27.8100000000015 0.024160647392273
27.8200000000015 0.0241606384515762
27.8300000000016 0.0241606280207634
27.8400000000016 0.0241606265306473
27.8500000000016 0.0241606548428535
27.8600000000016 0.0241606488823891
27.8700000000016 0.024160647392273
27.8800000000016 0.024160647392273
27.8900000000016 0.0241606280207634
27.9000000000016 0.0241606280207634
27.9100000000016 0.0241606250405312
27.9200000000016 0.0241606488823891
27.9300000000016 0.0241606488823891
27.9400000000016 0.0241606488823891
27.9500000000016 0.024160647392273
27.9600000000016 0.0241606488823891
27.9700000000016 0.0241606488823891
27.9800000000016 0.0241606503725052
27.9900000000016 0.0241606324911118
28.0000000000016 0.0241606295108795
28.0100000000016 0.0241606295108795
28.0200000000016 0.0241606265306473
28.0300000000016 0.0241606712341309
28.0400000000016 0.0241606503725052
28.0500000000016 0.0241606548428535
28.0600000000016 0.0241606265306473
28.0700000000016 0.0241606503725052
28.0800000000016 0.0241606265306473
28.0900000000016 0.0241606488823891
28.1000000000016 0.0241606503725052
28.1100000000016 0.0241606265306473
28.1200000000016 0.0241606488823891
28.1300000000016 0.0241606503725052
28.1400000000016 0.0241606265306473
28.1500000000016 0.0241606459021568
28.1600000000016 0.0241606280207634
28.1700000000016 0.024160647392273
28.1800000000016 0.0241606548428535
28.1900000000016 0.0241606563329697
28.2000000000016 0.0241606563329697
28.2100000000016 0.0241606310009956
28.2200000000016 0.0241606488823891
28.2300000000016 0.0241606488823891
28.2400000000016 0.0241606295108795
28.2500000000016 0.0241606265306473
28.2600000000016 0.0241606488823891
28.2700000000016 0.0241606503725052
28.2800000000016 0.0241606503725052
28.2900000000016 0.0241606280207634
28.3000000000016 0.0241606503725052
28.3100000000016 0.0241606503725052
28.3200000000016 0.0241606503725052
28.3300000000016 0.0241606503725052
28.3400000000016 0.0241606295108795
28.3500000000016 0.0241606265306473
28.3600000000016 0.0241606503725052
28.3700000000016 0.0241606503725052
28.3800000000016 0.0241606503725052
28.3900000000016 0.0241606503725052
28.4000000000016 0.0241606280207634
28.4100000000016 0.0241606444120407
28.4200000000016 0.0241606339812279
28.4300000000016 0.0241606339812279
28.4400000000016 0.0241606310009956
28.4500000000016 0.0241605922579765
28.4600000000016 0.0241606146097183
28.4700000000017 0.0241606563329697
28.4800000000017 0.0241606399416924
28.4900000000017 0.0241606399416924
28.5000000000017 0.024160635471344
28.5100000000017 0.0241606593132019
28.5200000000017 0.0241606369614601
28.5300000000017 0.024160635471344
28.5400000000017 0.0241606295108795
28.5500000000017 0.0241606369614601
28.5600000000017 0.0241606399416924
28.5700000000017 0.0241606399416924
28.5800000000017 0.0241606429219246
28.5900000000017 0.0241606310009956
28.6000000000017 0.0241606399416924
28.6100000000017 0.0241606429219246
28.6200000000017 0.0241606429219246
28.6300000000017 0.0241606295108795
28.6400000000017 0.0241606295108795
28.6500000000017 0.024160623550415
28.6600000000017 0.0241606339812279
28.6700000000017 0.0241606205701828
28.6800000000017 0.0241606310009956
28.6900000000017 0.0241606220602989
28.7000000000017 0.0241606310009956
28.7100000000017 0.0241606310009956
28.7200000000017 0.0241606190800667
28.7300000000017 0.0241606310009956
28.7400000000017 0.0241606310009956
28.7500000000017 0.0241606190800667
28.7600000000017 0.0241606310009956
28.7700000000017 0.0241606518626213
28.7800000000017 0.024160660803318
28.7900000000017 0.0241606324911118
28.8000000000017 0.0241606310009956
28.8100000000017 0.0241606310009956
28.8200000000017 0.0241606310009956
28.8300000000017 0.0241606280207634
28.8400000000017 0.0241606637835503
28.8500000000017 0.0241606310009956
28.8600000000017 0.0241606295108795
28.8700000000017 0.0241606310009956
28.8800000000017 0.0241606310009956
28.8900000000017 0.0241606310009956
28.9000000000017 0.0241606295108795
28.9100000000017 0.0241606190800667
28.9200000000017 0.0241606518626213
28.9300000000017 0.0241606637835503
28.9400000000017 0.0241606324911118
28.9500000000017 0.0241606742143631
28.9600000000017 0.0241606265306473
28.9700000000017 0.0241606324911118
28.9800000000017 0.0241606771945953
28.9900000000017 0.0241606414318085
29.0000000000017 0.0241606324911118
29.0100000000017 0.0241606265306473
29.0200000000017 0.0241606324911118
29.0300000000017 0.024160660803318
29.0400000000017 0.0241606265306473
29.0500000000017 0.0241606324911118
29.0600000000017 0.0241606742143631
29.0700000000017 0.024160635471344
29.0800000000017 0.0241606265306473
29.0900000000017 0.0241606265306473
29.1000000000017 0.0241606771945953
29.1100000000018 0.0241606324911118
29.1200000000018 0.0241606265306473
29.1300000000018 0.0241606637835503
29.1400000000018 0.0241606265306473
29.1500000000018 0.0241606265306473
29.1600000000018 0.024160660803318
29.1700000000018 0.0241606190800667
29.1800000000018 0.0241606548428535
29.1900000000018 0.024160660803318
29.2000000000018 0.0241606622934341
29.2100000000018 0.0241606190800667
29.2200000000018 0.0241606175899506
29.2300000000018 0.0241606444120407
29.2400000000018 0.024160660803318
29.2500000000018 0.0241606459021568
29.2600000000018 0.0241606414318085
29.2700000000018 0.0241606622934341
29.2800000000018 0.0241606429219246
29.2900000000018 0.0241606369614601
29.3000000000018 0.0241606444120407
29.3100000000018 0.024160635471344
29.3200000000018 0.0241606637835503
29.3300000000018 0.0241606220602989
29.3400000000018 0.0241606548428535
29.3500000000018 0.0241606622934341
29.3600000000018 0.0241606429219246
29.3700000000018 0.0241606369614601
29.3800000000018 0.0241606444120407
29.3900000000018 0.024160635471344
29.4000000000018 0.0241606429219246
29.4100000000018 0.0241606369614601
29.4200000000018 0.0241606175899506
29.4300000000018 0.024160635471344
29.4400000000018 0.0241606518626213
29.4500000000018 0.0241606459021568
29.4600000000018 0.0241606205701828
29.4700000000018 0.0241606518626213
29.4800000000018 0.0241606414318085
29.4900000000018 0.0241606518626213
29.5000000000018 0.024160623550415
29.5100000000018 0.0241606518626213
29.5200000000018 0.0241606414318085
29.5300000000018 0.0241606518626213
29.5400000000018 0.024160623550415
29.5500000000018 0.0241606518626213
29.5600000000018 0.0241606190800667
29.5700000000018 0.0241606518626213
29.5800000000018 0.0241606518626213
29.5900000000018 0.0241606190800667
29.6000000000018 0.0241606518626213
29.6100000000018 0.0241606414318085
29.6200000000018 0.0241606578230858
29.6300000000018 0.0241606459021568
29.6400000000018 0.024160623550415
29.6500000000018 0.0241606414318085
29.6600000000018 0.0241606459021568
29.6700000000018 0.0241606190800667
29.6800000000018 0.024160647392273
29.6900000000018 0.024160647392273
29.7000000000018 0.024160647392273
29.7100000000018 0.024160623550415
29.7200000000018 0.0241606131196022
29.7300000000018 0.0241606503725052
29.7400000000018 0.0241606622934341
29.7500000000019 0.0241606533527374
29.7600000000019 0.024160647392273
29.7700000000019 0.0241606637835503
29.7800000000019 0.0241606280207634
29.7900000000019 0.0241606429219246
29.8000000000019 0.024160623550415
29.8100000000019 0.0241606533527374
29.8200000000019 0.0241606637835503
29.8300000000019 0.0241606265306473
29.8400000000019 0.0241606250405312
29.8500000000019 0.024160623550415
29.8600000000019 0.0241606369614601
29.8700000000019 0.0241606459021568
29.8800000000019 0.024160635471344
29.8900000000019 0.0241606459021568
29.9000000000019 0.0241606459021568
29.9100000000019 0.024160635471344
29.9200000000019 0.0241606459021568
29.9300000000019 0.0241606459021568
29.9400000000019 0.024160635471344
29.9500000000019 0.0241606459021568
29.9600000000019 0.024160635471344
29.9700000000019 0.0241606459021568
29.9800000000019 0.0241606459021568
29.9900000000019 0.0241606324911118
30.0000000000019 0.0241606459021568
};
\addlegendentry{DDPG};
\end{axis}

\end{tikzpicture}

		\end{figure}
	\end{minipage}
	
\end{block}

%----------------------------------------------------------------------------------------

\begin{columns}[t,totalwidth=\twocolwid] % Split up the two columns wide column again

\begin{column}{\onecolwid} % The first column within column 2 (column 2.1)

%----------------------------------------------------------------------------------------
%	Section 5
%----------------------------------------------------------------------------------------
\vspace{-1cm}
\begin{variableblock}{PI Controller}{fg=black,bg=dblue!10}{fg=white,bg=blue!20!gray}
	\begin{figure}
		\resizebox{25cm}{!}{%----------- Create a fancy summing block
\tikzset{add/.style n args={4}{
		minimum width=6mm,
		path picture={
			\draw[black] 
			(path picture bounding box.south east) -- (path picture bounding box.north west)
			(path picture bounding box.south west) -- (path picture bounding box.north east);
			\node at ($(path picture bounding box.south)+(0,0.33)$)     {\tiny #1};
			\node at ($(path picture bounding box.west)+(0.33,0)$)      {\tiny #2};
			\node at ($(path picture bounding box.north)+(0,-0.33)$)    {\tiny #3};
			\node at ($(path picture bounding box.east)+(-0.33,0)$)     {\tiny #4};
		}
	}
}

%----------- Block style 1
\tikzstyle{block1} = [draw, line width=4pt, fill=blue!20, rectangle, 
minimum height=3em, minimum width=5em, node distance=7.5cm, line width=4pt]

%----------- Block style 2
\tikzstyle{block2} = [draw, fill=blue!20, rectangle, 
minimum height=3em, minimum width=3em, node distance=12.5cm, line width=4pt]

%----------- Sum style
\tikzstyle{sum} = [draw, fill=blue!20, circle, node distance=12cm, minimum size=2cm, line width=4pt]

%----------- Input style
\tikzstyle{input} = [coordinate, node distance=14cm]

%----------- Output style
\tikzstyle{output} = [coordinate, node distance=14cm]

%----------- Pin style
\tikzstyle{pinstyle} = [pin edge={to-,thin,black}]


\begin{tikzpicture}	
	% Initial position node
	\node [coordinate] (c1) {};
	
	
	% Create nodes for upper leg
	\node [sum, above of=c1, add={$-$}{}{+}{}, node distance=8cm] (sum4) {};
	\node [coordinate, above of=sum4, node distance=4cm] (c10) {};
	\node [coordinate, right of=c10, node distance=8cm] (c12) {};
	\node [coordinate, right of=sum4, node distance=8cm] (c2) {};
	\node [coordinate, above of=c2] (c4) {};
	\node [block2, below of=sum4, node distance=4cm] (r1) {$\boldsymbol{R_1}$};
	\node [coordinate, below of=r1, node distance=3cm] (c6) {};
	\node [coordinate, right of=c6, node distance=8cm] (c8) {};
	\node [block2, left of=sum4, node distance=6cm] (int1) {$\boldsymbol{\frac{K_{i_1}}{s}}$};
	\node [sum, left of=int1, add={+}{ }{+}{ }, node distance=6cm] (sum6) {};
	\node [block2, below of=sum6, node distance=4cm] (b1) {$\boldsymbol{b_1}$};
	
	
	% Connect nodes
	\draw [->, line width=4pt] (sum4) -- node [at end, label=right:{$\boldsymbol{X_1(s)}$}] {} (c2);
	\draw [->, line width=4pt] (r1) -- (sum4);
	\draw [->, line width=4pt] (b1) -- (sum6);
	\draw [->, line width=4pt] (sum6) -- (int1);
	\draw [->, line width=4pt] (int1) -- (sum4);
	\draw [->, line width=4pt] (c8) -| node [at start, label=right:{$\boldsymbol{\Delta F_1(s)}$}] {} (r1);
	\draw [->, line width=4pt] (c8) -| (b1);
	\draw [->, line width=4pt] (c12) -| node [at start, label=right:{$\boldsymbol{T(s)}$}] {} (sum4);
	\draw [->, line width=4pt] (c12) -| (sum6);
	
	% Create nodes for lower leg
	\node [sum, below of=c1, add={+}{}{$-$}{}, node distance=8cm] (sum5) {};
	\node [coordinate, below of=sum5, node distance=4cm] (c11) {};
	\node [coordinate, right of=c11, node distance=8cm] (c13) {};
	\node [coordinate, right of=sum5, node distance=8cm] (c3) {};
	\node [coordinate, above of=c3] (c5) {};
	\node [block2, above of=sum5, node distance=4cm] (r2) {$\boldsymbol{R_2}$};
	\node [coordinate, above of=r2, node distance=3cm] (c7) {};
	\node [coordinate, right of=c7, node distance=8cm] (c9) {};
	\node [block2, left of=sum5, node distance=6cm] (int2) {$\boldsymbol{\frac{K_{i_2}}{s}}$};
	\node [sum, left of=int2, add={+}{ }{+}{ }, node distance=6cm] (sum7) {};
	\node [block2, above of=sum7, node distance=4cm] (b2) {$\boldsymbol{b_2}$};
	
	
	% Connect nodes
	\draw [->, line width=4pt] (sum5) -- node [at end, label=right:{$\boldsymbol{X_6(s)}$}] {} (c3);
	\draw [->, line width=4pt] (r2) -- (sum5);
	\draw [->, line width=4pt] (b2) -- (sum7);
	\draw [->, line width=4pt] (sum7) -- (int2);
	\draw [->, line width=4pt] (int2) -- (sum5);
	\draw [->, line width=4pt] (c9) -| node [at start, label=right:{$\boldsymbol{\Delta F_2(s)}$}] {} (r2);
	\draw [->, line width=4pt] (c9) -| (b2);
	\draw [->, line width=4pt] (c13) -| node [at start, label=right:{$\boldsymbol{-T(s)}$}] {} (sum5);
	\draw [->, line width=4pt] (c13) -| (sum7);	
\end{tikzpicture}
}
		\caption{Block diagram of the PI controllers used to classically control power system frequency. Two feedback loop proportional integral (PI) controllers are used to control the frequency and the tie-line power flow. A single PI controller is connected to each power area.}
	\end{figure}
\end{variableblock}

%----------------------------------------------------------------------------------------

\end{column} % End of column 2.1

\begin{column}{\onecolwid} % The second column within column 2 (column 2.2)

%----------------------------------------------------------------------------------------
%	RESULTS
%----------------------------------------------------------------------------------------
\vspace{-1cm}
\begin{variableblock}{DDPG Controller}{fg=black,bg=dblue!10}{fg=white,bg=blue!20!gray}
	\begin{center}
		\textbf{Neural Network Architecture}
	\end{center}
	\begin{figure}
		\resizebox{25cm}{!}{%----------- Pin style
\tikzstyle{every pin edge}=[<-,shorten <=1pt]

%----------- Block style 1
\tikzstyle{neuron}=[draw,circle,fill=black!25,minimum size=17pt,inner sep=0pt]

%----------- Block style 1
\tikzstyle{input neuron}=[neuron, fill=white!80!green, minimum size=1cm]

%----------- Block style 1
\tikzstyle{output neuron}=[neuron, fill=white!80!blue, minimum size=1cm]

%----------- Block style 1
\tikzstyle{hidden neuron}=[neuron, fill=blue!20, minimum size=1cm]

%----------- Block style 1
\tikzstyle{annot} = [text width=4em, text centered]

% Define a distance to separate the layers of the network
\def\layersep{2*2.5cm}

\begin{tikzpicture}[shorten >=1pt,->, node distance=\layersep]
    
    % Draw the input layer nodes
    \foreach \i\y\j in {1/1/$\boldsymbol{x_4(t) = \Delta f_1}$, 2/2/$\boldsymbol{x_9(t) = \Delta f_2}$, 3/3/$\boldsymbol{x_5(t) = \Delta \texttt{TieLine}}$}
    % This is the same as writing \foreach \name / \y in {1/1,2/2,3/3,4/4}
        \node[input neuron, pin=left:\j] (I-\i) at (0,-2*\y) {};

    % Draw the hidden layer 1 nodes
    \foreach \name / \y in {1,...,5}
        \path[yshift=2*1cm]
            node[hidden neuron] (H1-\name) at (\layersep,-2*\y cm) {};
	
	% Draw the hidden layer 2 nodes
	\foreach \name / \y in {1,...,5}
		\path[yshift=2*1cm]
			node[hidden neuron] (H2-\name) at (2*\layersep,-2*\y cm) {};
	
    % Draw the output layer node
    \foreach \name / \y in {1/$\boldsymbol{u_1(t)}$,2/$\boldsymbol{u_2(t)}$}
    	\path[yshift=-2*0.5cm]
    		node[output neuron, pin={[pin edge={->}]right:\y}] (O-\name) at (3*\layersep,-2*\name cm) {};

    % Connect every node in the input layer with every node in the
    % hidden layer 1.
    \foreach \source in {1,...,3}
        \foreach \dest in {1,...,5}
            \path (I-\source) edge [color=gray] (H1-\dest);
	
	% Connect every node in hidden layer 1 with every node in
	% hidden layer 2.
	\foreach \source in {1,...,5}
		\foreach \dest in {1,...,5}
			\path (H1-\source) edge [color=gray] (H2-\dest);
	
    % Connect every node in the hidden layer with the output layer
    \foreach \source in {1,...,5}
    	\foreach \dest in {1,...,2}
        	\path (H2-\source) edge [color=gray] (O-\dest);
    
\end{tikzpicture}}
		\caption{ \ Indicative architecture of a neural network.}
	\end{figure}
\end{variableblock}

%----------------------------------------------------------------------------------------

\end{column} % End of column 2.2

\end{columns} % End of the split of column 2

\end{column} % End of the second column

\begin{column}{\sepwid}\end{column} % Empty spacer column

\begin{column}{\onecolwid} % The third column

%----------------------------------------------------------------------------------------
%	Section 7
%----------------------------------------------------------------------------------------

\begin{block}{Preliminary Experiment Setup}


\end{block}

\begin{block}{DDPG Controller Evolution}


\end{block}

\begin{block}{Research Direction}


\end{block}

%----------------------------------------------------------------------------------------

\end{column} % End of the third column

\end{columns} % End of all the columns in the poster

\end{frame} % End of the enclosing frame

\end{document}
