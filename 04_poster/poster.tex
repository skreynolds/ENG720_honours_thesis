%%%%%%%%%%%%%%%%%%%%%%%%%%%%%%%%%%%%%%%%%
% Jacobs Landscape Poster
% LaTeX Template
% Version 1.1 (14/06/14)
%
% Created by:
% Computational Physics and Biophysics Group, Jacobs University
% https://teamwork.jacobs-university.de:8443/confluence/display/CoPandBiG/LaTeX+Poster
% 
% Further modified by:
% Shane Reynolds
%
% License:
% CC BY-NC-SA 3.0 (http://creativecommons.org/licenses/by-nc-sa/3.0/)
%
%%%%%%%%%%%%%%%%%%%%%%%%%%%%%%%%%%%%%%%%%

%----------------------------------------------------------------------------------------
%	PACKAGES AND OTHER DOCUMENT CONFIGURATIONS
%----------------------------------------------------------------------------------------

\documentclass[final]{beamer}

\usepackage[scale=1.24]{beamerposter} % Use the beamerposter package for laying out the poster

\usetheme{confposter} % Use the confposter theme supplied with this template

\setbeamercolor{block title}{fg=black,bg=white} % Colors of the block titles
\setbeamercolor{block body}{fg=black,bg=white} % Colors of the body of blocks

\setbeamercolor{itemize item}{fg=black}
\setbeamertemplate{itemize item}[circle]

%\setbeamercolor{block alerted title}{fg=white,bg=green!20!gray} % Colors of the highlighted block titles
%\setbeamercolor{block alerted body}{fg=black,bg=dgreen!10} % Colors of the body of highlighted blocks
% Many more colors are available for use in beamerthemeconfposter.sty

\newenvironment{variableblock}[3]{%
  \setbeamercolor{block alerted body}{#2}
  \setbeamercolor{block alerted title}{#3}
  \begin{alertblock}{#1}}{\end{alertblock}}

%-----------------------------------------------------------
% Define the column widths and overall poster size
% To set effective sepwid, onecolwid and twocolwid values, first choose how many columns you want and how much separation you want between columns
% In this template, the separation width chosen is 0.024 of the paper width and a 4-column layout
% onecolwid should therefore be (1-(# of columns+1)*sepwid)/# of columns e.g. (1-(4+1)*0.024)/4 = 0.22
% Set twocolwid to be (2*onecolwid)+sepwid = 0.464
% Set threecolwid to be (3*onecolwid)+2*sepwid = 0.708

\newlength{\sepwid}
\newlength{\onecolwid}
\newlength{\twocolwid}
\newlength{\threecolwid}
\setlength{\paperwidth}{48in} % A0 width: 46.8in
\setlength{\paperheight}{36in} % A0 height: 33.1in
\setlength{\sepwid}{0.024\paperwidth} % Separation width (white space) between columns
\setlength{\onecolwid}{0.22\paperwidth} % Width of one column
\setlength{\twocolwid}{0.464\paperwidth} % Width of two columns
\setlength{\threecolwid}{0.708\paperwidth} % Width of three columns
\setlength{\topmargin}{-0.5in} % Reduce the top margin size
%-----------------------------------------------------------

\usepackage{graphicx}  % Required for including images

\usepackage{booktabs} % Top and bottom rules for tables

\usepackage{amsmath} % Required for math font

\usepackage{tikz} % Required for drawing sik pictures

% Import tikz libraries
\usetikzlibrary{shapes, arrows}
\usetikzlibrary{positioning, calc}

\graphicspath{{./figures/}}

%----------------------------------------------------------------------------------------
%	TITLE SECTION 
%----------------------------------------------------------------------------------------

\title{Automatic Generation Control of a Two Area Power System Using Deep Reinforcement Learning} % Poster title

\author{Shane Reynolds} % Author(s)

\institute{Charles Darwin University} % Institution(s)

%----------------------------------------------------------------------------------------

\begin{document}

\addtobeamertemplate{block end}{}{\vspace*{2ex}} % White space under blocks
\addtobeamertemplate{block alerted end}{}{\vspace*{2ex}} % White space under highlighted (alert) blocks

\setlength{\belowcaptionskip}{2ex} % White space under figures
\setlength\belowdisplayshortskip{2ex} % White space under equations

\begin{frame}[t] % The whole poster is enclosed in one beamer frame

\begin{columns}[t] % The whole poster consists of three major columns, the second of which is split into two columns twice - the [t] option aligns each column's content to the top

\begin{column}{\sepwid}\end{column} % Empty spacer column

\begin{column}{\onecolwid} % The first column

%----------------------------------------------------------------------------------------
%	Background
%----------------------------------------------------------------------------------------

\begin{block}{Background}
A generator's angular acceleration is governed by:
\begin{equation*}
	\boldsymbol{\Delta T = T_{mech} - T_{elec} = I \alpha}
\end{equation*}
\vspace{-2cm}
\begin{itemize}
	\setlength{\itemindent}{1em}
	\item \ If $\boldsymbol{\Delta T > 0}$, then $\boldsymbol{\alpha \uparrow}$ and $\boldsymbol{f}$ \textbf{(Hz)} $\uparrow$
	\item \ If $\boldsymbol{\Delta T < 0}$, then $\boldsymbol{\alpha \downarrow}$ and $\boldsymbol{f}$ \textbf{(Hz)} $\downarrow$
\end{itemize}
The Australian power network operates at 50 Hz.
\end{block}


%----------------------------------------------------------------------------------------
%	Section 1
%----------------------------------------------------------------------------------------

\begin{block}{Reinforcement Learning}
Reinforcement learning is a branch of machine learning concerned with an agent's sequential decision making to maximise cumulative expected reward.
	\begin{figure}
		% Additional styles
\tikzstyle{block} = [rectangle, draw, text width=8em, text centered, rounded corners, minimum height=4em, line width=4pt]
    
\tikzstyle{line} = [draw, -latex]



\begin{tikzpicture}[node distance = 6em, auto, thick]
    \node [block, fill=blue!25!gray, text=white] (Agent) {\textbf{Agent}};
    \node [block, below of=Agent, fill=green!25!gray, text=white] (Environment) {\textbf{Environment}};
    
     \path [line, line width=4pt] (Agent.0) --++ (4em,0em) |- node [near start]{$\boldsymbol{a_t}$} (Environment.0);
     \path [line, line width=4pt] (Environment.190) --++ (-6em,0em) |- node [near start] {$\boldsymbol{s_t}$} (Agent.170);
     \path [line, line width=4pt] (Environment.170) --++ (-4.25em,0em) |- node [near start, right] {$\boldsymbol{r_t}$} (Agent.190);
\end{tikzpicture}
	\end{figure}
The agent exists in some environment and at each time step observes state $\boldsymbol{s_t \in S}$; and takes an action $\boldsymbol{a_t \in A}$. Following this, the agent then receives a reward $\boldsymbol{r_t \in R: S \times A \times S \to [R_{min}, R_{max}]}$.
\end{block}


\begin{variableblock}{The Environment}{fg=black,bg=dgreen!10}{fg=white,bg=green!20!gray}
Two power areas connected via a transmission line. Each power area consists of: a governor controlled generator; and stochastic load demand.
	\begin{figure}
	\centering
	\begin{tikzpicture}
	
	% Area 1
	\draw [dashed, fill=blue!20] (0.5,0) circle (1.5cm);
		
	% Area 2
	\draw [dashed, fill=blue!20] (5.5,0) circle (1.5cm);
	
	% Area labels
	\node at (0.25,-1) {\scriptsize Area 1};
	\node at (5.75,-1) {\scriptsize Area 2};
	
	% Bus 1
	\draw [line width=0.5mm] (0,0) -- (1,0);
	\draw [line width=0.2mm] (0.5,0) -- (0.5,0.4);
	
	% Load 1
	\draw [->, line width=0.2mm] (0.1,0) -- (0.1,-0.4);
	\node at (-0.35,-0.2) {\tiny Gen. 1};
	
	% Gen 1
	\draw [fill=purple!20] (0.5,0.65) circle (0.25cm);
	\node at (-0.25,0.65) {\tiny Load 1};
	
	% Bus 2
	\draw [line width=0.5mm] (5,0) -- (6,0);
	\draw [line width=0.2mm] (5.5,0) -- (5.5,0.4);
	
	% Load 2
	\draw [->, line width=0.2mm] (5.9,0) -- (5.9,-0.4);
	\node at (6.35,-0.2) {\tiny Gen. 2};
	
	% Gen 2
	\draw [fill=purple!20] (5.5,0.65) circle (0.25cm);
	\node at (6.25,0.65) {\tiny Load 2};
	
	% Tie line 1 to 2
	\draw [line width=0.2mm] (0.9,0) -- (0.9,-0.2) -- (5.1,-0.2) -- (5.1,0);

\end{tikzpicture}

	\end{figure}
The control objective is to maintain inter-area power transfer, whilst regulating the frequency of each area.
\end{variableblock}

%------------------------------------------------



%----------------------------------------------------------------------------------------

\end{column} % End of the first column

\begin{column}{\sepwid}\end{column} % Empty spacer column

\begin{column}[t]{\twocolwid} % Begin a column which is two columns wide (column 2)



%----------------------------------------------------------------------------------------
%	Section 4
%----------------------------------------------------------------------------------------

\begin{variableblock}{The Environment in the Frequency \& Temporal Domains}{fg=black,bg=dgreen!10}{fg=white,bg=green!20!gray}
	\begin{figure}
	\resizebox{50cm}{!}{%----------- Create a fancy summing block
\tikzset{add/.style n args={4}{
		minimum width=6mm,
		path picture={
			\draw[black] 
			(path picture bounding box.south east) -- (path picture bounding box.north west)
			(path picture bounding box.south west) -- (path picture bounding box.north east);
			\node at ($(path picture bounding box.south)+(0,0.13)$)     {\tiny #1};
			\node at ($(path picture bounding box.west)+(0.13,0)$)      {\tiny #2};
			\node at ($(path picture bounding box.north)+(0,-0.13)$)    {\tiny #3};
			\node at ($(path picture bounding box.east)+(-0.13,0)$)     {\tiny #4};
		}
	}
}

%----------- Block style 1
\tikzstyle{block1} = [draw, fill=white!80!green, rectangle, 
minimum height=3em, minimum width=6em, node distance=2.5cm]

%----------- Block style 2
\tikzstyle{block2} = [draw, fill=white!80!green, rectangle, 
minimum height=3em, minimum width=3em, node distance=2.5cm]

%----------- Sum style
\tikzstyle{sum} = [draw, fill=white!80!green, circle, node distance=2cm]

%----------- Input style
\tikzstyle{input} = [coordinate, node distance=4cm]

%----------- Output style
\tikzstyle{output} = [coordinate, node distance=4cm]

%----------- Pin style
\tikzstyle{pinstyle} = [pin edge={to-,thin,black}]


\begin{tikzpicture}	
	% Tie line nodes
	\node [sum, add={$-$}{}{+}{ }] (sum1) {};
	\node [block1, right of=sum1, label=above:{Tie Line}] (tieline) {$\frac{2\pi T_{12}}{s}$};
	\node [output, right of=tieline, node distance=3cm] (out) {};
	\node [coordinate, above of=out, node distance=5cm] (c1) {};
	\node [coordinate, below of=out, node distance=5cm] (c2) {};
	\node [block2, left of=c2] (a12) {$-a_{12}$};
	
	% Position a reference coordinate for drawing
	\node [coordinate, left of=sum1, node distance=2.5cm] (c3) {};
	\node [coordinate, above of=c3, node distance=0.75cm] (c4) {};
	\node [coordinate, below of=c3, node distance=0.75cm] (c5) {};
	
	% Create nodes for upper leg
	\node [block1, above of=c3, node distance=3.5cm, label=above:{Gen. Load 1}] (genload1) {$\frac{K_{gl1}}{T_{gl1}s+1}$};
	\node [coordinate, right of=genload1, node distance=1.5cm] (c6) {};
	\node [sum, left of=genload1, add={$-$}{+}{$-$}{}, node distance=2.5cm] (sum2) {};
	\node [coordinate, below of=sum2] (p11) {};
	\node [coordinate, left of=p11, node distance=0.5cm, label=left:{$\Delta P_{L1}(s)$}] (p12) {};
	\node [block1, left of=sum2, node distance=3.5cm, label=above:{Turbine 1}] (turbine1) {$\frac{K_{t1}}{T_{t1}s+1}$};
	\node [block1, left of=turbine1, node distance=4.5cm, label=above:{Governor 1}] (governor1) {$\frac{K_{g1}}{T_{g1}s+1}$};
	\node [coordinate, left of=governor1, node distance=3cm] (c8) {};
	\node [coordinate, left of=c5, node distance=13.5cm] (c10) {};
	\node [coordinate, left of=c1, node distance=21.5cm] (c12) {};
	
	
	
	% Create nodes for lower leg
	\node [block1, below of=c3, node distance=3.5cm, label=above:{Gen. Load 2}] (genload2) {$\frac{K_{gl1}}{T_{gl1}s+1}$};
	\node [coordinate, right of=genload2, node distance=1.5cm] (c7) {};
	\node [sum, left of=genload2, add={$-$}{+}{$-$}{}, node distance=2.5cm] (sum3) {};
	\node [coordinate, above of=sum3] (p21) {};
	\node [coordinate, left of=p21, node distance=0.5cm, label=left:{$\Delta P_{L2}(s)$}] (p22) {};
	\node [block1, left of=sum3, node distance=3.5cm, label=above:{Turbine 2}] (turbine2) {$\frac{K_{t2}}{T_{t2}s+1}$};
	\node [block1, left of=turbine2, node distance=4.5cm, label=above:{Governor 2}] (governor2) {$\frac{K_{g2}}{T_{g2}s+1}$};
	\node [coordinate, left of=governor2, node distance=3cm] (c9) {};
	\node [coordinate, left of=c4, node distance=13.5cm] (c11) {};
	\node [coordinate, left of=a12, node distance=19cm] (c13) {};
	
	
	% Connect the tieline nodes
	\draw [->] (sum1) -- (tieline);
	\draw (tieline) -- node [label=below:{$X_5(s)$}] {} (out);
	
	% Connect nodes in upper block
	\draw (out) -- (c1);
	\draw [->] (c1) -| (sum2);
	
	\draw [->] (governor1) -- node [label=above:{$X_2(s)$}] {} (turbine1);
	\draw [->] (turbine1) -- node [label=above:{$X_3(s)$}] {} (sum2);
	\draw [->] (sum2) -- (genload1);
	\draw [->] (genload1) -| node [label=above:{$X_4(s)$}] {} (sum1);
	\draw (c6) |- (c4);
	\draw [->] (c8) -- node [label=above:{$U_1(s)$}] {} (governor1);
	\draw (p12) -- (p11);
	\draw [->] (p11) -- (sum2);
	\draw [->] (c4) -- (c11);
	\draw [->] (c1) -- (c12);
	
	
	% Connect nodes in lower block
	\draw (out) -- (c2);
	\draw [->] (c2) -- (a12);
	\draw [->] (a12) -| (sum3);
	\draw [->] (governor2) -- node [label=below:{$X_7(s)$}] {} (turbine2);
	\draw [->] (turbine2) -- node [label=below:{$X_8(s)$}] {} (sum3);
	\draw [->] (sum3) -- (genload2);
	\draw [->] (genload2) -| node [label=below:{$X_9(s)$}] {} (sum1);
	\draw (c7) |- (c5);
	\draw [->] (c9) -- node [label=below:{$U_2(s)$}] {} (governor2);
	\draw (p22) -- (p21);
	\draw [->] (p21) -- (sum3);
	\draw [->] (c5) -- (c10);
	\draw [->] (a12) -- (c13);
	
\end{tikzpicture}}
	\end{figure}
	
	\begin{center}
	\textbf{Two Area System ODE}\\
	\vspace{1cm}
	\begin{minipage}{0.45\linewidth}
	\small
		\begin{align*}
			\boldsymbol{\dot{x}_2(t)} &\boldsymbol{=} \boldsymbol{\frac{1}{T_{sg_1}}\big( K_{sg_1} u_1(t) - x_2(t) \big)} \\
			\boldsymbol{\dot{x}_3(t)} &\boldsymbol{=} \boldsymbol{\frac{1}{T_{t_1}} \big( K_{t_1} x_2(t) - x_3(t) \big)} \\
			\boldsymbol{\dot{x}_4(t)} &\boldsymbol{=} \boldsymbol{\frac{1}{T_{gl_1}} \bigg( K_{gl_1} \big( x_3(t) - x_5(t) - \Delta p_{L1}(t) \big) - x_4(t) \bigg)}
		\end{align*}
	\end{minipage}
	\hspace{0.5cm}
	\begin{minipage}{0.45\linewidth}
	\small
		\begin{align*}
			\boldsymbol{\dot{x}_7(t)} &\boldsymbol{=} \boldsymbol{\frac{1}{T_{sg_2}}\big( K_{sg_2} u_2(t) - x_7(t) \big)} \\
			\boldsymbol{\dot{x}_8(t)} &\boldsymbol{=} \boldsymbol{\frac{1}{T_{t_2}} \big( K_{t_2} x_7(t) - x_8(t) \big)} \\
			\boldsymbol{\dot{x}_9(t)} &\boldsymbol{=} \boldsymbol{\frac{1}{T_{gl_2}} \bigg( K_{gl_2} \big( x_8(t) - x_5(t) - \Delta p_{L2}(t) \big) - x_9(t) \bigg)}
		\end{align*}
	\end{minipage}
	\end{center}
	
\end{variableblock}

%----------------------------------------------------------------------------------------

\begin{columns}[t,totalwidth=\twocolwid] % Split up the two columns wide column again

\begin{column}{\onecolwid} % The first column within column 2 (column 2.1)

%----------------------------------------------------------------------------------------
%	Section 5
%----------------------------------------------------------------------------------------
\vspace{-1cm}
\begin{variableblock}{Classical PI Controller}{fg=black,bg=dblue!10}{fg=white,bg=blue!20!gray}
	\begin{figure}
		\resizebox{!}{21cm}{%----------- Create a fancy summing block
\tikzset{add/.style n args={4}{
		minimum width=6mm,
		path picture={
			\draw[black] 
			(path picture bounding box.south east) -- (path picture bounding box.north west)
			(path picture bounding box.south west) -- (path picture bounding box.north east);
			\node at ($(path picture bounding box.south)+(0,0.33)$)     {\tiny #1};
			\node at ($(path picture bounding box.west)+(0.33,0)$)      {\tiny #2};
			\node at ($(path picture bounding box.north)+(0,-0.33)$)    {\tiny #3};
			\node at ($(path picture bounding box.east)+(-0.33,0)$)     {\tiny #4};
		}
	}
}

%----------- Block style 1
\tikzstyle{block1} = [draw, line width=4pt, fill=blue!20, rectangle, 
minimum height=3em, minimum width=5em, node distance=7.5cm, line width=4pt]

%----------- Block style 2
\tikzstyle{block2} = [draw, fill=blue!20, rectangle, 
minimum height=3em, minimum width=3em, node distance=12.5cm, line width=4pt]

%----------- Sum style
\tikzstyle{sum} = [draw, fill=blue!20, circle, node distance=12cm, minimum size=2cm, line width=4pt]

%----------- Input style
\tikzstyle{input} = [coordinate, node distance=14cm]

%----------- Output style
\tikzstyle{output} = [coordinate, node distance=14cm]

%----------- Pin style
\tikzstyle{pinstyle} = [pin edge={to-,thin,black}]


\begin{tikzpicture}	
	% Initial position node
	\node [coordinate] (c1) {};
	
	
	% Create nodes for upper leg
	\node [sum, above of=c1, add={$-$}{}{+}{}, node distance=8cm] (sum4) {};
	\node [coordinate, above of=sum4, node distance=4cm] (c10) {};
	\node [coordinate, right of=c10, node distance=8cm] (c12) {};
	\node [coordinate, right of=sum4, node distance=8cm] (c2) {};
	\node [coordinate, above of=c2] (c4) {};
	\node [block2, below of=sum4, node distance=4cm] (r1) {$\boldsymbol{R_1}$};
	\node [coordinate, below of=r1, node distance=3cm] (c6) {};
	\node [coordinate, right of=c6, node distance=8cm] (c8) {};
	\node [block2, left of=sum4, node distance=6cm] (int1) {$\boldsymbol{\frac{K_{i_1}}{s}}$};
	\node [sum, left of=int1, add={+}{ }{+}{ }, node distance=6cm] (sum6) {};
	\node [block2, below of=sum6, node distance=4cm] (b1) {$\boldsymbol{b_1}$};
	
	
	% Connect nodes
	\draw [->, line width=4pt] (sum4) -- node [at end, label=right:{$\boldsymbol{X_1(s)}$}] {} (c2);
	\draw [->, line width=4pt] (r1) -- (sum4);
	\draw [->, line width=4pt] (b1) -- (sum6);
	\draw [->, line width=4pt] (sum6) -- (int1);
	\draw [->, line width=4pt] (int1) -- (sum4);
	\draw [->, line width=4pt] (c8) -| node [at start, label=right:{$\boldsymbol{\Delta F_1(s)}$}] {} (r1);
	\draw [->, line width=4pt] (c8) -| (b1);
	\draw [->, line width=4pt] (c12) -| node [at start, label=right:{$\boldsymbol{T(s)}$}] {} (sum4);
	\draw [->, line width=4pt] (c12) -| (sum6);
	
	% Create nodes for lower leg
	\node [sum, below of=c1, add={+}{}{$-$}{}, node distance=8cm] (sum5) {};
	\node [coordinate, below of=sum5, node distance=4cm] (c11) {};
	\node [coordinate, right of=c11, node distance=8cm] (c13) {};
	\node [coordinate, right of=sum5, node distance=8cm] (c3) {};
	\node [coordinate, above of=c3] (c5) {};
	\node [block2, above of=sum5, node distance=4cm] (r2) {$\boldsymbol{R_2}$};
	\node [coordinate, above of=r2, node distance=3cm] (c7) {};
	\node [coordinate, right of=c7, node distance=8cm] (c9) {};
	\node [block2, left of=sum5, node distance=6cm] (int2) {$\boldsymbol{\frac{K_{i_2}}{s}}$};
	\node [sum, left of=int2, add={+}{ }{+}{ }, node distance=6cm] (sum7) {};
	\node [block2, above of=sum7, node distance=4cm] (b2) {$\boldsymbol{b_2}$};
	
	
	% Connect nodes
	\draw [->, line width=4pt] (sum5) -- node [at end, label=right:{$\boldsymbol{X_6(s)}$}] {} (c3);
	\draw [->, line width=4pt] (r2) -- (sum5);
	\draw [->, line width=4pt] (b2) -- (sum7);
	\draw [->, line width=4pt] (sum7) -- (int2);
	\draw [->, line width=4pt] (int2) -- (sum5);
	\draw [->, line width=4pt] (c9) -| node [at start, label=right:{$\boldsymbol{\Delta F_2(s)}$}] {} (r2);
	\draw [->, line width=4pt] (c9) -| (b2);
	\draw [->, line width=4pt] (c13) -| node [at start, label=right:{$\boldsymbol{-T(s)}$}] {} (sum5);
	\draw [->, line width=4pt] (c13) -| (sum7);	
\end{tikzpicture}
}
	\end{figure}
	\begin{center}
		\textbf{ODE System}\\
		\vspace{0.75cm}
		\small
		$\boldsymbol{\dot{x}_1(t) = b_1 \Delta f_1(t) + x_5(t)}$ \ 
		$\boldsymbol{\dot{x}_6(t) = b_2 \Delta f_2(t) - x_5(t)}$
	\end{center}
\end{variableblock}

%----------------------------------------------------------------------------------------

\end{column} % End of column 2.1

\begin{column}{\onecolwid} % The second column within column 2 (column 2.2)

%----------------------------------------------------------------------------------------
%	RESULTS
%----------------------------------------------------------------------------------------
\vspace{-1cm}
\begin{variableblock}{DDPG Controller}{fg=black,bg=dblue!10}{fg=white,bg=blue!20!gray}
	\begin{center}
		\textbf{Neural Network Architecture}
	\end{center}
	\begin{figure}
		\centering
		\resizebox{25cm}{!}{%----------- Pin style
\tikzstyle{every pin edge}=[<-,shorten <=1pt]

%----------- Block style 1
\tikzstyle{neuron}=[draw,circle,fill=black!25,minimum size=17pt,inner sep=0pt]

%----------- Block style 1
\tikzstyle{input neuron}=[neuron, fill=white!80!green, minimum size=1cm]

%----------- Block style 1
\tikzstyle{output neuron}=[neuron, fill=white!80!blue, minimum size=1cm]

%----------- Block style 1
\tikzstyle{hidden neuron}=[neuron, fill=blue!20, minimum size=1cm]

%----------- Block style 1
\tikzstyle{annot} = [text width=4em, text centered]

% Define a distance to separate the layers of the network
\def\layersep{2*2.5cm}

\begin{tikzpicture}[shorten >=1pt,->, node distance=\layersep]
    
    % Draw the input layer nodes
    \foreach \i\y\j in {1/1/\textbf{FREQ. INPUT 1}, 2/2/\textbf{FREQ. INPUT 2}, 3/3/\textbf{TIELINE INPUT}}
    % This is the same as writing \foreach \name / \y in {1/1,2/2,3/3,4/4}
        \node[input neuron, pin=left:\j] (I-\i) at (0,-2*\y) {};

    % Draw the hidden layer 1 nodes
    \foreach \name / \y in {1,...,5}
        \path[yshift=2*1cm]
            node[hidden neuron] (H1-\name) at (\layersep,-2*\y cm) {};
	
	% Draw the hidden layer 2 nodes
	\foreach \name / \y in {1,...,5}
		\path[yshift=2*1cm]
			node[hidden neuron] (H2-\name) at (2*\layersep,-2*\y cm) {};
	
    % Draw the output layer node
    \foreach \name / \y in {1/\textbf{CTL. OUTPUT 1},2/\textbf{CTL. OUTPUT 2}}
    	\path[yshift=-2*0.5cm]
    		node[output neuron, pin={[pin edge={->}]right:\y}] (O-\name) at (3*\layersep,-2*\name cm) {};

    % Connect every node in the input layer with every node in the
    % hidden layer 1.
    \foreach \source in {1,...,3}
        \foreach \dest in {1,...,5}
            \path (I-\source) edge [color=gray] (H1-\dest);
	
	% Connect every node in hidden layer 1 with every node in
	% hidden layer 2.
	\foreach \source in {1,...,5}
		\foreach \dest in {1,...,5}
			\path (H1-\source) edge [color=gray] (H2-\dest);
	
    % Connect every node in the hidden layer with the output layer
    \foreach \source in {1,...,5}
    	\foreach \dest in {1,...,2}
        	\path (H2-\source) edge [color=gray] (O-\dest);
    
\end{tikzpicture}}
	\end{figure}
	\begin{center}
			\textbf{Agent Training Algorithm}
		\end{center}
\end{variableblock}

%----------------------------------------------------------------------------------------

\end{column} % End of column 2.2

\end{columns} % End of the split of column 2

\end{column} % End of the second column

\begin{column}{\sepwid}\end{column} % Empty spacer column

\begin{column}{\onecolwid} % The third column

%----------------------------------------------------------------------------------------
%	Section 7
%----------------------------------------------------------------------------------------

\begin{block}{Experiments}


\end{block}


\begin{block}{Results}


\end{block}

%----------------------------------------------------------------------------------------

\end{column} % End of the third column

\end{columns} % End of all the columns in the poster

\end{frame} % End of the enclosing frame

\end{document}
