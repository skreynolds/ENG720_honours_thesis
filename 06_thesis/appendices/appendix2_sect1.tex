\section{Tie Line Model}\label{fig:tie_line_model}
The power systems in each of the control areas for a two-area power system are comparable to the single area model shown in Figure \ref{fig:2110_single_area_pi_control_model}. In fact, the governor, and turbine models are interchangeable between the single area model and the two-area model. We do, however, still have make some changes --- these are as follows:
\begin{enumerate}
	\item model the interaction of the two power systems over the tie line; and
	\item re-analyse our generator-load demand model.
\end{enumerate}

The derivations presented below are motivated by Kothari \cite{Kothari2011}.

The transmission infrastructure, or tie line, allows for the flow of power from area 1 to area 2, and vice versa. For the convenience of this derivation, let any symbol with a subscript of 1 refer to power area 1 and those with a subscript 2 refer to power area 2. Now, letting the power angles of the area 1 generator and the area 2 generator be $(\delta_1)_0$ and $(\delta_2)_0$, respectively, we can write an expression for the power transported out of area 1 as:
\begin{equation}
	P_{tie, 1} = \frac{|V_1||V_2|}{X_{12}} \sin \big( (\delta_1)_0 - (\delta_2)_0 \big).
\end{equation}

Using incremental changes in $\delta_1$ $\delta_2$, we can determine an incremental change in the tie line power using a Taylor series expansion:
\begin{equation}
	\Delta P_{tie, 1}(pu) = T_{12} (\Delta \delta_1 - \Delta \delta_2). \label{eq:B01}
\end{equation}

Note that $T_{12}$ is the synchronising coefficient, and is mathematically expressed as:
\begin{equation}
	T_{12} = \frac{|V_1||V_2|}{P_{r1} X_{12}} \cos \big( (\delta_1)_0 - (\delta_2)_0 \big).
\end{equation}

Incremental power angles are integrals of incremental frequencies, owing to the link between  angular velocity and frequency. This allows the re-expression of Equation \ref{eq:B01} as:
\begin{equation}
	\Delta P_{tie, 1} = 2 \pi T_{12} \bigg( \int \Delta f_1 dt - \int \Delta f_2 dt \bigg). \label{eq:B02}
\end{equation}

The variables $\Delta f_1$ and $\Delta f_2$ are incremental frequency changes of areas 1 and 2, respectively.

Power can flow both ways over the tie line --- the ideas governing the flow from area 2 to area 1 are symmetrical to those presented above. Hence, we can write:
\begin{equation}
	\Delta P_{tie, 2} = 2 \pi T_{21} \bigg( \int \Delta f_2 dt - \int \Delta f_1 dt \bigg). \label{eq:B03}
\end{equation}

Note that the synchronising coefficient, $T_{21}$, can be expressed in terms of $T_{12}$, therefore:
\begin{equation}
	T_{21} = \frac{|V_2||V_1|}{P_{r2} X_{21}} \cos \big( (\delta_2)_0 - (\delta_1)_0 \big) = \bigg( \frac{P_{r1}}{P_{r2}} \bigg) T_{12} = a_{12} T_{12}. \label{eq:B04}
\end{equation}

Equations \ref{eq:B02} and \ref{eq:B03} describe tie line power flow from area 1 to area 2, and from area 2 to area 1, respectively. To link these expression back to the power system model, first take the inverse Laplace transform of Equation \ref{eq:B02} to provide:
\begin{equation}
	\Delta P_{tie,1}(s) = \frac{2 \pi T_{12}}{s} \times [\Delta F_1(s) - \Delta F_2(s)]. \label{eq:B05}
\end{equation}

Then take the inverse Laplace transform of Equation \ref{eq:B03}, which gives:
\begin{equation}
	\Delta P_{tie,2}(s) = - \frac{2 \pi a_{12} T_{12}}{s} \times [\Delta F_1(s) - \Delta F_2(s)] \label{eq:B06}
\end{equation}

Equation \ref{eq:B05} and \ref{eq:B06} describe the tieline models for power area 1 and 2, respectively. Note that the input for both \ref{eq:B05} and \ref{eq:B06} is $\Delta F_1(s) - \Delta F_2(s)$, and the transfer functions for each differ by a factor of $-a_{12}$. This symmetry allows for the use of a single tie-line model to represent the interconnection dynamics for a two area power system. The tie line block diagram is shown in Figure \ref{fig:B101_tie_line_model}.

\begin{figure}[h]
	\centering
	%----------- Create a fancy summing block
\tikzset{add/.style n args={4}{
		minimum width=6mm,
		path picture={
			\draw[black] 
			(path picture bounding box.south east) -- (path picture bounding box.north west)
			(path picture bounding box.south west) -- (path picture bounding box.north east);
			\node at ($(path picture bounding box.south)+(0,0.13)$)     {\tiny #1};
			\node at ($(path picture bounding box.west)+(0.13,0)$)      {\tiny #2};
			\node at ($(path picture bounding box.north)+(0,-0.13)$)    {\tiny #3};
			\node at ($(path picture bounding box.east)+(-0.13,0)$)     {\tiny #4};
		}
	}
}

%----------- Block style 1
\tikzstyle{block1} = [draw, fill=green!20, rectangle, 
minimum height=3em, minimum width=6em, node distance=2.5cm]

%----------- Block style 2
\tikzstyle{block2} = [draw, fill=green!20, rectangle, 
minimum height=3em, minimum width=3em, node distance=2.5cm]

%----------- Sum style
\tikzstyle{sum} = [draw, fill=green!20, circle, node distance=2cm]

%----------- Input style
\tikzstyle{input} = [coordinate, node distance=4cm]

%----------- Output style
\tikzstyle{output} = [coordinate, node distance=4cm]

\begin{tikzpicture}
	
	% Draw the nodes first
	\node [input] (input1) {};
	\node [sum,add={ }{+}{ }{$-$}, right of=input1, node distance=4cm] (sum) {};
	\node [right of=sum, node distance=6.55cm] (input2) {};
	\node [block1, above of=sum] (tieline) {$\frac{2 \pi T_{12}}{s}$};
	\node [block2, right of=tieline] (a12) {$-a_{12}$};
	\node [output, right of=a12] (P2) {};
	\node [output, left of=tieline] (P1) {};
	
	% Connect the nodes
	\draw [->] (input1) -- node [label=above:{$\Delta F_1(s)$}] {} (sum);
	\draw [->] (input2) -- node [label=above:{$\Delta F_2(s)$}] {} (sum);
	\draw [->] (sum) -- (tieline);
	\draw [->] (tieline) -- node[label=above:{$\Delta P_{tie,1}$}] {} (P1);
	\draw [->] (tieline) -- (a12);
	\draw [->] (a12) -- node[label=above:{$\Delta P_{tie,2}$}] {} (P2);
\end{tikzpicture}
	\caption[Tie line model]{A block diagram for the tie line connecting power area 1 and power area 2}
	\label{fig:B101_tie_line_model}
\end{figure}