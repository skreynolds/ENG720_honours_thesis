\chapter{Literature Review}
To appreciate historical developments in the field of load frequency control, this chapter first considers industry standard solutions, and the motivations for their continued use. Following this, alternative control approaches are considered including fuzzy logic control, genetic algorithms, and artificial neural networks. The chapter concludes with an analysis deep reinforcement learning applied to the load frequency control problem.

\section{Classical Controllers}\label{agc}
The first attempts at power system frequency control was using a flywheel governor attached to the synchronous machines \cite{Prabhat2005}. Originally, as discussed in \textsection \ref{sec:governor_model}, turbine governors were mechanically coupled to generator shafts providing a built-in proportional feedback control loop. As the frequency deviated from the generator's set point, the governor would open or close a steam valve to increase or decrease the rotational speed accordingly. This type of control is often referred to as the primary control loop \cite{Bevrani2011}, as discussed in \textsection \ref{sec:control_one_area_system}. Early synchronous machine operators found that primary loop governor control would require constant tuning to ensure that the power system was operating at the scheduled frequency \cite{Cohn1983}. Analysis undertaken using first-order linear models for the governor, turbine, and generation load control confirmed that governor primary loop control was able to arrest frequency deviations; however, was unable to return the system to scheduled operating frequencies \cite{Saadat2011}. Subsequently, a secondary feedback control loop was introduced which used the integral of frequency deviations, as discussed in \textsection \ref{ssec:control_two_area_system}. The PI control scheme constitutes the classical approach to solving the load frequency control problem \cite{Prabhat2005}.

Throughout the 1930s and 1940s, North American utility providers applied PI architectures to control the frequency of interconnected power systems comprised of two or more areas \cite{Cohn1983}. As detailed in \textsection \ref{ssec:control_two_area_system}, the aim was to regulate the frequency in each area in addition to preserving the power flow over the tie-lines interconnecting power system areas. In 1953, Concordia and Kirchmayer were first to present an empirical analysis on the dynamic performance of a PI controller tasked with the frequency control of a two area power system. They concluded that optimum control gains for the system existed; however, they could not determine these analytically \cite{Concordia1953}.

Cohn \cite{Cohn1971} and Aggarwal \textit{et al.} \cite{Aggarwal1968, Aggarwal1968a} undertook pioneering work to develop classical control approaches to work with power systems comprised of two or more control areas. Their work focused on controllers which were able to maintain the system frequency at scheduled values in multiple power areas, as well as maintain scheduled power flow over transmission infrastructure (tie lines) connecting the areas. Cohn's paper used a first-order linear system to model the tie line dynamics. The development of a tie line model allowed Cohn to use PI control architectures for each power system area, albeit with modified input signals. This work led to the development of the feedback signal called area control error (ACE). Cohn used ACE to ensure power systems were restored to the scheduled frequency, given a frequency deviation, and that unscheduled tie line power flows were minimised between neighbouring control areas \cite{Cohn1956}. Elgerd and Fosha were the first to deliver an optimal control concept for frequency control design of interconnected power systems \cite{Elgerd1970}. 

Classical power system frequency controllers remain a popular choice for industry given they are robust and simple and ...

They can be easily designed using Bode and Nyquist diagrams to obtain desired gain and phase margins. Root locus plots can also be used \cite{Ogat2010}. While these approaches are simple, well known, and easy for practical implementation, investigations using these approaches have resulted in control schemes that exhibit poor dynamic performance in the presence of parameter variations and nonlinearities \cite{Kundur1994, Elgerd1970, Bechert1977}.

Frequency controller analysis and design assumes plant models have linear dynamics; however, studies have shown that modern power systems display complex non-linear dynamics \cite{Concordia1957, Kwatny1975, Elgerd1994, Morsali2014}. Modern power systems are large-scale and comprised of multiple power generation sources such as thermal, hydro, and photovoltaic power. Commonly researched generator non-linearity include the effects of governor dead band (GDB) \cite{Concordia1957} and generator ramp constraints (GRC) \cite{Kwatny1975, Elgerd1994}. Moreover, modern power systems use high voltage direct current (HVDC) lines to export power over long distances, and they also feature energy storage systems such as pumped hydro or batteries \cite{Bevrani2011, Glover2012, Kothari2011, Kundur1994}. Both of these features display highly non-linear characteristics.

Linear ODE power system models capture underlying plant characteristics; however, these models are only valid within certain operating ranges. Non-linear plant characteristics mean that different linear ODE models are required as plant operating conditions change. Governor dead band is observed as a change in generator angular velocity for which there is no change in the governor valve position. GDB is generally attributed to backlash in the governor mechanism, and degrades LFC performance. GRC is a physical limitation of the turbine that imposes upper and lower boundaries on the rate of change in generating power from the turbine \cite{Morsali2014}. In recent years, frequency control methods using fuzzy logic, genetic algorithms (GAs), and artificial neural networks (ANNs), have attempted to address the problems that arise due to non-linearity.

\section{Fuzzy Logic Control}
Fuzzy logic control schemes are developed directly from power system domain experts or operators who control plant manually. Researchers have shown that a fuzzy gain scheduling PI controller can perform as well as a fixed gain controller, for frequency control of two and multi-area power systems. Additionally, it has been found that fuzzy controllers are simpler to implement \cite{Chang1997, Cam2005}. Yesil \textit{et al.} \cite{Yesil2004} proposed a self-tuning fuzzy PID controller for a two-area power system and noted improvements in controller transient performance when compared to a fuzzy gain scheduling PI controller.

\section{Genetic Algorithms}
Genetic algorithms are stochastic global search algorithms based on natural selection. In the context of power system control, GAs operate on a population of individuals. An individual is a set of control system parameters that are initially drawn at random and without knowledge of the task domain. Successive generations of individuals are developed using genetic operations such as recombination or mutation. The chance of an individual being selected for use in a genetic operation is based on an objective measure of fitness --- strong individuals are retained and weak individuals are discarded \cite{Fleming1993}.

Chang \textit{et al.} \cite{Chang1998} investigated using GA to determine fuzzy PI controller gains, which resulted in a control scheme which performed favourably when compared to a fixed-gain controller. Rekpreedapong \textit{et al.} \cite{Rerkpreedapong2003} took this further by optimising PI controller gains with GA while using linear matrix inequalities (LMI) constraints from a higher order controller. This research, performed on a three-area control system, was motivated by the belief higher order controllers are not practical for industry. Rekpreedapong \textit{et al.} concluded that the GA tuned PI controller, under LMI constraints, performed almost as well as a higher order control system. Research undertaken by Ghosal \cite{Ghoshal2004} concluded that PID control with gains optimised by GA provided better transient performance than PI control with gains optimised in the same way.

\section{Neural Networks}
Neural networks are systems that take input signals and by using many simple processing elements, produce output signals, as discussed in \textsection \ref{sec:deep_neural_networks}.

Beaufays \textit{et al.} \cite{Beaufays1994} demonstrated it was possible to used a neural network for frequency control in one and two-area power systems. A neural network was used instead of an integral controller in the classical structure and received a state variable input vector containing frequency deviation and tie-line power measurements instead of a single value ACE signal seen with classical controllers. The network was trained using gradient descent and the back propagation algorithms, resulting in better transient performance when compared with a classical PI controller. Using these results, Demiroren et al. \cite{Demiroren2001} went further by including non-linearity in the plant models. Specifically, governor deadband, reheater effects, and generating rate constraints are included and it was shown that the results obtained using the ANN controller outperformed the results of a standard PI classical control model for a two-area power system. Additional work confirmed these results for a larger four-area power system with thermal and hydro generation sources \cite{Zeynelgil2002}.

\section{Deep Reinforcement Learning}
A literature survey on deep reinforcement learning for power system applications, published in March 2020, highlighted that load frequency control is a critical aspect of operational control that is becoming more complex and challenging as renewable energy sources become more prevalent. Moreover, the survey only cited one study which focused on using DRL in the continuous action space for load frequency control \cite{Zhang2019}.

The study, published in 2019 by Yan and Xu, used a DDPG trained neural network to control the frequency of a single simulated power area consisting of a non-reheat steam turbine, wind turbine, and stochastic load demand. Yan and Xu concluded that the DDPG trained neural network was able to provide control action that resulted in reduced frequency deviation when compared to an optimally tuned PI controller \cite{Yan2019}.

