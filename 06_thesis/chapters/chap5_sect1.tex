\section{Environment Model} \label{ssec:env_modelling}
The two area power system was converted from the frequency domain to the temporal domain, as outlined in the approach from \textsection \ref{ssec:environment_and_pi}.

\begin{figure}[h]
	\centering
	\resizebox{\textwidth}{!}{%----------- Create a fancy summing block
\tikzset{add/.style n args={4}{
		minimum width=6mm,
		path picture={
			\draw[black] 
			(path picture bounding box.south east) -- (path picture bounding box.north west)
			(path picture bounding box.south west) -- (path picture bounding box.north east);
			\node at ($(path picture bounding box.south)+(0,0.13)$)     {\tiny #1};
			\node at ($(path picture bounding box.west)+(0.13,0)$)      {\tiny #2};
			\node at ($(path picture bounding box.north)+(0,-0.13)$)    {\tiny #3};
			\node at ($(path picture bounding box.east)+(-0.13,0)$)     {\tiny #4};
		}
	}
}

%----------- Block style 1
\tikzstyle{block1} = [draw, fill=white!80!green, rectangle, 
minimum height=3em, minimum width=6em, node distance=2.5cm]

%----------- Block style 2
\tikzstyle{block2} = [draw, fill=white!80!green, rectangle, 
minimum height=3em, minimum width=3em, node distance=2.5cm]

%----------- Sum style
\tikzstyle{sum} = [draw, fill=white!80!green, circle, node distance=2cm]

%----------- Input style
\tikzstyle{input} = [coordinate, node distance=4cm]

%----------- Output style
\tikzstyle{output} = [coordinate, node distance=4cm]

%----------- Pin style
\tikzstyle{pinstyle} = [pin edge={to-,thin,black}]


\begin{tikzpicture}	
	% Tie line nodes
	\node [sum, add={$-$}{}{+}{ }] (sum1) {};
	\node [block1, right of=sum1, label=above:{Tie Line}] (tieline) {$\frac{2\pi T_{12}}{s}$};
	\node [output, right of=tieline, node distance=3cm] (out) {};
	\node [coordinate, above of=out, node distance=5cm] (c1) {};
	\node [coordinate, below of=out, node distance=5cm] (c2) {};
	\node [block2, left of=c2] (a12) {$-a_{12}$};
	
	% Position a reference coordinate for drawing
	\node [coordinate, left of=sum1, node distance=2.5cm] (c3) {};
	\node [coordinate, above of=c3, node distance=0.75cm] (c4) {};
	\node [coordinate, below of=c3, node distance=0.75cm] (c5) {};
	
	% Create nodes for upper leg
	\node [block1, above of=c3, node distance=3.5cm, label=above:{Gen. Load 1}] (genload1) {$\frac{K_{gl1}}{T_{gl1}s+1}$};
	\node [coordinate, right of=genload1, node distance=1.5cm] (c6) {};
	\node [sum, left of=genload1, add={$-$}{+}{$-$}{}, node distance=2.5cm] (sum2) {};
	\node [coordinate, below of=sum2] (p11) {};
	\node [coordinate, left of=p11, node distance=0.5cm, label=left:{$\Delta P_{L1}(s)$}] (p12) {};
	\node [block1, left of=sum2, node distance=3.5cm, label=above:{Turbine 1}] (turbine1) {$\frac{K_{t1}}{T_{t1}s+1}$};
	\node [block1, left of=turbine1, node distance=4.5cm, label=above:{Governor 1}] (governor1) {$\frac{K_{g1}}{T_{g1}s+1}$};
	\node [coordinate, left of=governor1, node distance=3cm] (c8) {};
	\node [coordinate, left of=c5, node distance=13.5cm] (c10) {};
	\node [coordinate, left of=c1, node distance=21.5cm] (c12) {};
	
	
	
	% Create nodes for lower leg
	\node [block1, below of=c3, node distance=3.5cm, label=above:{Gen. Load 2}] (genload2) {$\frac{K_{gl1}}{T_{gl1}s+1}$};
	\node [coordinate, right of=genload2, node distance=1.5cm] (c7) {};
	\node [sum, left of=genload2, add={$-$}{+}{$-$}{}, node distance=2.5cm] (sum3) {};
	\node [coordinate, above of=sum3] (p21) {};
	\node [coordinate, left of=p21, node distance=0.5cm, label=left:{$\Delta P_{L2}(s)$}] (p22) {};
	\node [block1, left of=sum3, node distance=3.5cm, label=above:{Turbine 2}] (turbine2) {$\frac{K_{t2}}{T_{t2}s+1}$};
	\node [block1, left of=turbine2, node distance=4.5cm, label=above:{Governor 2}] (governor2) {$\frac{K_{g2}}{T_{g2}s+1}$};
	\node [coordinate, left of=governor2, node distance=3cm] (c9) {};
	\node [coordinate, left of=c4, node distance=13.5cm] (c11) {};
	\node [coordinate, left of=a12, node distance=19cm] (c13) {};
	
	
	% Connect the tieline nodes
	\draw [->] (sum1) -- (tieline);
	\draw (tieline) -- node [label=below:{$x_5(t)$}] {} (out);
	
	% Connect nodes in upper block
	\draw (out) -- (c1);
	\draw [->] (c1) -| (sum2);
	
	\draw [->] (governor1) -- node [label=above:{$x_2(t)$}] {} (turbine1);
	\draw [->] (turbine1) -- node [label=above:{$x_3(t)$}] {} (sum2);
	\draw [->] (sum2) -- (genload1);
	\draw [->] (genload1) -| node [label=above:{$x_4(t)$}] {} (sum1);
	\draw (c6) |- (c4);
	\draw [->] (c8) -- node [label=above:{$u_1(t)$}] {} (governor1);
	\draw (p12) -- (p11);
	\draw [->] (p11) -- (sum2);
	\draw [->] (c4) -- (c11);
	\draw [->] (c1) -- (c12);
	
	
	% Connect nodes in lower block
	\draw (out) -- (c2);
	\draw [->] (c2) -- (a12);
	\draw [->] (a12) -| (sum3);
	\draw [->] (governor2) -- node [label=below:{$x_7(t)$}] {} (turbine2);
	\draw [->] (turbine2) -- node [label=below:{$x_8(t)$}] {} (sum3);
	\draw [->] (sum3) -- (genload2);
	\draw [->] (genload2) -| node [label=below:{$x_9(t)$}] {} (sum1);
	\draw (c7) |- (c5);
	\draw [->] (c9) -- node [label=below:{$u_2(t)$}] {} (governor2);
	\draw (p22) -- (p21);
	\draw [->] (p21) -- (sum3);
	\draw [->] (c5) -- (c10);
	\draw [->] (a12) -- (c13);
	
\end{tikzpicture}}
	\caption[Two-area power system ODE derivation]{Variable assignment for a two area power system environment to model in the temporal domain}
	\label{4101_two_area_power_system_temporal_model}
\end{figure}

Suppose variables are assigned to the power system according to Figure \ref{4101_two_area_power_system_temporal_model}. The first order system of linear ordinary differential equations for the environment model is:
\begin{align}
	\dot{x}_2(t) &= \frac{1}{T_{sg_1}}\big( K_{sg_1} u_1(t) - x_2(t) \big) \label{eq:4101} \\
	\dot{x}_3(t) &= \frac{1}{T_{t_1}} \big( K_{t_1} x_2(t) - x_3(t) \big) \label{eq:4102} \\
	\dot{x}_4(t) &= \frac{1}{T_{gl_1}} \bigg( K_{gl_1} \big( x_3(t) - x_5(t) - \Delta p_{L1}(t) \big) - x_4(t) \bigg) \label{eq:4103}  \\
	\dot{x}_5(t) &= 2 \pi T_{12} \big( x_4(t) - x_9(t) \big) \label{eq:4104} \\
	\dot{x}_7(t) &= \frac{1}{T_{sg_2}}\big( K_{sg_2} u_2(t) - x_7(t) \big) \label{eq:4105} \\
	\dot{x}_8(t) &= \frac{1}{T_{t_2}} \big( K_{t_2} x_7(t) - x_8(t) \big) \label{eq:4106} \\
	\dot{x}_9(t) &= \frac{1}{T_{gl_2}} \bigg( K_{gl_2} \big( x_8(t) - x_5(t) - \Delta p_{L2}(t) \big) - x_9(t) \bigg) \label{eq:4107}
\end{align}

A full derivation of the first order linear system described by Equations \ref{eq:4101} to \ref{eq:4107} is described in Appendix \ref{app:C1_power_system_ode}. The equations were implemented as a method \verb|int_power_system_sim| in a Python class \verb|TwoAreaPowerSystemEnv|. Implementation for \verb|int_power_system_sim| is detailed in Appendix \ref{app:implementation_of_power_system_model}.

Model parameters were selected based on the most common environment parameters found in the literature review, as discussed in \textsection \ref{ssec:environment_and_pi}. The parameter value selection is shown in Table \ref{tab:5101_env_parameter_selection}.

\begin{table}[h]
	\centering
	\caption{Environment and controller parameters used for preliminary investigation experiments.}\label{tab:5101_env_parameter_selection}
	\begin{tabular}{llr}
		\toprule
		\textbf{Description} & \textbf{Parameter} & \textbf{Value} \\
		\midrule
		Speed Governor Gain & $K_{sg_1}$, $K_{sg_2}$ & 1 \\
		Speed Governor Time Constant & $T_{sg_1}$, $T_{sg_2}$ & 0.08 \\
		Turbine Gain & $K_{t_1}$, $K_{t_2}$ & 1 \\
		Turbine Time Constant & $T_{t_1}$, $T_{t_2}$ & 0.3 \\
		Generator Load Gain & $K_{gl_1}$, $K_{gl_2}$ & 120 \\
		Generator Load Time Constant & $T_{gl_1}$, $T_{gl_2}$ & 20 \\
		Tieline & $T_{12}$ & 0.1 \\
		Proportional Gain & $R_1$, $R_2$ & 2.4 \\
		Integral Gain & $K_{i_1}$, $K_{i_2}$, & -0.671 \\
		 & $b_1$, $b_2$ & 0.425 \\
		\bottomrule
	\end{tabular}\label{tab:5000}
\end{table}

A reward assigned to the agent at each timestep, for the purposes of the DDPG algorithm, was derived based on reward structures used by Yan \cite{Yan2020}. The final reward structure consisted of absolute deviations of the frequency for both power system areas, the absolute deviation of the tie-line power flow, and the absolute control effort provided by each controller.

The reward expression is shown in equation \ref{eq:5101}.

\begin{equation}
	\begin{split}
		\texttt{reward} = 	&- (|\texttt{Frequency 1}| + |\texttt{Frequency 2}|) \\
			  				&- (|\texttt{Tieline}|) \\
			  				&- (|\texttt{Control 1}| + |\texttt{Control 2}|)\label{eq:5101}
	\end{split}
\end{equation}