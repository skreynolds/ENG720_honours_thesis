\section{Environment Model} \label{ssec:env_modelling}
The two area power system was converted from the frequency domain to the temporal domain, as outlined in the approach from \textsection \ref{ssec:environment_and_pi}.

\begin{figure}[h]
	\centering
	\resizebox{\textwidth}{!}{\input{./figures/4101_two_area_power_system_temporal_model/two_area_power_system_temporal_model}}
	\caption[Two-area power system ODE derivation]{Variable assignment for a two area power system environment to model in the temporal domain}
	\label{4101_two_area_power_system_temporal_model}
\end{figure}

Suppose variables are assigned to the power system according to Figure \ref{4101_two_area_power_system_temporal_model}. The first order system of linear ordinary differential equations for the environment model is:
\begin{align}
	\dot{x}_2(t) &= \frac{1}{T_{sg_1}}\big( K_{sg_1} u_1(t) - x_2(t) \big) \label{eq:4101} \\
	\dot{x}_3(t) &= \frac{1}{T_{t_1}} \big( K_{t_1} x_2(t) - x_3(t) \big) \label{eq:4102} \\
	\dot{x}_4(t) &= \frac{1}{T_{gl_1}} \bigg( K_{gl_1} \big( x_3(t) - x_5(t) - \Delta p_{L1}(t) \big) - x_4(t) \bigg) \label{eq:4103}  \\
	\dot{x}_5(t) &= 2 \pi T_{12} \big( x_4(t) - x_9(t) \big) \label{eq:4104} \\
	\dot{x}_7(t) &= \frac{1}{T_{sg_2}}\big( K_{sg_2} u_2(t) - x_7(t) \big) \label{eq:4105} \\
	\dot{x}_8(t) &= \frac{1}{T_{t_2}} \big( K_{t_2} x_7(t) - x_8(t) \big) \label{eq:4106} \\
	\dot{x}_9(t) &= \frac{1}{T_{gl_2}} \bigg( K_{gl_2} \big( x_8(t) - x_5(t) - \Delta p_{L2}(t) \big) - x_9(t) \bigg) \label{eq:4107}
\end{align}

A full derivation of the first order linear system described by Equations \ref{eq:4101} to \ref{eq:4107} is described in Appendix \ref{app:C1_power_system_ode}. The equations were implemented as a method \verb|int_power_system_sim| in a Python class \verb|TwoAreaPowerSystemEnv|. Implementation for \verb|int_power_system_sim| is detailed in Appendix \ref{app:implementation_of_power_system_model}.

Model parameters were selected based on the most common environment parameters found in the literature review, as discussed in \textsection \ref{ssec:environment_and_pi}. The parameter value selection is shown in Table \ref{tab:5101_env_parameter_selection}.

\begin{table}[h]
	\centering
	\caption{Environment and controller parameters used for preliminary investigation experiments.}\label{tab:5101_env_parameter_selection}
	\begin{tabular}{llr}
		\toprule
		\textbf{Description} & \textbf{Parameter} & \textbf{Value} \\
		\midrule
		Speed Governor Gain & $K_{sg_1}$, $K_{sg_2}$ & 1 \\
		Speed Governor Time Constant & $T_{sg_1}$, $T_{sg_2}$ & 0.08 \\
		Turbine Gain & $K_{t_1}$, $K_{t_2}$ & 1 \\
		Turbine Time Constant & $T_{t_1}$, $T_{t_2}$ & 0.3 \\
		Generator Load Gain & $K_{gl_1}$, $K_{gl_2}$ & 120 \\
		Generator Load Time Constant & $T_{gl_1}$, $T_{gl_2}$ & 20 \\
		Tieline & $T_{12}$ & 0.1 \\
		Proportional Gain & $R_1$, $R_2$ & 2.4 \\
		Integral Gain & $K_{i_1}$, $K_{i_2}$, & -0.671 \\
		 & $b_1$, $b_2$ & 0.425 \\
		\bottomrule
	\end{tabular}\label{tab:5000}
\end{table}

A reward assigned to the agent at each timestep, for the purposes of the DDPG algorithm, was derived based on reward structures used by Yan \cite{Yan2020}. The final reward structure consisted of absolute deviations of the frequency for both power system areas, the absolute deviation of the tie-line power flow, and the absolute control effort provided by each controller.

The reward expression is shown in equation \ref{eq:5101}.

\begin{equation}
	\begin{split}
		\texttt{reward} = 	&- (|\texttt{Frequency 1}| + |\texttt{Frequency 2}|) \\
			  				&- (|\texttt{Tieline}|) \\
			  				&- (|\texttt{Control 1}| + |\texttt{Control 2}|)\label{eq:5101}
	\end{split}
\end{equation}