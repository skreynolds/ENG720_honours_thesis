\section{Development, Implementation, and Testing}
\subsection{Environment and PI Controller}
Training a neural network to act as a load frequency controller using DDPG requires a simulated model of a two area power system. A mathematical representation of the two area power system in the temporal domain will be developed by taking inverse Laplace transforms of frequency domain models from simulation experiments discussed in the literature review. The experimental design choice for developing the model in the temporal domain is motivated by the need to stop and start the simulation periodically. This is to allow the neural network to take simulation states as inputs, and issue control action outputs back to the simulation. Upon receiving a control action the simulation will iterate forward by some discrete time interval, using previous stopping states as initial conditions. This could not be achieved in the frequency domain as initial conditions are assumed to be zero.

A PI controller model will be developed to provide a performance benchmark comparison for the neural network. Whilst the PI controller could be developed in the frequency domain, for convenience it will be developed in the temporal domain to interact with the temporal model of the two area power system. The PI controller model will be mathematically specified using the same methodology described for the two area power system.

Model parameter values for the two area power system model will be selected based on a review of the most common model parameters observed in the literature review. Tuned model parameters for the PI controller will selected to ensure near optimal control performance. Developed models will be tested and verified against known results from the literature review to provide assurance of correct implementation.

\subsection{Neural Network, DDPG Controller, and Training Algorithm}


DEVELOPMENT OF NEURAL NETWORK

DEVELOPMENT OF DDPG CONTROLLER AND TRAINING ALGORITHM

\subsection{Software Implementation and Testing}
Environment models, controllers, neural networks, and training algorithms will be implemented in the Python programming language. This design choice is motivated by a number of factors:
\begin{itemize}
	\item \texttt{Scipy} is a Python library used for scientific computing. The library contains well developed numerical computation schemes and algorithms for the simulation of systems of ordinary differential equations.
	\item \texttt{PyTorch} is a Python based machine learning framework that is used extensively for neural network research prototyping. One of the main benefits of this library is that it provides access to graphical processing hardware which accelerates the neural network training process.
	\item \texttt{Gym} is a Python toolkit developed by OpenAI for developing reinforcement learning algorithms
	\item Python and the libraries mentioned above are open source, and as such do not require expensive licences for access.
\end{itemize}

HOW THE SOFTWARE WILL BE IMPLEMENTED USING AN OBJECT ORIENTED APPROACH

GENERAL HIERARCHY OF SOFTWARE IMPLEMENTATION

PROVIDE AN OUTLINE OF THE PROCESS THAT WILL BE USED TO DEVELOP SOFTWARE TO IMPLEMENT, AND ALSO INCLUDE A DESCRIPTION OF THE VERIFICATION PROCESS THAT NEEDS TO BE UNDERTAKEN IN ORDER TO TEST EACH OF THE IMPLEMENTATIONS THAT HAVE BEEN DEVELOPED.