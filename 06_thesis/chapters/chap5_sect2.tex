\section{Classical PI Controller Model}
The proportional integral (PI) controller was converted from the frequency domain to the temporal domain, as outlined in the approach from \textsection \ref{ssec:environment_and_pi}.

\begin{figure}[h]
	\begin{minipage}[b]{0.5\textwidth}
		\resizebox{7.0cm}{!}{%----------- Create a fancy summing block
\tikzset{add/.style n args={4}{
		minimum width=6mm,
		path picture={
			\draw[black] 
			(path picture bounding box.south east) -- (path picture bounding box.north west)
			(path picture bounding box.south west) -- (path picture bounding box.north east);
			\node at ($(path picture bounding box.south)+(0,0.13)$)     {\tiny #1};
			\node at ($(path picture bounding box.west)+(0.13,0)$)      {\tiny #2};
			\node at ($(path picture bounding box.north)+(0,-0.13)$)    {\tiny #3};
			\node at ($(path picture bounding box.east)+(-0.13,0)$)     {\tiny #4};
		}
	}
}

%----------- Block style 1
\tikzstyle{block1} = [draw, fill=white!80!green, rectangle, 
minimum height=3em, minimum width=6em, node distance=2.5cm]

%----------- Block style 2
\tikzstyle{block2} = [draw, fill=white!80!blue, rectangle, 
minimum height=3em, minimum width=3em, node distance=2.5cm]

%----------- Sum style
\tikzstyle{sum} = [draw, fill=white!80!blue, circle, node distance=2cm]

%----------- Input style
\tikzstyle{input} = [coordinate, node distance=4cm]

%----------- Output style
\tikzstyle{output} = [coordinate, node distance=4cm]

%----------- Pin style
\tikzstyle{pinstyle} = [pin edge={to-,thin,black}]

\begin{tikzpicture}	
	% Initial position node
	\node [coordinate] (c1) {};
	
	
	% Create nodes for upper leg
	\node [sum, above of=c1, add={$-$}{}{+}{}, node distance=3.5cm] (sum4) {};
	\node [coordinate, above of=sum4, node distance=1.25cm] (c10) {};
	\node [coordinate, right of=c10, node distance=2cm] (c12) {};
	\node [coordinate, right of=sum4, node distance=2cm] (c2) {};
	\node [coordinate, above of=c2] (c4) {};
	\node [block2, below of=sum4, node distance=1.75cm] (r1) {$R_1$};
	\node [coordinate, below of=r1] (c6) {};
	\node [coordinate, right of=c6, node distance=2cm] (c8) {};
	\node [block2, left of=sum4] (int1) {$\frac{K_{i_1}}{s}$};
	\node [sum, left of=int1, add={+}{ }{+}{ }] (sum6) {};
	\node [block2, below of=sum6, node distance=1.75cm] (b1) {$b_1$};
	
	
	% Connect nodes
	\draw [->] (sum4) -- node [at end, label=right:{$u_1(t)$}] {} (c2);
	\draw [->] (r1) -- (sum4);
	\draw [->] (b1) -- (sum6);
	\draw [->] (sum6) -- (int1);
	\draw [->] (int1) -- node [label=above:{$x_1(t)$}] {} (sum4);
	\draw [->] (c8) -| node [at start, label=right:{$x_4(t)$}] {} (r1);
	\draw [->] (c8) -| (b1);
	\draw [->] (c12) -| node [at start, label=right:{$x_5(t)$}] {} (sum4);
	\draw [->] (c12) -| (sum6);
	
\end{tikzpicture}}
		\caption[Area 1 PI controller ODE derivation]{Variable assignment for the PI controller for area 1 in order to model in the temporal domain}
		\label{fig:4102_two_area_pi_controller_temporal_1}
	\end{minipage}
	\hspace{0.1cm}
	\begin{minipage}[b]{0.5\textwidth}
		\resizebox{7.2cm}{!}{%----------- Create a fancy summing block
\tikzset{add/.style n args={4}{
		minimum width=6mm,
		path picture={
			\draw[black] 
			(path picture bounding box.south east) -- (path picture bounding box.north west)
			(path picture bounding box.south west) -- (path picture bounding box.north east);
			\node at ($(path picture bounding box.south)+(0,0.13)$)     {\tiny #1};
			\node at ($(path picture bounding box.west)+(0.13,0)$)      {\tiny #2};
			\node at ($(path picture bounding box.north)+(0,-0.13)$)    {\tiny #3};
			\node at ($(path picture bounding box.east)+(-0.13,0)$)     {\tiny #4};
		}
	}
}

%----------- Block style 1
\tikzstyle{block1} = [draw, fill=white!80!blue, rectangle, 
minimum height=3em, minimum width=6em, node distance=2.5cm]

%----------- Block style 2
\tikzstyle{block2} = [draw, fill=white!80!blue, rectangle, 
minimum height=3em, minimum width=3em, node distance=2.5cm]

%----------- Sum style
\tikzstyle{sum} = [draw, fill=white!80!blue, circle, node distance=2cm]

%----------- Input style
\tikzstyle{input} = [coordinate, node distance=4cm]

%----------- Output style
\tikzstyle{output} = [coordinate, node distance=4cm]

%----------- Pin style
\tikzstyle{pinstyle} = [pin edge={to-,thin,black}]

\begin{tikzpicture}	
	% Initial position node
	\node [coordinate] (c1) {};
	
	
	% Create nodes for lower leg
	\node [sum, below of=c1, add={+}{}{$-$}{}, node distance=3.5cm] (sum5) {};
	\node [coordinate, below of=sum5, node distance=1.25cm] (c11) {};
	\node [coordinate, right of=c11, node distance=2cm] (c13) {};
	\node [coordinate, right of=sum5, node distance=2cm] (c3) {};
	\node [coordinate, above of=c3] (c5) {};
	\node [block2, above of=sum5, node distance=1.75cm] (r2) {$R_2$};
	\node [coordinate, above of=r2] (c7) {};
	\node [coordinate, right of=c7, node distance=2cm] (c9) {};
	\node [block2, left of=sum5] (int2) {$\frac{K_{i_2}}{s}$};
	\node [sum, left of=int2, add={+}{ }{+}{ }] (sum7) {};
	\node [block2, above of=sum7, node distance=1.75cm] (b2) {$b_2$};
	
	
	% Connect nodes
	\draw [->] (sum5) -- node [at end, label=right:{$u_2(t)$}] {} (c3);
	\draw [->] (r2) -- (sum5);
	\draw [->] (b2) -- (sum7);
	\draw [->] (sum7) -- (int2);
	\draw [->] (int2) -- node [label=below:{$x_6(t)$}] {} (sum5);
	\draw [->] (c9) -| node [at start, label=right:{$x_9(t)$}] {} (r2);
	\draw [->] (c9) -| (b2);
	\draw [->] (c13) -| node [at start, label=right:{$-x_5(t)$}] {} (sum5);
	\draw [->] (c13) -| (sum7);
	
\end{tikzpicture}}
		\caption[Area 2 PI controller ODE derivation]{Variable assignment for the PI controller for area 2 in order to model in the temporal domain}
		\label{fig:4103_two_area_pi_controller_temporal_2}
	\end{minipage}
\end{figure}

Suppose variables are assigned to the controller for area 1 and the controller for area 2 according to Figures \ref{fig:4102_two_area_pi_controller_temporal_1} and \ref{fig:4103_two_area_pi_controller_temporal_2}, respectively. The first order system of linear differential equations for the PI controller are:
\begin{align}
	\dot{x}_1(t) &= b_1 \Delta f_1(t) + x_5(t) \label{eq:4108} \\
	\dot{x}_6(t) &= b_2 \Delta f_2(t) - x_5(t) \label{eq:4109}
\end{align}

Note that once the PI controller has stepped forward in time during simulation the actual control signals exported from the controller are $u_1(t)$ and $u_2(t)$. These are expressed as:
\begin{align}
	u_1(t) &= x_1(t) + x_5(t) - R_1 x_4(t) \\
	u_2(t) &= x_6(t) - x_5(t) - R_2 x_9(t) 
\end{align}

A full derivation of the first order linear system described by equations \ref{eq:4108} and \ref{eq:4109} is described in Appendix \ref{sec:temporal_domain_for_pi_controller}. The system of equations is implemented as a method \verb|int_control_system_sim| in a Python class \verb|ClassicalPiController|. Implementation for \verb|int_control_system_sim| is detailed in Appendix \ref{sec:implementation_of_controller}.

Model parameters were selected from research undertaken by Nanda and Kaul which optimised the controller using an integral squared error technique \cite{Nanda1977}. These PI controller values are widely used as a basis for comparison against experimental control techniques in the load frequency control literature. The parameter value selection for the PI controller is shown in Table \ref{tab:5201}.

\begin{table}[h]
	\centering
	\caption[PI controller parameters]{PI controller parameters used for simulation experiments.}\label{tab:5201_pi_parameter_selection}
	\begin{tabular}{llr}
		\toprule
		\textbf{Description} & \textbf{Parameter} & \textbf{Value} \\
		\midrule
		Proportional Gain & $R_1$, $R_2$ & 2.4 \\
		Integral Gain & $K_{i_1}$, $K_{i_2}$, & -0.671 \\
		 & $b_1$, $b_2$ & 0.425 \\
		\bottomrule
	\end{tabular}\label{tab:5201}
\end{table} 