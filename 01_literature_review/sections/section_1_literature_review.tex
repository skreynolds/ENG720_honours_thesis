\section{Literature Review}

% INTRODUCTORY PARAGRAPH
% what is the motivation for this research and what are the key questions that this paper is trying to answer
Traditional control structures for maintaining power system frequency, like automatic generation control (AGC), operate based on an assumption of linear plant models. An increasing proportion of photovoltaic sources of generation, and an increase in the use of high voltage direct current (HVDC) transmission lines to Australia's power network means power system dynamics are becoming more non-linear. This is driving a need to explore applications of novel control architectures to AGC for increased control performance under non-linear plant conditions. One such control architecture being investigated is Deep Reinforcement Learning (DRL). To fully grasp this problem two key areas of understanding are necessary. Firstly, it is important to know what AGC is and the control architectures that have been explored to address this problem. Secondly, knowledge of DRL and its historical applications to control problems is required to understand strengths, limitations, and underlying assumptions of the architecture. This review addresses both aspects in \S{}\S{} \ref{agc} and \ref{drl}, respectively.


\subsection{Automatic Generation Control}\label{agc}

% PARAGRAPH ON WHY AUTOMATIC GENERATION CONTROL IS NEEDED AND WAS DEVELOPED

% PARAGRAPH ON THE HISTORICAL DEVELOPMENT OF AUTOMATIC GENERATION CONTROL

% PARAGRAPH ON THE DIFFERENT TYPES OF CONTROLLERS THAT HAVE BEEN DEVELOPED FOR AGC

% MAY NEED MULTIPLE PARAGRAPHS HERE LISTING OUT EACH TYPE OF CONTROLLER AND EVALUATING THE RESULTS OF EACH

% PARAGRAPH ON RL CONTROLLERS THAT HAVE BEEN DEVELOPED

% PARAGRAPH ON GAPS THAT ARE PRESENT IN THE DRL SPACE FOR AGC - NO RESEARCH HAS BEEN PREVIOUSLY CONDUCTED HERE



\subsection{Deep Reinforcement Learning}\label{drl}

% PARAGRAPH ON WHAT REINFORCEMENT LEARNING IS AND WHAT DEEP REINFORCEMENT LEARNING IS

% PARAGRAPH ON WHAT HISTORICAL APPLICATIONS TO CONTROL ACTIVITIES

% PARAGRAPH ON 