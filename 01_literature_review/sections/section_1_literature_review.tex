\section{Literature Review}

% INTRODUCTORY PARAGRAPH
% what is the motivation for this research and what are the key questions that this paper is trying to answer
Traditional control structures for maintaining power system frequency, like automatic generation control (AGC), operate based on an assumption of linear plant models. An increasing proportion of photovoltaic sources of generation, and an increase in the use of high voltage direct current (HVDC) transmission lines to Australia's power network means power system dynamics are becoming more non-linear. This is driving a need to explore applications of novel control architectures to AGC for increased control performance under non-linear plant conditions. One such control architecture being investigated is Deep Reinforcement Learning (DRL). To fully grasp this problem two key areas of understanding are necessary. Firstly, it is important to know what AGC is and the control architectures that have previously been explored to address this problem. Secondly, knowledge of DRL and its historical applications to control problems is required to understand strengths, limitations, and underlying assumptions of the architecture. This review addresses both aspects in \S{}\S{} \ref{agc} and \ref{drl}, respectively.


\subsection{Automatic Generation Control}\label{agc}

% PARAGRAPH ON WHY AUTOMATIC GENERATION CONTROL IS NEEDED AND WAS ORIGINALLY DEVELOPED IN ANALOGUE FORM
Thomas Edison's commercial power station, commissioned in 1882 at 255-257 Pearl Street New York, was one of the first instances of grid connected generators to use a control system for power system frequency management. The facility consisted of six 100$\si{\kilo\watt}$ generators, each of which was equipped with a speed governor.

% What is load frequency control %
% Concerned with many control objectives but in particular the control of frequency on a real time basis, but with response times from a few seconds to a few minutes to meet th actual load within security and economy\\

% Analog controller development %
% Cohn did a lot of the pinoeering work in analog control system design

% Digital system development %
% Digital systems were developed in parallel with analog systems for a period of time replacing analog completely in modern power systems seen today

% Digital systems make reference to local control loops for frequency and referred to an primary control mechanisms. Frequency is kept near its nominal value using a proportional regulator action on a turbine valve. This type of control is widely used, however, it will leave a slight error for the controlled variable. In the analog days regulator set points we set manually.

% Local primary frequency control presents two main drawbacks: a substantial error may be left for frequency and, when a load change occurs, power 

% Classical control approaches were developed originally in Cohn's analog era, and adapted and improved upon in the subsequent digital era

% There are a number of competing techniques that have been explored with varying degrees of success. These include optimal controllers using state feedback models; sub-optimal controllers; and adaptive and self tuning controllers.





\subsection{Deep Reinforcement Learning}\label{drl}

% What is deep reinforcement learning when was it first used and what previous reserch is it derived from

% What reaserach has been undertaken in the theory on how to develop agents to perform sequential decision making

% What are the most common architectures and what are their benefits and limitations

% What are the main applications of deep reinformcement learning

% 