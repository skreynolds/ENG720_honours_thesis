\section{Literature Review}
Traditional control structures for maintaining power system frequency, like load frequency control (LFC) or automatic generation control (AGC), operate based on an assumption of linear plant models. An increasing proportion of photovoltaic sources of generation, and an increase in the use of high voltage direct current (HVDC) transmission lines to Australia's power network means power system dynamics are becoming more non-linear. This is driving a need to explore applications of novel control architectures to AGC for increased control performance under non-linear plant conditions. One such control architecture being investigated is Deep Reinforcement Learning (DRL). To fully grasp this problem two key areas of understanding are necessary. Firstly, it is important to know what LFC is and the control architectures that have previously been explored to address this problem. Secondly, knowledge of DRL and its historical applications to control problems is required to understand strengths, limitations, and underlying assumptions of the architecture.

This review addresses both aspects in \S{}\S{} \ref{agc} and \ref{drl}, respectively.


\subsection{Automatic Generation Control}\label{agc}
Since Thomas Edison's first commercial power station, commissioned in 1882 at 255-257 Pearl Street New York, the control of power frequency has been an important consideration for power generation \cite{Cohn1983}. In contrast to Edison's six 100$\si{\kilo\watt}$ generator facility, modern power systems are usually large-scale systems comprised of multiple generation sources such as thermal, hydro, and photovoltaic. Energy storage systems like pumped hydro and batteries are often featured, and high voltage direct current lines are often used when exporting power over long distances \cite{Bevrani2011, Glover2012, Kothari2011, Kundur1994}. Studies have shown that modern power systems display complex non-linear dynamics \cite{Concordia1957, Kwatny1975, Elgerd1994, Morsali2014}; however, the majority of research performed to date has used linearised models of single, two, or multi-area power systems \cite{Kothari2011, Saadat2011, Cohn1956, Bevrani2011, Wood2013, Elgerd1970, Kwatny1975}.

One of the first attempts to control the frequency of a single area power system under a linear model assumption was using a turbine governor with a proportional control loop. This is often referred to as primary control \cite{Bevrani2011}. The approach saw some success in arresting frequency deviations from the desired set point, but was ultimately found to be insufficient due to a persistent offset error from the set point \cite{Saadat}. Later research concluded that a supplementary integral control loop to the governor was required to provide sufficient frequency control \cite{Elgerd1970}. The combination of primary and supplementary, or proportional and integral control, is referred to as PI control or AGC. This scheme constitutes the classical approach to the solution of the LFC problem.

Cohn \cite{Cohn1971} and Aggarwal et al. \cite{Aggarwal1968, Aggarwal1968a} undertook pioneering work to develop classical control approaches to work with power systems comprised of two or more control areas. Cohn's papers referred to this as a tie line bias control strategy. One of the important things to come out of his work was the use of a control feedback signal called area control error (ACE). Cohn used ACE to ensure zero steady state error for frequency deviations, and to minimise unscheduled tie line power flows between neighbouring control areas \cite{Cohn1956}. Classical power system frequency controllers can be designed using Bode and Nyquist diagrams to obtain desired gain and phase margins. Root locus plots can also be used \cite{Ogat2010}. While these approaches are simple, well known, and easy for practical implementation, investigations using these approaches have resulted in control schemes that exhibit poor dynamic performance. This is especially true in the presence of parameter variations and nonlinearities \cite{Kundur1994, Elgerd1970, Bechert1977}.

Linear power system models capture some of the underlying plant characteristics; however, these models are only valid within certain operating ranges. Non-linear plant characteristics mean that different linear models are required as plant operating conditions change. Some of the more common non-linearities that impact LFC efforts include governor dead band (GDB) \cite{Concordia1957} and generator ramp constraint (GRC) \cite{Kwatny1975, Elgerd1994}. Governor dead band is observed as a change in generator angular velocity for which there is no change in the governor valve position. GDB is generally attributed to backlash in the governor mechanism, and degrades LFC performance (REFERENCE). Generation rate constraint is a physical limitation of the turbine which imposes upper and lower boundaries on the rate of change in generating power a turbine can produce \cite{Morsali2014}. In recent years, frequency control methods using fuzzy logic, genetic algorithms (GA), and artificial neural networks (ANN), have gone some way to addressing the problems which arise due to non-linearity.

Fuzzy logic control schemes are developed directly from power system domain experts or operators who control plant manually (REFERENCE). Researchers have shown that a fuzzy gain scheduling PI controller can perform as well as a fixed gain controller, for frequency control of two and multi-area power systems. Moreover, it was found fuzzy controllers are simpler to implement \cite{Chang1997, Cam2005}. Yesil et al. \cite{Yesil2004} proposed a self tuning fuzzy PID controller for a two area power system and noted improvements in controller transient performance when compared to a fuzzy gain scheduling PI controller.

Genetric algorithms are stochastic global search algorithms based on natural selection. In the context of power system control, GAs operate on a population of individuals. An individual is a set of control system parameters which are initially drawn at random and without knowledge of the task domain. Successive generations of individuals are developed using genetic operations such as recombination or mutation. An individuals chance of being selected for used in an genetic operation is based on an objective measure of fitness --- strong individuals are retained and weak individuals are discarded \cite{Fleming1993}.

Chang et al. \cite{Chang1998} investigated using GA to determine fuzzy PI controller gains, which resulted in a control scheme which performed favourably when compared to a fixed-gain controller. Rekpreedapong et al. \cite{Rerkpreedapong2003} took this one step further by optimally tuning PI controller gains with GA while using linear matrix inequalities (LMI) constraints from a higher order controller. This research, performed on a three area control system, was motivated by the belief higher order controllers are not practical for industry. Rekpreedapong et al. concluded that the GA tuned PI controller, under LMI constraints, performed almost as well a higher order control system. Research undertaken by Ghosal \cite{Ghoshal2004} concluded that PID control with gains optimised by GA provided better transient performance than PI control with gains optimised in the same way.

Artificial neural networks are systems that take input signals and, using many simple processing elements, produce output signals. The processing elements, or neurons, each have a number of internal parameters referred to as weights. Changing a weight will change the behaviour of a neuron. If many weights are changed, the behaviour of the ANN can be changed. The goal is to choose weights of the network in order to achieve the desired input/output relationship --- this is called training the network \cite{Nguyen1990}.

Beaufays et al. \cite{Beaufays1994} demonstrated it was possible to used a neural network for frequency control in one and two-area power systems. The ANN replaced the integral controller in the classical structure; however, employed a state variable vector input containing frequency deviation and tie-line power measurements instead of a single value ACE signal seen with classical controllers. The network was trained using a back propagation through time algorithm, and resulted in better transient performance when compared with a classical PI controller. Using these results, Demiroren et al. \cite{Demiroren2001} went further by including non-linearity in the plant models. Specifically, governor deadband, reheater effects, and generating rate constraints are included and it was shown that the results obtained using the ANN controller outperformed the results of a standard PI classical control model for a two-area power system. Research undertaken a year later confirmed these results for a larger four-area power system with thermal and hydro generation sources \cite{Zeynelgil2002}. 

% % % % % % % % % % % % % % % % % % % % % % % % % % % % % % % % % % % % % % % % % %
% Need to include some evaluation of the above fuzzy logic, GA, and ANN approaches
% % % % % % % % % % % % % % % % % % % % % % % % % % % % % % % % % % % % % % % % % %

\subsection{Deep Reinforcement Learning}\label{drl}

The AGC problem can be viewed as a stochastic multi-stage decision-making problem or a Markov Chain control problem, and algorithms have been presented for designing AGC based on a reinforcement learning approach \cite{Ahamed2002}.

% What is deep reinforcement learning when was it first used and what previous reserch is it derived from

% What reaserach has been undertaken in the theory on how to develop agents to perform sequential decision making

% What are the most common architectures and what are their benefits and limitations

% What are the main applications of deep reinformcement learning

%