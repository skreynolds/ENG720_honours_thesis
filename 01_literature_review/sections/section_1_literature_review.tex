\section{Literature Review}
Traditional control structures for maintaining power system frequency, like load frequency control (LFC) or automatic generation control (AGC), operate based on an assumption of linear plant models. An increasing proportion of photovoltaic sources of generation, and an increase in the use of high voltage direct current (HVDC) transmission lines to Australia's power network means power system dynamics are becoming more non-linear. This is driving a need to explore applications of novel control architectures to AGC for increased control performance under non-linear plant conditions. One such control architecture being investigated is Deep Reinforcement Learning (DRL). To fully grasp this problem two key areas of understanding are necessary. Firstly, it is important to know what LFC is and the control architectures that have previously been explored to address this problem. Secondly, knowledge of DRL and its historical applications to control problems is required to understand strengths, limitations, and underlying assumptions of the architecture. This review addresses both aspects in \S{}\S{} \ref{agc} and \ref{drl}, respectively.


\subsection{Automatic Generation Control}\label{agc}
Since Thomas Edison's first commercial power station, commissioned in 1882 at 255-257 Pearl Street New York, the control of power frequency has been an important consideration for power generation \cite{Cohn1983}. In contrast to Edison's six 100$\si{\kilo\watt}$ generator facility, modern power systems are usually large-scale systems comprised of multiple generation sources such as thermal, hydro, and photovoltaic. Energy storage systems like pumped hydro and batteries are often featured, and high voltage direct current lines can feature when exporting power over long distances \cite{Bevrani2011, Glover2012, Kothari2011, Kundur1994}. Studies have shown that modern power systems display complex non-linear dynamics (REFERENCE); however, the majority of research performed to date has used linearised models of single, two, or multi-area power systems \cite{Kothari2011, Saadat2011, Cohn1956, Bevrani2011, Wood2013, Elgerd1970, Kwatny1975}. One of the first attempts to control the frequency of a single area power system under a linear model assumption was using a turbine governor with a proportional control loop. This is referred to as primary control (REFERENCE). The approach saw some success in arresting frequency deviations from the desired set point, but was ultimately found to be insufficient due to a persistent offset error from the set point (REFERENCE). Later research concluded that a supplementary integral control loop to the governor was required to provide sufficient frequency control (REFERENCE). The combination of primary and supplementary, or proportional and integral control, is referred to as PI control (REFERECE). This scheme constitutes the classical approach to the solution of the LFC problem. In the 1950s Cohn (REFERENCE[12]) and Aggarwal et al. (REFERENCE[11]) undertook pioneering work to develop the classical control approach to work with power systems consisting of two or more control areas. Cohn's papers referred to this as a tie line bias control strategy. One of the important things to come out of his work was the use of a control feedback signal called area control error (ACE). Cohn used ACE to ensure zero steady state error for frequency deviations, and to minimise unscheduled tie line power flows between neighbouring control areas (REFERENCE). Classical power system frequency controllers can be designed using Bode and Nyquist diagrams to obtain desired gain and phase margins (REFERENCE). Root locus plots can also be used (REFERENCE). While these approaches are simple, well known, and easy for practical implementation, investigations using these approaches have resulted in control schemes that exhibit poor dynamic performance. This is especially true in the presence of parameter variations and nonlinearities (REFERENCE[16] and [58–60]).

Linear power system models capture some of the underlying plant characteristics; however, these models are only valid within certain operating ranges (REFERENCE). Non-linear plant characteristics mean that different linear models are required as plant operating conditions change. Some of the more common non-linearities referenced in the literature include governor ramp constraint (GRC) (REFERENCE [9,10]), governor dead band (REFERENCE), OTHER NON-LINEAR FACTORS THAT COULD BE REFERENCED. In recent years, frequency control methods using fuzzy logic, genetic algorithms, and artificial neural networks, have gone some way to addressing the problems which arise due to non-linearity.

Fuzzy logic control schemes are developed directly from power system domain experts or operators who control plant manually ([95–100] REFERENCE). Researchers have shown that a fuzzy gain scheduling PI controller can perform as well as a fixed gain controller, while being simpler to implement [96,97]. Yesil et al. (REFERENCE [98]) proposed a self tuning fuzzy PID controller for a two area power system and noted improvements in controller transient performance when compared to a fuzzy gain scheduling PI controller.

Genetic algorithms are search algorithms based on natural selection. Generations comprised of individual sets of control parameters are created, often randomly, without knowledge of the task domain. Successive generations are developed based the fitness of evaluated individuals --- strong individuals are retained and weak individuals are discarded. Chang et al. REFERENCE([101]) investigated using a genetic algorithm (GA) to determine fuzzy PI controller gains, which resulted in a control scheme which performed favourably when compared to a fixed-gain controller.