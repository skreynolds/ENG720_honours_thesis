\section{Scope}
The proposed research is only concerned with the task of load following using DRL and classical control agents. Load following is defined as maintaining system frequency within PWC's allowable region of 49.80$\si{\hertz}$ to 50.20$\si{\hertz}$ for normal operating conditions. Task involving frequency restoration following a disturbance event may be considered, time permitting. The key performance aspects that will assessed, for DRL and classical, are the controller's ability to:
\begin{itemize}
	\item maintain system frequency to the desired nominal 50$\si{\hertz}$ value;
	\item maintain the tie line power flow between control areas at the scheduled value.
\end{itemize}

The research will not consider power systems larger than a two area power system. Each power area will consist of one regulating generator, and one stochastically fluctuating demand profile. The research will not consider the control of system variables other than frequency, however, note that other system variables may be used as input features under agent training and inference regimes. Comparison of DRL agent performance will be made against theoretical models of classical control architectures. Performance against practical control architectures implemented by PWC (or other utilities) will not be considered. Research will be conducted in an entirely simulated environment. Agent performance on real hardware will not be explored.