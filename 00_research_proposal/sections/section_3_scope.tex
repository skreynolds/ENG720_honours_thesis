\section{Scope}
The proposed research is primarily concerned with the task of load following using DRL and classical control agents. Load following is defined as maintaining system frequency within PWC's allowable region of 49.80 to 50.20$\si{\hertz}$ for normal operating conditions. Tasks involving frequency restoration following a disturbance event may be considered, time permitting. The key performance aspects that will assessed are the controller's ability to:
\begin{itemize}
	\item maintain system frequency to the desired nominal 50$\si{\hertz}$ value;
	\item maintain the tie line power flow between control areas at a scheduled value.
\end{itemize}

The research will focus on two area power systems. Each power area will consist of one regulating generator, and one stochastically fluctuating demand profile. The research will primarily consider the role frequency in the system; however, other system variables may be used as input features under agent training and inference regimes. Comparison of DRL agent performance will be made against theoretical models of classical control architectures. Performance against practical control architectures implemented by PWC (or other utilities) will not be considered. Research will be conducted in a simulated environment. Agent performance on real hardware will not be explored.